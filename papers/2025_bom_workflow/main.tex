% !TeX program = lualatex

% --------- ① IEEEtran より前に NFSS を用意(ptm/phv/pcr を先に作る)---------
\RequirePackage{fontspec}                 % ← \documentclass より先
\setmainfont{TeX Gyre Termes}[Ligatures=TeX,NFSSFamily=ptm]
\setsansfont{TeX Gyre Heros}[Ligatures=TeX,NFSSFamily=phv]
\setmonofont{TeX Gyre Cursor}[NFSSFamily=pcr]

% --------- クラス -----------
\documentclass[conference]{IEEEtran}

% --------- 日本語 ----------
\usepackage{luatexja}
\usepackage{luatexja-fontspec}
% 太字形が揃っていて警告が少ない HaranoAji を既定に
\setmainjfont{HaranoAjiMincho}[Script=CJK, YokoFeatures={JFM=ujis}, TateFeatures={JFM=ujisv}]

% --------- 数式 ----------
\usepackage{newtxmath}

% --------- 他パッケージ ----------
\usepackage{amsmath}
\usepackage{graphicx}
\usepackage{xcolor}
\usepackage{cite}
\usepackage{bookmark}          % ← アウトライン再走警告の抑止
\usepackage[hidelinks]{hyperref}

% --------- ② 和文 small caps/italic 未定義 警告の無害化 ----------
% small caps/italic を存在する字形にサブスティチュート(警告を出さない)
\makeatletter
% JY3(横組)/ JT3(縦組)の小形(sc)を通常(n)へフォールバック
\DeclareFontShape{JY3}{mc}{m}{sc}{<-> ssub * mc/m/n}{}
\DeclareFontShape{JT3}{mc}{m}{sc}{<-> ssub * mc/m/n}{}
% 太字小形も同様
\DeclareFontShape{JY3}{mc}{b}{sc}{<-> ssub * mc/b/n}{}
\DeclareFontShape{JT3}{mc}{b}{sc}{<-> ssub * mc/b/n}{}
% 斜体(it)を立体(n)へフォールバック
\DeclareFontShape{JY3}{mc}{m}{it}{<-> ssub * mc/m/n}{}
\DeclareFontShape{JT3}{mc}{m}{it}{<-> ssub * mc/m/n}{}
\DeclareFontShape{JY3}{mc}{b}{it}{<-> ssub * mc/b/n}{}
\DeclareFontShape{JT3}{mc}{b}{it}{<-> ssub * mc/b/n}{}
\makeatother

% --------- ここから本文 ---------
\title{%
設計から量産部品発注に至る一般的実務フローとBOM運用ルールの体系化\\
\large Systematization of Workflow from Design to Mass Production Ordering and Rules for BOM Operation
}

\author{%
  \IEEEauthorblockN{三溝 真一 (Shinichi Samizo)}%
  \IEEEauthorblockA{独立系半導体研究者(元セイコーエプソン) / Independent Semiconductor Researcher (ex-Seiko Epson)\\%
  Email: \href{mailto:shin3t72@gmail.com}{shin3t72@gmail.com}\quad
  GitHub: \url{https://github.com/Samizo-AITL}}%
}

\begin{document}
\maketitle

\begin{abstract}
本研究は、製造業における「設計図面検討~量産部品発注」に至る一般的な実務プロセスを教育目的で抽象化し、部品表(BOM)を中核に据えた \textbf{部品コード体系(6桁+枝番)・属性統合・積み上げ管理} の一体フレームを提示する。環境(RoHS/REACH)、コスト、輸出管理(ECCN/HS/用途説明)をBOMで同時に扱う点に新規性がある。
\end{abstract}

\begin{IEEEkeywords}
BOM, Part Numbering, Roll-up Management, Export Control, RoHS, REACH, ECCN, HS Code, PLM, ERP
\end{IEEEkeywords}

\section{Introduction}
設計・調達・品質保証・輸出管理が分断されがちな現場において、暗黙知となりやすいコード運用と属性連携を形式知化する。本稿の目的は以下である。
\begin{enumerate}
  \item 一般的な設計~調達~発注フローの体系化
  \item BOM中心のデータ連携ルールの提示
  \item 教育・実務の両立性の検証
\end{enumerate}

\section{General Workflow}
設計図面検討 $\rightarrow$ 技術図面展開 $\rightarrow$ 部署配布 $\rightarrow$ BOM接続 $\rightarrow$ 調達・発注。  
環境・コスト・輸出属性を並列に付与・更新することで、監査対応力と教育効果を高める。

\section{BOM Generation and Structured Data}
BOMは単なる部品リストではなく、設計~調達~輸出をつなぐ共通言語である。  
親子関係・数量・参照図面ID・属性(環境/コスト/輸出)を最小単位で保持し、部品更新時に積み上げ判定を再実行する。

\section{Part Numbering System}
6桁コードを機能識別子、枝番を条件差管理とする。例:\texttt{ABCDEF-XX}。  
桁別カテゴリ:1=機械部品、2=電子部品、6=材料(SDS必須)、9=治具。  
機能が変わる場合は新しい6桁コードを発行する。

\section{Attributes and Export Control}
図面・環境・コスト・輸出・消防法を属性として統合。  
特に材料コード(先頭6)は危険物・環境・輸出の要配慮領域。  
SDS更新時は必ずBOMを再評価する。

\section{Roll-up Management}
部品属性 $\rightarrow$ サブAssy $\rightarrow$ 製品BOMへ集約。  
製品レベルで環境適合(RoHS/REACH)、コスト合計、輸出可否(ECCN/HS/用途)を同時判定する。

\section{Rules for BOM Operation}
設計変更時のルール:
\begin{itemize}
  \item 機能変化なし:枝番更新(軽微変更)
  \item 機能変化あり:新6桁コード+新図面
  \item 材料コード:SDS・消防法判定必須、更新時はロールアップ再評価
\end{itemize}

\section{Discussion}
教育的効果:新人・学生の俯瞰理解を支援。  
実務的効果:監査対応力・リスク低減。  
課題:ERP/PLM実装差、規制改訂への追随コスト。

\section{Conclusion}
BOMを統合情報基盤と再定義し、設計から輸出管理に至るプロセスを統合的に扱う教育・実務両立フレームを提示した。  
今後はAIによる属性自動付与や国際標準との接続を進める。

\section*{Acknowledgment}
本稿は公開可能な一般論に限定し、特定企業の機密情報は一切含まない。

\bibliographystyle{IEEEtran}
\bibliography{refs} % ← refs.bib を使わないならこの2行は外す

\section*{著者略歴 / Author Biography}
\noindent\textbf{三溝 真一}(Shinichi Samizo)は、信州大学大学院 工学系研究科 電気電子工学専攻にて修士号を取得した。その後、セイコーエプソン株式会社に勤務し、半導体ロジック/メモリ/高耐圧インテグレーション、そして、インクジェット薄膜ピエゾアクチュエータ及び PrecisionCore プリントヘッドの製品化に従事した。現在は独立系半導体研究者として、プロセス/デバイス教育、メモリアーキテクチャ、AIシステム統合などに取り組んでいる。連絡先: \href{mailto:shin3t72@gmail.com}{shin3t72@gmail.com}.
\end{document}
