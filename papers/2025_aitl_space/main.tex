\documentclass[conference]{IEEEtran}

% ---- Fonts & math (LuaLaTeX対応) ----
\usepackage{newtxtext,newtxmath}

% ---- Graphics & color ----
\usepackage{graphicx}
\usepackage[dvipsnames]{xcolor}

% ---- Spacing around floats (tight for 2 pages) ----
\setlength{\textfloatsep}{5pt plus 1pt minus 1pt}
\setlength{\floatsep}{5pt plus 1pt minus 1pt}
\setlength{\intextsep}{5pt plus 1pt minus 1pt}

% ---- Tables ----
\usepackage{booktabs}

% ---- Float control ----
\usepackage{placeins}

% ---- TikZ ----
\usepackage{tikz}
\usetikzlibrary{
  arrows.meta,
  positioning,
  fit,
  calc,
  shapes.geometric,
  shapes.misc
}

% ---- TikZ styles (NOTE: step -> flowstep to avoid key clash) ----
\tikzset{
  line/.style={-Latex, line width=0.45pt},
  box/.style={draw, rounded corners=2pt, align=center, inner sep=3pt, minimum height=7.2mm},
  smallbox/.style={box, minimum width=26mm},
  midbox/.style={box, minimum width=32mm},
  bigbox/.style={box, minimum width=64mm},
  flowstep/.style={box, minimum width=30mm} % <— was 32mm; a bit tighter
}

% ---- Convenience ----
\newcommand{\etal}{\textit{et~al.}}

% =========================================================
\begin{document}

\title{AITL on Space: A Robust Three-Layer Architecture\\
with a Tri-NVM Hierarchy (SRAM / MRAM / FRAM)\\
for Long-Duration Spacecraft Autonomy}

\author{\IEEEauthorblockN{Shinichi Samizo}\\
\IEEEauthorblockA{Independent Semiconductor Researcher\\
Former Engineer at Seiko Epson Corporation\\
Email: shin3t72@gmail.com\quad GitHub: \url{https://github.com/Samizo-AITL}}
}

\maketitle

\begin{abstract}
\noindent
We propose \textit{AITL on Space}, a robust three-layer control architecture (Robust Core, FSM Supervisor, AI Adaptor) integrated with a tri-NVM hierarchy (SRAM/MRAM/FRAM) and mapped to a 22\,nm FDSOI SoC. The contribution is a complete end-to-end design flow from mission-level specification to ASIC: requirements are formalized as JSON via EduController, synthesized by the AITL-H module, validated in FPGA HIL with fault injection, stress-tested through SystemDK FEM (thermal/radiation/packaging), and finally implemented as ASIC. This methodology enables resilient autonomy for long-duration spacecraft missions.
\end{abstract}

\section{Introduction}
Deep-space missions require ultra-robust control under total ionizing dose (TID), single event effects (SEE), and thermal cycling. Conventional PID\,+\,Flash architectures face lifetime limits due to charge-trap drift and endurance. We present \emph{AITL on Space}: a resilient three-layer architecture with a tri-NVM hierarchy and a reproducible design flow from specification to ASIC.

\section{Specification and Design Flow}
The process begins with \textbf{Mission Specification}. Requirements such as pointing accuracy, power stability, and thermal tolerance are captured in \textbf{EduController}, a model-based tool that exports plant matrices and weighting functions as JSON (portable across simulators). The JSON is consumed by \textbf{AITL-H}, which synthesizes an $H_\infty$ output-feedback controller $K$ with mixed-sensitivity weighting and generates a fixed-point implementation for RTL/FPGA/ASIC. The design then undergoes:
\begin{itemize}[itemsep=0pt,topsep=1pt,leftmargin=*]
  \item \textbf{FPGA HIL}: hardware-in-the-loop validation with SEU\,+\,outage injection; metrics include safe-mode entry $<\!1$\,s, recovery rate $\ge 99\%$, and ECC scrubbing efficiency.
  \item \textbf{SystemDK FEM}: co-simulation of thermal cycles, radiation effects, and packaging stress, closing the verification loop before silicon.
  \item \textbf{ASIC Mapping}: implementation on GlobalFoundries 22FDX FDSOI hardened for long-duration missions.
\end{itemize}
\textit{Toolchain at a Glance}. \emph{EduController} $\rightarrow$ spec\,$\rightarrow$\,JSON exporter (plant $A,B,C,D$, noise/disturbance models, weights $W_1,W_2,W_3$). \emph{AITL-H}: $H_\infty$ synthesizer (Riccati/LMI)\,$\rightarrow$\,fixed-point $K$. \emph{SystemDK FEM} = thermal/radiation/packaging derating \& memory scrubbing validation.

\section{System Architecture}
AITL consists of three layers:
\begin{itemize}[itemsep=0pt,topsep=1pt,leftmargin=*]
  \item \textbf{Robust Core}: $H_\infty$/MPC/SMC controllers for stability.
  \item \textbf{FSM Supervisor}: mode switching (Safe/Nominal/Recovery) with FDI/FDI\!I for fault management.
  \item \textbf{AI Adaptor}: long-term re-identification and drift compensation.
\end{itemize}
A tri-NVM hierarchy ensures persistence: SRAM for execution, MRAM for logs/code with ECC scrubbing and dual slots, and FRAM for safe boot and FSM states. Target SoC is 22\,nm FDSOI hardened for radiation and temperature stress.

\section{Mathematical Model and $H_\infty$ Design}
We consider an 11D discrete-time state-space plant coupling attitude (6), power bus (2), and thermal nodes (3):
\begin{align}
x_{k+1} &= A x_k + B u_k + E w_k,\\
y_k     &= C x_k + D u_k + v_k,
\end{align}
where $w_k$ and $v_k$ are disturbance and noise. The model extends to 20D by adding translational axes and bias states. Weights $(W_1,W_2,W_3)$ shape sensitivity, control effort, and complementary sensitivity. \textit{EduController} outputs them as JSON; \textit{AITL-H} synthesizes $K$ with robustness margins and exports a fixed-point realization for RTL/FPGA/ASIC.

\section{Verification Pipeline}
FPGA HIL injects SEUs and sensor outages. Metrics include safe-mode entry time ($<\!1$\,s), recovery rate ($\ge 99\%$), and ECC scrubbing efficiency. \textit{SystemDK FEM} validates thermal and radiation stress, ensuring packaging reliability before ASIC tape-out.

\section{Conclusion}
\textit{AITL on Space} combines robust control, supervisory safety, AI re-identification, and hardened memory. The proposed end-to-end flow—from mission specification to ASIC—provides a reproducible methodology for resilient autonomy in long-duration space missions.

% Balance last page columns around ref. 3 (tune as needed)
\IEEEtriggeratref{3}

\begin{thebibliography}{4}
\bibitem{doyle} J.~C. Doyle, B.~A. Francis, and A.~R. Tannenbaum, \emph{Feedback Control Theory}. Macmillan, 1992.
\bibitem{colinge} J.-P. Colinge, \emph{Silicon-on-Insulator Technology: Materials to VLSI}, 3rd ed. Springer, 2004.
\bibitem{wolf} W. Wolf, \emph{FPGA-Based System Design}. Prentice Hall, 2004.
\bibitem{rabey} J.~M. Rabaey, A. Chandrakasan, and B. Nikolić, \emph{Digital Integrated Circuits: A Design Perspective}, 2nd ed. Prentice Hall, 2003.
\end{thebibliography}

\section*{Author Biography}
\begingroup\small
Shinichi Samizo received the M.S. degree in Electrical and Electronic Engineering from Shinshu University, Japan. He worked at Seiko Epson Corporation as an engineer in semiconductor memory and mixed-signal device development, and contributed to inkjet MEMS actuators and PrecisionCore printhead technology. He is currently an independent semiconductor researcher focusing on process/device education, memory architecture, and AI system integration. Contact: \texttt{shin3t72@gmail.com}.
\endgroup

% --- 〈Biography のすぐ後に挿入〉 ------------------------
\FloatBarrier % ← ここまでを本文として確定させる

% Fig.1(片欄図だった内容を2欄図に)
\begin{figure*}[!b]
  \centering
  % 元のフロー図(TikZ or includegraphics)を \resizebox で幅合わせ
  % 例: \resizebox{0.60\textwidth}{!}{\input{fig_flow.tikz}}
  % あるいは画像の場合:
  % \includegraphics[width=0.60\textwidth]{fig1-flow.pdf}
  \resizebox{0.60\textwidth}{!}{%
    % === あなたの Fig.1 の tikzpicture 本文 ===
    % \begin{tikzpicture} ... \end{tikzpicture}
  }
  \vspace{-2mm}
  \caption{End-to-end design flow: Mission Spec $\rightarrow$ JSON (EduController) $\rightarrow$ AITL-H $\rightarrow$ FPGA HIL $\rightarrow$ FEM $\rightarrow$ ASIC.}
  \label{fig:flow}
\end{figure*}

% Fig.2(広い図:アーキテクチャ)— ページ下部に来やすいよう同じく [!b]
\begin{figure*}[!b]
  \centering
  \resizebox{0.90\textwidth}{!}{%
    % === あなたの Fig.2 の tikzpicture 本文(上段の大きい図)===
  }
  \vspace{-1mm}
  \caption{AITL architecture: three control layers with tri-NVM memory hierarchy. Orthogonal interconnects improve readability.}
  \label{fig:arch}
\end{figure*}

% Fig.3(小さめの閉ループ図)
\begin{figure*}[!b]
  \centering
  \resizebox{0.55\textwidth}{!}{%
    % === あなたの Fig.3 の tikzpicture 本文 ===
  }
  \vspace{-2mm}
  \caption{Closed-loop system for $H_\infty$ synthesis (compact single-column rendering).}
  \label{fig:closed}
\end{figure*}
% ---------------------------------------------------------
