%========================================
% main.tex  (LuaLaTeX / IEEEtran)
%========================================
\documentclass[conference]{IEEEtran}

% ---- Fonts & math (Latin) ----
\usepackage{newtxtext,newtxmath}

% ---- CJK (Japanese) for LuaLaTeX ----
\usepackage{luatexja}
\usepackage[match]{luatexja-fontspec}
% 和文フォント:TeX Live 標準(無ければ下の Noto を有効化)
\setmainjfont{HaranoAjiMincho}
\setsansjfont{HaranoAjiGothic}
% \setmainjfont{Noto Serif CJK JP}
% \setsansjfont{Noto Sans CJK JP}

% ---- Graphics & color ----
\usepackage{graphicx}
\usepackage[dvipsnames]{xcolor}

% ---- Links / URL ----
\usepackage[hidelinks]{hyperref}

% ---- Tables ----
\usepackage{booktabs}

% ---- Float control ----
\usepackage{placeins}

% ---- TikZ ----
\usepackage{tikz}
\usetikzlibrary{arrows.meta, positioning, calc, shapes.geometric, shapes.misc}

% ---- TikZ styles ----
\tikzset{
  line/.style       = {-{Latex[length=2.2mm,width=1.4mm]}, line width=0.45pt},
  box/.style        = {draw, rounded corners=1.8pt, align=center,
                       inner sep=2.5pt, minimum height=6.6mm},
  smallbox/.style   = {box, minimum width=24mm},
  midbox/.style     = {box, minimum width=30mm},
  bigbox/.style     = {box, minimum width=62mm},
  swim/.style       = {box, minimum height=46mm, fill=gray!5},
  title/.style      = {font=\bfseries, inner sep=0pt},
  sum/.style        = {draw, circle, inner sep=0.8pt, minimum size=3.4mm}
}

% ---- Tighten float spacing (2カラム最終ページ調整に役立つ) ----
\setlength{\textfloatsep}{5pt plus 1pt minus 1pt}
\setlength{\floatsep}{5pt plus 1pt minus 1pt}
\setlength{\intextsep}{5pt plus 1pt minus 1pt}

\newcommand{\etal}{\textit{et~al.}}

%========================================
\begin{document}

\title{AITL on Space: A Robust Three-Layer Architecture\\
with a Tri-NVM Hierarchy (SRAM / MRAM / FRAM)\\
for Long-Duration Spacecraft Autonomy}

\author{%
Shinichi Samizo\\
\emph{Independent Semiconductor Researcher}\\
Former Engineer at Seiko Epson Corporation\\
Email: shin3t72@gmail.com \quad GitHub: \url{https://github.com/Samizo-AITL}
}

\maketitle

\begin{abstract}
We propose \emph{AITL on Space}, a robust three-layer control architecture
(Robust Core, FSM Supervisor, AI Adaptor) integrated with a tri-NVM hierarchy
(SRAM/MRAM/FRAM) and mapped to a 22\,nm FDSOI SoC.
Requirements are formalized as JSON via \emph{EduController}, synthesized to an
$H_\infty$ controller \(K\) by \emph{AITL-H}, validated in FPGA HIL with fault
injection, and stress-tested through SystemDK FEM.
This end-to-end methodology enables resilient autonomy for long-duration missions.
\end{abstract}

\section{Introduction}
Deep-space missions require ultra-robust control under total ionizing dose,
single event effects (SEE), and thermal cycling.
Conventional PD\,+\,Flash architectures face lifetime limits due to cell
degradation. We present \emph{AITL on Space}:
a resilient three-layer architecture with a tri-NVM hierarchy and a
reproducible design flow from specification to ASIC.

\section{Specification and Design Flow}
The process begins with \textbf{Mission Specification}.
Requirements such as pointing accuracy, power stability, and thermal tolerance
are captured in \emph{EduController}, a model-based tool that exports plant
matrices and weighting functions as JSON (portable across simulators).
The JSON is consumed by \emph{AITL-H}, which synthesizes an
$H_\infty$ output-feedback controller \(K\) with mixed-sensitivity weights and
generates a fixed-point implementation for RTL/FPGA/ASIC.
Validation proceeds with:
\begin{itemize}
  \item \textbf{FPGA HIL}: hardware-in-the-loop validation with SEU\,+\,outage
  injection; metrics include safe-mode entry $<\!1$\,s, recovery rate $\ge 99\%$,
  and ECC scrubbing efficiency.
  \item \textbf{SystemDK FEM}: co-simulation of thermal cycles, radiation
  effects, and packaging stress, closing the verification loop before silicon.
  \item \textbf{ASIC Mapping}: implementation on GlobalFoundries 22FDX FDSOI
  hardened for long-duration missions.
\end{itemize}

\section{System Architecture}
The controller consists of three layers:
\textbf{Robust Core} ($H_\infty$/MPC/SMC),
\textbf{FSM Supervisor} (Safe/Nominal/Recovery; FDI/FDII),
and \textbf{AI Adaptor} (long-term re-identification and drift compensation).
A tri-NVM hierarchy ensures persistence: SRAM for execution,
MRAM for logs/code with ECC, and FRAM for safe boot and FSM states.
Target SoC is 22\,nm FDSOI for radiation and temperature margins.

\section{Mathematical Model and $H_\infty$ Design}
We consider a discrete LTI plant with disturbances \(w_k\) and noise \(v_k\),
and synthesize \(K\) using mixed-sensitivity shaping weights \(W_1,W_2,W_3\).
Design objective: \(\min \|T_{w\to z}\|_\infty\).

\section{Verification Pipeline}
FPGA HIL injects SEUs and sensor outages.
SystemDK FEM validates thermal and radiation stress.
Metrics: safe-mode $<\!1$\,s, recovery $\ge 99\%$, ECC scrubbing efficiency.
This closes the loop prior to ASIC tape-out.

% --- 〈Biography のすぐ後に挿入〉 ------------------------
\FloatBarrier % ← ここまでを本文として確定

% === Fig.1: End-to-end design flow ==========================
\begin{figure}[t]
\centering
\begin{tikzpicture}[node distance=10mm]
  % ノード
  \node[smallbox] (edu) {EduController\\\small モデル化 \& JSON出力};
  \node[smallbox, right=12mm of edu] (json) {JSON\\\small $A,B,C,D$, $W_1,W_2,W_3$};
  \node[smallbox, right=12mm of json] (aitl) {AITL-H\\\small $H_\infty$合成 $\to$ 固定小数点$K$};
  \node[smallbox, right=12mm of aitl] (hil) {FPGA HIL\\\small SEU/欠測注入, 安全率評価};
  \node[smallbox, right=12mm of hil] (fem) {SystemDK FEM\\\small 熱/放射/実装 ストレス};
  \node[smallbox, right=12mm of fem] (asic) {ASIC\\\small 22FDX FDSOI};

  % 矢印
  \draw[line] (edu) -- (json);
  \draw[line] (json) -- (aitl);
  \draw[line] (aitl) -- (hil);
  \draw[line] (hil) -- (fem);
  \draw[line] (fem) -- (asic);

  % 下部の注記
  \node[align=center, below=6mm of hil] (metrics)
    {\small 指標:セーフモード$<\!1$\,s,回復率$\ge 99\%$,ECCスクラビング効率};

  % 図枠タイトル
  \node[title, above=6mm of aitl] {End-to-end Design Flow};
\end{tikzpicture}
\caption{End-to-end design flow from mission spec to ASIC.}
\label{fig:flow}
\end{figure}

% === Fig.2: AITL architecture ===============================
\begin{figure}[t]
\centering
\begin{tikzpicture}[node distance=6mm]
  \tikzset{
    colA/.style={fill=gray!6},
    colB/.style={fill=gray!6},
    colC/.style={fill=gray!6},
    title/.style={font=\bfseries, inner sep=1pt}
  }

  % 全体枠
  \node[bigbox, minimum width=84mm, minimum height=56mm, align=left] (soc) { };

  % レイヤ枠
  \node[bigbox, colA, minimum width=26mm, minimum height=46mm, anchor=west, xshift=3mm] (robust) at (soc.west) {};
  \node[bigbox, colB, minimum width=26mm, minimum height=46mm, right=3mm of robust] (fsm) {};
  \node[bigbox, colC, minimum width=26mm, minimum height=46mm, right=3mm of fsm] (ai) {};

  % 見出し
  \node[title] at ([yshift=+22mm]robust.center) {Robust Core};
  \node[title] at ([yshift=+22mm]fsm.center)   {FSM Supervisor};
  \node[title] at ([yshift=+22mm]ai.center)    {AI Adaptor};

  % Robust Core
  \node[smallbox, align=center] (rc1) at ([yshift=+6mm]robust.center) {$H_\infty$/MPC/SMC\\\small 安定化制御};
  \node[smallbox, below=4mm of rc1] (rc2) {観測器/整形重み\\\small $W_1,W_2,W_3$};
  \node[smallbox, below=4mm of rc2] (rc3) {固定小数点$K$\\\small RTL/FPGA/ASIC};

  % FSM
  \node[smallbox] (fs1) at ([yshift=+12mm]fsm.center) {モード管理\\\small Safe/Nominal/Recovery};
  \node[smallbox, below=4mm of fs1] (fs2) {FDI/FDII\\\small 故障検出・隔離};
  \node[smallbox, below=4mm of fs2] (fs3) {セーフブート\\\small ウォッチドッグ};

  % AI
  \node[smallbox] (ai1) at ([yshift=+12mm]ai.center) {長期リID/ドリフト補償};
  \node[smallbox, below=4mm of ai1] (ai2) {パラメタ更新\\\small 演算は低頻度};
  \node[smallbox, below=4mm of ai2] (ai3) {検証ゲート\\\small 安全境界内のみ反映};

  % 信号の流れ
  \draw[line] (rc1.east) -- (fs1.west);
  \draw[line] (fs1.east) -- (ai1.west);
  \draw[line] (ai3.west) -- ++(-6mm,0) |- (rc3.east);

  % Tri-NVM
  \node[bigbox, minimum width=20mm, minimum height=46mm, anchor=east, xshift=-3mm, align=center] (nvm) at (soc.east) {};
  \node[title, above=0mm of nvm] {Tri-NVM};

  \node[smallbox, minimum width=18mm, align=center] (sram) at ([yshift=+12mm]nvm.center) {SRAM\\\small 実行用};
  \node[smallbox, minimum width=18mm, below=3mm of sram] (mram) {MRAM\\\small ログ/コード\\ECC};
  \node[smallbox, minimum width=18mm, below=3mm of mram] (fram) {FRAM\\\small セーフブート/FSM};

  % メモリ接続
  \draw[line] (rc3.east) -- ++(4mm,0) |- (sram.west);
  \draw[line] (fs3.east) -- ++(6mm,0) |- (fram.west);
  \draw[line] (ai2.east) -- ++(10mm,0) |- (mram.west);

  % タイトル
  \node[title] at ([yshift=+30mm]soc.north) {AITL Architecture on 22\,nm FDSOI SoC};
\end{tikzpicture}
\caption{AITL architecture: three layers with a tri-NVM hierarchy.}
\label{fig:arch}
\end{figure}

% === Fig.3: Closed-loop block diagram =======================
\begin{figure}[t]
\centering
\begin{tikzpicture}[node distance=10mm]
  % ブロック
  \node[smallbox, minimum width=22mm] (K) {$K$};
  \node[smallbox, right=22mm of K, minimum width=26mm] (P) {$P$};

  % サマ
  \node[sum, left=10mm of K] (sumin) {$\Sigma$};
  \node[sum, right=12mm of P] (sumy) {$\Sigma$};

  % 配線:u, y ループ
  \draw[line] (sumin) -- (K);
  \draw[line] (K) -- node[above] {$u$} (P);
  \draw[line] (P) -- node[above] {$y$} (sumy);

  % フィードバック
  \draw[line] (sumy) |- ++(0,-12mm) -| node[pos=0.25, below] {$-$} (sumin);

  % 外乱/雑音/性能
  \node[left=12mm of sumin] (r) {};
  \draw[line] (r.center) -- node[above] {$r$} (sumin);

  \node[above=10mm of P] (w) {};
  \draw[line] (w.center) -- node[right] {$w$} (P.north);

  \node[below=10mm of sumy] (v) {};
  \draw[line] (v.center) -- node[right] {$v$} (sumy.south);

  % z 出力
  \node[smallbox, below=8mm of P] (W) {$W$};
  \draw[line] (P.south) -- (W.north);
  \node[right=14mm of W] (z) {};
  \draw[line] (W) -- node[above] {$z$} (z.center);

  % 注釈
  \node[align=center, font=\small, below=14mm of K]
    {設計:$H_\infty$ で $\|T_{w\to z}\|_\infty$ を最小化};
\end{tikzpicture}
\caption{Closed-loop structure used for robust design.}
\label{fig:loop}
\end{figure}

\section*{Conclusion}
\emph{AITL on Space} combines robust control, supervisory safety, AI
re-identification, and hardened memory. The end-to-end flow—from mission
specification to ASIC—provides a reproducible methodology for resilient
autonomy in long-duration space missions.

% ----------------- References -----------------
\begin{thebibliography}{1}
\bibitem{doyle}
J. C. Doyle, B. A. Francis, and A. R. Tannenbaum,
\emph{Feedback Control Theory}. Macmillan, 1992.

\bibitem{colinge}
J.-P. Colinge, \emph{Silicon-on-Insulator Technology: Materials to VLSI},
3rd ed. Springer, 2004.

\bibitem{wolf}
W. Wolf, \emph{FPGA-Based System Design}. Prentice Hall, 2004.

\bibitem{rabaey}
J. M. Rabaey, A. Chandrakasan, and B. Nikoli\'{c},
\emph{Digital Integrated Circuits: A Design Perspective}, 2nd ed. Prentice Hall, 2003.
\end{thebibliography}

% ----------------- Author Biography -----------------
\section*{Author Biography}
Shinichi Samizo received the M.S. degree in Electrical and Electronic
Engineering from Shinshu University, Japan. He worked at Seiko Epson Corporation
as an engineer in semiconductor memory and mixed-signal device development, and
contributed to inkjet MEMS actuators and PrecisionCore printhead technology.
He is currently an independent semiconductor researcher focusing on
process/device education, memory architecture, and AI system integration.\\
\emph{Contact:} \href{mailto:shin3t72@gmail.com}{shin3t72@gmail.com}.

\end{document}
