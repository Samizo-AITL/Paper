\documentclass[conference]{IEEEtran}

% ==== Packages ====
\usepackage{graphicx}
\usepackage{amsmath,amssymb}
\usepackage{hyperref}
\usepackage{siunitx}
\usepackage{booktabs}
\usepackage{tikz}                 % for placeholder frames
\usepackage{newtxtext,newtxmath}  % Times-like fonts for IEEE style (LuaLaTeX OK)

% ==== Safe include (handles underscores + compiles even if file missing) ====
\newcommand{\safeincludegraphics}[2][]{%
  \begingroup
  \edef\fname{\detokenize{#2}}% neutralize _ and other specials
  \IfFileExists{\fname}{%
    \includegraphics[#1]{\fname}%
  }{%
    \begin{tikzpicture}
      \node[draw,minimum width=0.98\linewidth, minimum height=4cm, align=center]
      {\small \texttt{\fname} not found\\(placeholder)};
    \end{tikzpicture}%
  }%
  \endgroup
}

% ==== Title/Author ====
\title{AITL on Space: A Robust Three-Layer Architecture with a Tri-NVM Hierarchy (SRAM / MRAM / FRAM) for Long-Duration Spacecraft Autonomy}

\author{%
  \IEEEauthorblockN{Shinichi Samizo}
  \IEEEauthorblockA{Independent Semiconductor Researcher\\
  Former Engineer at Seiko Epson Corporation\\
  Email: \href{mailto:shin3t72@gmail.com}{shin3t72@gmail.com}\\
  GitHub: \url{https://github.com/Samizo-AITL}}%
}

\begin{document}
\maketitle

\begin{abstract}
We propose \textbf{AITL on Space}, a three-layer control architecture (Robust Core, FSM Supervisor, AI Adaptor) implemented on a 22\,nm FDSOI SoC with a hardened \textbf{Tri-NVM hierarchy} (SRAM / MRAM / FRAM). The system targets ultra-robust autonomy under radiation, thermal cycling, and long-term drift. This paper outlines the architecture, state-space model (8--20\,D), H$\infty$ mixed-sensitivity design flow, and a verification pipeline from FPGA HIL to ASIC.
\end{abstract}

\section{Introduction}
Long-duration missions require high availability under TID/SEE and thermal cycles. Conventional PID+Flash architectures face reliability limits. We summarize related work and motivate AITL on Space.

\section{System Architecture}
AITL comprises: (i) Robust Core (H$\infty$/MPC/SMC), (ii) FSM Supervisor (Safe/Nominal/Recovery, FDI/FDII), and (iii) AI Adaptor for long-term re-identification. A \textbf{Tri-NVM hierarchy}---SRAM for execution, MRAM for logs/code (ECC+scrub, A/B slots), FRAM for safe boot---ensures persistence.

\begin{figure}[t]
  \centering
  \safeincludegraphics[width=\linewidth]{figs/arch_block.pdf}
  \caption{AITL on Space architecture with Robust Core, Supervisor FSM, AI Adaptor, and the Tri-NVM hierarchy.}
  \label{fig:arch_block}
\end{figure}

\section{Mathematical Model}
We use an 11D PoC plant that couples attitude (6), power bus (2), and thermal nodes (3). The discrete-time model is
\begin{align}
x_{k+1} &= A x_k + B u_k + E w_k, \quad
y_k = C x_k + D u_k + v_k.
\end{align}
Extensions scale to 20D with translational axes and bias states.

\begin{figure}[t]
  \centering
  \safeincludegraphics[width=0.9\linewidth]{figs/state_space.pdf}
  \caption{PoC 11D state-space model coupling attitude (6), power bus (2), and thermal nodes (3).}
  \label{fig:state_space}
\end{figure}

\section{H$\infty$ Mixed-Sensitivity Design}
Weights $(W_1,W_2,W_3)$ shape sensitivity, effort, and complementary sensitivity. EduController exports JSON plant/weights; AITL-H synthesizes output-feedback $K$ and fixed-point realization.

\section{Verification Pipeline}
FPGA HIL injects SEU bursts and sensor outages; metrics include safe-mode time ($\leq 1\,\mathrm{s}$), recovery rate ($\geq 99\%$), and ECC statistics. Physical design proceeds to 22FDX tape-out; SystemDK FEM closes thermal/packaging.

\begin{figure}[t]
  \centering
  \safeincludegraphics[width=\linewidth]{figs/pipeline.pdf}
  \caption{Verification pipeline from JSON design to RTL, FPGA HIL, and ASIC; SystemDK closes the loop with thermal and radiation scenarios.}
  \label{fig:pipeline}
\end{figure}

\section{Conclusion}
AITL on Space offers a practical path to resilient autonomy for deep-space missions.

\bibliographystyle{ieeetr}
\begin{thebibliography}{9}
\bibitem{hinf} Doyle, G. et al., \emph{Feedback Control Theory}.
\bibitem{fds} Colinge, J.-P., \emph{Silicon-on-Insulator Technology}.
\end{thebibliography}

% ===== Biography =====
\section*{Author Biography}
\textbf{Shinichi Samizo} received the M.S. degree in Electrical and Electronic
Engineering from Shinshu University, Japan. He worked at Seiko Epson
Corporation as an engineer in semiconductor memory and mixed-signal
device development, and contributed to inkjet MEMS actuators and
PrecisionCore printhead technology. He is currently an independent
semiconductor researcher focusing on process/device education, memory
architecture, and AI system integration. Contact:
\href{mailto:shin3t72@gmail.com}{shin3t72@gmail.com}.

\end{document}
