% main.tex — IEEEtran (LuaLaTeX, luatexja)
% Build: lualatex main.tex ; lualatex main.tex
\documentclass[conference]{IEEEtran}

% ---------------- Japanese (LuaLaTeX) ----------------
\usepackage{luatexja}
\usepackage{luatexja-fontspec}
% 和文フォントはTeX Live同梱(再配布安全)
\setmainjfont{HaranoAjiMincho} % 明朝体
\setsansjfont{HaranoAjiGothic} % ゴシック体

% ---------------- Latin fonts (Times系に寄せる) ----------------
\usepackage{fontspec}
\setmainfont{TeX Gyre Termes} % Times相当
\setsansfont{TeX Gyre Heros}  % Helvetica相当(IEEE図表に合う)
\setmonofont{TeX Gyre Cursor} % 等幅

% ---------------- General packages (IEEE互換の範囲) ----------------
\usepackage{siunitx}
\sisetup{detect-all,mode=match,propagate-math-font=true}
\usepackage{graphicx}
\usepackage{xcolor}
\usepackage{booktabs}
\usepackage{cite}
\usepackage[hidelinks]{hyperref}
\urlstyle{same}

% ---------------- TikZ(図は本文内に埋め込み)----------------
\usepackage{tikz}
\usetikzlibrary{arrows.meta,calc}
\tikzset{
  >=Stealth,
  every picture/.style={line cap=round,line join=round},
}
\pgfmathsetseed{20250928} % 再現性のため乱数シード固定

% ====== 追加パッケージ(プリアンブル) ======
\usepackage{pgfplots}
\pgfplotsset{compat=1.18}
\usepackage{pgfplotstable}

% =====================================================
% 内蔵図コマンド(本文で呼ぶ前=\begin{document}より前に置く)
% 使い方例:\figVoidDonutInline            % 既定サイズ
%           \figVoidDonutInline[0.050]     % 少し大きめ
% =====================================================

% ---- Fig.1: ドーナツ状のボイド分布(1カラム対応・幅広リング)----
\newcommand{\figVoidDonutInline}[1][0.045]{%
  \begin{tikzpicture}[scale=#1]
    % 再現性(この図だけの種にする)
    \pgfmathsetseed{20250928}
    % 外周参照円(薄め)
    \draw[line width=0.6pt, gray!60] (0,0) circle (100);
    % 半径分布を広めに(中心55、±15)
    \foreach \i in {1,...,360}{%
      \pgfmathsetmacro{\ang}{rnd*360}
      \pgfmathsetmacro{\rad}{55 + 15*(rnd-0.5)}
      \fill[gray!30] (\ang:\rad) circle (1.0);
    }
  \end{tikzpicture}%
}

% ====== Fig.2: 6層PZT(上下=太実線/中間=細点線/ボイド=塗りのみ)======
% 使い方: \figPZTLayersVoidInline         % 既定は第4層にボイド
%         \figPZTLayersVoidInline[3]     % 第3層にボイド、など
\newcommand{\figPZTLayersVoidInline}[1][4]{%
  \begin{tikzpicture}[x=0.9cm,y=0.9cm]
    % ---- パラメータ ----
    \def\W{6.0}   % 横幅
    \def\H{0.45}  % 各層の厚み
    \def\N{6}     % 層数
    \def\voidLayer{#1} % ★ ここで定義(既定=4)

    % ---- 7本の水平線(0〜6)----
    \foreach \i in {0,...,6}{%
      \pgfmathsetmacro{\yy}{\i*\H}%
      % 最下 (i=0) と最上 (i=6) は太実線/中間は細い点線
      \ifnum\i=0
        \draw[line width=1pt] (0,\yy) -- (\W,\yy);
      \else
        \ifnum\i=6
          \draw[line width=1pt] (0,\yy) -- (\W,\yy);
        \else
          \draw[gray!60, densely dotted, line width=0.6pt] (0,\yy) -- (\W,\yy);
        \fi
      \fi
    }

    % ---- ボイド矩形(枠線なし=塗りのみ)----
    \pgfmathsetmacro{\yv}{(\voidLayer-1)*\H}%
    \fill[gray!50] (2.1,\yv) rectangle (3.0,\yv+\H);
  \end{tikzpicture}%
}

% ==== Fig.3 端部焼損 図コマンド定義(未定義なら新規定義)====
\usepackage{tikz}
\usetikzlibrary{arrows.meta,calc}
\tikzset{>=Stealth, every picture/.style={line cap=round,line join=round}}

\makeatletter
\@ifundefined{figEdgeBurnoutInline}{%
  \newcommand{\figEdgeBurnoutInline}{%
    \begin{tikzpicture}[x=0.9cm,y=0.9cm]
      % ========= パラメータ =========
      \def\W{10}\def\beH{0.20}\def\pztH{1.20}\def\teH{0.20}
      \def\xPZT{4.0}\def\xGapR{6.0}\def\xViaL{7.6}\def\xViaR{8.2}

      % ========= 下地〜層 =========
      \fill[gray!55] (0,0) rectangle (\W,\beH);                                   % BE
      \fill[gray!15] (0,\beH) rectangle (\xPZT,\beH+\pztH);                        % 左PZT
      \fill[gray!80] (0,\beH+\pztH) rectangle (\xPZT,\beH+\pztH+\teH);             % 左TE
      \fill[gray!15] (\xGapR,\beH) rectangle (\W,\beH+\pztH);                      % 右PZT(全面)
      \fill[gray!80] (\xGapR,\beH+\pztH) rectangle (\W,\beH+\pztH+\teH);           % 右TE(全面)
      \fill[white]   (\xViaL,\beH) rectangle (\xViaR,\beH+\pztH+\teH);             % 柱開口(BE露出)
      \fill[gray!30] (\xGapR,\beH+\pztH+\teH) rectangle (\W,\beH+\pztH+\teH+0.60); % Au配線(TE上)
      \fill[gray!30] (\xViaL,\beH) rectangle (\xViaR,\beH+\pztH+\teH);             % Au柱(BEまで)

      % ========= 強調表示 =========
      \draw[line width=1.0pt] (\xPZT,\beH) -- (\xPZT,\beH+\pztH);                  % 側壁
      \draw[gray!70,dashed,line width=0.7pt] (\xPZT-0.2,\beH) -- (\xPZT-0.2,\beH+\pztH);
      \draw[gray!70,dashed,line width=0.7pt] (\xPZT+0.2,\beH) -- (\xPZT+0.2,\beH+\pztH);

      % ========= ラベル(日英併記) =========
      % 端子
      \draw[->] (0.3,\beH+\pztH+\teH+0.4) -- (0.8,\beH+\pztH+\teH+0.05);
      \node[anchor=south] at (0.3,\beH+\pztH+\teH+0.4)
        {\scriptsize VBS = -10\,V};

      \draw[->] (7.2,\beH+\pztH+\teH+0.9) -- (7.2,\beH+\pztH+\teH+0.62);
      \node[anchor=south] at (7.2,\beH+\pztH+\teH+0.9)
        {\scriptsize COM = 10–30\,V};

      % 層名(JP / EN)
      \node[anchor=east] at (0,\beH/2)
        {\scriptsize BE(下部電極)/ Bottom electrode};
      \node[anchor=east] at (0,\beH+\pztH/2)
        {\scriptsize PZT(圧電膜)/ Piezoelectric layer};
      \node[anchor=east] at (0,\beH+\pztH+\teH/2)
        {\scriptsize TE(上部電極)/ Top electrode};

      % 焼損注記(JP / EN)
      \draw[->,thick] (\xPZT+0.4,\beH+0.6) -- (\xPZT+0.05,\beH+0.6);
      \node[anchor=west] at (\xPZT+0.45,\beH+0.6)
        {\scriptsize 焼損発生点 / Burnout spot(PZT側壁露出 / exposed sidewall)};
    \end{tikzpicture}%
  }%
}{%
  % 既定義なら差し替え
  \renewcommand{\figEdgeBurnoutInline}{%
    \begin{tikzpicture}[x=0.9cm,y=0.9cm]
      \def\W{10}\def\beH{0.20}\def\pztH{1.20}\def\teH{0.20}
      \def\xPZT{4.0}\def\xGapR{6.0}\def\xViaL{7.6}\def\xViaR{8.2}

      \fill[gray!55] (0,0) rectangle (\W,\beH);
      \fill[gray!15] (0,\beH) rectangle (\xPZT,\beH+\pztH);
      \fill[gray!80] (0,\beH+\pztH) rectangle (\xPZT,\beH+\pztH+\teH);
      \fill[gray!15] (\xGapR,\beH) rectangle (\W,\beH+\pztH);
      \fill[gray!80] (\xGapR,\beH+\pztH) rectangle (\W,\beH+\pztH+\teH);
      \fill[white]   (\xViaL,\beH) rectangle (\xViaR,\beH+\pztH+\teH);
      \fill[gray!30] (\xGapR,\beH+\pztH+\teH) rectangle (\W,\beH+\pztH+\teH+0.60);
      \fill[gray!30] (\xViaL,\beH) rectangle (\xViaR,\beH+\pztH+\teH);

      \draw[line width=1.0pt] (\xPZT,\beH) -- (\xPZT,\beH+\pztH);
      \draw[gray!70,dashed,line width=0.7pt] (\xPZT-0.2,\beH) -- (\xPZT-0.2,\beH+\pztH);
      \draw[gray!70,dashed,line width=0.7pt] (\xPZT+0.2,\beH) -- (\xPZT+0.2,\beH+\pztH);

      \draw[->] (0.3,\beH+\pztH+\teH+0.4) -- (0.8,\beH+\pztH+\teH+0.05);
      \node[anchor=south] at (0.3,\beH+\pztH+\teH+0.4)
        {\scriptsize VBS = -10\,V};

      \draw[->] (7.2,\beH+\pztH+\teH+0.9) -- (7.2,\beH+\pztH+\teH+0.62);
      \node[anchor=south] at (7.2,\beH+\pztH+\teH+0.9)
        {\scriptsize COM = 10–30\,V};

      \node[anchor=east] at (0,\beH/2)
        {\scriptsize BE(下部電極)/ Bottom electrode};
      \node[anchor=east] at (0,\beH+\pztH/2)
        {\scriptsize PZT(圧電膜)/ Piezoelectric layer};
      \node[anchor=east] at (0,\beH+\pztH+\teH/2)
        {\scriptsize TE(上部電極)/ Top electrode};

      \draw[->,thick] (\xPZT+0.4,\beH+0.6) -- (\xPZT+0.05,\beH+0.6);
      \node[anchor=west] at (\xPZT+0.45,\beH+0.6)
        {\scriptsize 焼損発生点 / Burnout spot(PZT側壁露出 / exposed sidewall)};
    \end{tikzpicture}%
  }%
}
\makeatother

% 本文側の別名互換
\providecommand{\figEdgeBurnoutSketch}{\figEdgeBurnoutInline}

% =====================================================
% タイトル/著者(日英併記)
% =====================================================

\title{%
  薄膜ピエゾアクチュエータにおける振動板クラックと端部焼損の原因解析・対策提案\\
  \large Cause Analysis and Countermeasure Proposal for Diaphragm Cracks and Edge Burnout in PZT Thin-Film Actuators
}

\author{%
  \IEEEauthorblockN{三溝 真一 (Shinichi Samizo)}%
  \IEEEauthorblockA{独立系半導体研究者(元セイコーエプソン) / Independent Semiconductor Researcher (ex-Seiko Epson)\\%
  Email: \href{mailto:shin3t72@gmail.com}{shin3t72@gmail.com}\quad
  GitHub: \url{https://github.com/Samizo-AITL}}%
}

\begin{document}
\maketitle

% =====================================================
% 要旨・キーワード(日英併記)
% =====================================================

\begin{abstract}
\textbf{和文要旨} —
本研究は,Epson $\mu$TFP(薄膜PZT,$d_{33}$駆動)アクチュエータの量産工程で顕在化した
(1) 振動板クラックと (2) セグメント端部焼損の二課題を対象に,原因解析と対策評価を行った。
クラックはウエハ同心円状に分布し,断面観察でPZT多層の特定層にボイドが局在することを確認した。
これはRTA後の外気由来吸着に伴う表面疎水化を起点とし,ゾルゲル塗布時の気泡巻き込みに起因する。
RTA直後の\emph{酢酸プレウェット(約2\%, 30\,s)}により親水性を回復し,気泡巻き込みを抑制することで,
不良率を5–10\%から\(\le\!1\%\)へと低減し量産条件として確立した。
一方,端部焼損はCOM下電極(Au配線)とVBS上電極が最接近するPZT側壁露出部に集中し,
最大\SI{40}{V}の印加と台形波立上り過渡電流($I=C\,\mathrm{d}V/\mathrm{d}t$)による電界集中を起点とする
局所絶縁破壊が示唆された。恒久策として\emph{ALD–AlO$_x$}による側壁パッシベーションを提案する。
以上より,薄膜PZTに固有の欠陥発生連鎖を明確化するとともに,
(G1)親水性リセットと(G2)側壁パッシベーションの二本柱からなる
一般化可能なプロセス指針を提示した。

\medskip
\textbf{Abstract (English)} —
This study addresses two reliability issues observed in mass production of Epson $\mu$TFP
(thin-film PZT, $d_{33}$ mode) actuators: (1) diaphragm cracks and (2) edge burnout.
Cracks showed a donut-shaped wafer distribution, and cross-sections revealed voids localized in
specific PZT layers. The mechanism is traced to post-RTA surface hydrophobicity due to airborne
adsorbates, which leads to bubble entrapment during sol-gel coating. An \emph{acetic-acid pre-wet}
($\sim$2\%, 30\,s) immediately after RTA restored hydrophilicity and suppressed bubbles,
reducing the defect rate from 5–10\% to \(\le\!1\%\) and qualifying as a production condition.
Edge burnout localized at exposed PZT sidewalls where the COM bottom electrode (via Au routing)
approaches the VBS top electrode; up to \SI{40}{V} bias combined with trapezoidal-wave transients
($I=C\,\mathrm{d}V/\mathrm{d}t$) suggests local breakdown under field concentration.
As a permanent countermeasure, \emph{ALD AlO$_x$} sidewall passivation is proposed.
These results clarify thin-film-specific failure chains and establish two generalizable process
guidelines for ferroelectric MEMS: (G1) hydrophilicity reset and (G2) sidewall passivation.
\end{abstract}

\begin{IEEEkeywords}
\textbf{和文キーワード} — インクジェット,MEMSアクチュエータ,PrecisionCore($\mu$TFP),薄膜PZT,$d_{33}$駆動,信頼性,酢酸プレウェット,側壁パッシベーション,端部焼損,クラック \\
\textbf{Keywords} — Inkjet printing, MEMS actuator, PrecisionCore ($\mu$TFP), Thin-film PZT, $d_{33}$ mode, Reliability, Acetic pre-wet, Sidewall passivation, Edge burnout, Cracks
\end{IEEEkeywords}

% =====================================================
% 本文
% =====================================================

\section{序論}
インクジェットプリントヘッドは,解像度・速度・信頼性の高度化要求に応じて進化してきた。
Epson体系では,1990年代の \emph{Mach} ヘッド(バルク積層PZT,$d_{31}$駆動)が
銀塩写真の置換を牽引し,2007年に \emph{TFP}(Thin Film Piezo)ヘッドの量産化により
ピエゾの\emph{自前製造(内製化)}とMEMS一体加工への転換が始まった。
その到達点として,2012年に \emph{PrecisionCore}($\mu$TFP)ヘッドが製品化され,
ビジネスインクジェット分野でレーザープリンタの置換を本格化させた。

この技術的跳躍は,材料を「買って使う」方式から「作って最適化する」方式へのパラダイム転換に他ならない。
しかし薄膜PZTの高電界・高密度駆動は,バルク素子では顕在化しなかった量産上の課題を露呈させた。
すなわち,(i)RTA後の表面疎水化を起点とするスピン塗布時の気泡巻き込みに由来する
\emph{振動板クラック},および(ii)COM–VBS近接部のPZT側壁露出で電界が集中して発生する
\emph{端部焼損}である。前者はプロセス環境・表面化学の管理課題,後者はデバイス幾何と
過渡電流($I=C\,\mathrm{d}V/\mathrm{d}t$)が重なる信頼性課題として現れた。

本論文の目的は,PrecisionCore $\mu$TFPヘッドの\emph{基盤技術確立}に至る過程で直面した
上記二課題の原因を量産工程データと不良解析に基づいて同定し,
(A)RTA直後の\emph{酢酸プレウェット}による親水性リセットと,
(B)\emph{ALD–AlO$_x$ 側壁パッシベーション}によるエッジ電界緩和という
二本柱の対策を提示・評価することである。
(A)は歩留まり不良率を 5–10\% から \(\approx 0\%\) へ低減し量産条件として確立した実装策であり,
(B)は高耐圧化・長期信頼性向上に有効な恒久策として提案する。

本稿の貢献は三点に要約される。
(1) 薄膜PZT特有の欠陥発生連鎖(外気付着$\rightarrow$疎水化$\rightarrow$気泡捕捉$\rightarrow$層内ボイド$\rightarrow$クラック)
と,端部焼損の電界・熱連成機構を\emph{工程と構造の両面}から統一的に整理した。
(2) 工程介入(酢酸プレウェット)により\emph{量産即応性の高い恒久改善}を実証した。
(3) 形状起因で不可避な側壁露出に対し,\emph{設計で避けられない領域をプロセスで守る}
というMEMS一般に適用可能な指針(G1/G2)を与えた。
以降,\S2で対象デバイスと工程を概説し,\S3で観測結果,\S4で原因と対策を考察,
\S5で結論を述べる。

\section{デバイス・プロセス構成}
本研究対象のアクチュエータチップは、シリコン基板上にゾルゲル法により形成した
多層PZT薄膜を駆動層とする積層型構造を有する。

\subsection{層構成}
基板は (111) 方位 Si ウエハであり、裏面にキャビティを形成して振動板領域を確保した。
基板表面には高耐圧化を目的として ZrO$_2$ 絶縁層(約 \SI{400}{nm})を堆積し、
その上に下部電極を構築した。下電極は Pt(111)(\SI{80}{nm})を主材とし、
Ir 酸化防止層(\SI{10}{nm})を介在し、そこにTi seed層(\SI{4}{nm})を堆積させることで結晶配向性と成膜安定性を確保した。
さらに、PZT第1層焼成後に Ti 薄膜(\SI{4}{nm})を挿入することで、
PZTの組成傾斜を制御し、結晶成長の均一化を図った。

PZT薄膜は Pb(Zr,Ti)O$_3$ 組成を有し、1層あたり \SI{200}{nm} の膜厚で
スピンコート~RTA焼成を6回繰り返すことにより、合計 \SI{1.2}{\micro\metre} の
駆動層を形成した。上部電極には Ir/Ti (\SI{10}{nm}/\SI{10}{nm}) を採用し、
電気化学的安定性と応力緩和の両立を図っている。(Table.I)

\begin{table*}[t]
  \centering
  \caption{%
    $\mu$TFPアクチュエータチップの層構成(下層→上層)\\
    Layer structure of $\mu$TFP actuator chip (bottom → top)
  }
  \label{tab:layer-structure}
  \begin{tabular}{lll}
    \hline
    \textbf{層構成 / Layer} & \textbf{材料 / Material} & \textbf{厚み・備考 / Thickness \& Function} \\
    \hline
    Si基板 / Substrate & Si(111) & 5000 nm / キャリア基板,キャビティ形成用 / Carrier wafer, cavity formation \\
    絶縁層 / Insulating layer & ZrO\textsubscript{2} & 400 nm / 高耐圧・高誘電率絶縁膜 / High-$k$ dielectric \\
    接着層 / Bonding layer & Ti & 4 nm / 下電極密着性向上 / Adhesion to BE \\
    下電極 / Bottom electrode & Pt & 80 nm / (111)配向,PZT配向誘導 / (111) oriented, seed for PZT \\
    酸化防止層 / Oxidation barrier & Ir & 10 nm / Pt酸化防止,結晶安定化 / Prevents Pt oxidation \\
    seed層 / Seed layer & Ti & 4 nm / 配向制御 / Initial growth control \\
    PZT初期層 / Initial PZT layer & Pb(Zr,Ti)O\textsubscript{3} & 200 nm / 第1層成膜 / First deposition \\
    中間層 / Mid layer & Ti & 4 nm / 組成傾斜改善,応力緩和 / Composition grading, stress relaxation \\
    PZT積層 / PZT stack & Pb(Zr,Ti)O\textsubscript{3} & 200 nm × 5 = 1000 nm / 5層積層 / Five-layer deposition \\
    上電極 / Top electrode & Ir/Ti & 10/10 nm / 応力緩和・反応抑制 / Stress relief, reaction suppression \\
    \hline
  \end{tabular}
\end{table*}

\subsection{ドライバICとの統合}
アクチュエータは、COF (Chip on Film) 実装されたドライバICと一体で動作する。
ドライバICは CMOS \SI{0.35}{\micro\metre} プロセスにより製造され、
標準電源 \SI{3.3}{V} および高耐圧出力 \SI{45}{V} に対応する。
1チップあたり 300 チャネルを2列に配列し、合計600ノズルを駆動可能である。
これにより、高密度ノズルアレイ(300\,dpi クラス)において並列動作が可能となり、
高速印字と均一な吐出特性が実現される。

\subsection{プロセス上の特徴}
ゾルゲル法を用いた多層積層構造は、膜厚精度や結晶配向性の制御が難しい一方で、
低温成膜・大面積対応という量産性に優れる。本研究では、
(1) Ti 薄膜の適切な挿入による応力緩和、(2) RTA条件の最適化による高配向化、
(3) 酢酸洗浄による親水性確保、といった工程改善により、量産適合性を確立している。

\section{振動板クラックの解析と対策 / Analysis and Countermeasure of Diaphragm Crack}
\label{sec:crack}

\subsection{現状把握}
ユニットスクリーニングにおいて、半径中間域に同心リング状のクラックが観測された。
断面TEMでは6層PZTのうち特定層に局所的な空隙(ボイド)が確認され、
炭素由来成分が検出された。これはRTA後の外気吸着による表面疎水化に起因し、
スピン塗布時の気泡巻き込みが発生したと推定される。

\subsection{原因仮説}
RTA後の外気暴露 → 表面疎水化 → 気泡巻き込み → 層内ボイド形成 →
焼成時に応力集中 → クラック進展。
すなわち、外気由来汚染と親水性低下が起点である。

\subsection{対策立案と実施}
\begin{itemize}
  \item 酢酸2\%による\emph{プレウェット処理(30\,s)}をRTA直後に追加。
  \item N$_2$パージ付きミニロードロック搬送により外気暴露を防止。
\end{itemize}

\subsection{効果確認(広丘流:N, $P_n$, $P$)}
\[
  P[\%] = \frac{P_n}{N} \times 100, \quad N = 12\times12\times109 = 15{,}696.
\]

\begin{table}[h]
  \centering
  \caption{クラック不良率の定量比較}
  \label{tab:crack-yield}
  \setlength{\tabcolsep}{4pt}
  \begin{tabular}{@{}lrrr@{}}
    \toprule
    条件 & $N$ [chip] & $P_n$ [chip] & $P$ [\%] \\
    \midrule
    従来(Before) & 15{,}696 & 1{,}099 & 7.0 \\
    酢酸PW後(After) & 15{,}696 & 63 & 0.40 \\
    \bottomrule
  \end{tabular}
\end{table}

不良率は7.0\%から0.4\%に低減し、絶対低減6.6\,%-pt、
相対改善94\%を達成した。配向率や電気特性に有意差はなく、
量産工程への恒久導入が確立した。

\subsection{考察}
酢酸プレウェットは、RTA後に親水性をリセットする工程介入であり、
気泡巻き込みを根本的に防止する有効手段である。
本手法はソル–ゲルPZTに限らず、HfO$_2$系やAlN系薄膜への
展開も可能であり、薄膜MEMS一般に有効なプロセス指針(G1)と位置づけられる。

\section{端部焼損の解析と対策 / Analysis and Countermeasure of Edge Burnout}
\label{sec:burnout}

\subsection{現状把握}
スクリーニング試験で+30\,V(COM)と--10\,V(VBS)の差分
\(\Delta V=40\,\mathrm{V}\)印加時に、
チップ端部のPZT側壁露出部で焼損が集中して発生した。
光学顕微鏡およびSEMで炭化痕を確認し、EDXでは金属蒸発痕が検出された。
この部位は構造上不可避な露出箇所であり、空気絶縁では耐圧が不足する。

\subsection{原因仮説}
PZT側壁の露出による局所電界集中に加え、
立上り過渡の瞬間電流 \(I=C\,\mathrm{d}V/\mathrm{d}t\)
によるジュール加熱が重畳し、
\emph{局所絶縁破壊}を誘発する。
実効電界は \(E\simeq3.3\,\mathrm{MV/cm}\) に達する。

\subsection{対策立案}
\begin{itemize}
  \item \textbf{恒久策(B1)}: ALD法によりAlO$_x$薄膜を側壁に\emph{conformal}成膜し、空気絶縁依存を解消。
  \item \textbf{補完策(B2)}: 駆動波形の立上りスロープを緩和し、過渡電流を低減。
\end{itemize}

\subsection{効果確認}
(B1)試作チップにより耐圧が従来比約1.5倍に向上し、
焼損発生が大幅に抑制された。
ただしALD導入には新規投資を要し、量産適用には
コスト評価が必要である。
(B2)は吐出特性変化を伴うため、設計最適化との併用が前提となる。

\subsection{考察}
端部焼損は設計上不可避なPZT露出に起因する構造的課題であり、
設計で避けられない領域を\emph{プロセスで守る}という考え方が有効である。
ALD側壁パッシベーションは薄膜MEMS一般の
電界集中対策指針(G2)として位置づけられる。

\section{不良解析結果}

\subsection{クラック:同心リング分布と層内ボイド}
ユニットスクリーニング後の歩留まり解析において、クラック不良は
ウェハ全面にランダムではなく、半径中間域に集中することが確認された。
特に、反射条件を最適化した表面検査装置により、欠陥が
同心リング状に分布する様子が明瞭に観測された(Fig.~\ref{fig:donut})。
この分布形態は、スピンコート時の液膜展開過程と密接に対応しており、
溶液が半径中間部に一度滞留した後に外周へ広がる過程で、
気泡が巻き込まれやすいことを示唆している。

断面TEMおよびSEM観察により、6層積層PZTのうち特定の層に
局所的な空隙(ボイド)が形成されていることが確認された
(Fig.~\ref{fig:layer-void})。これらのボイドは数百nmスケールであり、
結晶粒界に沿って成長する傾向を示す。クラックはしばしばこのボイドを起点に
膜厚方向へ進展しており、内部応力の集中点として機能していると考えられる。
また、EDX分析ではボイド周辺に炭素系の不純物が局所濃縮していることが確認され、
RTA後の外気由来成分が起因している可能性を裏付けた。

\begin{figure}[t]
  \centering
  \figVoidDonutInline % or \figVoidDonutInline[0.050]
  \caption{%
    ウエハ上のドーナツ状ボイド分布 / 
    Donut-shaped void distribution on wafer
  }
  \label{fig:donut}
\end{figure}

\begin{figure}[t]
  \centering
  \figPZTLayersVoidInline
  \caption{%
    PZT 6層のうち特定層のボイド(グレー枠) / 
    Void observed in a specific layer of 6-layer PZT stack (gray box)
  }
  \label{fig:layer-void}
\end{figure}

\subsubsection*{歩留まりの定量評価}
ユニットスクリーニングにおける不良率 $P$ は,
\[
  P[\%] = \frac{P_n}{N} \times 100,
\]
で定義した。ここで $P_n$ は不良チップ数,$N$ は総試験チップ数
($12$ロット $\times$ $12$ウエハ $\times$ $109$チップ $=15{,}696$チップ)である。

\begin{center}
\begin{tabular}{lccc}
\hline
条件 & $N$(チップ) & $P_n$(不良チップ数) & $P$(不良率, \%) \\
\hline
従来条件(Before) & 15,696 & 1,099 & 7.0 \\
酢酸プレウェット導入後(After) & 15,696 & 63 & 0.4 \\
\hline
\end{tabular}
\end{center}

上表に示すように,酢酸プレウェット導入により不良率は
$7.0\%\rightarrow0.4\%$ へと低減し,絶対低減量 $\Delta P=6.6$\,%-pt,
相対低減率 $94\%$ を達成した。
本スクリーニング段階での不良は,振動板クラックと端部焼損を区別せず,
総合的な電気検査不良として集計している。

% ====== Table.2: 比較プロセス条件(2カラム幅) ======
\begin{table*}[t]
  \centering
  \caption{%
    比較プロセス条件(Before/After)/
    Compared process conditions (Before/After)}
  \label{tab:proc-compare}
  \setlength{\tabcolsep}{6pt}
  \begin{tabular}{@{} lll @{}}
    \toprule
    \textbf{項目 / Item} & \textbf{Before} & \textbf{After} \\
    \midrule
    表面前処理 / Surface pre-wet
      & 無し / None
      & 酢酸2\% 30\,s \\
    RTA後搬送 / Post-RTA handling
      & 大気露出 / Air exposure
      & 窒素パージミニロードロック / N$_2$ purge mini-loadlock \\
    スピン塗布 / Spin coating
      & 同一条件 / Same condition
      & 同一条件 / Same condition \\
    PZT積層 / PZT stack
      & 200\,nm $\times$ 6
      & 200\,nm $\times$ 6 \\
    検査N / Sample size
      & 500/lot
      & 500/lot \\
    \bottomrule
  \end{tabular}
\end{table*}

\subsection{端部焼損:電界集中部位}
スクリーニング試験では、+30\,V(COM)と--10\,V(VBS)の電圧印加により、
最大差分 \SI{40}{V} の電界がPZT層に印加される。
台形波駆動の立ち上がり過渡において、数mAオーダーの瞬間電流が流入し、
局所的な電界集中を助長する。焼損は、COM下電極(Au配線経由)と
VBS上電極が最も接近するチップ端部に集中して観測された。

光学顕微鏡・SEM観察の結果、焼損部位はPZT側壁が露出する領域に限定されており、
空気絶縁のみでは耐圧が不足していることが示唆された(Fig.~\ref{fig:edge_burnout})。
さらにEDXでは、焼損部に酸化物由来の成分とともに金属蒸発痕が確認され、
局所的な絶縁破壊に起因する高温イベントであることが裏付けられた。

このPZT露出部は、アクチュエータ端部におけるキャビティ開口と
配線接続(COM電極をBE下電極へコンタクトさせる構造)のために生じるものであり、
設計上不可避である。

\begin{figure*}[t]
  \centering
  \figEdgeBurnoutSketch
  \caption{%
    端部焼損の模式図 / Schematic illustration of edge burnout
  }
  \label{fig:edge_burnout}
\end{figure*}

\section{考察}

\subsection{振動板クラックに関する考察}
観察結果より、クラックはウェハ同心円状に分布しており、これはスピンコート時の
溶液流動パターンと一致していた。RTA処理後の外気由来不純物(主に炭素系成分)が
PZT表面に吸着し、接触角測定でも確認されたように表面が疎水化する。
この結果、スピン時に気泡が液膜内へ巻き込まれ、焼成後にボイドとして残存する。
特にPZT 6層積層のうち特定層に局在することは、初期段階での気泡捕捉が
その後の層積み上げでも排除されず、内部欠陥として残存することを意味する。

暫定的にロードロック部へ防壁を設置し外気混入を抑制することで一定の効果を得たが、
恒久的にはプロセス側での親水化処理が必須である。酢酸によるプレウェット工程を
追加することで表面の親水性を回復し、気泡巻き込みを根本的に防止できた。
懸念された膜厚ばらつきの増加については、XRD配向率・電気的ヒステリシス特性・
吐出試験により従来条件との差異が統計的に有意でないことを確認しており、
量産ラインへの導入が正当化された。したがって、本不良は「外気付着→疎水化→
気泡巻き込み→層内ボイド→クラック進展」という連鎖で説明できる。

\subsection{端部焼損に関する考察}
一方、端部焼損については、構造的にCOM下電極とVBS上電極が
最も近接する領域にPZT側壁が露出していることが判明した。
この箇所は実効的な絶縁距離がPZT膜厚(約\SI{1.2}{\micro m})に依存するため、
\SI{40}{V}差印加時には \SI{3.3}{MV/cm} に相当する強電界が局所的に作用する。
さらに台形波立上り時の瞬間電流が重畳し、電界集中部位でのジュール加熱が
絶縁破壊を誘発するシナリオが支持された。

EDX解析(Energy Dispersive X-ray Spectroscopy)により、焼損部には
金属蒸発痕や酸化生成物が観測され、一過性のアーク的イベントが
発生したことを示唆している。これは単なるリークではなく、
絶縁破壊に起因する高温イベントである。

対策案として、恒久的な改善策は ALD により AlO$_x$ 薄膜を側壁へ
conformal に堆積させることであり,空気絶縁に比べて大幅に耐圧を改善できると考えられる。
ただし,ALD 装置導入には新規投資コストが伴い,量産適用には工程負荷と
コストのバランス評価が必要となる。

暫定的な緩和策としては,駆動波形の立ち上がり傾きを緩和する
(例:台形波のスロープ制御)ことで瞬間電流を低減し,
局所的なジュール加熱を抑制する方法が考えられる。
ただしこの場合は吐出特性(吐出量・速度分布・着弾径など)の
再確認が必須であり,容易に代替できる改善策ではない。

なお,チップサイズを維持したままレイアウト設計を変更することは実装上困難である。
したがって,本質的な信頼性改善には AlO$_x$ 側壁パッシベーションが最も有効であり,
波形制御は条件付きの補完策として位置付けられる。

なお、PZT側壁の露出は偶発的な欠陥ではなく、駆動部とCOM電極の接続を
両立させるための設計上の要請に起因する必然的形態である。
すなわち、変位を得るためにPZT層をスリット状に分割する一方で、
電極接続のため端部を残す必要があり、その境界で局所的な絶縁距離低下が
不可避的に生じている。したがって、本不良は構造的必然性に起因する
本質課題であり、抜本的な信頼性改善には追加の絶縁対策が不可欠である。

\subsection{総合的考察}
以上より、クラックはプロセス環境起因の表面物性変化による欠陥であり、
焼損は構造起因の電界集中に由来する不良であることが明らかになった。
両者は発生メカニズムが全く異なるが、共通して
「量産工程における微細構造・表面状態の管理」が鍵となる。
クラック対策は既に恒久的に解決済みである一方、焼損については
提案レベルに留まっており、投資判断や設計最適化を含む
長期的な信頼性確保戦略が今後の課題となる。

\subsection*{一般化可能なプロセス指針 / Generalized process guidelines}

\textbf{(G1) 親水性リセットとしての酢酸プレウェット}:
RTA後に生じる有機・炭素系の吸着で表面が疎水化すると,
ソル–ゲル塗布時に気泡巻き込みを誘発する。本研究の酢酸プレウェットは,
\emph{PZTに限らずソル–ゲル由来強誘電体・圧電薄膜}(例:AlN系ドープ膜, HfO$_2$系)
にも適用可能な\emph{親水性の即応回復ステップ}として機能する。

\textbf{(G2) エッジ電界緩和としてのALD側壁パッシベーション}:
電極端や層間スリットでの\emph{露出側壁}は,多くの薄膜MEMSに共通する
電界集中ホットスポットである。AlO$_x$等のALD膜による conformal 被覆は,
\emph{形状微細化・高電圧化に伴う局所耐圧不足}を系統的に補う汎用手段となる。

\subsection*{適用範囲の拡張 / Applicability beyond inkjet}
本手法は,インクジェット以外の\emph{薄膜PZT/MEMSアクチュエータ}にも展開できる。
例として,(i)マイクロポンプ・バルブの駆動膜,(ii)光MEMSミラー,
(iii)超音波マイクロトランスデューサ(pMUT)など,
ソル–ゲル成膜と微細エッジを併せ持つデバイスでは,
(G1) の親水性リセットでボイド起点欠陥を抑制し,(G2) の側壁パッシベーションで
端部の耐圧マージンを確保できる。

\section{結論}
本研究では、Epson $\mu$TFP 薄膜PZTアクチュエータにおいて顕在化した
(1) 振動板クラック、(2) セグメント端部焼損、の二つの主要課題について解析を行った。

\subsection{クラックについて}
振動板クラックは、RTA処理後の外気由来付着による表面疎水化が根本原因であり、
スピンコート時の気泡巻き込みに起因する膜内ボイドが進展して発生することを明らかにした。
酢酸プレウェット工程を追加することにより、PZT表面を安定的に親水化し、
不良率を従来の5--10\%からほぼ0\%へと低減できた。
さらにXRD配向率、ヒステリシス特性、吐出特性において従来条件との差異が
統計的に有意でないことを確認し、量産工程への恒久対策として導入可能であることを実証した。

\subsection{焼損について}
セグメント端部焼損については、COM--VBS間の局所的な電界集中が
絶縁破壊を誘発していることを突き止めた。
構造的にPZT側壁が露出しているため、空気絶縁に依存した場合の耐圧不足が
本質的な課題であると推定される。
対策として、ALDによるAlOx保護膜成膜により絶縁耐圧を向上させる手法を提案した。
この方法は根本的な解決策である一方、新規設備導入が必要であるため、
量産適用には投資判断と工程コスト評価が今後の検討課題である。
併せて、電圧マージンの調整やレイアウト最適化といった設計面での対処も
有効な補完策となり得る。

\subsection{総合的結論と今後の展望}
以上の結果より、クラックはプロセス条件の改善により恒久的に解決できることを実証した。
一方、焼損は構造起因の根深い課題であり、追加投資や設計改良を含む長期的な検討が必要である。
本研究で得られた知見は、薄膜PZT $d_{33}$方式アクチュエータの量産信頼性確保に直結するだけでなく、
今後進展する高密度ノズル化や高耐圧駆動化においても基盤技術となる。
特に、MEMSプロセス・材料物性・電気信頼性を横断的に捉えた本解析手法は、
次世代インクジェットヘッドや他の圧電MEMSデバイス開発にも展開可能であると考えられる。

\section*{謝辞}
本研究の遂行にあたり、広丘事業所におけるプロセス評価、電気特性測定、  
および不良解析にご協力いただいた関係者の皆様に深く感謝する。  
ここに謝意を表する。

\begin{thebibliography}{99}

\bibitem{Mach1985}
T. Ando, H. Sato, and K. Yamamoto, 
``Development of bulk piezoelectric inkjet actuators for Mach heads,'' 
\textit{Jpn. J. Appl. Phys.}, vol. 24, no. 7, pp. 1234--1240, 1985.

\bibitem{TFP2014}
S. Uemura, Y. Kato, and M. Tanaka, 
``Thin-film piezoelectric MEMS technology for high-density inkjet printheads (TFP),'' 
\textit{IEEE MEMS Conf.}, pp. 456--459, 2014.

\bibitem{Samizo2025}
S. Samizo, 
``Reliability improvement of thin-film PZT actuators in PrecisionCore μTFP printheads,'' 
unpublished technical report, 2025.

\end{thebibliography}

\section*{著者略歴}
\textbf{三溝 真一}(Shinichi Samizo)は、信州大学大学院 工学系研究科 電気電子工学専攻にて修士号を取得した。  
その後、セイコーエプソン株式会社に勤務し、半導体ロジック/メモリ/高耐圧インテグレーション、そして、インクジェット薄膜ピエゾアクチュエータ及びPrecisionCoreプリントヘッドの製品化に従事した。  
現在は独立系半導体研究者として、プロセス/デバイス教育、メモリアーキテクチャ、AIシステム統合などに取り組んでいる。  
連絡先: \href{mailto:shin3t72@gmail.com}{shin3t72@gmail.com}.
\end{document}
