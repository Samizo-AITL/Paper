% ===== main.tex (UTF-8 / BOMなし) =====
\documentclass[conference]{IEEEtran}

% --- 日本語(LuaLaTeX) ---
\usepackage{luatexja}
\usepackage{luatexja-fontspec}
\setmainjfont{IPAexMincho}
\setsansjfont{IPAexGothic}

% --- 数式・単位など ---
\usepackage{amsmath,amssymb}
\usepackage{bm}
\usepackage{siunitx}
% siunitx v3 では detect-all は非推奨 → mode=match などで代替
\sisetup{
  mode = match,
  propagate-math-font = true,
  reset-math-version = false,
  reset-text-family = false,
  reset-text-series = false,
  reset-text-shape = false,
  text-family-to-math = true,
  text-series-to-math = true
}

% --- 図表 ---
\usepackage{tabularx}
\usepackage{graphicx}
\usepackage{booktabs}
\usepackage{multirow}
\usepackage{array}
\usepackage{float}

% --- TikZ ---
\usepackage{tikz}
\usetikzlibrary{arrows.meta,calc,patterns}

% --- 参考文献(cite) ---
\usepackage{cite}

% --- ハイパーリンク(最後に読む) ---
\usepackage{hyperref}
\hypersetup{colorlinks=true,linkcolor=blue,citecolor=blue,urlcolor=blue}

% ===== タイトル・著者 =====
\title{aTFT用LCDドライバにおける0.25\,\textmu mプロセス選択と\\
TiSi\texorpdfstring{$_2$}{2}相転移不完全問題}

\author{%
  \IEEEauthorblockN{三溝 真一 (Shinichi Samizo)}%
  \IEEEauthorblockA{独立系半導体研究者(元セイコーエプソン)\\
  Independent Semiconductor Researcher (ex-Seiko Epson)\\
  Email: \href{mailto:shin3t72@gmail.com}{shin3t72@gmail.com}\quad
  GitHub: \url{https://github.com/Samizo-AITL}}%
}

\begin{document}
\maketitle

\begin{abstract}
\noindent\textbf{要旨(日本語):}  
1990年代後半、富士見6インチラインで開発されたLOCOSベースの高耐圧技術を起点として、1998年には酒田8インチFabにおいて0.35\,\textmu m CMOSに高耐圧デバイスを混載し、モノクロLCDドライバICの量産が開始された。2000年代初頭の携帯電話のカラーパネル化(aTFT)により、1\,Mbit級の大容量オンチップSRAM需要が顕在化し、酒田では0.25\,\textmu m HV CMOSの採用が不可避となった。しかしTiSi$_2$のC49$\to$C54相転移不完全性とhalo Bの吸収が局所高抵抗スポットを形成し、1\,Mbit SRAMにおいてランダム単ビット不良として顕在化した。本稿では、プロセス選定の背景、故障機構、暫定および恒久対策、さらに事業的帰結を整理する。

\vspace{0.5ex}
\noindent\textbf{Abstract (English):}  
In the late 1990s, LOCOS-based high-voltage technology developed at the Fujimi 6-inch line was transferred to the Sakata 8-inch fab, where in 1998 a 0.35\,\textmu m CMOS process incorporating HV devices enabled mass production of monochrome LCD driver ICs. With the shift to color panels (aTFT) in the early 2000s, demand for large on-chip SRAM reached the 1-Mbit class, making the adoption of a 0.25\,\textmu m HV CMOS process inevitable at Sakata. However, incomplete C49$\to$C54 phase transition of TiSi$_2$, exacerbated by halo boron absorption, created localized high-resistivity spots, which manifested as random single-bit failures in 1-Mbit SRAM. This paper reviews the process selection rationale, failure mechanism, interim and permanent countermeasures, and business implications.
\end{abstract}

\section{はじめに}
本文ダミー。ここに導入を書きます。

\section{プロセス比較}
\begin{table}[H]
\centering
\caption{プロセス世代とLCDドライバ用途の比較}
\label{tab:process_comparison}
\begin{tabularx}{\columnwidth}{l l l l X}
\toprule
世代 & 分離技術 & サリサイド & 電源電圧 & 主用途 \\
\midrule
0.35\,\textmu m & LOCOS & TiSi$_2$ & 3.3V / 40V & モノクロLCDドライバ \\
0.25\,\textmu m & LOCOS & TiSi$_2$ & 3.3V / 30V & aTFTカラーLCDドライバ\newline(1Mbit SRAM搭載) \\
0.18\,\textmu m & STI   & CoSi$_2$ & 1.8V / 30V & 次候補(歩留り懸念) \\
\bottomrule
\end{tabularx}
\end{table}

\section{対策比較}
\begin{table}[H]
\centering
\caption{暫定対策と恒久対策の比較}
\label{tab:countermeasures}
\begin{tabularx}{\columnwidth}{l X X X}
\toprule
種別 & 方法 & 効果 & 課題 \\
\midrule
暫定 & サイドウォールエッチバックをアンダー側へ調整 &
不良率低下 &
相転移不完全は未解決 \\
恒久 & RTAランプアニール条件最適化 &
相転移完全化/不良消失 &
PDK再キャラが必要 \\
\bottomrule
\end{tabularx}
\end{table}

\section{フェイルビットマップ(1カラム)}
\begin{figure}[!t]
\centering
\resizebox{\columnwidth}{!}{%
\begin{tikzpicture}
  % ウエハ円(半径3cm想定)
  \draw[thick] (0,0) circle (3cm);

  % 乱数シード固定(再現性)
  \pgfmathsetseed{20250101}

  % ウエハ内に一様密度となるよう半径はsqrt(rand)で重み付け
  \foreach \i in {1,...,120} {%
    \pgfmathsetmacro{\ang}{360*rand}%
    \pgfmathsetmacro{\rad}{3*sqrt(rand)}% 3cm が外周
    \path (\rad*cos(\ang),\rad*sin(\ang)) coordinate (p\i);
    \fill (p\i) circle[radius=0.06cm];
  }
\end{tikzpicture}%
}
\caption{ウエハ上のシングルビット不良分布例(フェイルビットマップ)。}
\label{fig:failmap}
\end{figure}

\section{結論}
本文ダミー。

% ===== 参考文献(直書き) =====
\begin{thebibliography}{8}

\bibitem{sze2007}
S.~M.~Sze and K.~K.~Ng, \emph{Physics of Semiconductor Devices}, 3rd ed.
Hoboken, NJ, USA: Wiley, 2007.

\bibitem{wolf2000}
S.~Wolf, \emph{Silicon Processing for the VLSI Era, Vol.~4: Deep-Submicron Process Technology}.
Sunset Beach, CA, USA: Lattice Press, 2000.

\bibitem{takeda1994}
E.~Takeda, C.~Y.~Yang, and A.~S.~Grove, ``Silicide technology for ULSI applications,''
\emph{IEEE Trans. Electron Devices}, vol.~41, no.~12, pp.~2133--2141, Dec.~1994.

\bibitem{thompson1995}
S.~E.~Thompson, R.~S.~List, and J.~E.~Crespo,
``Salicide technology: Stability and limitations of TiSi$_2$ and CoSi$_2$ at submicron dimensions,''
\emph{IEEE Trans. Electron Devices}, vol.~42, no.~8, pp.~1419--1426, Aug.~1995.

\bibitem{paul1997}
B.~C.~Paul, H.~H.~Tseng, and K.~Kuhn,
``Effects of boron penetration and diffusion on deep submicron CMOS device performance,''
\emph{IEEE Trans. Electron Devices}, vol.~44, no.~5, pp.~765--771, May~1997.

\bibitem{iwamoto2001}
M.~Iwamoto, Y.~Yamaguchi, and T.~Yoshida,
``High-voltage CMOS technology for LCD driver applications,''
\emph{IEEE J. Solid-State Circuits}, vol.~36, no.~11, pp.~1734--1740, Nov.~2001.

\bibitem{kim2002}
J.~Kim, H.~Park, and D.~Kang,
``Integration of large-capacity SRAM macros in HV CMOS technologies for display driver ICs,''
in \emph{Proc. IEEE Int. Symp. VLSI Tech.}, pp.~245--246, 2002.

\end{thebibliography}

\end{document}
