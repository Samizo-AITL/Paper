\documentclass[conference]{IEEEtran}
\IEEEoverridecommandlockouts
\usepackage{amsmath,amssymb}
\usepackage{siunitx}
\usepackage{graphicx}
\usepackage{subfig}
\usepackage{booktabs}
\usepackage{cite}

\title{Low-Cost Integration of 1.8-V FeFET on 0.18-\texorpdfstring{$\mu$m}{µm} CMOS:\\
+1 Mask and a Single ALD Tool, with Comprehensive Reliability Assessment}

\author{\IEEEauthorblockN{Shinichi Samizo}
\IEEEauthorblockA{Independent Semiconductor Researcher\\
Former Engineer at Seiko Epson Corporation\\
Email: shin372@gmail.com, GitHub: \texttt{https://github.com/Samizo-AITL}}}

\begin{document}
\maketitle

\begin{abstract}
Ferroelectric FETs (FeFETs) are promising CMOS-compatible embedded non-volatile memories (NVMs).  
This paper demonstrates a 1.8 V FeFET module integrated on a legacy 0.18 $\mu$m CMOS process with only one additional mask and a single ALD tool.  
Fabricated devices show endurance exceeding $10^{5}$ program/erase cycles and retention longer than 10 years at 85\,$^\circ$C.  
Reliability was characterized on FeCAP/FeFET structures: time-zero dielectric breakdown (TZDB), time-dependent dielectric breakdown (TDDB), endurance, and retention.  
The approach provides a cost-effective path to extend mature-node lifetimes and to enable embedded NVM for automotive/industrial/IoT, while high-temperature retention remains the key limiter.
\end{abstract}

\section{Introduction}
FeFETs based on HfO$_2$ have gained traction as CMOS-compatible NVMs.  
Most prior work targets advanced nodes; however, mature nodes ($\sim$0.18 $\mu$m) remain widely used in automotive/industrial markets where long supply lifetimes and low cost are critical.  

This work contributes:  
(1) a +1 mask low-cost module,  
(2) only one ALD tool added to the line,  
(3) a yield-friendly SRAM+FeFET system usage model, and  
(4) comprehensive reliability evidence on FeCAP/FeFET.

\section{Process Integration}
Baseline is a 0.18 $\mu$m CMOS platform (1.8 V core, optional 3.3 V I/O).  
The FeFET module is inserted after poly definition and silicide/RTA, requiring minimal line modification.  

\subsection{Process Flow}
Fig.~1 shows the additional process steps.  
\subsection{Cross Section}
The HZO/Al$_2$O$_3$/TiN stack is illustrated in Fig.~2.

\section{Devices and Methods}
Test structures include FeCAPs (flat/comb) and 100~µm $\times$ 100~µm FeFET cells.  
Programming used $\pm$2.3–2.7 V, 1–50 µs pulses. Keysight B1500A and a manual probe were used.

\subsection{Protocols}
\begin{itemize}
  \item TZDB: constant ramp at 0.1 V/s, RT–125\,$^\circ$C.  
  \item TDDB: constant-voltage stress at $\pm$2.3/2.5/2.7 V, 85\,$^\circ$C and 125\,$^\circ$C.  
  \item Endurance: $\pm$2.5 V, 10 µs, 10 kHz up to $10^5$ cycles.  
  \item Retention: 25\,$^\circ$C, 85\,$^\circ$C, 125\,$^\circ$C, with Arrhenius extrapolation.  
\end{itemize}

\section{Results}

\subsection{Reliability Overview}
All reliability metrics are summarized in Figs.~3–6.

\subsection{TZDB}
CDF plots (Fig.~3a) show breakdown distributions consistent with uniform dielectric thickness.

\subsection{TDDB (CDF and Weibull)}
Fig.~4a presents cumulative distribution function (CDF) results under multiple stress conditions.  
Subsequent Weibull plots (Fig.~4b) extract slope ($\beta$) and lifetime ($\eta$), enabling lifetime projections.  

\subsection{Endurance}
Endurance up to $10^5$ cycles verified.  
The memory window shrinks 20–30\% (Fig.~5).  
Compact fit is $\Delta V_{\text{th}}(N)=1.12 - 0.05\log_{10}N$.

\subsection{Retention}
Arrhenius extrapolation with $E_a \approx 1.1$ eV predicts:  
$>$100 years @25\,$^\circ$C, $>$10 years @85\,$^\circ$C, and only months @150\,$^\circ$C (Fig.~6).  
The extracted activation energy matches oxygen vacancy diffusion in HfO$_2$, confirming physical validity.

\section{System Architecture (SRAM + FeFET)}
The SoC uses a single 1.8 V core domain for logic, SRAM, and FeFET access.  
Write/erase pulses ($\pm$2.3–2.7 V, 1–50 µs) are generated by an on-chip charge pump.  
A backup controller copies SRAM to the FeFET array on power-fail detection and restores them at power-up.  
Optional 3.3 V domain supports I/O and AMS blocks (ADC/DAC, LDO).  
Figs.~7–8 illustrate the architecture and backup/restore flow.

\section{Discussion}
The HZO/Al$_2$O$_3$/TiN stack provides sufficient reliability for industrial/consumer eNVM at 85\,$^\circ$C.  

\subsection{High-Temperature Challenges}
For $>$125\,$^\circ$C automotive use, further improvements are required:  
\begin{itemize}
  \item interfacial layer (IL) optimization,  
  \item crystallinity control,  
  \item minimizing oxygen vacancy generation,  
  \item refresh/rewrite schemes,  
  \item error correction code (ECC).  
\end{itemize}

\subsection{System Relevance}
The combination of $10^5$ endurance cycles and 10-year retention @85\,$^\circ$C aligns with mature-node product lifetimes, supporting adoption in industrial SoCs.

\section{Conclusion}
We realized a FeFET module on 0.18 µm CMOS with one extra mask and one ALD tool.  
Devices exhibit $>10^5$ cycles and $>10$ years retention at 85\,$^\circ$C.  
This method extends mature-node lifetimes and enables cost-effective embedded NVM for automotive/industrial/IoT.

\section*{Acknowledgment}
The author thanks collaborators for valuable discussions.

\bibliographystyle{IEEEtran}
\begin{thebibliography}{99}
\bibitem{Boe}
T. Böscke et al., Appl. Phys. Lett., vol. 99, p. 102903, 2011.  
\bibitem{Mull}
J. Müller et al., Appl. Phys. Lett., vol. 99, p. 112901, 2012.  
\bibitem{Miko}
T. Mikolajick et al., J. Appl. Phys., vol. 125, p. 204103, 2019.  
\bibitem{Mull2}
J. Müller et al., IEEE Trans. Electron Devices, vol. 62, no. 12, pp. 4158–4166, 2015.  
\bibitem{Park}
J. Park et al., IEEE Electron Device Lett., vol. 41, no. 5, pp. 711–714, 2020.  
\bibitem{Naka}
H. Nakamura et al., IEEE Trans. Device Mater. Rel., vol. 3, no. 4, pp. 132–136, 2003.  
\bibitem{Yama}
K. Yamazaki et al., Jpn. J. Appl. Phys., vol. 57, 04FB07, 2018.  
\end{thebibliography}

\begin{IEEEbiography}{Shinichi Samizo}
Shinichi Samizo has over 25 years of experience in semiconductor process integration and actuator development.  
After studying control theory and EM modeling in academia, he joined Seiko Epson in 1997 and worked on 0.35–0.18 µm CMOS logic/memory/HV integration, DRAM, and LCD drivers.  
Later he contributed to PZT actuator development and the PrecisionCore inkjet head.  
He is currently an independent researcher, publishing educational materials via the “Project Design Hub”.
\end{IEEEbiography}

\end{document}
