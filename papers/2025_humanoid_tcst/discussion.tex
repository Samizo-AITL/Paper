\section{Discussion}
Table~\ref{tab:humanoid_comparison} compares the proposed Samizo-AITL PoC
with two representative humanoid platforms: Boston Dynamics \textit{Atlas} 
and Tesla \textit{Optimus}.
Atlas excels at dynamic acrobatics, while Optimus emphasizes scalable
industrial deployment. In contrast, the proposed PoC targets autonomy,
fault tolerance, and sustainable operation.

A first distinctive feature is the integration of LLMs into the hierarchical
control loop. Instead of replacing classical controllers, the LLM layer
generates goals, interprets anomalies, and provides conversational interfaces.
This complements the FSM for supervisory logic and PID/state-space methods
for stabilization, creating a hybrid architecture that combines the safety
of model-based control with the adaptability of data-driven intelligence.

A second differentiator is energy autonomy. The PoC integrates
piezoelectric, photovoltaic, and regenerative harvesting,
allowing up to 20\% of the power budget to be sustained
without external charging. This contrasts with Atlas and Optimus,
which rely exclusively on batteries. Together with FRAM/EEPROM-based
checkpoint-and-resume, the system ensures resilient operation
in remote or resource-limited environments.

A third contribution is educational reproducibility.
All specifications, models, and proof-of-concept results are openly published
in bilingual (Japanese–English) format on GitHub Pages.
This open-science approach enables replication, lowers barriers for students,
and positions the PoC as both a research prototype and an instructional
benchmark in control engineering education.

Overall, the proposed system demonstrates that hybrid architectures
can extend humanoid robotics beyond performance and manufacturability,
toward autonomy, resilience, and sustainable deployment.
