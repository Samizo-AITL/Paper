\section{Conclusion}
This paper presented a flagship proof-of-concept humanoid robot control system
that integrates finite state machines (FSM), proportional-integral-derivative (PID) control,
state-space methods, and large language models (LLMs) within a cross-node chipset architecture.
The architecture spans a 22 nm SoC for inference and control, a 0.18 µm AMS sensor hub,
and a 0.35 µm LDMOS power drive with external GaN/MOSFET integration.
SystemDK-based validation confirmed posture recovery within 200 ms,
gait stability improved by 30\%, and energy efficiency gains of 15\%.
Self-harvesting contributed up to 20\% of the total power budget,
while checkpoint-and-resume capability enabled robust and fast mission continuity.

The main contributions of this work are as follows:
\begin{itemize}
  \item Introduction of a hierarchical control framework
        that combines FSM, PID/state-space, and LLM layers
        for enhanced autonomy and fault tolerance;
  \item Demonstration of cross-node semiconductor co-design
        integrating digital, AMS, and power technologies in one system;
  \item Experimental validation of resilience and sustainability
        through posture recovery, gait stability, and energy harvesting KPIs;
  \item Open publication of models and PoC results,
        reinforcing reproducibility and educational impact.
\end{itemize}

Future work will extend this PoC toward larger-scale prototypes
with high-torque actuation via GaN integration, optimized energy harvesting,
and deployment in real-world field scenarios such as mountainous
or disaster environments.
Beyond robotics, the proposed hybrid control concept suggests
a broader paradigm for intelligent systems,
bridging classical model-based control and modern AI-driven reasoning.
