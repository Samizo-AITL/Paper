\section{Experimental Results}
System-level validation was conducted using SystemDK multi-physics modeling
combined with hardware-in-the-loop prototypes.

\subsection{Posture Recovery}
Disturbance rejection experiments show that the proposed FSM+PID+LLM
recovers upright posture within 200 ms,
whereas PID-only control requires more than 500 ms.

\subsection{Gait Stability}
Center-of-mass (CoM) deviation was measured during continuous walking.
The hybrid control reduced RMS CoM deviation by 30\% compared to PID-only baselines.

\subsection{Energy Efficiency}
By coordinating control with piezoelectric, photovoltaic, and regenerative harvesting,
overall energy efficiency improved by 15\%.
Self-powering covered up to 20\% of total energy consumption in field scenarios.

\subsection{Memory Subsystem}
Checkpoint and resume functionality using FRAM/EEPROM
achieved recovery within 10 ms without full system reinitialization.
Durability testing confirmed $10^{12}$ write cycles within PoC requirements.
