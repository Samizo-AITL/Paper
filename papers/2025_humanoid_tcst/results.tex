\section{Experimental Results}
System-level validation was performed using SystemDK multi-physics modeling
and hardware-in-the-loop prototypes. The evaluation focused on disturbance recovery,
gait stability, energy efficiency, memory subsystem performance, and comparison
with existing humanoid platforms.

\subsection{Posture Recovery}
Disturbance rejection tests were conducted on a flat surface under controlled lateral pushes 
applied at the torso level during continuous walking with a gait cycle of 0.8~s. 
In the SystemDK co-simulation, sensor quantization noise, encoder jitter, thermal drift, 
and mechanical stress effects were modeled using vendor-specified parameters 
for AMS (0.18~µm) and LDMOS (0.35~µm) technologies. 
Each experiment was repeated ten times, and recovery times were averaged. 
Results indicate that the proposed FSM+PID+LLM controller
restores upright posture within $200 \pm 15$~ms,
compared to $520 \pm 25$~ms with PID-only control.
This demonstrates a statistically consistent more than twofold improvement in recovery speed.

\subsection{Gait Stability}
Center-of-mass (CoM) deviation was measured during continuous walking. 
The hybrid architecture reduced RMS CoM deviation by approximately $30\% \pm 4\%$
relative to the PID-only baseline, 
confirming enhanced whole-body coordination across multiple trials.

\subsection{Energy Efficiency}
By combining classical control with piezoelectric, photovoltaic, and regenerative harvesting,
the system achieved an average energy efficiency improvement of $15\% \pm 3\%$. 
In field scenarios, self-harvesting contributed up to 20\% of the total power budget,
significantly extending operational duration without external charging.

\subsection{Memory Subsystem}
Checkpoint-and-resume functionality using FRAM/EEPROM-based storage
enabled system recovery within $10 \pm 1$~ms without full reinitialization. 
Endurance tests validated $10^{12}$ write cycles,
satisfying durability requirements for continuous PoC operation.

\subsection{Comparison with Existing Humanoids}
Table~\ref{tab:humanoid_comparison} compares the proposed Samizo-AITL PoC 
with Boston Dynamics Atlas and Tesla Optimus.
Unlike Atlas, which prioritizes dynamic acrobatics,
and Optimus, which targets scalable industrial deployment,
the proposed system emphasizes autonomy, fault tolerance,
and energy self-sufficiency.

\begin{table*}[t]
\caption{Comparison of world-leading humanoid robots. 
Atlas excels in dynamic acrobatics, Optimus prioritizes scalable industrial deployment, 
and the proposed Samizo-AITL PoC emphasizes autonomy, fault tolerance, and sustainable energy use.}
\label{tab:humanoid_comparison}
\centering
\renewcommand{\arraystretch}{1.2}
\footnotesize
\begin{tabular}{p{2.5cm} p{2.5cm} p{2.5cm} p{6.5cm}}
\toprule
\textbf{Feature} & \textbf{Atlas} & \textbf{Optimus} & \textbf{Samizo-AITL PoC} \\
\midrule
Goal & Research (dynamic demos) & Mass production for logistics & Educational culmination; autonomy + fault tolerance \\
Control & Dynamic (jumps/flips) & Simple walking/manipulation & FSM + PID + State-space + LLM \\
Disturbance Recovery & Robust & Limited & Posture recovery $\leq 200 \pm 15$ ms \\
Conversation & None & Planned & Natural via LLM \\
Person Recognition & None & Not implemented & Face + voiceprint \\
Navigation & Experimental & Planned factory nav. & SLAM + voice command \\
Damage Tolerance & Stops after falls & Not implemented & Continues with remaining actuators \\
Power Output & Battery + hydraulics & Internal battery & LDMOS + GaN/MOSFET (high torque) \\
Energy Autonomy & Battery only & Battery only & Piezo + PV + regenerative harvesting \\
Openness & Closed demos & Partially open & Open, bilingual on GitHub Pages \\
\bottomrule
\end{tabular}
\end{table*}
