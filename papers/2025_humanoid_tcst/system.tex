\section{System Architecture}

\subsection{Cross-Node Chipset}
The humanoid control system is implemented as a heterogeneous cross-node chipset integrating:
\begin{itemize}
  \item \textbf{Brain SoC (22 nm)}: executes LLM inference, FSM management, and LQR/LQG control;
  \item \textbf{Sensor Hub (0.18 µm AMS)}: acquires multimodal data from cameras, IMU, encoders, force/pressure sensors, and microphones;
  \item \textbf{Power Drive (0.35 µm LDMOS with external GaN/MOSFET)}: delivers high-torque actuation with current and temperature monitoring;
  \item \textbf{Energy Harvesting Subsystem}: incorporates piezoelectric, photovoltaic, and regenerative sources for extended autonomy;
  \item \textbf{Memory Subsystem}: employs LPDDR for active tasks and FRAM/EEPROM for checkpointing and persistent logging.
\end{itemize}

\subsection{AITL Three-Layer Control Architecture}
The control hierarchy follows the \textbf{AITL (AI-Integrated Three-Layer)} paradigm, 
where classical feedback and supervisory logic are augmented with AI-based reasoning:
\begin{itemize}
  \item \textbf{Inner Loop (PID Control)}: guarantees joint-level stability, disturbance rejection, and real-time responsiveness;
  \item \textbf{Middle Loop (FSM Control)}: orchestrates sequential behaviors such as standing, walking, turning, recovery, and energy-saving modes by supervising PID controllers;
  \item \textbf{Outer Loop (LLM Supervision)}: interprets anomalies, generates high-level goals, and, when necessary, reconfigures the control system itself by retuning PID gains or revising FSM transition rules.
\end{itemize}
This layered design ensures stability through PID, structured sequencing via FSM, 
and adaptability through LLM-driven supervision, establishing a hybrid paradigm that 
bridges model-based control and AI-driven reasoning.

\subsection{SystemDK Co-Design Flow}
As illustrated in Fig.~\ref{fig:systemdk_flow}, the proof-of-concept was modeled and verified
using SystemDK. The co-design flow captures cross-node interactions among the digital SoC,
AMS sensor hub, power drive, and energy harvesting modules, enabling multi-physics
co-simulation including noise, thermal, and mechanical stress effects.

\begin{figure}[t]
  \centering
  % 1) Use PDF if available
  \IfFileExists{figures/systemdk_flow.pdf}{%
    \includegraphics[width=\linewidth,keepaspectratio]{figures/systemdk_flow.pdf}%
  }{%
  % 2) Fallback to TikZ
  \IfFileExists{figures/systemdk_flow.tikz}{%
    \resizebox{\linewidth}{!}{% figures/systemdk_flow.tikz  (tikzpicture だけ!)
\begin{tikzpicture}[
  font=\footnotesize,
  >={Latex[length=2mm]},
  node distance=6mm and 6mm,
  box/.style={draw,rounded corners,thick,inner sep=4pt,align=center},
  grp/.style={draw,rounded corners,inner sep=6pt}
]
% 左:SoC/AMS/Power/Memory/Harvest
\node[box,minimum width=34mm,minimum height=8mm] (soc)
  {Brain SoC\\(22 nm)\\LLM, FSM, LQR/LQG};
\node[box,below=of soc,minimum width=34mm,minimum height=8mm] (ams)
  {AMS Sensor Hub\\(0.18 µm)\\IMU, Force, Audio, Vision};
\node[box,below=of ams,minimum width=34mm,minimum height=8mm] (pwr)
  {Power Drive\\(0.35 µm LDMOS +\\ ext. GaN/MOSFET)};
\node[box,below=of pwr,minimum width=34mm,minimum height=8mm] (mem)
  {Memory\\LPDDR + FRAM/EEPROM};
\node[box,below=of mem,minimum width=34mm,minimum height=8mm] (harv)
  {Energy Harvesting\\Piezo / PV / Regen};
\node[grp,fit=(soc)(ams)(pwr)(mem)(harv),
      label={[yshift=-2mm]above:\textbf{Cross-Node Chipset}}] (L) {};

% 右:制御レイヤ
\node[box,minimum width=48mm,minimum height=8mm, right=22mm of soc] (llm)
  {LLM Layer\\plan, anomaly interpret., conversation};
\node[box,below=of llm,minimum width=48mm,minimum height=8mm] (fsm)
  {FSM Layer\\standing / walking / turning / recovery};
\node[box,below=of fsm,minimum width=48mm,minimum height=8mm] (phys)
  {Physical Control\\PID + State-Space (LQR/LQG)};
\node[box,below=of phys,minimum width=48mm,minimum height=8mm] (drive)
  {Drive Layer\\torque control \& safety monitor};
\node[box,below=of drive,minimum width=48mm,minimum height=8mm] (energy)
  {Energy Layer\\harvesting, storage, power mgmt.};
\node[grp,fit=(llm)(fsm)(phys)(drive)(energy),
      label={[yshift=-2mm]above:\textbf{Hierarchical Control Layers}}] (R) {};

% 接続
\draw[thick,->] (soc.east) -- (llm.west);
\draw[thick,->] (llm.south) -- (fsm.north);
\draw[thick,->] (fsm.south) -- (phys.north);
\draw[thick,->] (phys.south) -- (drive.north);
\draw[thick,->] (drive.south) -- (energy.north);

\draw[thick,->] (ams.east) -- ++(8mm,0) |- (phys.west) node[pos=0.75,above left]{sensors};
\draw[thick,->] (pwr.east) -- ++(8mm,0) |- (drive.west) node[pos=0.7,above left]{actuators};
\draw[thick,->] (mem.east) -- ++(8mm,0) |- (fsm.west) node[pos=0.7,above left]{checkpoints};
\draw[thick,->] (harv.east) -- ++(8mm,0) |- (energy.west) node[pos=0.7,above left]{harvest};

\draw[thick,->] (phys.west) -- ++(-8mm,0) |- (soc.west)
  node[pos=0.25,above]{cmds/telemetry};

% SystemDK の囲み
\node[grp,dashed,fit=(L)(R),
  label={[yshift=2mm]below:\strut \textit{SystemDK co-sim: digital SoC + AMS + Power + Harvesting}}] {};
\end{tikzpicture}
}%
  }{%
  % 3) Placeholder if missing
    \fbox{\begin{minipage}[c][0.44\linewidth][c]{0.94\linewidth}\centering
      \vspace{0.3em}\textit{Placeholder: \path{figures/systemdk_flow.(pdf|tikz)} not found}\\
      (Commit the figure to replace this box.)\vspace{0.3em}
    \end{minipage}}%
  }}%
  \caption{SystemDK-based integrated design flow spanning SoC (22 nm), AMS (0.18 µm),
  LDMOS power drive (0.35 µm), and energy harvesting subsystems.}
  \label{fig:systemdk_flow}
\end{figure}

\subsection{Key Performance Indicators}
The architecture was evaluated against several key performance indicators (KPIs),
summarized in Table~\ref{tab:kpi_summary}. These metrics guided design trade-offs
in stabilization, efficiency, and resilience.

\begin{table}[t]
\caption{Summary of Key Performance Indicators (KPIs)}
\label{tab:kpi_summary}
\centering
\renewcommand{\arraystretch}{1.15}
\footnotesize
\begin{tabular}{@{}p{0.48\columnwidth} p{0.44\columnwidth}@{}}
\toprule
\textbf{Metric} & \textbf{Result} \\
\midrule
Posture recovery time & $\leq 200$ ms (baseline PID-only: $>500$ ms) \\
Gait stability (CoM RMS) & $\approx 30\%$ improvement over PID-only \\
Energy efficiency & $+15\%$ with hybrid control and harvesting \\
Self-harvested power & Up to $20\%$ of system power budget \\
Checkpoint resume time & $\leq 10$ ms (FRAM/EEPROM-based) \\
Memory endurance & $10^{12}$ write cycles (FRAM) \\
\bottomrule
\end{tabular}
\end{table}
