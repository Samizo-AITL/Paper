% main.tex : Post-CFET paper (IEEEtran, two-column)

\documentclass[conference]{IEEEtran}

% ---------- Packages ----------
\usepackage{newtxtext,newtxmath}   % Times系フォント
\usepackage{xcolor}
\usepackage{graphicx}
\usepackage{booktabs}
\usepackage{multirow}
\usepackage{standalone}
\usepackage{gincltex}              % \input TikZ 図
\usepackage{tikz}
\usetikzlibrary{positioning,fit,shapes,arrows.meta,mindmap,trees,calc}
\usepackage{siunitx}
\usepackage{url}
\usepackage{hyperref}
\usepackage{placeins}              % float barrier (section optionは外す)

% ---------- Macros ----------
\newcommand{\figpath}{figures}     % 図ディレクトリのショートカット
\newcommand{\tikzcol}[2][\linewidth]{%
  \resizebox{#1}{!}{\input{#2}}}

% ---------- Title & Author ----------
\title{Post-CFET Device Architectures: Materials, Integration, and Design Perspectives}

\author{
\IEEEauthorblockN{Shinichi Samizo}
\IEEEauthorblockA{Independent Semiconductor Researcher\\
Project Design Hub, Samizo-AITL\\
\textit{Email:} \href{mailto:shin3t72@gmail.com}{shin3t72@gmail.com}\\
\textit{GitHub:} \href{https://github.com/Samizo-AITL}{Samizo-AITL}}
}

\begin{document}

\maketitle

% ---------- Abstract ----------
\begin{abstract}
CMOS scaling has evolved from Planar MOSFETs to FinFETs, Gate-All-Around (GAA) nanosheets, and Complementary FETs (CFETs). 
CFET improves electrostatic control and mitigates wiring bottlenecks, but silicon is approaching its material and thermal limits. 
This paper reviews post-CFET device candidates including two-dimensional (2D) material FETs, monolithic 3D integration, spintronics/quantum devices, and heterogeneous atomic-scale integration. 
Their physical principles, fabrication challenges, experimental demonstrations, reliability concerns, application domains, and implications for design and education are compared.
\end{abstract}

% ---------- Sections ----------
\section{Introduction}
The semiconductor industry has advanced for more than five decades through device scaling and structural innovation. 
The trajectory from Planar MOSFET $\rightarrow$ FinFET $\rightarrow$ GAA $\rightarrow$ CFET represents the pursuit of enhanced electrostatic control and integration efficiency. 
However, mobility degradation, leakage, wiring delay, and thermal density have become limiting factors, demanding exploration of post-CFET technologies.

\section{Evolution from CMOS to Post-CFET}
Fig.~\ref{fig:evolution} summarizes the historical pathway of CMOS device evolution.

\subsection{Planar MOSFET to FinFET}
Short-channel effects and leakage became critical below 45\,nm. Tri-gate FinFETs improved gate control and became the industry standard.

\subsection{FinFET to GAA}
Cell-height constraints required nanosheets fully surrounded by gates. GAA provides stronger electrostatics but increases process variability.

\subsection{GAA to CFET}
CFET vertically stacks nFET and pFET, improving cell density and reducing interconnect delay. Challenges include heat removal and process complexity.

\subsection{Beyond CFET}
Silicon material limits and wiring dominance demand novel materials, new integration schemes, and alternative state variables (spin, photon, bio).

% ---------- Wide Figure (moved to page 2) ----------
\begin{figure*}[!t]
  \centering
  \tikzcol[0.9\textwidth]{\figpath/evolution_tree.tex}
  \caption{Evolution tree: CMOS $\rightarrow$ CFET $\rightarrow$ post-CFET candidates.}
  \label{fig:evolution}
\end{figure*}

\section{Post-CFET Candidate Technologies}

\subsection{2D Material FETs}
\textbf{Demonstrations:} MoS\textsubscript{2} FET ($L_g$ = 12 nm, IEDM 2023) with $I_{on}/I_{off}=10^7$, SS = 65 mV/dec.  
\textbf{Challenges:} High contact resistance ($R_c \approx 1$ k$\Omega \cdot \mu$m), film non-uniformity (5–10\%), interface traps.  
\textbf{Applications:} Ultra-low-power IoT, flexible electronics, bio-sensing.  
\textbf{Design impact:} Immature SPICE models, limited statistical variability data.

\subsection{Monolithic 3D Integration (M3D)}
\textbf{Demonstrations:} SRAM stacking (IEDM 2019): delay −30\%, area −40\%. AI SoC (Nat. Electronics 2022): energy efficiency +1.7×.  
\textbf{Challenges:} Low-temperature processing ($<450^{\circ}$C), inter-layer $V_{th}$ variation, thermal hotspots $>1$ W/mm$^2$.  
\textbf{Applications:} AI accelerators, memory-centric computing.  
\textbf{Design impact:} Requires 3D P\&R EDA, thermal/mechanical co-simulation.

\subsection{Spintronics / Quantum Devices}
\textbf{Demonstrations:} STT-MRAM endurance $10^{12}$ cycles (IBM), SOT-MRAM write current −40\%, topological FET on/off $=10^3$ at RT.  
\textbf{Challenges:} CMOS compatibility; reduce write current from mA to $\mu$A.  
\textbf{Applications:} Neuromorphic computing, radiation-hardened systems, in-memory logic.  
\textbf{Design impact:} Logic–memory fusion beyond von Neumann architecture.

\subsection{Heterogeneous Atomic-Scale Integration}
\textbf{Demonstrations:} Si+MoS\textsubscript{2} photodetector, responsivity 200 mA/W at 1.55 $\mu$m (Nat. Photonics 2020); CMOS+MEMS sensor chips.  
\textbf{Challenges:} Interface stability, lattice/thermal mismatch, yield.  
\textbf{Applications:} Optical interconnect, medical sensing, aerospace.  
\textbf{Design impact:} Cross-domain EDA required (electrical + optical + mechanical).

\begin{table}[!t]
\centering
\caption{Comparison of post-CFET candidate technologies}
\label{tab:comparison}
\begin{tabular}{lcccc}
\toprule
Tech. & Demonstrations & Rc/Thermal & Reliability & Applications \\
\midrule
2D-FET & $I_{on}/I_{off}=10^7$ & 1k$\Omega\cdot\mu$m & Film var. & IoT/Flex/Bio \\
M3D    & Delay −30\%, +1.7× & $<450^\circ$C & $V_{th}$ shift & AI/Memory \\
Spin   & MRAM $10^{12}$ & RT stability & Write stress & Neuro/Space \\
Hetero & Si+MoS$_2$ PD & PD 200 mA/W & Intf. mismatch & Optic/Medical \\
\bottomrule
\end{tabular}
\end{table}

\section{Comparison Matrix}
Table~\ref{tab:comparison} compares the four candidate technologies.

\section{Design and Educational Perspectives}
Future EDA must integrate multi-physics: heat, stress, quantum, and cross-domain effects. 
Educational curricula should include scaling history, candidate technology reviews, multi-physics simulations, case studies, and system-level design integration.

\section{Future Scenarios}
\begin{itemize}
  \item \textbf{2030s (early):} Lab-scale demonstrations of 2D+CFET and M3D+2D hybrids.
  \item \textbf{2030s (late):} Partial adoption in IoT/AI edge devices.
  \item \textbf{2040s:} Mainstream in HPC and aerospace; fusion of spintronics and M3D enabling non-volatile 3D logic–memory.
\end{itemize}

\section{Conclusion}
Post-CFET represents a paradigm shift from structural scaling to material innovation, integration schemes, alternative physical states, and heterogeneous fusion. 
It holds educational, design, and industrial significance, bridging the gap beyond the silicon era.

% ---------- Additional Figures (end of paper) ----------
\FloatBarrier

\begin{figure*}[!t]
  \centering
  \tikzcol[0.95\textwidth]{\figpath/block_diagram.tex}
  \caption{Conceptual block diagram of candidate device/integration options.}
  \label{fig:block}
\end{figure*}

\begin{figure*}[!t]
  \centering
  \tikzcol[0.95\textwidth]{\figpath/mindmap.tex}
  \caption{Post-CFET technology mind map.}
  \label{fig:mindmap}
\end{figure*}

\begin{figure*}[!t]
  \centering
  \tikzcol[0.95\textwidth]{\figpath/roadmap.tex}
  \caption{2030–2045 roadmap (materials, integration, applications, EDA).}
  \label{fig:roadmap}
\end{figure*}

% ---------- References ----------
\begin{thebibliography}{1}
\bibitem{irds2024} IRDS, \emph{International Roadmap for Devices and Systems}, 2024.
\bibitem{takagi2023} S. Takagi, et al., IEDM Tech Digest, 2023.
\bibitem{liu2022} Z. Liu, et al., \emph{Nature Electronics}, 2022.
\bibitem{ferr2020} A. Fert, et al., \emph{Rev. Mod. Phys.}, 2019.
\bibitem{wong2020} H. S. P. Wong, \emph{Nat. Rev. Mater.}, 2020.
\bibitem{batude2019} P. Batude, et al., IEDM, 2019.
\end{thebibliography}

% ---------- Biography ----------
\begin{IEEEbiography}{Shinichi Samizo}
received the M.S. degree in Electrical and Electronic Engineering from Shinshu University, Japan. 
He worked at Seiko Epson Corporation on semiconductor memory and mixed-signal device development, and contributed to inkjet MEMS actuators and PrecisionCore printhead technology. 
He is currently an independent semiconductor researcher focusing on process/device education, memory architecture, and AI system integration.  
Contact: \href{mailto:shin3t72@gmail.com}{shin3t72@gmail.com}, GitHub: \href{https://github.com/Samizo-AITL}{Samizo-AITL}.
\end{IEEEbiography}

\end{document}
