% --------------------------------------------------
% Post-CFET Device Architecture Paper (English Version, Revised)
% --------------------------------------------------
\documentclass[conference]{IEEEtran}
\IEEEoverridecommandlockouts

% ---------------- Packages (order tuned for IEEEtran + XeLaTeX) --------------
\usepackage{newtxtext,newtxmath}     % Times-like text/math
\usepackage{graphicx}
\usepackage{booktabs}
\usepackage{multirow}
\usepackage{standalone}              % \input tikz figures
\usepackage{tikz}
\usetikzlibrary{shapes,arrows.meta,positioning,fit,mindmap,calc}

% TikZ styles
\tikzset{
  blk/.style={rectangle,rounded corners,draw=black,fill=blue!10,
              minimum width=2.8cm,minimum height=0.8cm,align=center,font=\small},
  flowarrow/.style={-{Latex[length=3mm,width=2mm]},thick},
  mod/.style={rectangle,draw=black,fill=white,
              minimum width=2.4cm,minimum height=0.8cm,align=center,font=\small},
  bubble/.style={draw, rounded corners, fill=blue!7, align=left,
                 inner sep=2.5pt, font=\scriptsize},
  milestone/.style={circle, draw, fill=black, minimum size=2pt, inner sep=0pt},
  arrow/.style={-{Latex[length=3mm,width=2mm]}, thick}
}

% ---- siunitx (v3) ----
\usepackage{siunitx}
\sisetup{
  mode = match,
  propagate-math-font = true,
  per-mode = symbol,
  range-phrase = --,
  uncertainty-mode = separate,
  text-family-to-math = true,
  text-series-to-math = true
}
\DeclareSIUnit{\decade}{dec}

% 図を列幅にフィット
\newcommand{\tikzcol}[2][\columnwidth]{\resizebox{#1}{!}{\input{#2}}}

\usepackage{url}
\usepackage[hidelinks]{hyperref}
\graphicspath{{figures/}}

% ---------------- Title & Author --------------------------------------------
\title{Post-CFET Device Architectures: Materials, Integration, and Design Perspectives}

\author{
\IEEEauthorblockN{Shinichi Samizo}
\IEEEauthorblockA{Independent Semiconductor Researcher\\
Project Design Hub, Samizo-AITL\\
\textit{Email:} \href{mailto:shin3t72@gmail.com}{shin3t72@gmail.com}\quad
\textit{GitHub:} \href{https://github.com/Samizo-AITL}{Samizo-AITL}}
}

\begin{document}
\maketitle

% ---------------- Abstract ---------------------------------------------------
\begin{abstract}
CMOS scaling has evolved from Planar MOSFETs to FinFETs, Gate-All-Around (GAA) nanosheets, and Complementary FETs (CFETs). CFET improves electrostatic control and mitigates wiring bottlenecks, but silicon is approaching its material and thermal limits. This paper reviews \textbf{post-CFET device candidates} including \textit{two-dimensional (2D) material FETs, monolithic 3D integration, spintronics/quantum devices, and heterogeneous atomic-scale integration}. Their physical principles, fabrication challenges, experimental demonstrations, reliability concerns, application domains, and implications for design and education are compared.
\end{abstract}

% ---------------- 1. Introduction -------------------------------------------
\section{Introduction}
The semiconductor industry has advanced for more than five decades through device scaling and structural innovation.
The trajectory from Planar MOSFET $\rightarrow$ FinFET $\rightarrow$ GAA $\rightarrow$ CFET represents the pursuit of enhanced electrostatic control and integration efficiency.
However, mobility degradation, leakage, wiring delay, and thermal density have become limiting factors, demanding exploration of \textbf{post-CFET technologies}.

% ---------------- 2. Evolution Path -----------------------------------------
\section{Evolution from CMOS to Post-CFET}
Fig.~\ref{fig:evolution} summarizes the historical pathway of CMOS device evolution.

\begin{figure}[t]
  \centering
  \tikzcol{figures/evolution_tree.tex}
  \caption{Evolution tree: CMOS $\rightarrow$ CFET $\rightarrow$ post-CFET candidates.}
  \label{fig:evolution}
\end{figure}

\subsection{Planar MOSFET to FinFET}
Short-channel effects and leakage became critical below \SI{45}{\nano\meter}. Tri-gate FinFETs improved gate control and became the industry standard.

\subsection{FinFET to GAA}
Cell-height constraints required nanosheets fully surrounded by gates. GAA provides stronger electrostatics but increases process variability.

\subsection{GAA to CFET}
CFET vertically stacks nFET and pFET, improving cell density and reducing interconnect delay. Challenges include heat removal and process complexity.

\subsection{Beyond CFET}
Silicon material limits and wiring dominance demand novel materials, new integration schemes, and alternative state variables (spin, photon, bio).

% ---------------- 3. Candidates ---------------------------------------------
\section{Post-CFET Candidate Technologies}

\subsection{2D Material FETs}
\begin{itemize}
  \item \textbf{Demonstrations:} MoS$_2$ FET ($L_g=\SI{12}{\nano\meter}$, IEDM~2023) with Ion/Ioff=$10^7$, SS=\SI{65}{\milli\volt\per\decade}.
  \item \textbf{Challenges:} High contact resistance ($R_c \approx \SI{1}{\kilo\ohm\micro\meter}$), film non-uniformity (5--10\%), interface traps.
  \item \textbf{Applications:} Ultra-low-power IoT, flexible electronics, bio-sensing.
  \item \textbf{Design impact:} Immature SPICE models, limited statistical variability data.
\end{itemize}

\subsection{Monolithic 3D Integration (M3D)}
\begin{itemize}
  \item \textbf{Demonstrations:} SRAM stacking (IEDM~2019): delay $-30\%$, area $-40\%$. AI SoC (Nat.~Electronics~2022): energy efficiency $+1.7\times$.
  \item \textbf{Challenges:} Low-temperature processing ($<\SI{450}{\celsius}$), inter-layer $V_\mathrm{th}$ variation, thermal hotspots $>\SI{1}{\watt\per\square\milli\meter}$.
  \item \textbf{Applications:} AI accelerators, memory-centric computing.
  \item \textbf{Design impact:} Requires 3D P\&R EDA, thermal/mechanical co-simulation.
\end{itemize}

\subsection{Spintronics / Quantum Devices}
\begin{itemize}
  \item \textbf{Demonstrations:} STT-MRAM endurance $10^{12}$ cycles (IBM), SOT-MRAM write current $-40\%$, topological FET on/off $=10^3$ at room temperature.
  \item \textbf{Challenges:} CMOS compatibility; reduce write current from mA to \si{\micro\ampere}.
  \item \textbf{Applications:} Neuromorphic computing, radiation-hardened space systems, in-memory logic.
  \item \textbf{Design impact:} Logic–memory fusion beyond von Neumann architecture.
\end{itemize}

\subsection{Heterogeneous Atomic-Scale Integration}
\begin{itemize}
  \item \textbf{Demonstrations:} Si+MoS$_2$ photodetector, responsivity \SI{200}{\milli\ampere\per\watt} at \SI{1.55}{\micro\meter} (Nat.~Photonics~2020); CMOS+MEMS sensor chips.
  \item \textbf{Challenges:} Interface stability, lattice/thermal mismatch, yield.
  \item \textbf{Applications:} Optical interconnect, medical sensing, aerospace.
  \item \textbf{Design impact:} Cross-domain EDA required (electrical + optical + mechanical).
\end{itemize}

% ---------------- Block diagram (wide) --------------------------------------
\begin{figure*}[t]
  \centering
  \tikzcol[.95\textwidth]{figures/block_diagram.tex}
  \caption{Conceptual block diagrams of candidate device/integration options.}
\end{figure*}

% ---------------- Mindmap (wide) --------------------------------------------
\begin{figure*}[t]
  \centering
  \tikzcol[.95\textwidth]{figures/mindmap.tex}
  \caption{Post-CFET technology mind map.}
\end{figure*}

% ---------------- 4. Comparison Table ---------------------------------------
\section{Comparison Matrix}
Table~\ref{tab:matrix} compares the four candidate technologies.

\begin{table}[t]
\centering
\caption{Comparison of post-CFET candidate technologies}
\label{tab:matrix}
\begin{tabular}{@{}l l l l l c@{}}
\toprule
Tech. & Demonstrations & Rc/Thermal & Reliability & Applications & TRL \\
\midrule
2D-FET & Ion/Ioff $=10^7$, SS=\SI{65}{mV/dec} & \SI{1}{k\ohm\micro\meter} & Film variation & IoT/Flex/Sensor & 3--5 \\
M3D    & Delay $-30\%$, Area $-40\%$          & $<\SI{450}{\celsius}$    & $V_\mathrm{th}$ shift & AI/Memory       & 4--6 \\
Spin   & MRAM $10^{12}$ cycles                & RT stability             & Write stress  & Neuro/Space/IMC  & 3--5 \\
Hetero & Si+MoS$_2$ PD \SI{200}{mA/W}         & Interface limits         & Yield         & Optics/Medical    & 2--4 \\
\bottomrule
\end{tabular}
\end{table}

% ---------------- Roadmap (wide) --------------------------------------------
\begin{figure*}[t]
  \centering
  \tikzcol[.95\textwidth]{figures/roadmap.tex}
  \caption{2030--2045 roadmap (materials, integration, applications, EDA).}
\end{figure*}

% ---------------- 5. Design/Education --------------------------------------
\section{Design and Educational Perspectives}
Future EDA must integrate multi-physics: heat, stress, quantum, and cross-domain effects.
Educational curricula should include: (1) scaling history, (2) candidate technology reviews, (3) multi-physics simulations, (4) case studies, and (5) system-level design integration.

% ---------------- 6. Scenarios ----------------------------------------------
\section{Future Scenarios}
\begin{itemize}
  \item \textbf{2030s (early):} Lab-scale demonstrations of 2D+CFET and M3D+2D hybrids.
  \item \textbf{2030s (late):} Partial adoption in IoT/AI edge devices.
  \item \textbf{2040s:} Mainstream in HPC and aerospace; fusion of spintronics and M3D enabling non-volatile 3D logic–memory.
\end{itemize}

% ---------------- 7. Conclusion ---------------------------------------------
\section{Conclusion}
Post-CFET represents a paradigm shift from structural scaling to material innovation, integration schemes, alternative physical states, and heterogeneous fusion. It holds educational, design, and industrial significance, bridging the gap beyond the silicon era.

% ---------------- References -------------------------------------------------
\begin{thebibliography}{00}
\bibitem{irds2024} IRDS, \emph{International Roadmap for Devices and Systems}, 2024.
\bibitem{takagi2023} S.~Takagi \emph{et al.}, IEDM Tech Digest, 2023.
\bibitem{liu2022} Z.~Liu \emph{et al.}, \emph{Nature Electronics}, 2022.
\bibitem{fert2019} A.~Fert \emph{et al.}, \emph{Rev. Mod. Phys.}, 2019.
\bibitem{wong2020} H.-S.~P.~Wong, \emph{Nat. Rev. Mater.}, 2020.
\bibitem{batude2019} P.~Batude \emph{et al.}, IEDM, 2019.
\end{thebibliography}

% ---------------- Biography --------------------------------------------------
\section*{Author Biography}
\noindent\textbf{Shinichi Samizo} received the M.S. degree in Electrical and Electronic Engineering from Shinshu University, Japan. He worked at Seiko Epson Corporation on semiconductor memory and mixed-signal device development, and contributed to inkjet MEMS actuators and PrecisionCore printhead technology. He is currently an independent semiconductor researcher focusing on process/device education, memory architecture, and AI system integration.\\
\textbf{Contact:} \href{mailto:shin3t72@gmail.com}{shin3t72@gmail.com}, GitHub: \href{https://github.com/Samizo-AITL}{Samizo-AITL}.
\end{document}
