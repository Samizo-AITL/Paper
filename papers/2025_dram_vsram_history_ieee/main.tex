\documentclass[conference]{IEEEtran}
% ========= 日本語対応(LuaLaTeX推奨) =========
\usepackage{luatexja}
\usepackage{luatexja-fontspec}
% HaranoAjiフォント(TeX Live 標準)
\setmainjfont{HaranoAjiMincho}
\setsansjfont{HaranoAjiGothic}

% ========= 一般パッケージ =========
\usepackage{amsmath,amssymb}
\usepackage{graphicx}
\usepackage{booktabs}
\usepackage{url}
\usepackage[hidelinks]{hyperref}
\usepackage{cite}

% ========= タイトル =========
\title{DRAM技術導入とその戦略的位置づけ(1997--2001)\\
\large 酒田FabにおけるDRAM/PSRAMとロジック展開の連関}

% ========= 著者情報(ご指定のブロックをそのまま利用) =========
\author{%
  \IEEEauthorblockN{三溝 真一 (Shinichi Samizo)}%
  \IEEEauthorblockA{独立系半導体研究者(元セイコーエプソン)\\%
  Independent Semiconductor Researcher (ex-Seiko Epson)\\%
  Email: \href{mailto:shin3t72@gmail.com}{shin3t72@gmail.com}\\%
  GitHub: \url{https://github.com/Samizo-AITL}}%
}

\begin{document}
\maketitle

\begin{abstract}
本論文は,1997年から2001年にかけてセイコーエプソン酒田事業所が
三菱電機からの技術移管を通じて \mbox{0.5\,$\mu$m} $\rightarrow$ \mbox{0.35\,$\mu$m} $\rightarrow$ \mbox{0.25\,$\mu$m} のDRAMプロセスを短期間で習得し,
VSRAM(疑似SRAM)を経由して液晶ドライバー等のロジック/混載CMOSへ展開した
技術的・戦略的過程を,筆者の実体験に基づき整理する。
主要な不良モード(Pause/Disturb Refresh)の物理起源と対策,および
量産歩留まりの推移を示し,得られたプロセス知見が高耐圧混載CMOSに
どのように接続されたかを考察する。
\end{abstract}

\begin{IEEEkeywords}
DRAM, VSRAM/PSRAM, 0.25\,$\mu$m, リテンション, ディスターブ, 酒田Fab, 技術移管, 高耐圧混載CMOS
\end{IEEEkeywords}

\section{序論 / Introduction}
1997年,当時の半導体産業はWindows~95の普及や8インチラインの普及に伴い急伸した。
本研究は,酒田FabにおけるDRAM導入を「目的」ではなく「手段」として位置づけ,
先端プロセスの自前化からロジック展開へ接続した実践知を記録するものである。

\section{第1章:0.5\,\texorpdfstring{$\mu$m}{μm} と 0.35\,\texorpdfstring{$\mu$m}{μm} 世代の立ち上げ}
\subsection{0.5\,$\mu$m 16M DRAM}
熊本Fabで確立されたプロセスを導入し,初期から安定した量産歩留まりを確保した。

\subsection{0.35\,$\mu$m 64M DRAM:洗浄フロー差異と「鏡写し」}
初期試作30ロット超で形状不良が多発。徹底調査の結果,
洗浄フロー差異(硫酸過水の省略)が原因と判明し,
熊本プロセスの完全\emph{鏡写し}で解消した。

\section{第2章:0.25\,\texorpdfstring{$\mu$m}{μm} 世代64M DRAMの立ち上げ}
\subsection{SCF方式と初期歩留まり}
Short Cycle Feedback(SCF)で条件最適化を高速回転。
本番ロットで歩留まり$\sim$65\%を達成。

\subsection{保持時間モデルとリーク起源}
保持時間は,
\begin{equation}
\tau = \frac{C_{\mathrm{cell}} \cdot V_{\mathrm{cell}}}{I_{\mathrm{leak}}}
\end{equation}
で表され,\,$I_{\mathrm{leak}}$ の支配因子としてジャンクションリークを特定。
プラズマダメージ低減(低パワーアッシング,犠牲酸化)等で
80\%台後半まで改善。

\section{第3章:VSRAM(2001年)— Pause/Disturb対策と歩留まり改善}
低消費・高温(90\,$^\circ$C)要求によりPause/Disturbが顕在化。
HF洗浄回数最小化,バックバイアス強化($V_{bs}=-3$\,V),
ゲートCD中心値の厳密管理等で,30\%$\rightarrow$80\%台へ改善。

\section{第4章:0.18\,\texorpdfstring{$\mu$m}{μm} トレンチ系の評価と断念}
NANYA 0.18\,$\mu$m(東芝系)では90\,$^\circ$C保持で不十分。
モバイル用途への適用を断念し,液晶ドライバーの高耐圧混載CMOSへ
集中する戦略へ移行。

\section{結論}
DRAMは目的ではなく手段であり,獲得した先端プロセス知見は
液晶ドライバー等のコア事業での差別化に直結した。

\bibliographystyle{IEEEtran}
\bibliography{refs}

% ===== 著者略歴(IEEEtran conference向け:無番号セクション) =====
\section*{著者略歴}
\noindent\textbf{三溝 真一 (Shinichi Samizo)} は、信州大学大学院 工学系研究科 電気電子工学専攻にて修士号を取得した。
その後、セイコーエプソン株式会社に勤務し、半導体ロジック/メモリ/高耐圧インテグレーション、さらにインクジェット薄膜ピエゾアクチュエータおよび PrecisionCore プリントヘッドの製品化に従事した。
現在は独立系半導体研究者として、プロセス/デバイス教育、メモリアーキテクチャ、AIシステム統合などの研究に取り組んでいる。
連絡先: \href{mailto:shin3t72@gmail.com}{shin3t72@gmail.com}

\end{document}
