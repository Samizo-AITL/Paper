\documentclass[conference]{IEEEtran}

% ========= 日本語対応(LuaLaTeX推奨) =========
\usepackage{luatexja}
\usepackage{luatexja-fontspec}
\setmainjfont{HaranoAjiMincho}
\setsansjfont{HaranoAjiGothic}

% ========= 一般パッケージ =========
\usepackage{amsmath,amssymb}
\usepackage{graphicx}
\usepackage{booktabs}
\usepackage{array}
\usepackage{url}
\usepackage[hidelinks]{hyperref}
\usepackage{cite}

% ========= 図・グラフ =========
\usepackage{tikz}
\usetikzlibrary{arrows.meta,decorations.pathreplacing,calc,patterns}
\usepackage{pgfplots}
\pgfplotsset{compat=1.17}

% ========= タイトル =========
\title{DRAM技術導入とその戦略的位置づけ(1997--2001)\\
\large 酒田FabにおけるDRAM/PSRAMとロジック展開の関連}

\author{%
  \IEEEauthorblockN{三溝 真一 (Shinichi Samizo)}%
  \IEEEauthorblockA{独立系半導体研究者(元セイコーエプソン)\\%
  Independent Semiconductor Researcher (ex-Seiko Epson)\\%
  Email: \href{mailto:shin3t72@gmail.com}{shin3t72@gmail.com}\\%
  GitHub: \url{https://github.com/Samizo-AITL}}%
}

\begin{document}
\maketitle

\begin{abstract}
\textbf{(日本語)}\\
本論文は,1997年から2001年にかけてセイコーエプソン酒田事業所が
三菱電機からの技術移管を通じて \mbox{0.5\,$\mu$m} $\rightarrow$ \mbox{0.35\,$\mu$m} $\rightarrow$ \mbox{0.25\,$\mu$m} の
DRAMプロセスを短期間で習得し,得られた知見を
先端ロジックや高耐圧混載CMOSへ展開して液晶ドライバー製品化に結びつけた
技術的・戦略的過程を,筆者の実体験に基づき整理する。
主要な不良モード(Pause/Disturb Refresh)の物理起源と対策,
量産歩留まりの推移を示し,獲得知見がその後の事業へ
どのように接続されたかを考察する。\\[1ex]

\textbf{(English)}\\
This paper reviews 1997–2001, when Seiko Epson’s Sakata Fab
assimilated DRAM processes (0.5\,$\mu$m → 0.35\,$\mu$m → 0.25\,$\mu$m) transferred from Mitsubishi Electric.
The acquired know-how was extended beyond DRAM to advanced logic
and high-voltage mixed CMOS, leading to LCD driver products.
Key failure modes (Pause/Disturb Refresh), countermeasures, and yield evolution
are summarized based on the author’s on-site experience.
\end{abstract}

\begin{IEEEkeywords}
DRAM, VSRAM/PSRAM, 0.25\,$\mu$m process, retention failure, disturb failure, Sakata Fab, technology transfer, high-voltage mixed CMOS, LCD driver

\hspace{1em}(日本語)DRAM,VSRAM/PSRAM,0.25\,$\mu$mプロセス,リテンション不良,ディスターブ不良,酒田Fab,技術移管,高耐圧混載CMOS,液晶ドライバー
\end{IEEEkeywords}

% ========== 本文開始 ==========
\section{序論}
1997年,当時の半導体産業は \textit{Windows~95} の普及や
Intel~Pentium~II の登場を契機に拡大局面にあった。
製造面では8インチウェーハと \SI{0.35}{\micro\meter} 世代の量産移行が進み,
DRAMとロジックLSIで国際競争が激化していた。

セイコーエプソンは山形県酒田市の8インチFab(酒田事業所)で,
三菱電機からの技術移管を通じて
\SI{0.5}{\micro\meter} $\rightarrow$ \SI{0.35}{\micro\meter} $\rightarrow$ \SI{0.25}{\micro\meter}
の三世代DRAMプロセスを短期間に習得した。  
狙いはDRAM単独で競争優位を築くことではなく,
DRAMを\emph{手段}として最新プロセスの自前化を図り,
その成果を外販・内製の両面に横展開する点にあった。  

具体的には,
\begin{itemize}
  \item \textbf{外部向けファンドリビジネス}:Xilinx 等ファブレス顧客との協働による先端ロジック展開,
  \item \textbf{社内展開}:自社SoCおよびマイコン製品への応用,
  \item \textbf{高耐圧混載CMOS/液晶ドライバー}:ディスプレイ用デバイスでの競争優位確立
\end{itemize}
という三本柱への知見移転を目的としていた。

本研究の貢献は以下に要約される。
\begin{itemize}
  \item \textbf{因果の定式化}:
    保持時間 $\tau=C_{\mathrm{cell}}V_{\mathrm{cell}}/I_{\mathrm{leak}}$ に基づき,
    Pause/Disturb 不良をプラズマ誘起欠陥と接合リークに帰着。
  \item \textbf{工程対策の一般化}:
    O$_2$アッシング停止 $\rightarrow$ 硫酸ウェット剥離,HF最小化,
    $V_{bs}$強化,ゲートCD管理を核とした再現性の高い改善指針。
  \item \textbf{移管運用指針}:
    二次因子(洗浄順序・装置・ケミカル・搬送)まで\emph{鏡写し}を徹底する立上げ原則。
  \item \textbf{事業接続性}:
    DRAMで得た表面・接合・CD制御の知見を,
    ファンドリロジック,社内SoC/マイコン,および高耐圧混載CMOS/LCDドライバーへ移転。
\end{itemize}

戦略的流れは以下の通りである:
\begin{enumerate}
  \item DRAM技術導入
  \item Xilinx 等ファブレス顧客との協働による先端ロジック・ファンドリ展開
  \item 社内SoC/マイコン製品展開
  \item 高耐圧混載による液晶ドライバー市場シェア獲得
\end{enumerate}
