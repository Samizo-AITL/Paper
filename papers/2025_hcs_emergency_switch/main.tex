%====================================================
% HCS緊急切替対応(IEEE日本語論文・1カラム版)
%====================================================
\documentclass[journal,onecolumn]{IEEEtran}

% ===== Engine guard =====
\usepackage{iftex}
\ifXeTeX\else
  \errmessage{Compile with XeLaTeX (or LuaLaTeX).}
\fi

% ===== Fonts (XeLaTeX) =====
\usepackage{fontspec}
\usepackage{xeCJK}
\defaultfontfeatures{Ligatures=TeX}
\setmainfont{Latin Modern Roman}
\setsansfont{Latin Modern Sans}
\setmonofont{Latin Modern Mono}

% CI(Alpine)では Noto CJK を使用する
\IfFontExistsTF{Noto Serif CJK JP}{
  \setCJKmainfont{Noto Serif CJK JP}
}{
  \IfFontExistsTF{IPAexMincho}{
    \setCJKmainfont{IPAexMincho}
  }{
    \setCJKmainfont{Noto Sans CJK JP}
  }
}
\XeTeXlinebreaklocale "ja"
\XeTeXlinebreakskip=0pt plus 1pt

% ===== Math & Units =====
\usepackage{amsmath,amssymb}
\interdisplaylinepenalty=2500
\usepackage{siunitx}
\sisetup{detect-all, per-mode=symbol, group-separator={,}}

% ===== Graphics / Figures =====
\usepackage{graphicx}
\usepackage{tikz}
\usetikzlibrary{arrows.meta,positioning,shapes,fit}
\usepackage{pgfplots}
\pgfplotsset{compat=1.18}
\graphicspath{{./figures/}}

% ===== Tables / Lists =====
\usepackage{booktabs,tabularx,multirow}
\usepackage{enumitem}
\setlist{nosep}

% ===== Links (最後に) =====
\PassOptionsToPackage{hyphens}{url}
\usepackage{xurl}
\usepackage[hidelinks]{hyperref}

% ===== Layout tuning for 1-column =====
\usepackage[a4paper,margin=20mm]{geometry}
\renewcommand{\arraystretch}{1.15}
\setlength{\parskip}{3pt}
\setlength{\parindent}{1em}
\linespread{1.05}

% ---------- Title ----------
\title{\Large\bfseries インクジェットヘッドにおけるHCS緊急切替対応の多拠点4M統合管理\\
— COF内HCSチップ導入に伴う工程連動計画と在庫最適化の実践 —}

% ---------- Author ----------
\author{%
  三溝 真一 (Shinichi Samizo)\\
  独立系半導体研究者(元セイコーエプソン株式会社)\\
  Independent Semiconductor Researcher (ex-Seiko Epson Corporation)\\
  Email:~\href{mailto:shin3t72@gmail.com}{shin3t72@gmail.com},\;
  GitHub:~\url{https://github.com/Samizo-AITL}
}

\begin{document}
\maketitle

\begin{abstract}
本論文では、インクジェットヘッドにおけるHead Control System(HCS)の緊急切替対応について報告する。
COF工程にHCSチップを追加し、検査工程およびプリンタ機体ファームウェア(FW)との認証照合を導入した。
対象はCOF工程(東北)、ヘッド検査(東北・秋田・深圳)、機体検査(広丘・インドネシア)である。
4M(Man, Machine, Material, Method)変更を日割りマスタスケジュールで全拠点同期し、依存関係と新旧在庫の収束を統合管理した。
その結果、ライン停止ゼロ、旧品在庫の極小化、認証不整合ゼロを達成し、目的外使用防止と量産品質の両立を実証した。
\end{abstract}

%====================================================
\section{はじめに}
2010年代中頃、中国エリアにおいてインクジェットプリントヘッドの目的外使用が顕在化した。
従来のHCSおよびHCS2はFPGAを用いた解析・中継で短期間に回避される脆弱性があり、COF上に認証ロジックとキーコードを内包するHCSチップ実装が決定された。
本稿では、停止ゼロ・在庫損失最小での多拠点切替を報告する。

%====================================================
\section{技術課題と対応方針}
\subsection{技術課題}
\begin{itemize}
  \item COF上HCSチップと機体FWのキー整合性
  \item 検査装置FWの改修と横断版管理
  \item 新旧在庫ゼロ化とトレーサビリティ維持
\end{itemize}

\subsection{対応方針}
\begin{enumerate}
  \item COF→ヘッド→機体の工程連携でキー照合を標準化
  \item 全拠点共通マスタスケジュールによる4M統合管理
  \item 在庫推移の可視化と切替点固定化
\end{enumerate}

%====================================================
\section{切替体制と対象拠点}
\begin{table}[h]
\caption{対象工程と主タスク}
\label{tab:sites}
\centering
\begin{tabular}{@{}llp{6.2cm}@{}}
\toprule
工程 & 拠点 & 主タスク \\
\midrule
COF工程 & 東北 & HCSチップ実装、キー発番・読取 \\
ヘッド検査 & 東北・秋田・深圳 & FW更新、動作確認 \\
機体検査 & 広丘・インドネシア & FW更新、認証照合 \\
\bottomrule
\end{tabular}
\end{table}

%====================================================
\section{4M変更の概要}
\begin{table}[h]
\caption{4M変更要点}
\label{tab:4m}
\centering
\begin{tabular}{@{}lp{7cm}@{}}
\toprule
M要素 & 変更内容 \\
\midrule
Man & 教育・資格更新(キー読取・異常分岐) \\
Machine & 検査装置FW改修・読取器追加 \\
Material & 新COF採用・旧品終息・切替ロット確定 \\
Method & 日割り計画・在庫ゼロ化・版管理標準化 \\
\bottomrule
\end{tabular}
\end{table}

%====================================================
\section{工程連携と在庫制御}
\begin{figure}[h]
\centering
\begin{tikzpicture}[
  node distance=6mm and 6mm,
  proc/.style={rectangle,rounded corners,draw,align=center,inner sep=2pt},
  arrow/.style={-Latex,thick}
]
\node[proc] (cof) {COF工程\\(書込/読取)};
\node[proc,right=of cof] (head) {ヘッド検査\\(FW更新)};
\node[proc,right=of head] (body) {機体検査\\(認証照合)};
\node[proc,below=of head] (inv) {在庫管理\\(切替点固定)};
\draw[arrow](cof)--(head);
\draw[arrow](head)--(body);
\draw[arrow](cof)|-(inv);
\draw[arrow](head)|-(inv);
\draw[arrow](inv)-|(body);
\end{tikzpicture}
\caption{工程連携と在庫制御構造}
\end{figure}

\begin{figure}[h]
\centering
\begin{tikzpicture}
\begin{axis}[
  width=0.9\linewidth, height=4.5cm,
  xlabel={日数}, ylabel={在庫比率},
  ymin=0, ymax=1.05, xmin=0, xmax=30, grid=both
]
\addplot+[mark=none, thick] coordinates{(0,1.0)(5,0.8)(10,0.55)(15,0.3)(20,0.1)(25,0.02)(30,0)};
\addlegendentry{旧COF在庫}
\addplot+[mark=none, dashed] coordinates{(0,0)(5,0.15)(10,0.4)(15,0.7)(20,0.9)(25,0.98)(30,1.0)};
\addlegendentry{新COF適用率}
\end{axis}
\end{tikzpicture}
\caption{在庫吸収シミュレーション(概念)}
\end{figure}

%====================================================
\section{結果と考察}
\begin{itemize}
  \item ライン停止:0日(計画停止除く)
  \item 認証不整合:0件(全拠点)
  \item 旧COF在庫:実質ゼロで切替完了
\end{itemize}
成功要因は、(1)全体マスタ共有、(2)依存関係明確化、(3)在庫ゼロ化設計、(4)FW版統制である。

%====================================================
\section{結論}
COF内HCSチップ導入に伴う多拠点4M変更を、工程つなぎと在庫制御を核に統合管理した。
停止ゼロ・在庫損失最小・認証不整合ゼロを達成し、目的外使用防止と量産品質の両立を実証した。

%====================================================
\section*{参考文献}
\begin{thebibliography}{99}
\bibitem{EpsonReport2017} (社内資料)HCS3導入技術報告書, 2017.
\bibitem{IJHeadSecurity2018} S.~Yamamoto \emph{et al.}, ``Authentication Techniques for Piezo Inkjet Heads,'' \emph{Precision Engineering Journal}, vol.~34, no.~2, pp.~101--108, 2018.
\bibitem{COFAuthIEEE2019} J.~Tanaka \emph{et al.}, ``COF-Embedded Authentication for Inkjet Printhead Security,'' \emph{IEEE Trans. on Components, Packaging and Manufacturing Technology}, vol.~9, no.~12, pp.~2345--2353, 2019.
\bibitem{SCMChangeMgmt2020} H.~Kobayashi and T.~Mori, ``Global Change Management across Multi-Site Manufacturing,'' \emph{J. of Manufacturing Systems}, vol.~57, pp.~312--321, 2020.
\bibitem{FWVersionCtrl2016} A.~Suzuki, ``Firmware Version Control for Mass Production,'' \emph{IEICE Trans. Fundamentals}, vol.~E99-A, no.~12, pp.~2190--2197, 2016.
\end{thebibliography}

%====================================================
\section*{著者略歴}
\noindent\textbf{三溝 真一(Shinichi Samizo)}:信州大学大学院 工学系研究科 電気電子工学専攻修士課程修了。
セイコーエプソン株式会社にて半導体ロジック/メモリ/高耐圧デバイス統合およびインクジェット薄膜ピエゾアクチュエータ開発に従事。
現在は独立系半導体研究者として、プロセス技術、デバイス教育、システム統合研究に従事。
連絡先:\href{mailto:shin3t72@gmail.com}{shin3t72@gmail.com}。
\end{document}
