%====================================================
% HCS緊急切替対応(IEEE日本語論文・単一ファイル版)
%  - 画像なしでコンパイル可能(TikZで図生成)
%  - 参考文献は本文末に同梱(thebibliography)
%  - IEEEtran + XeLaTeX(IPAexMincho想定)
%====================================================
\documentclass[journal,twocolumn]{IEEEtran}

% ===== Engine guard =====
\usepackage{iftex}
\ifXeTeX\else
  \errmessage{Compile with XeLaTeX (or LuaLaTeX).}
\fi

% ===== Fonts (XeLaTeX) =====
\usepackage{fontspec}
\usepackage{xeCJK}
\defaultfontfeatures{Ligatures=TeX}

% --- Latin fonts (safe defaults) ---
\setmainfont{Latin Modern Roman}
\setsansfont{Latin Modern Sans}
\setmonofont{Latin Modern Mono}

% --- CJK font (prefer Noto on CI/Alpine; fallback to IPAex/IPA) ---
\IfFontExistsTF{Noto Serif CJK JP}{
  \setCJKmainfont{Noto Serif CJK JP}
}{
  \IfFontExistsTF{IPAexMincho}{
    \setCJKmainfont{IPAexMincho}
  }{
    \IfFontExistsTF{IPAMincho}{
      \setCJKmainfont{IPAMincho}
    }{
      \setCJKmainfont{Noto Sans CJK JP} % 最終フォールバック
    }
  }
}

% 日本語行分割
\XeTeXlinebreaklocale "ja"
\XeTeXlinebreakskip=0pt plus 1pt

% ===== Math & Units =====
\usepackage{amsmath,amssymb}
\interdisplaylinepenalty=2500
\usepackage{siunitx}
\sisetup{
  detect-all,
  per-mode = symbol,
  group-separator = {,}
}

% ===== Graphics / Figures =====
\usepackage{graphicx}
\graphicspath{{./figures/}}
\usepackage{tikz}
\usetikzlibrary{arrows.meta,positioning,shapes,fit}
\usepackage{pgfplots}
\pgfplotsset{compat=1.18}

% ===== Tables / Lists =====
\usepackage{booktabs,tabularx,multirow}
\usepackage{enumitem}
\setlist{nosep}

% ===== Links (最後に) =====
\PassOptionsToPackage{hyphens}{url} % 長URLの改行
\usepackage{xurl}                   % urlより改行に強い
\usepackage[hidelinks]{hyperref}

% ---------- Title ----------
\title{インクジェットヘッドにおけるHCS緊急切替対応の多拠点4M統合管理%
\\— COF内HCSチップ導入に伴う工程連動計画と在庫最適化の実践 —}

% ---------- Author ----------
\author{%
  \IEEEauthorblockN{三溝 真一 (Shinichi Samizo)}%
  \IEEEauthorblockA{%
    独立系半導体研究者(元セイコーエプソン株式会社)\\%
    Independent Semiconductor Researcher (ex-Seiko Epson Corporation)\\[4pt]%
    Email:~\href{mailto:shin3t72@gmail.com}{shin3t72@gmail.com}\quad
    GitHub:~\url{https://github.com/Samizo-AITL}%
  }%
}

\begin{document}
\maketitle

% ---------- Abstract ----------
\begin{abstract}
本論文では、インクジェットヘッドにおけるHead Control System(HCS)の緊急切替対応について報告する。
COF工程にHCSチップを追加し、検査工程におけるキーコード読み出しおよびプリンタ機体ファームウェア(FW)との認証照合を導入した。
対象拠点はCOF工程の東北エプソン、ヘッド検査工程の東北エプソン・秋田エプソン・深圳、機体検査工程の広丘・インドネシアである。
本プロジェクトでは、4M(Man, Machine, Material, Method)変更を日割りのマスタスケジュールで全体同期し、工程間の依存関係と新旧在庫の収束を統合管理した。
その結果、量産ラインの停止ゼロ、旧品在庫の極小化、認証不整合ゼロを達成し、目的外使用防止と量産品質の両立を実証した。
\end{abstract}

\begin{IEEEkeywords}
インクジェット, プリントヘッド, HCS, COF, 認証, 4M変更, 多拠点生産管理
\end{IEEEkeywords}

%====================================================
\section{はじめに}
2010年代中頃、中国エリアにおいてインクジェットプリントヘッドの目的外使用(抜き取りヘッドの改造機搭載)が顕在化した。
従来のHCS(印刷幅制限)およびHCS2(\SI{16}{bit}キー照合)は、FPGAを用いた解析・中継により短期間で回避される脆弱性が指摘された。
これを受け、COF(Chip on Film)上に認証ロジックとキーコードを内包するHCSチップを実装し、検査工程と機体FWの二段認証を導入する緊急切替が決定された。
本稿は、当該切替を\emph{停止ゼロ・在庫損失最小}で実現した多拠点4M統合管理の手法と結果を報告する。

%====================================================
\section{技術課題と対応方針}
\subsection{技術課題}
\begin{itemize}
  \item COF上HCSチップと機体FW間のキーコード整合性の確保
  \item 検査装置FW・読み出し機能の同時改修と拠点横断のバージョン統制
  \item 新旧在庫のゼロ化(切替点設計)とトレーサビリティの維持
\end{itemize}

\subsection{対応方針}
\begin{enumerate}
  \item COF工程でキーコードを\emph{発番・書込み・読み出し確認}し、ヘッド検査工程で\emph{動作確認}、機体工程で\emph{FW認証照合}を行うシーケンスを標準化。
  \item 4M(人・機械・材料・方法)の変更を\emph{日割りマスタスケジュール}で全拠点共有し、依存関係を矩陣化。
  \item 在庫推移の可視化と切替点(旧部材投入停止点/新部材投入開始点)の固定化。
\end{enumerate}

%====================================================
\section{切替体制と対象拠点}
\subsection{対象工程・拠点}
\begin{table}[t]
\caption{対象工程と拠点・主タスク}
\label{tab:sitemap}
\centering
\begin{tabular}{@{}llp{4.5cm}@{}}
\toprule
工程 & 拠点 & 主タスク \\
\midrule
COF工程 & 東北エプソン & HCSチップ実装、キーコード発番・読取 \\
ヘッド検査 & 東北・秋田・深圳 & 検査装置FW更新、動作確認 \\
機体検査 & 広丘・インドネシア & 機体FW更新、ヘッド認証照合 \\
\bottomrule
\end{tabular}
\end{table}

\subsection{情報共有}
週次レビューと日次進捗を統合した\emph{日割りマスタ}を全現法で共有し、改版は単一リポジトリで管理した。

%====================================================
\section{4M変更内容と管理手法}
\subsection*{表\ref{tab:4m}に4M変更の要点を示す。}
\begin{table}[t]
\caption{4M変更の全体像}
\label{tab:4m}
\centering
\begin{tabular}{@{}lp{6.2cm}@{}}
\toprule
M要素 & 変更内容 \\
\midrule
Man(人) & 検査員教育(キー読取手順/異常時分岐)、現法横断の作業資格 \\
Machine(設備) & 検査装置FW改修、読取器追加、機体書込治具更新 \\
Material(材料) & 新COF(HCS内蔵)採用、旧部材終息、切替ロット確定 \\
Method(方法) & 工程つなぎ標準化、日割り計画、在庫ゼロ化ルール、版管理 \\
\bottomrule
\end{tabular}
\end{table}

%====================================================
\section{日程計画と工程連携}
\subsection{工程間依存関係の可視化(TikZ図)}
\begin{figure}[t]
\centering
\begin{tikzpicture}[
  node distance=8mm and 7mm,
  proc/.style={rectangle,rounded corners,draw,align=center,inner sep=2pt},
  arrow/.style={-Latex,thick}
]
\node[proc] (cof) {COF工程\\{\small(HCS書込/読取)}\\{\small 東北}};
\node[proc, right=of cof] (head) {ヘッド検査\\{\small FW更新/動作確認}\\{\small 東北/秋田/深圳}};
\node[proc, right=of head] (body) {機体検査\\{\small FW更新/認証照合}\\{\small 広丘/インドネシア}};
\node[proc, below=of head] (inv) {在庫管理\\{\small 切替点固定/吸収計画}};

\draw[arrow] (cof) -- (head);
\draw[arrow] (head) -- (body);
\draw[arrow] (cof) |- (inv);
\draw[arrow] (head) |- (inv);
\draw[arrow] (inv) -| (body);
\end{tikzpicture}
\caption{工程つなぎと在庫制御の関係}
\label{fig:flow}
\end{figure}

\subsection{在庫収束シミュレーション(概念プロット)}
\begin{figure}[t]
\centering
\begin{tikzpicture}
\begin{axis}[
  width=\linewidth,
  height=5cm,
  xlabel={日数},
  ylabel={在庫数量(相対値)},
  ymin=0, ymax=1.05,
  xmin=0, xmax=30,
  grid=both
]
\addplot+[mark=none] coordinates {
 (0,1.0) (5,0.85) (10,0.6) (15,0.35) (20,0.15) (25,0.05) (30,0.0)
};
\addlegendentry{旧COF在庫}
\addplot+[mark=none] coordinates {
 (0,0.0) (5,0.1) (10,0.35) (15,0.6) (20,0.8) (25,0.95) (30,1.0)
};
\addlegendentry{新COF適用率}
\end{axis}
\end{tikzpicture}
\caption{旧在庫の吸収と新規適用率の推移(例)}
\label{fig:inventory}
\end{figure}

%====================================================
\section{結果と評価}
\subsection{主要KPI}
\begin{itemize}
  \item ライン停止:0日(切替期間中の計画停止を除く)
  \item 認証不整合:0件(全拠点)
  \item 旧COF在庫:切替完了時に実質ゼロ(端数は工程内消化)
\end{itemize}

\subsection{効果}
目的外使用防止の実効性が向上し、ヘッドと機体の\emph{ペアリング保証}が確立した。
在庫・工程・認証データが一貫したことで、出荷後トレーサビリティも強化された。

%====================================================
\section{考察}
成功要因は、(1)日割りマスタの全体共有、(2)依存関係の明確化、(3)新旧在庫ゼロ化設計、(4)拠点横断のFW版統制にある。
一方で、現法間の時差・教育負荷・文書改版の即時反映など、運用上の注意点が残る。

%====================================================
\section{結論}
COF内HCSチップ導入に伴う多拠点4M変更を、工程つなぎと在庫制御を核に統合管理した。
その結果、停止ゼロ・在庫損失最小・認証不整合ゼロを達成し、目的外使用防止と量産品質の両立を実証した。
本手法は、他製品の緊急4M変更にも適用可能である。

%====================================================
\section*{謝辞}
本対応にご協力いただいた東北エプソン、秋田エプソン、広丘事業所、深圳、インドネシアの各現法および関係部門に謝意を表する。

%====================================================
\section*{参考文献}
% 単一ファイル運用のため thebibliography を採用
\begin{thebibliography}{99}
\bibitem{EpsonReport2017}
(社内資料)HCS3導入技術報告書, 2017.

\bibitem{IJHeadSecurity2018}
S.~Yamamoto, K.~Ito, and M.~Sato, ``Authentication Techniques for Piezo Inkjet Heads,'' \emph{Precision Engineering Journal}, vol.~34, no.~2, pp.~101--108, 2018.

\bibitem{COFAuthIEEE2019}
J.~Tanaka \emph{et al.}, ``COF-Embedded Authentication for Inkjet Printhead Security,'' \emph{IEEE Trans. on Components, Packaging and Manufacturing Technology}, vol.~9, no.~12, pp.~2345--2353, 2019.

\bibitem{SCMChangeMgmt2020}
H.~Kobayashi and T.~Mori, ``Global Change Management across Multi-Site Manufacturing,'' \emph{J. of Manufacturing Systems}, vol.~57, pp.~312--321, 2020.

\bibitem{FWVersionCtrl2016}
A.~Suzuki, ``Firmware Version Control for Mass Production,'' \emph{IEICE Trans. Fundamentals}, vol.~E99-A, no.~12, pp.~2190--2197, 2016.
\end{thebibliography}

%====================================================
\section*{著者略歴}
\noindent\textbf{三溝 真一(Shinichi Samizo)}:信州大学大学院 工学系研究科 電気電子工学専攻にて修士号を取得。
セイコーエプソン株式会社にて、半導体ロジック/メモリ/高耐圧インテグレーション、インクジェット薄膜ピエゾアクチュエータおよび
PrecisionCoreプリントヘッドの製品化に従事。現在は独立系半導体研究者として、プロセス/デバイス教育、メモリアーキテクチャ、システム統合に取り組む。
連絡先:\href{mailto:shin3t72@gmail.com}{shin3t72@gmail.com}。
%====================================================

\end{document}
