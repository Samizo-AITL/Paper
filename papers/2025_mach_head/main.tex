\documentclass[conference]{IEEEtran}
\usepackage{luatexja}
\usepackage{graphicx}
\usepackage{amsmath}
\usepackage{url}
\usepackage{hyperref}

\begin{document}

\title{Sn-Biはんだメッキ断絶に伴う\\Mach世代インクジェットヘッドの接合方式移行事例}

\author{%
  \IEEEauthorblockN{三溝 真一 (Shinichi Samizo)}%
  \IEEEauthorblockA{独立系半導体研究者(元セイコーエプソン)\\%
  Independent Semiconductor Researcher (ex-Seiko Epson)\\%
  Email: \href{mailto:shin3t72@gmail.com}{shin3t72@gmail.com}\\%
  GitHub: \url{https://github.com/Samizo-AITL}}%
}

\maketitle

\begin{abstract}
本稿では、Sn-Biはんだメッキ工場の閉鎖に伴い、Mach世代インクジェットヘッドの接合方式をSn-BiリフローからACFおよびはんだ接合へ移行した事例について報告する。PrecisionCore登場直後の時期にあたり、実質エプソン全機種に関わる大規模な4M変更となった。本事例は、外的制約下において「不完全でも成立する解決策」を選択し、事業継続性を確保した実装技術の一例である。
\end{abstract}

\section{背景}
Mach世代インクジェットヘッドは、d31モードバルク積層PZTアクチュエータを採用していた。当初はCOF配線との接合にSn-Biはんだメッキによるリフロー方式を用い量産化していた。しかし、Sn-Biメッキ工場の閉鎖・ディスコン確定により、この方式での継続生産は不可能となった。当時はPrecisionCoreが市場投入された直後であったが、量産主力は依然としてMach世代であり、実質エプソン全機種がMachに依存していたため、影響は全社的であった。

\section{接合方式移行}
Sn-Bi断絶を受け、COF端子構造を見直し、出力端子(PZT側)と入力端子(基板側)で異なる接合方式を導入した。

\subsection{出力端子(COF→PZT)}
接合方式はACF (Anisotropic Conductive Film) 接合である。採用理由は微細ピッチ適合性とPZT電極との親和性であり、構造はCOF Sn端子–ACF導電粒子–PZT電極からなる。

\subsection{入力端子(COF→基板)}
接合方式ははんだ接合である。採用理由は電気的・機械的強度の確保、従来基板側プロセスの延長であり、構造はCOF Sn端子–基板側はんだ–基板パッドからなる。

この二重構造により、Sn-Bi断絶後も量産可能な接続体系を確立した。

\section{切り替えの背景と状況}
コンシューマ(水系インク)用ヘッドは先行してACF化済みであり、量産に支障はなかった。一方、溶剤インク対応ヘッドはACF膨潤リスクが懸念され、切り替えが遅れていた。そこにSn-Biディスコンが重なり、残存在庫も限られる中でショーテージ目前、待ったなしの危機となった。このため、溶剤インクヘッドについてもリスクを抱えたまま強制的にACFへ切り替える突貫プロジェクトが立ち上がり、調達部も全面参画し、材料確保・工程切替・量産立ち上げを同時並行で推進した。

\section{製造管理の難易度}
対象は単一機種ではなく複数の溶剤インクヘッド機種であった。さらに、COFはエプソン製・シャープ製・東芝製と複数社から供給されていた。接合工程もCOFとPZT接合、COFと基板接合の二系統が存在した。加えて、製造拠点は東北エプソン・秋田エプソンと分かれていた。したがって、複数機種×複数サプライヤ×複数拠点に跨る条件調整と切替統制が必要であり、製造コントロールの難易度は極めて高かった。

\section{評価試験(他部門実施)}
ACF接合の溶剤膨潤リスクについては、他部門が恒温槽+インク同梱タッパー試験を実施した。筆者の役割はACF接合ヘッドのサンプル提供に限定された。詳細条件は不明だが、以下の結果が報告された。
\begin{itemize}
 \item 蓋あり(蒸気滞留環境): 膨潤により接合不良、ヘッド不動作
 \item 蓋なし(蒸気逃げあり): 正常動作を維持
\end{itemize}

\section{顧客対応(他部門実施)}
代表例としてミマキ(Mimaki)など産業用途顧客が存在した。顧客との交渉・合意形成は他部門が担当した。方針は以下の通りである。
\begin{itemize}
 \item Sn-Bi再利用は不可能 → ACF採用
 \item 溶剤用途での不具合リスクは事前説明
 \item 不具合発生時はヘッド交換で対応
\end{itemize}
この条件で顧客了承を得て、事業継続を確保できた。

\section{結論}
本件はPrecisionCore登場直後に行われたが、当時の量産主力はMach世代であり、実質的にエプソン全機種に関わる大規模な4M変更(Man, Machine, Material, Method)であった。Sn-Bi断絶により接合方式は実質的にACF一択となった。溶剤用途では膨潤リスクを抱えるが、「不完全でも使える方法を確立する」ことを優先した。その結果、コンシューマ用途では問題なく、産業用途ではヘッド交換を条件に供給を継続することができ、エプソン全機種の事業継続性を確保した。

\section*{著者略歴}
\textbf{三溝 真一}(Shinichi Samizo)は、信州大学大学院 工学系研究科 電気電子工学専攻にて修士号を取得した。  
その後、セイコーエプソン株式会社に勤務し、半導体ロジック/メモリ/高耐圧インテグレーション、さらにインクジェット薄膜ピエゾアクチュエータおよびPrecisionCoreプリントヘッドの製品化に従事した。  
現在は独立系半導体研究者として、プロセス/デバイス教育、メモリアーキテクチャ、AIシステム統合などの研究に取り組んでいる。  
連絡先: \href{mailto:shin3t72@gmail.com}{shin3t72@gmail.com}.

\end{document}
