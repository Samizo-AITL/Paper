\documentclass[conference]{IEEEtran}

% ==== 推奨パッケージ構成(LuaLaTeX + IEEEtran + TikZ 安定運用)====

% --- 日本語/基本環境 ---
\usepackage{luatexja}
\usepackage{graphicx}
\usepackage{amsmath}
\usepackage{array}
\usepackage{booktabs}
\usepackage{tabularx}

% --- 色とTikZ(塗り分け・構造図に必須)---
\usepackage{xcolor}
\usepackage{tikz}
\usetikzlibrary{arrows.meta,positioning,shapes,fit}

% --- フォント設定 ---
\usepackage{luatexja-fontspec}
\setmainfont{TeX Gyre Termes}      % IEEEtran向けTimes系
\setsansfont{TeX Gyre Heros}
\setmonofont{TeX Gyre Cursor}
\setmainjfont{IPAexMincho}
\setsansjfont{IPAexGothic}
\ltjsetparameter{jacharrange={-2,-3}, autoxspacing=true, autospacing=true}

% --- 引用・単位・リンク ---
\usepackage{cite}
\usepackage{siunitx}
\sisetup{detect-all}
\usepackage{url}
\usepackage{hyperref}
\def\UrlBreaks{\do\/\do-\do\_}
\hypersetup{hidelinks}

% --- 表まわり設定 ---
\newcolumntype{Y}{>{\raggedright\arraybackslash}X}
\renewcommand\tabularxcolumn[1]{m{#1}}

% --- 段組バランス ---
\usepackage{balance}
% \balance  % 結論直前で使用

%========================================================
\begin{document}

\title{Sn-Biはんだメッキ断絶に伴う\\Mach世代インクジェットヘッドの接合方式移行事例}

\author{%
  \IEEEauthorblockN{三溝 真一 (Shinichi Samizo)}%
  \IEEEauthorblockA{独立系半導体研究者(元セイコーエプソン)\\%
  Independent Semiconductor Researcher (ex-Seiko Epson)\\%
  Email: \href{mailto:shin3t72@gmail.com}{shin3t72@gmail.com}\\%
  GitHub: \url{https://github.com/Samizo-AITL}}%
}

\maketitle

\begin{abstract}
本稿では、Sn-Biはんだメッキの製造停止により、Mach世代インクジェットヘッドの接合方式をSn-BiリフローからACFおよびはんだ接合へ移行した事例を報告する。PrecisionCore登場直後の時期にあたり、実質的にエプソン全機種に関わる大規模な4M変更となった。本事例は、外的制約下において「完全解ではなくとも成立する解」を選択し、事業継続性を確保した実装技術の一例である。
\end{abstract}

\begin{IEEEkeywords}
インクジェットヘッド, Sn-Biはんだ, ACF接合, 実装技術, 4M変更
\end{IEEEkeywords}

\vspace{1em}
\noindent
\textbf{Abstract (English):}  
This paper reports a case study of the bonding method transition in Mach generation inkjet heads, from Sn-Bi solder reflow to ACF and solder bonding, following the discontinuation of the Sn-Bi plating process.  
The transition occurred shortly after the introduction of PrecisionCore, when Mach heads still supported most Epson printer models.  
It represented a large-scale 4M (Man, Machine, Material, Method) change across the product line, demonstrating that, under external constraints, an “incomplete but workable solution” was chosen to ensure business continuity.

%========================================================
\section{背景}

Mach世代インクジェットヘッドは、エプソンが2000年代後半から2010年代にかけて量産展開した主力プリントヘッドであり、d$_{31}$モードで動作するバルク積層PZT素子列を採用していた。  
この方式は比較的高い吐出圧力と安定したインク吐出を実現できるため、コンシューマ用途から産業用途まで幅広く適用されていた。  

当初、COF(Chip on Film)配線とPZT素子列との接合には、Sn–Biはんだメッキを用いたリフロー方式が採用されていた。  
Sn–Bi系はんだは低融点(約\SI{139}{\degreeCelsius})であり、PZTやCOFへの熱ダメージを抑制できること、また量産工程として安定した実績があることから選択された。  

しかし、外部要因としてSn–Biメッキ工場の閉鎖が確定し、供給ディスコン(discontinuation)が避けられない状況となった。  
Sn–BiメッキはCOF端子形成の根幹であり、代替材料が存在しなかったため、この時点で従来の量産方式は継続不可能となった。  

さらに、この時期はPrecisionCoreが市場投入された直後であったが、量産規模はまだ限定的であり、事業の主力は依然としてMach世代ヘッドであった。  
特にコンシューマプリンタから産業用溶剤インク機まで、ほぼすべてのエプソンプリンタ製品群がMachヘッドに依存していたため、この接合方式問題は単一機種に留まらず、\textbf{実質的にエプソン全機種に影響する全社的課題}となった。  

結果として、Sn–Bi断絶への対応は「技術課題」であると同時に、「事業継続リスク」そのものであり、短期間での大規模な4M変更(Man, Machine, Material, Method)を余儀なくされる状況に至った。

%========================================================
\section{Machヘッド構造概要}

Mach世代インクジェットヘッドは、エプソンが2000年代後半から2010年代にかけて量産展開した代表的なピエゾ駆動ヘッドであり、d$_{31}$モードで動作する積層型PZT素子を多数並列に配置した構造を有する。  
各PZT素子は水平方向に変位し、その変位を下方のキャビティ膜へ伝達することでインクを加圧する。  
Machの特徴は、外部駆動ICによって電圧印加を受ける\textbf{受動的変換素子群}として設計されている点にある。  
すなわち、制御回路やセンサを内包する能動デバイスではなく、外部信号に応答する単純で堅牢な機能素子列として構成されている。

%--------------------------------------------------------
\subsection{PZT素子および内部構造}
PZT素子は、COM電極層とGND電極層が交互に積層されたモノリシック構造を持つ。  
電界印加によりd$_{31}$モードで水平方向に歪み、その変位がキャビティ膜を駆動する。  
両端の側面にはGND電極が共通化され、各COM電極は独立して駆動される。  
素子列は櫛歯状に加工され、\SI{180}{dpi}(ピッチ約\SI{0.141}{mm})程度で配列される。  
電気的には「GND共通・COM独立」のループを多数並列化した構成となる。

\begin{figure}[t]
\centering
\resizebox{\columnwidth}{!}{
\begin{tikzpicture}[x=1mm,y=1mm,font=\scriptsize,>=Latex]
\fill[gray!08] (0,2) rectangle (36,40);
\node[anchor=west] at (38,22){PZT素子(櫛歯列の1ユニット模式)};
\foreach \y in {4,7,10,13,16,19,22,25,28,31,34,37}{
  \draw[gray!60] (2,\y) -- (34,\y);
}
\fill[black!40] (0,14) rectangle (2,28);
\fill[black!40] (34,14) rectangle (36,28);
\node[anchor=east] at (0,28){GND};
\node[anchor=west] at (36,28){GND};
\fill[black!15] (17,28) rectangle (19,40);
\node at (18,36){COM};
\draw[->,thick] (18,1) -- (18,-4) node[below]{d$_{31}$変位(水平方向歪み)};
\fill[gray!50] (12,0) rectangle (24,2);
\node[anchor=north] at (18,0){取付面(機械接合・絶縁)};
\end{tikzpicture}}
\caption{櫛歯状の積層PZT素子模式図。内部はCOM/GND電極を交互積層し,両側がGND共通。COMは独立駆動され,d$_{31}$モードで水平方向に変位する。外部端子はNi/Auまたは焼成Ag仕上げが想定される(推定)。}
\label{fig:pzt_part}
\end{figure}

%--------------------------------------------------------
\subsection{COF配線と接続構造}
COF(Chip on Film)は、PZT素子群の上面に実装され、外部ICからの信号伝達および電極接続を担う。  
COF上には複数のCOMラインと両端のGNDラインが形成されており、下端ではACF(Anisotropic Conductive Film)を介してPZT電極群と接続される。  
上端はSn–Ag–Cu(SAC)系はんだにより基板(駆動IC実装側)と接続され、電気ループが閉じる。  
この構成により、PZT駆動回路はCOF層内で完結し、外装構造から電気的に分離されている。

\begin{figure}[t]
\centering
\resizebox{\columnwidth}{!}{
\begin{tikzpicture}[x=1mm,y=1mm,font=\scriptsize,>=Latex]
\fill[gray!15] (0,0) rectangle (70,28);
\node[anchor=west] at (72,14){COF(Sn/Cu構造)};
\fill[gray!35] (5,24) rectangle (65,26);
\node[anchor=west] at (66,25){上端:SACリフロー(PCB接続)};
\fill[gray!35] (5,2) rectangle (65,4);
\node[anchor=west] at (66,3){下端:ACF接合(PZT接続)};
\fill[black!40] (8,4) rectangle (12,24);
\fill[black!15] (20,4) rectangle (22,24);
\fill[black!15] (26,4) rectangle (28,24);
\fill[black!15] (32,4) rectangle (34,24);
\fill[black!15] (38,4) rectangle (40,24);
\fill[black!40] (58,4) rectangle (62,24);
\node[anchor=north] at (10,4){GND};
\node[anchor=north] at (21,4){COM};
\node[anchor=north] at (27,4){COM};
\node[anchor=north] at (33,4){COM};
\node[anchor=north] at (39,4){COM};
\node[anchor=north] at (60,4){GND};
\draw[gray!60,thick] (42,10) rectangle (56,18);
\node at (49,14){IC};
\end{tikzpicture}}
\caption{COF配線構造。上端はPCBと\textbf{SACリフロー接合},下端はPZTと\textbf{ACF接合}。縦配線は両端GND,中央に複数のCOMラインを持つ。COF端子材はSn/Cu構造。}
\label{fig:cof_part}
\end{figure}

%--------------------------------------------------------
\subsection{キャビティおよび機械構造}
PZT素子群の下方には、シリコンエッチングにより形成されたキャビティ構造体(キャビアセンブリ)が配置されている。  
キャビティ膜にはステンレス製の島(SUS島)が形成され、PZT素子の下端がこの島上に接合される。  
PZT変位はSUS島を介してキャビティ膜へ伝達され、インク室内の圧力変化として変換される。  
SUS島およびキャビティ膜は機械的支持体として機能し、電気的には完全に絶縁されている。

\begin{figure}[t]
\centering
\resizebox{\columnwidth}{!}{
\begin{tikzpicture}[x=1mm,y=1mm,font=\scriptsize,>=Latex]
\fill[gray!10] (0,2) rectangle (80,5);
\fill[gray!20] (0,5) rectangle (80,6.2);
\draw[dashed,gray] (4,2.5) -- (76,2.5) node[anchor=west] at (80,2.5){インクキャビティ};
\fill[gray!55] (35,6.2) rectangle (45,7.8);
\node[anchor=west] at (45.5,7.2){SUS島(機械支持・絶縁)};
\fill[gray!08] (37,7.8) rectangle (43,32);
\foreach \y in {9.5,12,14.5,17,19.5,22,24.5,27,29.5}{
  \draw[gray!50] (37,\y)--(43,\y);
}
\fill[black!40] (37,15) rectangle (38,25);
\fill[black!40] (42,15) rectangle (43,25);
\fill[black!15] (39.5,28) rectangle (40.5,32);
\node[anchor=east] at (37,25){GND}; 
\node[anchor=west] at (43,25){GND};
\node at (40,30.8){COM};
\fill[black!30] (36,32) rectangle (44,33);
\node[anchor=west] at (44.5,32.5){ACF接合部(COF→PZT)};
\fill[gray!15] (34,33) rectangle (46,52);
\fill[black!40] (35,35) rectangle (36,50);
\fill[black!15] (39,35) rectangle (41,50);
\fill[black!40] (44,35) rectangle (45,50);
\node[anchor=east] at (35,50){GND}; 
\node at (40,50){COM}; 
\node[anchor=west] at (45,50){GND};
\fill[gray!35] (30,52) rectangle (50,54);
\fill[gray!60] (38.5,51.2) rectangle (41.5,52);
\node[anchor=west] at (50.5,53){PCB(入力)/SACはんだ接合};
\end{tikzpicture}}
\caption{ピエゾアセンブリ(COF+PZT)とキャビアセンブリ(Si+膜+SUS島)の断面模式。上側は入力:\textbf{COF–PCBリフロー接合},下側は出力:\textbf{COF–PZT ACF接合}。PZT外部端子はNi/Auまたは焼成Ag(推定)。}
\label{fig:assembly_section}
\end{figure}

%--------------------------------------------------------
\subsection{電気的ループ構成}
各PZT素子の電気ループは次のように形成される。  
\begin{itemize}
  \item 外部IC出力 $\rightarrow$ COF上のCOMライン $\rightarrow$ ACF $\rightarrow$ PZT内部COM電極
  \item PZT内部GND電極 $\rightarrow$ 側面メタライズ $\rightarrow$ COF両端GNDライン $\rightarrow$ IC基準GND
\end{itemize}
このループはCOFおよびPZT内部で完結しており、キャビティや筐体は関与しない。  
電気的独立性を保つことで、信号クロストークの抑制と駆動安定性を確保している。

\begin{table}[t]
\centering
\footnotesize
\caption{Machヘッド主要構成要素と機能整理}
\label{tab:mach-structure}
\renewcommand{\arraystretch}{1.1}
\begin{tabularx}{\columnwidth}{@{}p{2.2cm} Y Y@{}}
\toprule
構成要素 & 主な機能 & 接続・電気的状態 \\
\midrule
PZT素子(積層体) & 電気‐機械変換(d$_{31}$変形) & COM/GND交互積層,両端GND共通。外部端子はNi/Auまたは焼成Ag(推定)。 \\
COF(Chip on Film) & 信号伝達・接合インタフェース & 下端:ACFでPZTと接続,上端:SACリフローで基板と接続。Sn/Cu端子構造。 \\
ACF層 & 導電接合(異方導電) & Sn–Bi代替材,微細ピッチ対応。 \\
駆動基板(IC含む) & 各COM信号生成・制御 & PCB端子はSAC仕上げ。GND基準共通。 \\
SUS島/キャビティ膜 & 機械支持・圧力伝達 & 機械的に接続,電気的に絶縁。 \\
\bottomrule
\end{tabularx}
\end{table}

このようにMachヘッドは、電気駆動系と流体変位系を完全に分離しつつ、COF–PZT–キャビティの三層構成により安定した吐出を実現している。  
その設計思想は、後のPrecisionCoreでMEMS化・統合化される以前の、実装技術アプローチの完成形と位置づけられる。

%========================================================
\section{接合方式移行}

Sn–Biはんだメッキの供給停止に伴い、Mach世代ヘッドのCOF端子構造を全面的に見直した。  
従来は「Sn–Biメッキ+リフロー」による単一方式であったが、代替材料の特性と用途条件を考慮し、  
出力端子(PZT側)と入力端子(基板側)で異なる接合方式を採用する方針に切り替えた。  
これは、ひとつの方式で全用途を代替することが困難であったため、  
機能要求ごとに適した方式を分離して採用する設計判断である。

%--------------------------------------------------------
\subsection{出力端子(COF→PZT)}
PZT電極側の接合方式は、ACF(Anisotropic Conductive Film)接合へ移行した。  
主な採用理由は以下の通りである。

\begin{itemize}
  \item PZT電極表面(Ni/Auまたは焼成Ag)との親和性が高いこと  
  \item 微細ピッチ端子(\SI{100}{\micro m}級)に対応可能であること  
  \item 既にコンシューマ機用途での採用実績があり、信頼性データを再利用できたこと
\end{itemize}

接続構造は「COF Sn/Cu端子–ACF導電粒子–PZT Ni/Au端子」となり、  
Sn–Biを介さずに導電・機械接続を確立した。  
COF端子材はSn/Cu化され、ACF層との密着性を確保する表面処理条件が最適化された。

%--------------------------------------------------------
\subsection{入力端子(COF→基板)}
基板(PCB)側の接続には、従来方式を踏襲したSAC(Sn–Ag–Cu)系はんだ接合を採用した。  
採用理由は以下の通りである。

\begin{itemize}
  \item 電気的・機械的な接合強度を安定的に確保できること  
  \item 基板側の既存プロセス(SACリフロー条件)を維持できること  
  \item Sn–Biの代替としてSn単独めっきでも工程互換性が高いこと
\end{itemize}

接続構造は「COF Sn/Cu端子–SACはんだ–PCB Cuパッド」となり、  
Sn–Biを含まない完全リードフリー構成で実現された。

%--------------------------------------------------------
\subsection{二重構造の意義}
この結果、Mach世代ヘッドのCOF構造は、  
出力側(PZT接続)=ACF方式、入力側(PCB接続)=SACリフロー方式の  
\textbf{二重接合構造(Dual Bonding System)}へと再設計された。  

これにより、Sn–Bi断絶後も量産を継続可能な接続体系が確立された。  
本方式の特徴は、単一の材料や工法に依存せず、  
機能要求(微細ピッチ/高強度)に応じて接合手段を柔軟に分離した点にある。  

\begin{table}[t]
\centering
\footnotesize
\caption{Mach世代ヘッドの二重接合方式(COF出力/入力の整理)}
\label{tab:dual-bond}
\renewcommand{\arraystretch}{1.1}
\begin{tabularx}{\columnwidth}{@{}p{2.1cm} p{1.2cm} Y Y@{}}
\toprule
接続インタフェース & 方式 & 構造(概略) & 採用理由(要点) \\
\midrule
COF $\rightarrow$ PZT(出力) & ACF &
COF Sn/Cu端子–ACF導電粒子–PZT Ni/Au端子 &
微細ピッチ対応、PZT電極との高い表面親和性、コンシューマ用途実績 \\
\addlinespace[2pt]
COF $\rightarrow$ PCB(入力) & SACリフロー &
COF Sn/Cu端子–SACはんだ–PCB Cuパッド &
接合強度安定、既存量産設備流用、リードフリー化対応 \\
\bottomrule
\end{tabularx}
\end{table}

%========================================================
\section{切り替えの背景と状況}

当初、COFのSn–Biメッキは安定供給されていたが、  
2010年代初頭に外部メッキ工場の閉鎖が決定し、Sn–Biラインの稼働が停止した。  
これにより、代替供給源が存在しないまま在庫逼迫が発生し、  
短期間での全機種切り替えが不可避となった。  

Sn–BiメッキはCOF端子形成の基幹工程であり、  
この供給停止は単なる材料切替ではなく、量産継続の可否に直結する事業リスクであった。  
Machヘッドは当時、コンシューマ機から産業用溶剤機まで、  
ほぼ全てのエプソンプリンタ製品群に採用されていたため、  
この問題は単一製品に留まらず、\textbf{事業全体の供給リスク}として認識された。  

%--------------------------------------------------------
\subsection{用途別の課題}
コンシューマ(家庭用・水系インク)モデルでは、すでにACF接合への移行実績があり、  
Sn–Biディスコンの影響は限定的であった。  
一方、溶剤インク対応の産業用途では、ACF樹脂の溶剤膨潤リスクが指摘されており、  
長期信頼性データが不足していたため、切替判断が難航した。  

加えて、溶剤系ヘッドは高温高湿環境やインク蒸気濃度の高い条件で使用されるため、  
ACFの導電粒子保持層が化学的膨潤を起こす懸念があり、  
評価試験と工程承認が遅延していた。

%--------------------------------------------------------
\subsection{緊急対応プロジェクト}
Sn–Bi供給停止の確定により、残存在庫での操業継続は数カ月以内に限界を迎える見通しとなった。  
そこで、溶剤インク機種を対象とした緊急プロジェクトが発足し、  
リスクを許容した上でACF接合への突貫切替を実施する方針が採られた。  

このプロジェクトは、開発部門だけでなく、調達・製造・品質保証部門を横断した全社体制で構築された。  
COF端子仕様、接合条件、代替ACF材料の調達、量産ライン切替を同時並行で進め、  
わずか数カ月の間に量産再開可能な状態へ到達した。  

結果として、Sn–Bi断絶による供給停止は回避され、  
Machヘッドを中心とする主要製品の出荷が継続された。

\begin{table}[t]
\centering
\footnotesize
\caption{Sn–Bi断絶を起点とした接合方式切替フェーズ整理}
\label{tab:flow}
\renewcommand{\arraystretch}{1.1}
\begin{tabularx}{\columnwidth}{@{}p{1.9cm} Y Y@{}}
\toprule
フェーズ & 主な出来事・判断 & 主なアクション(4M対応) \\
\midrule
従来運用 &
Sn–Biメッキ+リフロー方式で安定量産 &
— \\
\addlinespace[2pt]
外因顕在化 &
Sn–Biメッキ工場閉鎖/供給停止確定 &
緊急代替検討プロジェクト立上げ \\
\addlinespace[2pt]
リスク評価 &
ACFの溶剤膨潤リスクを評価中(長期試験) &
加速試験データ収集中/判断保留 \\
\addlinespace[2pt]
方式決定 &
一方式では代替困難 → 二重接合構造採用 &
出力=ACF(COF→PZT),入力=SAC(COF→PCB) \\
\addlinespace[2pt]
4M変更実施 &
Man:教育・標準化/Machine:治具条件更新/Material:ACF・SAC切替/Method:量産条件再確立 &
複数拠点で統一化対応 \\
\addlinespace[2pt]
結果 &
供給停止を回避/全機種で生産再開 &
事業継続性を確保 \\
\bottomrule
\end{tabularx}
\end{table}

%========================================================
\section{製造管理の難易度}

本件の接合方式移行は、単一機種に閉じた変更ではなく、  
複数の溶剤系Machヘッドを同時に対象とする全社的プロジェクトであった。  
加えて、COFサプライヤ、製造拠点、設備条件が多岐にわたっていたため、  
製造管理は極めて高い複雑性を伴った。

%--------------------------------------------------------
\subsection{機種間のばらつき}
Machヘッドには、インク系・ノズル密度・駆動電圧条件が異なる複数機種が存在した。  
機種ごとにCOFのピッチやパッド寸法が異なり、ACF層の圧着条件や温度プロファイルも個別最適が必要であった。  
そのため、機種別に条件マップを作成し、  
「共通化可能な要素」と「機種固有の要素」を整理するパラメトリック管理が導入された。

%--------------------------------------------------------
\subsection{サプライヤ間の差異}
COFはエプソン社内製、シャープ製、東芝製の三系統で供給されており、  
それぞれ端子構造や表面処理仕様(Sn/Cu厚、Ni下地、酸化防止層)が異なっていた。  
この差異により、ACFおよびSACの濡れ性・初期導通抵抗が変動したため、  
サプライヤ別の接合ウィンドウを個別に設定した。  
条件統制の中心は「表面処理条件 × 圧着温度 × 保持時間」の三要素であり、  
これを統合条件表として社内共通管理した。

%--------------------------------------------------------
\subsection{拠点間の差異}
生産拠点は主に東北エプソンと秋田エプソンの2拠点で構成され、  
それぞれ異なる装置構成・品質保証体制・技能者スキルを持っていた。  
両拠点の工程統一を目的として、  
・条件書の標準化、  
・教育・認定制度の共通化、  
・初期ロット監査および立上げ支援チームの巡回  
が順次実施された。  

初期段階では歩留まり差や接合不良モード(導通不安定・剥離・樹脂溢出)に差が見られたが、  
フィードバックループの確立により、量産安定性が最終的に均一化された。

\begin{table}[t]
\centering
\footnotesize
\caption{製造管理における三重の複雑性と統制ポイント}
\label{tab:complexity}
\renewcommand{\arraystretch}{1.1}
\begin{tabularx}{\columnwidth}{@{}p{2.3cm} Y Y@{}}
\toprule
複雑性の軸 & 主な要因 & 統制ポイント \\
\midrule
機種の多様性 &
ノズル密度・ピッチ差/電圧条件差 &
条件マップ化、共通項と差分項の明示化 \\
\addlinespace[2pt]
サプライヤ差 &
COF表面処理仕様差(Sn厚、Ni下地、Cu構造) &
表面処理別の圧着条件ウィンドウ管理 \\
\addlinespace[2pt]
拠点差 &
設備構成・スキル差/品質保証体制差 &
標準条件書化、教育・監査、初期ロット監視 \\
\bottomrule
\end{tabularx}
\end{table}

%--------------------------------------------------------
\subsection{統合的プロセス管理}
最終的に、接合条件・材料ロット・生産拠点を紐付けて追跡する  
「統合ロットトレーサビリティ」が導入された。  
これにより、COF端子仕様やACFロットの差が歩留まりに及ぼす影響を、  
工程横断で可視化する仕組みが確立された。  

このように、機種差・供給差・拠点差の三重構造を統制するため、  
単なる材料変更ではなく、製造プロセス全体を再構築する活動となった。  
本プロジェクトの製造管理は、実装技術よりもむしろ「事業継続戦略の実行フェーズ」としての性格を持っていた。

%========================================================
\section{評価試験(他部門実施)}

ACF接合方式は微細ピッチ対応やプロセス簡素化の利点を持つ一方で、  
樹脂層を介した異方導電構造を採用するため、  
溶剤インク環境下での化学的安定性に課題があった。  
特に、溶剤蒸気による膨潤や吸収が導電粒子保持層に影響を及ぼす可能性が指摘され、  
実使用条件下での信頼性検証が求められた。

%--------------------------------------------------------
\subsection{試験目的と設計}
本評価は、溶剤蒸気環境におけるACF接合界面の安定性を確認することを目的とした。  
他部門が主導し、恒温槽を用いた加速暴露試験として実施された。  

試験設計は以下の通りである。

\begin{itemize}
  \item サンプル:ACF接合済みMachヘッド(溶剤系構造)
  \item 条件:溶剤インクを同梱し、恒温槽中に放置(高温環境)
  \item 評価モード:密閉容器(蓋あり)/開放容器(蓋なし)の2条件比較
  \item 評価項目:外観変化、導通抵抗、駆動動作確認
\end{itemize}

筆者は試験実施者ではなく、評価用サンプル提供のみを担当した。  
したがって、ここでは報告書に基づく結果要約を示す。

%--------------------------------------------------------
\subsection{試験結果}
試験結果を表\ref{tab:solvent}に示す。  
密閉環境(蓋あり)では溶剤蒸気が滞留し、  
ACF樹脂層が吸収膨潤を起こして導通抵抗が上昇、  
最終的に回路断線に至った。  

一方、開放環境(蓋なし)では蒸気拡散により膨潤が抑制され、  
接合界面の導通および機能は維持された。  
この結果から、ACFの溶剤耐性は蒸気濃度環境に大きく依存することが確認された。

\begin{table}[t]
\centering
\footnotesize
\caption{ACF接合の溶剤蒸気環境下評価結果(他部門試験)}
\label{tab:solvent}
\renewcommand{\arraystretch}{1.1}
\begin{tabularx}{\columnwidth}{@{}p{2.4cm} Y p{2.2cm}@{}}
\toprule
試験条件 & 観察結果(要約) & 動作状態 \\
\midrule
密閉(蓋あり) &
ACF樹脂層が溶剤蒸気を吸収し膨潤。導通抵抗上昇、接合界面劣化。 &
不動作(断線) \\
\addlinespace[2pt]
開放(蓋なし) &
蒸気が拡散し、ACF樹脂の膨潤は抑制。導通安定、外観変化なし。 &
正常動作(維持) \\
\bottomrule
\end{tabularx}
\end{table}

%--------------------------------------------------------
\subsection{技術的判断}
試験結果から、ACF接合は密閉・高蒸気環境下で膨潤劣化を起こす可能性が示唆された。  
しかし、実際のプリンタ運用環境は完全密閉ではなく、  
開放系であるため蒸気滞留は限定的と判断された。  

したがって、量産適用にあたっては以下の暫定判断が下された。

\begin{itemize}
  \item 溶剤環境でも実使用条件ではACF膨潤の影響は限定的。  
  \item 信頼性完全保証は困難だが、動作上は成立する。  
  \item 不具合発生時にはヘッド交換で対応可能とする暫定運用を容認。
\end{itemize}

この判断により、Sn–Bi断絶下でも供給停止を回避し、  
実装リスクを許容範囲内で管理したうえで量産継続が実現された。

%========================================================
\section{顧客対応(他部門実施)}

ACF接合方式への切替は、技術的リスクを伴う暫定解であったため、  
特に溶剤系ヘッドを使用する産業用途の顧客対応が重要な課題となった。  
代表的な影響先はミマキエンジニアリング(Mimaki)をはじめとする産業プリンタメーカーであり、  
長期動作安定性を前提とする製品群であったため、  
接合方式変更は慎重な説明と合意形成を必要とした。

%--------------------------------------------------------
\subsection{説明および合意形成の基本方針}
顧客対応はエプソン社内の他部門(営業・品質保証)が主導し、  
技術部門は変更内容の提供と技術的裏付けを担当した。  
基本方針は以下の三点に整理される。

\begin{enumerate}
  \item Sn–Biメッキの供給停止は不可避であり、既存プロセスの維持は不可能であることを明示する。  
  \item 代替方式としてACF接合を提案し、溶剤環境下での膨潤リスクを正直に共有する。  
  \item 完全保証は困難であるが、動作は成立すること、また不具合時には交換対応を行う体制を提示する。  
\end{enumerate}

この三原則に基づき、顧客との協議では「技術的制約を前提にした供給継続」を主眼に置いた。  
技術的に完璧な解決策ではないことを明確にしつつ、  
代替が存在しない現実を共有し、事業継続を最優先とする判断を促した。

%--------------------------------------------------------
\subsection{交渉経過と合意内容}
顧客側では、ACF接合による長期信頼性低下の懸念が挙げられた。  
これに対し、エプソン側は以下の補足説明を行った。

\begin{itemize}
  \item ACF接合は既に家庭用(水系)モデルで量産実績があり、短期的な信頼性は確保されている。  
  \item 溶剤環境で膨潤リスクがあるものの、開放系構造下では問題が発生しないことが評価試験で確認されている。  
  \item 不具合発生時には製品保証範囲内でヘッド交換を行う運用を組み込む。  
\end{itemize}

これらの説明を踏まえ、顧客は以下の条件付きで承認した。

\begin{enumerate}
  \item ACF接合方式の採用を承認する。  
  \item 溶剤環境下における劣化が発生した場合、交換対応を前提とする。  
  \item 長期信頼性データが整い次第、順次再評価を行う。  
\end{enumerate}

この合意により、製品供給の継続が正式に認められ、  
エプソン側は量産出荷を中断することなく維持することができた。

%--------------------------------------------------------
\subsection{意義と位置づけ}
この対応は、技術的に完全ではない暫定解を前提にしても、  
「リスクを明示した上で顧客と透明性を持って合意形成を行う」ことで、  
事業継続を確保できることを示した実例である。  

特に本件では、  
\textbf{(1)技術上の制約を正直に開示し、(2)代替手段を提示し、(3)リスク管理策を明文化する}  
という三段階の交渉プロセスが有効に機能した。  
その結果、顧客・メーカー双方が納得のうえで暫定解を受け入れる形となり、  
企業間の信頼関係を維持したまま量産体制を継続することができた。

%========================================================
\section{結論}

本件の接合方式移行は、PrecisionCore登場直後の時期に実施されたが、  
当時の量産主力は依然としてMach世代ヘッドであった。  
したがって、単一製品の改修に留まらず、  
\textbf{実質的にエプソン全機種に関わる全社的な4M変更(Man, Machine, Material, Method)}  
として展開された点に大きな特徴がある。  

Sn–Biはんだメッキの供給停止により、従来のリフロー接合法は継続不能となった。  
この制約の下で、COF構造を全面的に見直し、  
出力側(COF→PZT)にはACF、入力側(COF→PCB)にはSAC(Sn–Ag–Cu)を採用する  
「二重接合構造」を短期間で確立したことは、技術的な大きな成果であった。  

溶剤環境下ではACF膨潤リスクという不確定要素が残ったが、  
実使用環境を踏まえた評価結果に基づき、  
「動作は成立し、不具合時には交換対応を行う」という運用条件で  
量産を継続する判断が下された。  
この判断は、完璧ではないが現実的な「成立解(feasible solution)」として機能し、  
事業の連続性を維持した。

\begin{itemize}
  \item コンシューマ(水系インク)用途では、ACF量産が安定して継続された。  
  \item 産業用(溶剤系)用途では、条件付き運用下で供給を維持できた。  
  \item 材料・設備・人員を跨いだ全社的対応により、供給停止を回避した。  
\end{itemize}

本事例の技術的意義は、単なる材料置換ではなく、  
\textbf{用途ごとに異なる接合方式を選択・併用する柔軟な実装設計体系を構築した}  
点にある。  
また、事業的意義としては、完全解を追求するのではなく、  
「不完全でも成立する現実解を提示し、供給責任を果たす」判断を下した点が挙げられる。  

この経験から得られる普遍的教訓は、  
\textbf{外部制約下においては技術的最適解よりも“継続可能な実用解”を優先すべき場面が存在する}  
ということである。  
すなわち、限られた時間・材料・工程の中でも、  
「止めないこと」そのものが最も重要な技術成果であるという点を再確認させた事例であった。

%--------------------------------------------------------
\section*{謝辞}

本報告に関連する評価試験、顧客交渉、材料調達、製造条件切替に携わった  
エプソン社内の関係部門、ならびにCOFサプライヤ各社の多大な協力に深く感謝する。  
特に、溶剤系ヘッド切替時に迅速な判断と対応を実施いただいた  
調達・製造・品質保証部門に謝意を表する。

%--------------------------------------------------------
\section*{著者略歴}

\textbf{三溝 真一}(Shinichi Samizo)は、信州大学大学院 工学系研究科 電気電子工学専攻にて修士号を取得。  
セイコーエプソン株式会社にて、半導体ロジック・メモリ統合設計、  
インクジェットヘッドのPZT駆動構造およびPrecisionCore製品化を担当。  
現在は独立系半導体研究者として、プロセスデバイス教育・AI制御アーキテクチャ研究に従事。  
連絡先:\href{mailto:shin3t72@gmail.com}{shin3t72@gmail.com}.

