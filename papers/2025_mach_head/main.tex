\documentclass[conference]{IEEEtran}

% ==== 追加推奨(LuaLaTeX + IEEEtran + TikZ を安定運用)====

% --- 基本環境 ---
\usepackage{luatexja}              % 日本語処理の基本
\usepackage{graphicx}
\usepackage{amsmath}
\usepackage{array}
\usepackage{booktabs}
\usepackage{tabularx}

% --- 色とTikZ(塗り分けに必須)---
\usepackage{xcolor}
\usepackage{tikz}
\usetikzlibrary{arrows.meta,positioning,shapes,fit}

% --- フォント設定 ---
\usepackage{luatexja-fontspec}
\setmainfont{TeX Gyre Termes}      % IEEEtran向けTimes系
\setsansfont{TeX Gyre Heros}
\setmonofont{TeX Gyre Cursor}
\setmainjfont{IPAexMincho}         % 好みで変更可
\setsansjfont{IPAexGothic}
\ltjsetparameter{jacharrange={-2,-3}, autoxspacing=true, autospacing=true}

% --- 引用・参照 ---
\usepackage{cite}

% --- 単位と数値整形 ---
\usepackage{siunitx}
\sisetup{detect-all}

% --- URL・ハイパーリンク ---
\usepackage{url}
\usepackage{hyperref}              % ★ hypersetup より前に必ず置く
\def\UrlBreaks{\do\/\do-\do\_}
\hypersetup{hidelinks}             % ★ ここに置くのが正解

% --- 表関連設定(tabularx 読み込み後に置く)---
\newcolumntype{Y}{>{\raggedright\arraybackslash}X}
\renewcommand\tabularxcolumn[1]{m{#1}}

% --- 段落バランス(2段組の最終ページ空白防止)---
\usepackage{balance}
% \balance  % 結論直前で有効化

\begin{document}

\title{Sn-Biはんだメッキ断絶に伴う\\Mach世代インクジェットヘッドの接合方式移行事例}

\author{%
  \IEEEauthorblockN{三溝 真一 (Shinichi Samizo)}%
  \IEEEauthorblockA{独立系半導体研究者(元セイコーエプソン)\\%
  Independent Semiconductor Researcher (ex-Seiko Epson)\\%
  Email: \href{mailto:shin3t72@gmail.com}{shin3t72@gmail.com}\\%
  GitHub: \url{https://github.com/Samizo-AITL}}%
}

\maketitle

\begin{abstract}
本稿では、Sn-Biはんだメッキ工場の閉鎖に伴い、Mach世代インクジェットヘッドの接合方式をSn-BiリフローからACFおよびはんだ接合へ移行した事例について報告する。PrecisionCore登場直後の時期にあたり、実質エプソン全機種に関わる大規模な4M変更となった。本事例は、外的制約下において「不完全でも成立する解決策」を選択し、事業継続性を確保した実装技術の一例である。
\end{abstract}

\begin{IEEEkeywords}
インクジェットヘッド, Sn-Biはんだ, ACF接合, 実装技術, 4M変更
\end{IEEEkeywords}

\vspace{1em}
\noindent
\textbf{Abstract (English):}  
This paper reports a case study of the bonding method transition in Mach generation inkjet heads, from Sn-Bi solder reflow to ACF and solder bonding, due to the shutdown of the Sn-Bi plating facility.  
The transition took place shortly after the introduction of PrecisionCore, at a time when most Epson printers still relied on Mach heads.  
It represented a large-scale 4M (Man, Machine, Material, Method) change across the entire product line.  
This case demonstrates an example of implementation technology in which, under external constraints, an "incomplete but workable solution" was chosen to ensure business continuity.

\section{背景}
Mach世代インクジェットヘッドは、エプソンが2000年代後半から2010年代にかけて量産展開した主力プリントヘッドであり、d31モードバルク積層PZTアクチュエータを採用していた。この方式は、比較的高い吐出圧力と安定したインク吐出を実現できるため、コンシューマ用途から産業用途まで幅広く適用されていた。  

当初、COF(Chip on Film)配線とPZTアクチュエータとの接合には、Sn-Biはんだメッキを用いたリフロー方式が採用されていた。Sn-Bi系はんだは低融点であり、PZT基板やCOFへの熱ダメージを抑制できること、また量産工程として安定した実績があることから選択された。  

しかし、外部要因としてSn-Biメッキ工場の閉鎖が確定し、供給ディスコン(discontinuation)が避けられない状況となった。Sn-BiメッキはCOF端子形成の根幹であり、代替材料が存在しなかったため、この時点で従来の量産方式は継続不可能となった。  

さらに、この時期はPrecisionCoreが市場投入された直後であったが、量産規模はまだ限定的であり、事業の主力は依然としてMach世代ヘッドであった。特にコンシューマプリンタから産業用溶剤インク機まで、ほぼ全てのエプソンプリンタ製品群がMachヘッドに依存していたため、この接合方式問題は単一機種に留まらず、\textbf{実質的にエプソン全機種に影響する全社的課題}となった。  

結果として、Sn-Bi断絶への対応は「技術課題」であると同時に、「事業継続リスク」そのものであり、短期間での大規模な4M変更(Man, Machine, Material, Method)を余儀なくされる状況に至った。

\section{接合方式移行}
Sn-Bi断絶を受け、COF端子構造を全面的に見直し、出力端子(PZT側)と入力端子(基板側)で異なる接合方式を導入した。これは「一方式で全てを代替する」ことが困難であったため、用途と要求特性に応じて最適な方式を振り分けるという設計判断である。

\subsection{出力端子(COF→PZT)}
接合方式はACF (Anisotropic Conductive Film) 接合を採用した。主な理由は以下の通りである。  
\begin{itemize}
  \item PZT電極との高い表面親和性を確保できること  
  \item 微細ピッチ端子に対応可能であり、Sn-Bi代替として実用性があること  
  \item すでにコンシューマ用途で実績があり、信頼性データを一定程度参照可能であったこと  
\end{itemize}
接続構造は「COF Sn端子–ACF導電粒子–PZT電極」となり、Sn-Biを介さずに電気的導通を実現した。

\subsection{入力端子(COF→基板)}
接合方式は従来の延長上にあるはんだ接合を選択した。理由は以下の通りである。  
\begin{itemize}
  \item 電気的および機械的な接合強度を安定して確保できること  
  \item 基板側の既存プロセスを大きく変更せずに適用可能であること  
  \item Sn-BiではなくSn単独メッキに対応した基板側はんだ材料が利用可能であったこと  
\end{itemize}
接続構造は「COF Sn端子–基板側はんだ–基板パッド」となり、リフローにより安定した接続が実現された。

\subsection{二重構造の意義}
出力側にACF、入力側にハンダという二重構造の導入により、Sn-Bi断絶後も量産可能な接続体系を確立した。この方式は「単一の代替材料に依存せず、要求仕様に応じて方式を分離する」という点で、外的制約下における柔軟な実装戦略の一例となった。

\begin{table}[t]
\centering
\footnotesize
\caption{Mach世代ヘッドの二重接合方式(COF出力/入力の整理)}
\label{tab:dual-bond}
\renewcommand{\arraystretch}{1.1}
\begin{tabularx}{\columnwidth}{@{}p{1.95cm} p{1.25cm} Y Y@{}}
\toprule
接続インタフェース & 方式 & 構造(簡略) & 採用理由(要点) \\
\midrule
COF $\rightarrow$ PZT(出力) & ACF &
COF Sn端子–ACF導電粒子–PZT電極 &
微細ピッチ適合、PZT電極との親和性、コンシューマ用途での実績参照可 \\
\addlinespace[2pt]
COF $\rightarrow$ 基板(入力) & はんだ &
COF Sn端子–基板側はんだ–基板パッド &
電気・機械強度が安定、既存基板実装プロセスを活かせる \\
\bottomrule
\end{tabularx}
\end{table}

\section{切り替えの背景と状況}
コンシューマ(水系インク)用ヘッドは先行してACF接合へ切り替え済みであり、量産に支障はなかった。一方で、溶剤インク対応ヘッドについてはACF樹脂の溶剤膨潤リスクが懸念され、評価や信頼性確認が遅れたため、切り替えが後回しとなっていた。  

しかし、ここにSn-Biディスコンの確定が重なり、残存在庫も限られる中で、ショーテージが目前に迫る状況となった。つまり「リスクを抱えたままでも切り替えを断行しなければ、製品供給そのものが途絶する」という、待ったなしの危機的状況であった。  

このため、溶剤インクヘッドについてもリスクを許容しつつ強制的にACFへ切り替える突貫プロジェクトが発足した。本プロジェクトには開発部門のみならず、調達部も全面的に参画し、材料供給確保・代替部材調達・製造工程切替・量産立ち上げを同時並行で進める体制が構築された。  

このようにして、Sn-Bi断絶を外的トリガーとした全社的な緊急対応が開始され、Mach世代ヘッドの全機種を対象とする大規模な接合方式移行が実行された。

\begin{table}[t]
\centering
\footnotesize
\caption{Sn-Bi断絶を起点とした接合方式切替のフェーズ整理}
\label{tab:flow}
\renewcommand{\arraystretch}{1.1}
\begin{tabularx}{\columnwidth}{@{}p{1.8cm} Y Y@{}}
\toprule
フェーズ & 主要イベント/判断 & 主なアクション(4M対応) \\
\midrule
従来運用 &
Sn-Biメッキ+リフローで安定量産 &
— \\
\addlinespace[2pt]
外因顕在化 &
Sn-Biメッキ工場閉鎖・ディスコン確定/在庫逼迫 &
代替検討プロジェクト立上げ(緊急) \\
\addlinespace[2pt]
リスク評価 &
溶剤インクでACF膨潤懸念(密閉>開放) &
加速試験(他部門):密閉×不良/開放×維持 \\
\addlinespace[2pt]
方式選択 &
一方式では両立困難 → 二重構造を採用 &
出力=ACF(COF→PZT)、入力=はんだ(COF→基板) \\
\addlinespace[2pt]
4M変更 &
Man:教育・標準化/Machine:設備条件最適化/Material:ACF・はんだ・COF Sn化/Method:量産条件確立 &
複数機種×複数サプライヤ×複数拠点の統制 \\
\addlinespace[2pt]
結果 &
コンシューマ用途:問題なし/産業用途:交換対応条件で供給継続 &
事業継続性を確保 \\
\bottomrule
\end{tabularx}
\end{table}

% ====== 2カラムの決定版 図(スケッチ準拠) ======
\begin{figure*}[t]
\centering
\resizebox{\textwidth}{!}{%
\begin{tikzpicture}[x=1mm,y=1mm,>=Latex,
  blk/.style={draw, line width=0.35pt},
  si/.style={blk, fill=black!5},
  pzt/.style={blk, fill=black!12},
  film/.style={blk, fill=black!2},
  cavity/.style={blk, fill=white},
  cof/.style={blk, fill=black!8},
  icpkg/.style={blk, fill=black!20, rounded corners=0.8mm},
  bondACF/.style={blk, fill=black!28},      % 出力:ACF
  bondSolder/.style={blk, fill=black!45},   % 入力:はんだ
  lab/.style={font=\scriptsize}
]

% ===== 下地:Siとキャビティ =====
% Si台座(左右の“脚”を残して中央にキャビティ)
\draw[si] (0,0) rectangle (160,10);                       % Si全体ベース
\draw[cavity] (40,0.3) rectangle (120,10);                % キャビティ開口
\node[lab] at (80,2.2) {キャビティ};
\node[lab, anchor=west] at (121,2.2) {Si(基板)};

% キャビティ上の保護/絶縁フィルム(PPT?)
\draw[film] (0,10) rectangle (160,11);
\node[lab, anchor=west] at (161,10.5) {保護フィルム(PPT?)};

% ===== PZT(中央) =====
\draw[pzt] (72,11) rectangle (88,65);
\node[lab, anchor=west] at (89,40) {バルクPZT};

% ===== COF(左側の縦配線)+IC =====
\draw[cof] (55,11) rectangle (60,65);                     % COF縦ストリップ
\node[lab, rotate=90] at (57.5,40) {COF};
% 上部ICパッケージ(COFに実装)
\draw[icpkg] (50,54) rectangle (60,63);
\node[lab, anchor=east] at (49.8,58.5) {IC};

% ===== 出力側:COF→PZT の ACF接合(スケッチ赤部) =====
\draw[bondACF] (60,33) rectangle (72,37);                 % ACFブリッジ
% ラベルと矢印
\draw[->,line width=0.35pt] (100,50) -- (72,35.5);
\node[lab, align=left, anchor=west] at (101,50)
{出力(COF$\rightarrow$PZT)\\ACF接合(導電粒子)};

% ===== 入力側:COF→基板 の はんだ接合(下部) =====
% COF下端からSiのパッドへ小さく接続
\draw[bondSolder] (56,12.0) rectangle (59,13.6);
\draw[blk] (57.5,11) -- (57.5,12.0);                      % COF下端と接続
\node[lab, anchor=east] at (47.5,12.8) {入力(COF$\rightarrow$基板) はんだ接合(Sn系)};
\draw[->,line width=0.35pt] (47.8,12.8) -- (56,12.8);

% ===== 電圧/配線の注記 =====
\draw[->,line width=0.35pt] (57.5,65) -- (57.5,71);
\node[lab, anchor=south] at (57.5,71) {配線(信号 / $-10$V 等)};

% ===== 溶剤蒸気 膨潤リスク(ACF付近) =====
\draw[->,line width=0.35pt] (122,38) -- (72,36.5);
\node[lab, align=left, anchor=west] at (123,38)
{溶剤蒸気環境\\ACF樹脂の膨潤リスク};

% ===== 凡例 =====
\begin{scope}[shift={(0,77)}]
  \node[lab, anchor=west] at (0,0) {凡例:};
  \draw[cof] (12,-3) rectangle (20,3);   \node[lab, anchor=west] at (21,0) {COF};
  \draw[pzt] (34,-3) rectangle (42,3);   \node[lab, anchor=west] at (43,0) {PZT};
  \draw[si]  (56,-3) rectangle (64,3);   \node[lab, anchor=west] at (65,0) {Si};
  \draw[film](78,-3) rectangle (86,3);   \node[lab, anchor=west] at (87,0) {保護フィルム};
  \draw[bondACF] (104,-3) rectangle (112,3); \node[lab, anchor=west] at (113,0) {ACF接合(出力)};
  \draw[bondSolder] (134,-3) rectangle (142,3); \node[lab, anchor=west] at (143,0) {はんだ接合(入力)};
\end{scope}

% ===== 下部ラベル(I/O分離を強調) =====
\node[lab, anchor=east] at (30,36) {I/O分離:出力=ACF/入力=はんだ};

\end{tikzpicture}
}% end resizebox
\caption{Mach世代インクジェットヘッドの断面概念図(スケッチ準拠)。\;左のCOFから上部ICに接続し、\textbf{出力側(COF$\rightarrow$PZT)はACF}、\textbf{入力側(COF$\rightarrow$基板)ははんだ}で接合する二重方式。中央のバルクPZTはキャビティ上に配置され、保護フィルム(PPT?)を介してSi基板に載る。溶剤蒸気環境ではACF樹脂の膨潤リスクがあるため、開放環境の設計配慮が必要。}
\label{fig:mach_dual_bond_xsec}
\end{figure*}

\section{製造管理の難易度}
本件は単一機種に閉じた問題ではなく、複数の溶剤インク対応ヘッド機種を同時に対象とする必要があった。さらに、COFはエプソン製・シャープ製・東芝製と複数の外部サプライヤから供給されており、それぞれの端子構造・表面処理条件・品質保証基準が微妙に異なっていた。したがって、単純な一律切替ではなく、サプライヤごとに接合条件の最適化が求められた。  

また、接合工程は「COFとPZT接合」と「COFと基板接合」の二系統が存在し、それぞれにACF/はんだの適用方式が異なっていた。PZT側では微細ピッチかつ材料親和性を考慮したACFが必須であり、基板側では電気的・機械的強度を担保するためのはんだ接合が要求された。両者のプロセス条件を矛盾なく成立させることが技術的課題となった。  

さらに、製造拠点は東北エプソン・秋田エプソンの複数現法に分散していた。それぞれ異なる生産ライン設備、品質保証体制、作業員スキルレベルを持っており、統一的な切替を行うには全拠点間での工程条件共有・標準化が不可欠であった。特に量産初期には拠点ごとの歩留まり差や不良モードのばらつきが顕在化し、それを吸収するための工程調整やフィードバックループが繰り返された。  

したがって、本プロジェクトの製造管理は「複数機種 × 複数サプライヤ × 複数拠点」という三重の複雑性を同時に扱う必要があり、単なる材料置換ではなく、事業全体を巻き込む大規模なプロセス統制活動であった。結果として、全社横断的な製造・調達・品質保証の協調がなければ成立しない高難度の切替プロジェクトとなった。

\begin{table}[t]
\centering
\footnotesize
\caption{製造管理における三重の複雑性と統制ポイント}
\label{tab:complexity}
\renewcommand{\arraystretch}{1.1}
\begin{tabularx}{\columnwidth}{@{}p{2.4cm} Y Y@{}}
\toprule
複雑性の軸 & 具体例 & 統制ポイント \\
\midrule
機種の多様性 &
複数の溶剤インクヘッド機種が同時対象 &
機種別の接合条件マップ化/量産条件の共通化と差分管理 \\
\addlinespace[2pt]
サプライヤ差 &
COF:エプソン/シャープ/東芝(端子構造・表面処理差) &
サプライヤ別の最適条件と受入基準の明確化 \\
\addlinespace[2pt]
拠点差 &
東北エプソン/秋田エプソン(設備・QA・スキル差) &
標準作業・教育・立上げ監査/歩留まりフィードバックループ \\
\bottomrule
\end{tabularx}
\end{table}

\section{評価試験(他部門実施)}
ACF接合は微細ピッチ適合性に優れる一方で、樹脂材料を用いるため溶剤インクとの化学的相互作用が懸念された。特に、溶剤蒸気による膨潤や劣化が接合界面に与える影響は未知数であり、信頼性リスク評価が不可欠であった。  

このため、他部門により恒温槽を用いた加速試験が実施された。試験は「ACF接合ヘッドと溶剤インクを同梱したタッパー容器を恒温環境下に放置する」という単純化されたモデル評価である。筆者自身は試験を実施しておらず、役割はACF接合ヘッドのサンプル提供に限定された。試験条件(温度、時間、サンプル数)の詳細は手元に残っていないが、社内報告では以下の結果が示されている。

\begin{itemize}
 \item 蓋あり(密閉環境で蒸気滞留): ACF樹脂が膨潤し、接合部の電気的導通が失われ、ヘッドは不動作となった。
 \item 蓋なし(開放環境で蒸気拡散あり): ACFの膨潤は抑制され、接合機能は維持され、ヘッドは正常動作を継続した。
\end{itemize}

これらの結果は、ACF接合が溶剤インクの蒸気濃度環境に強く依存することを示している。密閉条件下では蒸気が滞留し、ACF内部に吸収・膨潤が進行するため、接合不良を引き起こす。一方、蒸気が逃げる開放環境では膨潤が抑制され、実用上の動作は維持できることが確認された。  

ただし、実際のプリンタ使用環境は完全密閉ではなく、むしろ開放系に近い。このため「実環境では蓋なし条件に近い挙動を期待できるが、溶剤系では膨潤リスクがゼロではない」という判断に至った。すなわち、ACF採用は不完全解であるものの、代替が存在しない中で「動作は成立する」ことを確認した意義は大きかった。

\begin{table}[t]
\centering
\footnotesize
\caption{ACF接合の溶剤蒸気環境評価(他部門試験の要約)}
\label{tab:solvent}
\renewcommand{\arraystretch}{1.1}
\begin{tabularx}{\columnwidth}{@{}p{2.2cm} Y p{1.9cm}@{}}
\toprule
条件 & 観察された現象(要約) & 動作結果 \\
\midrule
蓋あり(密閉,蒸気滞留) &
ACF樹脂の膨潤/接合界面の導通不安定化・劣化 &
不動作 \\
\addlinespace[2pt]
蓋なし(開放,蒸気拡散) &
膨潤は抑制/接合機能は維持 &
正常動作 \\
\bottomrule
\end{tabularx}
\end{table}

\section{顧客対応(他部門実施)}
本件の影響を強く受けたのは、溶剤インクを用いる産業用途の顧客群であった。代表例としてミマキエンジニアリング(Mimaki)が挙げられるが、他にも同様の条件下で製品を展開している複数顧客が存在した。これらの顧客は高い印字信頼性と長期安定稼働を要求しており、接合方式変更によるリスクは極めて重大な懸念事項であった。  

顧客との交渉・合意形成は他部門が担当したが、その基本方針は以下に整理できる。  

\begin{itemize}
 \item Sn-Bi再利用は不可能であり、代替はACF接合方式しか存在しないことを説明した。  
 \item ACFは溶剤環境下で膨潤リスクを伴うことを事前に明示し、技術的限界を共有した。  
 \item リスクマネジメント策として「不具合が発生した場合は迅速にヘッド交換を行う」ことを条件とした。  
\end{itemize}

結果として、顧客は「完全な解決策ではないが、事業を継続するために必要な措置」と理解し、ACF方式での供給を受け入れた。この合意により、エプソン側は製品供給を止めることなく継続可能となり、顧客側も事業を維持することができた。  

この対応は、技術的に完全無欠なソリューションが得られない状況下においても、リスクを明示し、代替策を提示し、顧客と合意形成を図ることで事業継続性を確保する一つの実践例となった。

\section{結論}
本件の接合方式移行は、PrecisionCoreが市場に投入された直後の時期に実施されたが、当時の量産主力は依然としてMach世代であった。このため、単一製品に留まらず、\textbf{実質的にエプソン全機種に関わる大規模な4M変更(Man, Machine, Material, Method)} となった。  

Sn-Biはんだメッキのディスコンにより、従来のリフロー接合法は存続不可能となり、接合方式は事実上ACFに一本化せざるを得なかった。溶剤用途におけるACF膨潤リスクは完全には解決できなかったものの、「不完全でも成立する方法を提示し、製品供給を維持する」という判断が下された。  

その結果、以下の成果が得られた。  
\begin{itemize}
 \item コンシューマ(水系インク)用途では、問題なく量産継続が可能であった。  
 \item 産業用(溶剤インク)用途では、リスクを抱えつつも「不具合発生時のヘッド交換対応」を条件に供給を継続できた。  
 \item 全社横断的な調達・製造・品質保証部門の連携により、突発的な外部要因に対しても事業継続性を確保できることが実証された。  
\end{itemize}

本事例の技術的意義は、単なる材料置換にとどまらず、\textbf{複数の接合方式を役割ごとに組み合わせることで、新たな実装体系を短期間で確立した}点にある。また、事業的意義としては、完全解ではなくとも「成立可能な現実解」を選択し、顧客との合意形成を通じて供給責任を果たした点が挙げられる。  

さらに、本事例から得られる普遍的な教訓として、\textbf{外部制約下では技術的最適解よりも「事業を継続可能とする実用解」の提示が優先される場合がある}ことが明らかになった。すなわち、「不完全でも使える方法」を確立することが、製品開発や生産技術において重要な戦略の一つであることを示している。

\section*{謝辞}
本研究に関連する評価試験や顧客交渉を担当されたエプソン社内の関連部門、ならびにCOFサプライヤ各社の協力に深く感謝する。また、溶剤インクヘッド切替に際して調達・製造・品質保証部門が迅速に対応いただいたことをここに記して謝意を表する。

\begin{thebibliography}{99}
\bibitem{ACF_tech}
S. Kim, H. Lee, and J. Park, “Reliability of ACF bonding for fine-pitch interconnects,” \textit{IEEE Trans. Compon. Packag. Technol.}, vol. 33, no. 4, pp. 957–964, Dec. 2010.

\bibitem{IJ_piezo}
M. Nakagawa, T. Iwase, and K. Yoshino, “Development of piezoelectric inkjet printhead technologies,” \textit{Jpn. J. Appl. Phys.}, vol. 50, no. 9, pp. 09PA01, 2011.

\bibitem{Pb_free}
Y. Suganuma, \textit{Lead-Free Soldering in Electronics: Science, Technology, and Environmental Impact}. New York: Marcel Dekker, 2004.

\bibitem{Epson_head}
H. Okada, T. Kato, and K. Kimura, “High-density piezoelectric inkjet printhead with MEMS-based precision structure,” in \textit{Proc. IEEE MEMS}, 2013, pp. 451–454.

\bibitem{SnBi_issue}
J. H. Lau, \textit{Solder Joint Reliability of BGA, CSP, Flip Chip, and Fine Pitch SMT Assemblies}. New York: McGraw-Hill, 1997.
\end{thebibliography}

\section*{著者略歴}
\textbf{三溝 真一}(Shinichi Samizo)は、信州大学大学院 工学系研究科 電気電子工学専攻にて修士号を取得した。  
その後、セイコーエプソン株式会社に勤務し、半導体ロジック/メモリ/高耐圧インテグレーション、さらにインクジェット薄膜ピエゾアクチュエータおよびPrecisionCoreプリントヘッドの製品化に従事した。  
現在は独立系半導体研究者として、プロセス/デバイス教育、メモリアーキテクチャ、AIシステム統合などの研究に取り組んでいる。  
連絡先: \href{mailto:shin3t72@gmail.com}{shin3t72@gmail.com}.

\end{document}
