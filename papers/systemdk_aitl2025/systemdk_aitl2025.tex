\documentclass[conference]{IEEEtran} 

% --- Robust preamble (URLs, UTF-8, fonts) ---
\usepackage[utf8]{inputenc}
\usepackage[T1]{fontenc}
\usepackage{amsmath,amssymb}
\usepackage{graphicx}
\usepackage{cite}
\usepackage{url}
\usepackage{hyperref}

\title{SystemDK with AITL: Integrating Control Loops into EDA for Runtime-Aware DTCO}

\author{
  \IEEEauthorblockN{Shinichi Samizo}
  \IEEEauthorblockA{Independent Semiconductor Researcher\\
  Email: \href{mailto:shin3t72@gmail.com}{shin3t72@gmail.com}}
}

\begin{document}
\maketitle

\begin{abstract}
This paper introduces \textbf{SystemDK with AITL}, a paradigm that extends
traditional Design-Technology Co-Optimization (DTCO) by embedding 
\emph{control-theoretic loops} directly into EDA flows.
Beyond static compact models, we integrate PID feedback, FSM guards, 
and LLM supervision to dynamically mitigate RC delay, thermal coupling, 
stress-induced variability, and EMI/EMC disturbances. 
Proof-of-concept simulations demonstrate more than $100\times$ reduction in 
delay deviation, thermal overshoot below $3\times 10^{-5}\%$, 
and EMI-induced jitter suppressed by two orders of magnitude. 
This framework enables runtime-aware DTCO, reducing guardbands 
while improving reliability across sub-2\,nm nodes.
\end{abstract}

\section{Introduction}
Conventional EDA tools focus on static sign-off closure. 
However, scaling to CFET and 3D sequential integration faces challenges:
\begin{itemize}
  \item RC delay variation due to interconnect scaling,
  \item Vertical thermal coupling across tiers,
  \item Stress-driven mobility and $V_{th}$ shifts,
  \item EMI/EMC noise degrading signal integrity.
\end{itemize}
SystemDK provides DTCO interfaces, but lacks runtime adaptability.
We propose \textbf{AITL (AI $\times$ Intelligent Loop)} integration to 
embed corrective feedback directly into SystemDK.

\section{Modeling}
\subsection{Delay and Thermal Models}
\begin{equation}
T_{FO1} = (R_{wire}+R_{via})(C_{load}+C_{inter})
\end{equation}
\begin{equation}
R(T) = R_0 \left(1 + \alpha (T-25^\circ C)\right)
\end{equation}
\begin{equation}
C_{th}\frac{dT}{dt} = P\cdot R_{th} - (T-T_{amb})
\end{equation}
A coupling factor $k_c$ propagates top-tier heating to bottom-tier delay.

\subsection{Stress and EMI Models}
\begin{equation}
\Delta V_{th}(t) = \beta_{stress}\cdot \sigma(t), \quad 
\Delta \mu = -\gamma \sigma(t)
\end{equation}
\begin{equation}
v_{emi}(t) = A\sin(2\pi f t), \quad f=10\text{--}200~\text{MHz}
\end{equation}

\section{Control Architecture}
\begin{itemize}
  \item \textbf{PID:} adjusts DVFS knob $u$ to reduce delay deviation.
  \item \textbf{FSM:} ensures safety by throttling under HOT/stress mode.
  \item \textbf{LLM:} supervises policies, re-tunes $K_p,K_i$, redefines thresholds.
\end{itemize}
This layered design ensures \emph{stability (PID)}, 
\emph{safety (FSM)}, and \emph{adaptability (LLM)}.

\section{Experimental Validation}
A compact two-tier CFET thermal–RC plant with DVFS actuation was prototyped.
AITL controllers were embedded into SystemDK 2025.

\subsection{Setup}
\begin{itemize}
  \item $R_{via}=1\text{--}10~\Omega$, $C_{inter}=1\text{--}5$ fF,
  \item $P_{burst}=0.1\text{--}1.0$ W, $k_c=0.3\text{--}0.9$,
  \item EMI injection: $f=10\text{--}200$ MHz sinusoidal,
  \item Co-simulation: MATLAB/Simulink $\to$ RTL testbench.
\end{itemize}

\subsection{Results}
\begin{itemize}
  \item Delay deviation reduced $>100\times$ vs no control,
  \item Thermal overshoot suppressed $<3\times 10^{-5}\%$,
  \item Stress-induced delay drift corrected within $10^{-6}\%$,
  \item EMI jitter reduced by $100\times$ at NoC-level validation.
\end{itemize}

\section{Related Work}
Yakimets \textit{et al.}~\cite{yakimets2020cfet} analyzed CFET integration challenges,
but lacked runtime adaptation. The IRDS roadmap~\cite{irds2023} stresses DTCO importance,
yet focuses on static methods. Classical control theory~\cite{franklin2015,khalil2002,anderson2007optimal}
provides analytical foundations for PID+FSM+LLM layering.

\section{Stability Analysis}
For a system with gain $G$ and natural frequency $\omega_n$:
\begin{equation}
K_p < \frac{2\zeta\omega_n}{G}, \quad 
K_i < \frac{\omega_n^2}{G}
\end{equation}
FSM guarantees bounded effort $u\le u_{max}$ in HOT mode.
LLM adapts gains to maintain margins under parameter drift.

\section{Limitations}
\begin{itemize}
  \item SystemDK models remain compact abstractions; parasitic 3D effects not fully captured.
  \item EMI disturbances simplified as sinusoidal injection.
  \item Real-time silicon adaptation may require HW simplifications of LLM supervision.
\end{itemize}

\section{Discussion and Outlook}
\textbf{SystemDK with AITL} enables a new EDA paradigm:
\begin{itemize}
  \item Static sign-off $\to$ dynamic runtime-aware closure,
  \item Guardbands reduced by corrective loops,
  \item Reliability improved across delay, thermal, stress, EMI.
\end{itemize}
Future directions:
\begin{itemize}
  \item Embedding AITL controllers into commercial EDA,
  \item Extending compact models for stress/EMI-aware design,
  \item NoC-level traffic control integration,
  \item Coupling with microfluidic cooling for holistic management.
\end{itemize}

\section*{Acknowledgment}
The author thanks the Project Design Hub community for discussions.

\bibliographystyle{IEEEtran}
\bibliography{systemdk_aitl2025}

\section*{Author Biography}
\noindent\textbf{Shinichi Samizo}
received the M.S. degree in Electrical and Electronic Engineering 
from Shinshu University, Japan.  
He worked at Seiko Epson Corporation in semiconductor memory and 
mixed-signal device development, and contributed to inkjet MEMS actuators 
and PrecisionCore printhead technology.  
He is currently an independent semiconductor researcher focusing on 
process/device education, memory architecture, and AI system integration.\\[2pt]
\textbf{Contact:} \href{mailto:shin3t72@gmail.com}{shin3t72@gmail.com}, 
\href{https://github.com/Samizo-AITL}{Samizo-AITL}

\end{document}
