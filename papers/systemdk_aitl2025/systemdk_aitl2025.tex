\documentclass[conference]{IEEEtran} 

% --- Robust preamble (URLs, UTF-8, fonts) ---
\pdfoutput=1                  % 強制的にPDF出力
\usepackage[utf8]{inputenc}
\usepackage[T1]{fontenc}
\usepackage{amsmath,amssymb}
\usepackage{graphicx}
\usepackage{cite}
\usepackage{url}
\usepackage{hyperref}
\usepackage{times}            % IEEE推奨フォント

\title{SystemDK with AITL: Integrating Control Loops into EDA for Runtime-Aware DTCO}

\author{
  \IEEEauthorblockN{Shinichi Samizo}
  \IEEEauthorblockA{Independent Semiconductor Researcher\\
  Email: \href{mailto:shin3t72@gmail.com}{shin3t72@gmail.com}}
}

\begin{document}
\maketitle

\begin{abstract}
This paper introduces \textbf{SystemDK with AITL}, a paradigm that extends 
traditional Design-Technology Co-Optimization (DTCO) by embedding 
\emph{control-theoretic loops} directly into EDA flows. 
Beyond static compact models, we integrate PID feedback, FSM guards, 
and LLM supervision to dynamically mitigate RC delay, thermal coupling, 
stress-induced variability, and EMI/EMC disturbances. 
This framework enables runtime-aware DTCO, reducing guardbands while 
improving reliability across sub-2\,nm nodes.
\end{abstract}

\section{Introduction}
Conventional EDA focuses on static design closure. 
However, future nodes face dynamic challenges: 
\begin{itemize}
  \item RC delay variation due to interconnect scaling,
  \item Thermal coupling in 3D CFET and sequential integration,
  \item Stress-driven mobility and threshold voltage shifts,
  \item EMI/EMC noise affecting signal integrity.
\end{itemize}
SystemDK provides DTCO interfaces, but lacks runtime adaptability.
We propose extending it with \textbf{AITL (AI $\times$ Intelligent Loop)} 
to bring control feedback directly into SystemDK.

For related CFET modeling challenges and DTCO context, 
see \cite{yakimets2020cfet,irds2023}.  
Classical control foundations are based on \cite{franklin2015feedback}.

\section{Architecture: SystemDK with AITL}
\subsection{Control Loop Integration}
\begin{itemize}
  \item \textbf{PID:} compensates RC delay and thermal deviation.
  \item \textbf{FSM:} ensures safety by throttling under HOT/stress conditions.
  \item \textbf{LLM:} supervises policies, adapts gains, and redefines thresholds.
\end{itemize}

\subsection{Physical Domains}
\begin{itemize}
  \item \textbf{Thermal-aware device models}: compact RC network with feedback.
  \item \textbf{Stress-augmented RC models}: variability corrected online.
  \item \textbf{EMI disturbance injection}: evaluated at RTL/NoC simulation.
  \item \textbf{Co-simulation path}: MATLAB/Simulink $\to$ RTL testbench.
\end{itemize}

\section{Experimental Validation}
We prototyped a two-tier CFET thermal–RC plant with DVFS actuation.  
SystemDK 2025 was extended with AITL controllers.  
Simulation parameters:
\begin{itemize}
  \item $R_{via}=1\text{--}10~\Omega$, $C_{inter}=1\text{--}5$ fF,
  \item $P_{burst}=0.1\text{--}1.0$ W, coupling $k_c=0.3\text{--}0.9$,
  \item EMI injection: sinusoidal $f=10\text{--}200$ MHz.
\end{itemize}
Results show:
\begin{itemize}
  \item Delay deviation suppressed $>100\times$ vs no-control,
  \item Thermal overshoot $<3\times 10^{-5}\%$,
  \item Stress-induced delay shift compensated within $10^{-6}\%$,
  \item EMI-induced jitter reduced by 2 orders of magnitude.
\end{itemize}

\section{Discussion and Outlook}
\textbf{SystemDK with AITL} enables a new EDA paradigm:
\begin{itemize}
  \item From \emph{static sign-off} $\to$ \emph{dynamic runtime-aware design}.
  \item Guardbands can be reduced by embedding corrective loops.
  \item Reliability improves by co-optimizing delay, thermal, stress, and EMI.
\end{itemize}
Future extensions:
\begin{itemize}
  \item Embedding AITL controllers into commercial EDA flows,
  \item Extending compact models for stress and EMI-aware design,
  \item Integration with NoC-level traffic control,
  \item Coupling with microfluidic cooling for holistic thermal management.
\end{itemize}

\section*{Acknowledgment}
The author thanks the Project Design Hub community for discussions.

\bibliographystyle{IEEEtran}
\bibliography{systemdk_aitl2025}

\section*{Author Biography}
\noindent\textbf{Shinichi Samizo}
received the M.S. degree in Electrical and Electronic Engineering from Shinshu University, Japan.  
He worked at Seiko Epson Corporation as an engineer in semiconductor memory and mixed-signal device development, and contributed to inkjet MEMS actuators and PrecisionCore printhead technology.  
He is currently an independent semiconductor researcher focusing on process/device education, memory architecture, and AI system integration.\\[2pt]
\textbf{Contact:} \href{mailto:shin3t72@gmail.com}{shin3t72@gmail.com}, 
\href{https://github.com/Samizo-AITL}{Samizo-AITL}

\end{document}
