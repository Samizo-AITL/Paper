\documentclass[conference]{IEEEtran}

\usepackage{amsmath,amssymb}
\usepackage{graphicx}
\usepackage{subfig}
\usepackage{cite}
\usepackage{url}

\begin{document}

\title{SystemDK with AITL: Physics-Aware Runtime DTCO via PID, FSM, and LLM Integration}

\author{\IEEEauthorblockN{Shinichi Samizo}
\IEEEauthorblockA{Independent Semiconductor Researcher \\
Email: shin3t72@gmail.com}
}

\maketitle

\begin{abstract}
This paper introduces SystemDK with AITL,
a paradigm that extends traditional Design-Technology Co-Optimization (DTCO) 
by embedding \textit{control-theoretic loops} directly into EDA flows. 
Beyond static compact models, we integrate PID feedback, FSM guards, and LLM supervision to dynamically mitigate RC delay, thermal coupling, stress-induced variability, and EMI/EMC disturbances. 
In addition, FEM analysis (thermal, stress, EM) and S-parameter measurements are injected into synthesis, P\&R, and STA to ensure physics-aware closure. 
Proof-of-concept simulations demonstrate over 100$\times$ reduction in delay deviation, thermal overshoot below $3 \times 10^{-5}\%$, and EMI-induced jitter suppressed by two orders of magnitude. 
This framework enables runtime-aware DTCO, reducing guardbands while improving reliability across sub-2 nm nodes.
\end{abstract}

\section{Introduction}
Conventional EDA tools focus on static sign-off closure. 
However, scaling to CFET and 3D sequential integration introduces \textit{dynamic runtime effects}:
\begin{itemize}
    \item RC delay variation due to interconnect scaling,
    \item Vertical thermal coupling across stacked tiers,
    \item Stress-driven mobility and $V_{th}$ shifts,
    \item EMI/EMC noise degrading timing and signal integrity.
\end{itemize}

SystemDK provides DTCO interfaces, but lacks runtime adaptability. 
We propose \textbf{AITL (AI $\times$ Intelligent Loop)} integration to embed corrective feedback directly into SystemDK.

\section{Modeling}
The delay and thermal behavior of CFET interconnects are governed by resistive, capacitive, and thermal RC dynamics. 
Compact models are extended with stress-induced, EMI, and transmission disturbance terms.

\subsection{Delay and Thermal Models}
FO1 delay is:
\begin{equation}
T_{FO1} = (R_{wire}+R_{via})(C_{load}+C_{inter}),
\end{equation}
where $R_{via}$ dominates at scaled nodes due to aspect ratio.
Temperature dependence:
\begin{equation}
R(T) = R_0 \left(1 + \alpha(T-25^{\circ}C)\right),
\end{equation}
with $\alpha$ as TCR. Thermal dynamics:
\begin{equation}
C_{th}\frac{dT}{dt} = P \cdot R_{th} - (T - T_{amb}),
\end{equation}
where vertical coupling $k_c$ propagates heating.

\subsection{Stress and EMI Models}
Stress perturbs device parameters:
\begin{equation}
\Delta V_{th}(t) = \beta_{stress}\cdot \sigma(t), \quad \Delta \mu = -\gamma \cdot \sigma(t).
\end{equation}
EMI injection:
\begin{equation}
v_{emi}(t) = A\sin(2\pi ft), \quad f=10\text{--}200~\text{MHz}.
\end{equation}

\subsection{Network Analyzer Models}
Interconnect transmission is modeled by $S$-parameters:
\begin{equation}
H(f) = S_{21}(f), \quad f=1\text{--}40~\text{GHz}.
\end{equation}

\section{Control Architecture}
A three-layered controller (PID, FSM, LLM) is proposed:
\begin{itemize}
    \item \textbf{PID}: compensates delay deviations by adjusting DVFS knob $u$,
    \item \textbf{FSM}: enforces safety with $u_{max}$ bounds,
    \item \textbf{LLM}: supervises, adapts ($K_p,K_i$), redefines thresholds.
\end{itemize}

FSM+LLM supervision is synthesized into Verilog RTL, integrated into logic synthesis and P\&R with FEM/S-parameter feedback.

\begin{figure}[t]
    \centering
    \includegraphics[width=0.95\columnwidth]{fig1.pdf}
    \caption{Supervisory PID+FSM+LLM control architecture integrated with the EDA flow.}
\end{figure}

\section{Experimental Validation}
Two-tier CFET thermal–RC plant with DVFS actuation was prototyped.  
AITL controllers were integrated in SystemDK 2025.

\subsection{Setup}
\begin{itemize}
    \item $R_{via}=1$–$10~\Omega$, $C_{inter}=1$–$5$ fF,
    \item $P_{burst}=0.1$–$1.0$ W, $k_c=0.3$–$0.9$,
    \item EMI: 10–200 MHz sinusoidal,
    \item Co-sim: MATLAB/Simulink $\rightarrow$ RTL testbench.
\end{itemize}

\subsection{Results}
\begin{itemize}
    \item Delay deviation reduced $>$100$\times$ vs baseline,
    \item Thermal overshoot $<3 \times 10^{-5}\%$,
    \item Stress-induced drift compensated within $10^{-6}\%$,
    \item EMI jitter reduced 100$\times$ in NoC simulation.
\end{itemize}

\begin{figure}[t]
  \centering
  % (a)
  \includegraphics[width=0.9\columnwidth]{fig2a.pdf}\\[10pt]
  (a) Suppression vs. $k_c$ and $P_{burst}$ (FEM co-sim).\\[15pt]

  % (b)
  \includegraphics[width=0.9\columnwidth]{fig2b.pdf}\\[10pt]
  (b) Delay vs. time (No control / PID / PID+FSM+LLM).\\[15pt]

  % (c)
  \includegraphics[width=0.7\columnwidth]{fig2c.pdf}\\
  (c) EMI-induced jitter suppression (normalized RMS).

  \caption{Experimental results under AITL control (synthetic but representative).}
\end{figure}

\section{Related Work}
Yakimets \textit{et al.}~\cite{yakimets2020} studied CFET integration but lacked runtime adaptation.  
IRDS~\cite{irds2023} emphasized DTCO but with static flows.  
Control theory~\cite{franklin2015,anderson2007,khalil2002} provides foundation.  
EMI compliance follows IEC~\cite{iec2019}.  
Commercial tools (e.g., Synopsys PrimeTime, Cadence Tempus) focus on static sign-off, motivating runtime-aware extensions.

\section{Stability Analysis}
PID loop must satisfy:
\begin{equation}
K_p < \frac{2\zeta \omega_n}{G}, \quad K_i < \frac{\omega_n^2}{G},
\end{equation}
FSM bounds control effort $u \le u_{max}$.  
LLM adapts gains to maintain Lyapunov stability margins.

\section{Limitations}
\begin{itemize}
    \item Compact models omit parasitic 3D effects,
    \item EMI modeled as simple sinusoid,
    \item Hardware constraints may limit real-time LLM supervision.
\end{itemize}

\section{Discussion and Outlook}
SystemDK with AITL reframes EDA:
\begin{itemize}
    \item Static sign-off $\rightarrow$ dynamic runtime closure,
    \item Guardbands $\rightarrow$ adaptive loops,
    \item Reliability $\rightarrow$ cross-domain resilience (delay, thermal, stress, EMI).
\end{itemize}

Future work:
\begin{enumerate}
    \item Embed AITL into commercial EDA,
    \item Extend compact models (stress/EMI-aware),
    \item Integrate with NoC traffic controllers,
    \item Couple with microfluidic cooling for holistic DTCO,
    \item Package as educational framework (Edusemi).
\end{enumerate}

\section*{Acknowledgment}
The author thanks the Project Design Hub community for discussions.

\bibliographystyle{IEEEtran}
\begin{thebibliography}{1}

\bibitem{yakimets2020}
D.~Yakimets \textit{et al.}, ``Challenges for cfet (complementary fet) integration,'' 
in \textit{Proc. IEEE IEDM}, 2020, pp. 19.1.1--19.1.4.

\bibitem{irds2023}
IRDS, ``International roadmap for devices and systems (irds) 2023,'' 2023. [Online]. Available: \url{https://irds.ieee.org/roadmap-2023}

\bibitem{franklin2015}
G.~F. Franklin, J.~D. Powell, and A.~Emami-Naeini, \textit{Feedback Control of Dynamic Systems}, 7th ed. Pearson, 2015.

\bibitem{khalil2002}
H.~K. Khalil, \textit{Nonlinear Systems}. Prentice Hall, 2002.

\bibitem{anderson2007}
B.~D.~O. Anderson and J.~B. Moore, \textit{Optimal Control: Linear Quadratic Methods}. Dover, 2007.

\bibitem{iec2019}
IEC, ``Electromagnetic Compatibility (EMC) --- Part 4: Testing and Measurement Techniques,'' IEC Std. 61000-4, 2019.

\end{thebibliography}

\begin{IEEEbiography}{Shinichi Samizo}
received the M.S. degree in Electrical and Electronic Engineering from Shinshu University, Japan. He worked at Seiko Epson Corporation in semiconductor memory and mixed-signal device development, and contributed to inkjet MEMS actuators and PrecisionCore printhead technology. He is now an independent semiconductor researcher focusing on process/device education, memory architecture, and AI system integration. \\
\textbf{Contact:} shin3t72@gmail.com, Samizo-AITL
\end{IEEEbiography}

\end{document}
