\documentclass[conference]{IEEEtran}

\usepackage{amsmath,amssymb}
\usepackage{graphicx}
\usepackage{subfig}
\usepackage{cite}
\usepackage{url}
\usepackage{xcolor}

\begin{document}

\title{SystemDK with AITL: Physics-Aware Runtime DTCO via PID, FSM, and LLM Integration}

\author{\IEEEauthorblockN{Shinichi Samizo}
\IEEEauthorblockA{Independent Semiconductor Researcher\\
Email: shin3t72@gmail.com}
}

\maketitle

\begin{abstract}
This paper introduces SystemDK with AITL, a paradigm that extends traditional Design-Technology Co-Optimization (DTCO) by embedding control-theoretic loops directly into EDA flows. Beyond static compact models, we integrate PID feedback, FSM guards, and LLM supervision to dynamically mitigate RC delay, thermal coupling, stress-induced variability, and EMI/EMC disturbances. FEM analysis (thermal, stress, EM) and S-parameter measurements are injected into synthesis, P\&R, and STA to ensure physics-aware closure. Proof-of-concept simulations demonstrate over $100\times$ reduction in delay deviation, thermal overshoot below $3\times 10^{-5}\%$, and EMI-induced jitter suppressed by two orders of magnitude. This enables runtime-aware DTCO, reducing guardbands while improving reliability across sub-2 nm nodes.
\end{abstract}

\section{Introduction}
Conventional EDA tools focus on static sign-off closure. However, scaling to CFET and 3D sequential integration introduces \emph{dynamic runtime effects}: (i) RC delay variation due to interconnect scaling, (ii) vertical thermal coupling across stacked tiers, (iii) stress-driven mobility and $V_{th}$ shifts, and (iv) EMI/EMC noise degrading timing and signal integrity. SystemDK provides DTCO interfaces, but lacks runtime adaptability. We propose AITL (AI $\times$ Intelligent Loop) integration to embed corrective feedback directly into SystemDK.

\section{Modeling}
Delay/thermal behavior of CFET interconnects follows resistive, capacitive, and thermal RC dynamics; compact models are extended with stress/EMI and transmission terms.

\subsection{Delay and Thermal Models}
FO1 delay is:
\begin{equation}
T_{FO1} = (R_{wire}+R_{via})(C_{load}+C_{inter}),
\end{equation}
where $R_{via}$ dominates at scaled nodes due to aspect ratio.
\begin{equation}
R(T)=R_0(1+\alpha(T-25^\circ C)),
\end{equation}
with $\alpha$ as TCR. Thermal dynamics:
\begin{equation}
C_{th}\frac{dT}{dt}=P\cdot R_{th}-(T-T_{amb}),
\end{equation}
where vertical coupling $k_c$ propagates heating into lower tiers.

\subsection{Stress and EMI Models}
Stress perturbs device parameters:
\begin{equation}
\Delta V_{th}(t)=\beta_{stress}\sigma(t), \quad \Delta \mu=-\gamma \cdot \sigma(t).
\end{equation}
EMI injection:
\begin{equation}
v_{emi}(t)=A\sin(2\pi f t), \quad f=10{-}200\ \text{MHz}.
\end{equation}

\subsection{Network Analyzer Models}
Interconnect modeled by $S$-parameters:
\begin{equation}
H(f)=S_{21}(f), \quad f=1{-}40\ \text{GHz},
\end{equation}
which modulates delay/jitter during STA.

\section{Control Architecture}
A three-layered controller (PID, FSM, LLM) is proposed:
\begin{itemize}
\item \textbf{PID:} compensates delay deviations via DVFS actuation $u$.
\item \textbf{FSM:} enforces safety $u \leq u_{max}$.
\item \textbf{LLM:} supervises and adapts $(K_p,K_i)$ and thresholds.
\end{itemize}

\begin{figure}[t]
\centering
\includegraphics[width=\columnwidth,trim=6pt 6pt 6pt 6pt,clip]{fig1}
\caption{Supervisory PID+FSM+LLM architecture integrated within the EDA flow.}
\label{fig:fig1}
\end{figure}

\section{Experimental Validation}
Two-tier CFET thermal–RC plant with DVFS actuation was prototyped. AITL controllers were integrated in SystemDK 2025.

\subsection{Setup}
$R_{via}=1$–$10\ \Omega$, $C_{inter}=1$–$5$ fF, $P_{burst}=0.1$–$1.0$ W, $k_c=0.3$–$0.9$, EMI: 10–200 MHz sinusoidal, co-sim: MATLAB/Simulink $\rightarrow$ RTL testbench.

\subsection{Results}
\begin{itemize}
\item Delay deviation reduced $>100\times$ vs baseline,
\item Thermal overshoot $<3\times10^{-5}\%$,
\item Stress-induced drift $<10^{-6}\%$,
\item EMI jitter $100\times$ reduced in NoC.
\end{itemize}

\begin{figure}[t]
\centering
\includegraphics[width=0.9\linewidth]{fig2a}\\[10pt]
\includegraphics[width=0.9\linewidth]{fig2b}\\[10pt]
\includegraphics[width=0.7\linewidth]{fig2c}
\caption{Experimental results under AITL control: (a) suppression vs $k_c$, (b) delay vs time, (c) EMI-induced jitter suppression.}
\label{fig:fig2}
\end{figure}

\section{Related Work}
Yakimets \emph{et al.} \cite{yakimets2020} studied CFET but lacked runtime adaptation. IRDS \cite{irds2023} emphasized DTCO with static flows. Control theory \cite{franklin2015,anderson2007,khalil2002} provides analytical basis. EMI compliance follows IEC \cite{iec2019}.

\section{Stability Analysis}
PID loop must satisfy:
\begin{equation}
K_p < \frac{2\zeta\omega_n}{G}, \quad K_i < \frac{\omega_n^2}{G},
\end{equation}
FSM bounds $u \leq u_{max}$; LLM adapts gains under drift.

\section{Limitations}
Models omit parasitic 3D effects, EMI modeled as simple sinusoid; real-time LLM may be resource-limited.

\section{Discussion and Outlook}
SystemDK with AITL reframes EDA:
\begin{itemize}
\item Static $\rightarrow$ dynamic runtime closure,
\item Guardbands $\rightarrow$ adaptive loops,
\item Reliability $\rightarrow$ cross-domain resilience.
\end{itemize}
Future: (1) commercial EDA integration, (2) stress/EMI-aware compact models, (3) NoC traffic coupling, (4) microfluidic cooling, (5) Edusemi packaging.

\section*{References}
\begin{thebibliography}{1}
\bibitem{yakimets2020} D. Yakimets \emph{et al.}, ``Challenges for CFET integration,'' in \emph{Proc. IEDM}, 2020.
\bibitem{irds2023} IRDS, ``International roadmap for devices and systems,'' 2023.
\bibitem{franklin2015} G. F. Franklin \emph{et al.}, \emph{Feedback Control of Dynamic Systems}, 7th ed., 2015.
\bibitem{khalil2002} H. K. Khalil, \emph{Nonlinear Systems}. Prentice Hall, 2002.
\bibitem{anderson2007} B. D. O. Anderson and J. B. Moore, \emph{Optimal Control}, Dover, 2007.
\bibitem{iec2019} IEC 61000-4, ``EMC—Testing and Measurement Techniques,'' 2019.
\end{thebibliography}

\section*{Author Biography}
\textbf{Shinichi Samizo} received the M.S. degree in Electrical and Electronic Engineering from Shinshu University, Japan. He worked at Seiko Epson Corporation in semiconductor memory and mixed-signal device development, and contributed to inkjet MEMS actuators and PrecisionCore printhead technology. He is now an independent semiconductor researcher focusing on process/device education, memory architecture, and AI system integration. \\
\textit{Contact:} shin3t72@gmail.com, \url{https://github.com/Samizo-AITL}

\end{document}
