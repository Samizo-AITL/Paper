\section{Introduction}
Inkjet printing has become a versatile technology in both industrial and
consumer applications, with piezoelectric lead zirconate titanate (PZT)
actuators serving as the dominant driver mechanism.
Despite extensive research on lead-free piezoelectrics such as
(K,Na)NbO$_3$ (KNN) and Bi$_{0.5}$Na$_{0.5}$TiO$_3$ (BNT), these
alternatives have not replaced PZT in mainstream printing due to
insufficient compatibility in key metrics such as piezoelectric
coefficient $d_{33}$, long-term durability, and cost-effective
manufacturing.

In biomedical applications, however, the design requirements differ
fundamentally.
Rather than extreme durability and maximum actuation strength,
biocompatibility, chemical safety, and moderate performance are the
primary needs.
Typical bio-printing tasks---including cell patterning, protein array
generation, and hydrogel dispensing---require droplet volumes in the
range of 1--10 pL and endurance on the order of $10^6$ shots, which are
significantly less demanding than industrial printing standards.

This work therefore explores a Bio-Inkjet (Bio-IJ) architecture based on
bulk KNN actuators, combined with chip-on-film (COF) driver integration
and silicon cavity bonding.
By tailoring the system to biomedical rather than industrial
requirements, the proposed approach provides a feasible lead-free
alternative that aligns with environmental regulations and biological
safety.
