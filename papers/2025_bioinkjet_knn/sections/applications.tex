\section{Applications}
The proposed Bio-Inkjet architecture enables several key biomedical
applications in which moderate actuation performance, precise droplet
control, and biocompatibility are prioritized over extreme durability:
\begin{itemize}
  \item \textbf{Cell patterning}: 
        Controlled deposition of living cells into predefined patterns
        for tissue engineering and regenerative medicine.
        Gentle actuation and droplet volumes of 1--10~\si{\pico\liter}
        support high cell viability, with survival rates above 80\%
        reported under comparable shear stress conditions.
  \item \textbf{Protein and DNA microarrays}: 
        Picoliter-scale dispensing of biomolecules onto functionalized
        substrates for high-throughput screening, diagnostics, and
        drug discovery.
        The ability to generate uniform, sub-100~\si{\micro\meter}
        spots is critical for assay reproducibility.
  \item \textbf{Hydrogel 3D printing}: 
        Layer-by-layer deposition of bio-compatible hydrogels, followed
        by UV or thermal curing, to fabricate soft scaffolds for cell
        culture and organ-on-chip platforms.
        Precise droplet placement ensures structural fidelity and
        material homogeneity.
\end{itemize}

These use cases demonstrate that the moderate performance of bulk KNN
actuators---picoliter droplet generation at voltages below
$\pm 50$~V---is sufficient to meet biomedical requirements.
Here, droplet volume control, biocompatibility, and integration with
fluidic handling systems are far more critical than the billion-cycle
endurance demanded in industrial printing.
