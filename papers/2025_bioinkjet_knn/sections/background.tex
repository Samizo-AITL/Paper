\section{Background: Pb-free Piezoelectrics}
Research on lead-free piezoelectrics has been pursued for more than two
decades as part of global efforts to replace lead zirconate titanate
(PZT) with environmentally benign alternatives.
Among the most prominent candidates are (K,Na)NbO$_3$ (KNN),
Bi$_{0.5}$Na$_{0.5}$TiO$_3$ (BNT), and Sc-doped AlN (ScAlN).

KNN typically exhibits a piezoelectric coefficient $d_{33}$ in the range
of 150--250~pm/V, with optimized compositions reaching up to
300--400~pm/V, and a Curie temperature exceeding 400~\si{\celsius}.
These properties make it suitable for moderate-displacement actuators.
In contrast, ScAlN provides a more modest $d_{33}$ of
20--30~pm/V, but its excellent CMOS compatibility and feasibility for
thin-film MEMS integration have positioned it as an attractive material
for micro-scale devices.
BNT and related systems have also been investigated, though issues such
as depolarization and processing complexity have limited their adoption.

Despite these advances, industrial printing applications have not
embraced Pb-free piezoelectrics.
The primary obstacle has been the expectation of \textit{full PZT
compatibility}---requiring identical performance in terms of strain
output, durability over $10^9$ cycles, and manufacturing cost.
Such stringent requirements exceed the current capabilities of KNN,
BNT, or ScAlN.

In biomedical printing, however, these limitations are less critical.
Because the performance demands are moderate and biocompatibility is a
non-negotiable requirement, Pb-free piezoelectrics can be reconsidered
as practical actuator materials for Bio-Inkjet systems.
