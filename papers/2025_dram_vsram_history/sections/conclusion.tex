\section{Conclusion}

The 0.25-\si{\micro\meter} DRAM ramp at Epson's Sakata fab demonstrated how DRAM can function as a \textbf{strategic vehicle} to internalize advanced process technologies and redeploy them to core products. A data-driven cycle—observation, hypothesis, countermeasure, and verification—eliminated a subtle plasma-damage-induced junction leakage and stabilized yield ($\sim$65\% $\rightarrow$ $\sim$80\%).

Built on that foundation, the 0.25-\si{\micro\meter} VSRAM was mass-produced in 2001 and enabled the first camera phone, even though it initially shipped at $\sim$30\% yield before improving to 80–90\%. At 0.18~\si{\micro\meter}, a trench-capacitor VSRAM was abandoned due to larger junction area and insufficient 90~\si{\celsius} retention, marking practical limits of 1T–1C-based pseudo-SRAM for mobile. 

Strategically, stepping off commodity memory competition while retaining the acquired process/design/reliability know-how strengthened Epson's display driver, ASIC, and logic businesses. This case offers a concrete template for technology-transfer ramp-ups that prioritize \emph{capability acquisition and reuse} over direct DRAM commercialization.
