% ---------- Conclusion ----------
\section{Conclusion}
The 0.25-\si{\micro\meter} DRAM ramp at Epson’s Sakata fab demonstrated how DRAM can function as a strategic vehicle to internalize advanced process technologies and redeploy them into core businesses rather than as a direct product line. Rapid SCF-based ramp-up, defect analysis, yield improvement, and verification eliminated a subtle plasma-damage–induced junction leakage and stabilized yield ($\sim$65\% $\rightarrow$ $\sim$80\%). 

On that basis, the 0.25-\si{\micro\meter} VSRAM was mass-produced in 2001 and enabled the world’s first camera-equipped mobile phone. Although launched at only $\sim$30\% yield, market entry was prioritized, and subsequent countermeasures improved production yield to 80--90\% while sustaining reliability. At 0.18-\si{\micro\meter}, trench-capacitor VSRAM was abandoned because of excessive junction leakage and insufficient 90~°C retention, exposing the inherent limits of 1T-1C pseudo-SRAM for mobile applications. 

Strategically, exiting commodity DRAM competition while retaining acquired process/design/reliability know-how strengthened Epson’s display driver, ASIC, and logic businesses. This case thus provides a concrete template for technology-transfer ramp-ups that prioritize capability acquisition and reuse over direct DRAM commercialization, and it remains a valuable archive and educational resource for modern process engineers.
