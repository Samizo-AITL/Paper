\section{Failure Analysis and Yield Improvement}

\subsection{Initial Observation}
The first production lots showed $\sim$65\% yield. Wafer test was dominated by \textbf{Pause Refresh Fail (Bin5)}. Defects appeared as uniformly scattered single-bit errors across the wafer (weak clustering, no edge/line signature). Storage-node capacitance met spec; SEM cross-sections at failed cells revealed no structural anomaly. Other CDs/films/electricals were within spec.

\subsection{Hypothesis (Failure Model)}
Directly measurable leakages were normal, suggesting a subtle leakage path. We hypothesized increased leakage at the \textbf{storage-node contact $n^+/p^-$ junction}. After gate etch, a remnant gate oxide on S/D active is repeatedly exposed to resist-stripping \emph{ashing} during multiple LDD steps. Cumulative plasma damage makes the oxide locally porous and can extend damage into the diffusion, creating minute leakage paths. This explains random single-bit distribution without visible structural defects.

\subsection{Countermeasures}
\begin{itemize}
  \item \textbf{Process}: Replace resist stripping in LDD steps from plasma ashing to \textbf{wet stripping (sulfuric-based)} to eliminate plasma damage. 
  \item \textbf{Integration hygiene}: Confirm downstream photo cleanliness and avoid residue risks with the wet strip.
\end{itemize}

\subsection{Effectiveness}
Yield improved from $\sim$65\% to \textbf{$\sim$80\%}. Uniformly scattered single-bit fails decreased markedly. Burn-in and retention/reliability passed; the final recipe was fixed for volume production.

% (Optional) If you want to plot yield-by-lot from CSV:
\begin{figure}[t]
\centering
\pgfplotstableread[col sep=comma]{data/yield_lot.csv}\yieldtbl
\begin{tikzpicture}
\begin{axis}[
  width=\columnwidth, height=0.58\columnwidth,
  xlabel={Lot Index}, ylabel={Yield [\%]},
  ymin=0, ymax=100, grid=both, mark=*]
\addplot table[x=lot,y=yield]{\yieldtbl};
\end{axis}
\end{tikzpicture}
\caption{Yield evolution during the 0.25-\si{\micro\meter} ramp.}
\label{fig:yield}
\end{figure}
