\begin{abstract}
In 1998, a 3rd-generation 64-Mbit DRAM was transferred and ramped on a 0.25-\textmu m process using short-cycle feedback (SCF) and a parallel introduction of production and margin lots. The initial yield was $\sim$65\%, dominated by retention-related failures under pause-refresh (Bin-5) and disturb-refresh (Bin-6) tests. Process tracing indicated cumulative plasma damage from resist ashing after WSA-ET and multiple LDD steps as a primary root cause. By shifting resist stripping to wet processes and strengthening body back-bias, the yield improved to $\sim$80\% and passed long-term reliability. We position this as a practical ramp-up design framework and discuss its continuity to modern phenomena (e.g., row-disturb/row-hammer) and its educational value.
\end{abstract}
