\documentclass[conference]{IEEEtran}
\usepackage{amsmath, amssymb, graphicx, cite, hyperref}

% ---------- Title ----------
\title{SkyEdge: Secure High-Altitude Drone Platform Integrating $H_\infty$ Control, Domestic Devices, and Advanced Mechanical Design}

% ---------- Author ----------
\author{
\IEEEauthorblockN{Shinichi Samizo}
\IEEEauthorblockA{Independent Semiconductor Researcher \\
Project Design Hub, Samizo-AITL \\
\textit{Email:} \href{mailto:shin3t72@gmail.com}{shin3t72@gmail.com} \\
\textit{GitHub:} \href{https://github.com/Samizo-AITL}{Samizo-AITL}}
}

\begin{document}
\maketitle

% ---------- Abstract ----------
\begin{abstract}
This paper presents the foundational design of \emph{SkyEdge}, 
a secure high-altitude unmanned aerial vehicle (UAV) platform that 
integrates $H_\infty$ control, domestic devices, and a variable-pitch 
mechanical structure. The proposed framework provides robust disturbance 
rejection, secure hardware implementation, and reliable flight capability 
up to 10,000 m altitude. We describe the control architecture, device 
integration, and mechanical design, and we outline evaluation plans 
toward a proof-of-concept prototype.
\end{abstract}

% ---------- Keywords ----------
\begin{IEEEkeywords}
UAV, robust control, $H_\infty$, variable-pitch rotor, secure systems, high-altitude flight
\end{IEEEkeywords}

% ---------- Sections ----------
\section{Introduction}
Unmanned aerial vehicles (UAVs) are increasingly important in defense, 
disaster response, and environmental monitoring. However, most commercial 
systems are limited to altitudes below 3,000 m, and many rely on 
foreign-made devices with security concerns. This paper aims to establish 
a domestic, secure UAV platform capable of robust operation in 
high-altitude environments.

\section{Related Work}
Control approaches such as PID, adaptive control, and sliding-mode 
control have been applied to UAVs, but robust performance under strong 
disturbances remains challenging. $H_\infty$ control has potential for 
disturbance rejection. Existing high-altitude UAV programs (NASA Helios, 
JAXA HAPS) demonstrate feasibility but rely on specialized designs. 
Security aspects and domestic device integration remain underexplored.

\section{System Architecture}
The proposed architecture integrates three layers:
\begin{itemize}
    \item \textbf{$H_\infty$ Control:} ensures robustness against gusts 
    up to 20--30 m/s.
    \item \textbf{FSM:} manages mode transitions (normal, high-altitude, 
    comm-loss, emergency return).
    \item \textbf{LLM:} assists adaptive redesign of control policies in 
    unforeseen conditions (simulation environment).
\end{itemize}

\section{Device Integration}
The device layer includes a domestic SoC (65 nm FDSOI), LDMOS motor 
drivers, high-rate IMU (1 kHz), and secure modules (TPM + PQC). 
Control cycle is maintained at $\leq 1.0$ ms with ESC response $\leq 100 
\, \mu$s. Estimated BOM cost is 596,700 JPY per unit (with reduction in 
mass production).

\section{Mechanical Design}
The UAV has a 700--900 mm frame, CFRP structure, and a variable-pitch 
20-inch rotor system. At take-off weight 6.38 kg, the thrust-to-weight 
ratio is $\approx 2.82$. Variable pitch enables adaptation from sea 
level to 10,000 m (RPM from $\approx 8,339$ to $14,353$). Servo torque 
requirement is 0.62 N$\cdot$m with safety factor of 2--3.

\section{Evaluation Plan}
Planned evaluations include wind-tunnel testing, low-temperature chamber 
tests, redundancy verification, and communication robustness under 
jamming scenarios. A proof-of-concept schedule has been drafted for 
2025--2026.

\section{Conclusion}
We presented the SkyEdge architecture integrating $H_\infty$ control, 
domestic devices, and mechanical design for secure high-altitude UAVs. 
The proposed system addresses robustness, security, and reliability at 
10,000 m operation. Future work includes prototype development and 
extension to underwater vehicles (\emph{SeaEdge}), toward a unified 
autonomous platform for defense, disaster response, GX, and education.

% ---------- References ----------
\bibliographystyle{IEEEtran}
\bibliography{refs}

\end{document}
