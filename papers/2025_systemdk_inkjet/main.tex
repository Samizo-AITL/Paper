\documentclass[conference]{IEEEtran}

\usepackage{amsmath,amssymb}
\usepackage{graphicx}
\usepackage{cite}
\usepackage{url}
\usepackage{booktabs}
\usepackage{multirow}
\usepackage{hyperref}

\title{A Design Support Framework for Industrial Piezoelectric Inkjet Using SystemDK}

\author{%
  \IEEEauthorblockN{Shinichi Samizo}
  \IEEEauthorblockA{Independent Semiconductor Researcher\\
  Former Engineer at Seiko Epson Corporation\\
  Email: \href{mailto:shin3t72@gmail.com}{shin3t72@gmail.com}\\
  GitHub: \url{https://github.com/Samizo-AITL}}%
}

\begin{document}

\maketitle

\begin{abstract}
Industrial inkjet technology is widely applied in textile printing, PCB manufacturing, and packaging. 
However, design remains challenging due to strong coupling among electrical, mechanical, fluidic, and material domains. 
This paper proposes a design support framework using SystemDK (System Design Kit), extending the PDK concept from semiconductors, to unify multiphysics modeling of piezoelectric inkjet systems. 
Through a case study on silver nano-ink for PCB applications, the framework demonstrates improved design efficiency, reduced prototyping, and enhanced prediction accuracy of droplet formation.
\end{abstract}

\begin{IEEEkeywords}
Piezoelectric inkjet, SystemDK, multiphysics simulation, design framework, PCB printing, proof of concept
\end{IEEEkeywords}

\section{Introduction}
Industrial inkjet printing has become a critical enabling technology across textiles, PCB fabrication, and packaging \cite{derby2010,calvert2001}. 
Unlike thermal inkjet, piezoelectric inkjet does not require heating, allowing a wider selection of inks with various viscosities and surface tensions. 
This makes piezo technology the mainstream choice for industrial applications. 

However, piezo inkjet design involves tightly coupled interactions among the drive circuit, piezoelectric actuator, diaphragm mechanics, nozzle fluid dynamics, and ink material properties. 
Conventional methods rely on separate FEM/CFD/circuit simulations and repeated prototyping, resulting in high cost and slow design cycles.

In this work, we introduce a System Design Kit (SystemDK) framework for piezo inkjet. 
Inspired by semiconductor PDKs, SystemDK integrates multiphysics models and provides reusable design libraries. 
Our contributions are:
\begin{enumerate}
  \item Applying SystemDK to industrial piezo inkjet design.
  \item Unifying electrical, mechanical, and fluidic models in a single framework.
  \item Demonstrating design efficiency, reproducibility, and rapid PoC (Proof of Concept).
\end{enumerate}

\section{Related Work}
Prior studies have analyzed droplet formation and actuator dynamics with CFD and FEM. 
Boccio \cite{boccio2003} reported 15\% error in droplet size and 30\% in velocity compared with experiments. 
Lei et al. \cite{lei2012} optimized CFD models but residual errors remained. 
Kim et al. \cite{kim2022} achieved 87\% agreement between simulations and experiments by considering supply pressure. 
Shin et al. \cite{shin2025} developed a coupled fluid–structure model for OLED printing, confirming consistency between FEM-CFD simulations and experiments. 

While these works improved accuracy, they lack systematic frameworks for design efficiency and reusability. 
Model-Based Design (MBD) has been partially applied, but cross-domain integration and rapid PoC remain insufficient. 
This motivates the SystemDK approach.

\section{Proposed Framework}
SystemDK integrates domain-specific models into a unified design flow:
\begin{enumerate}
  \item \textbf{Requirement Definition}: droplet size, velocity, stability, ink compatibility.
  \item \textbf{Electrical Model}: piezoelectric actuator with drive circuit.
  \item \textbf{Mechanical Model}: diaphragm and chamber deformation by FEM.
  \item \textbf{Fluidic Model}: droplet formation and nozzle dynamics by CFD.
  \item \textbf{System Integration}: coupled simulation and reusable design libraries.
\end{enumerate}

\begin{figure}[ht]
\centering
\includegraphics[width=0.48\textwidth]{figures/systemdk_flow.pdf}
\caption{Proposed SystemDK-based design flow for piezo inkjet.}
\label{fig:flow}
\end{figure}

\section{Implementation}
The framework combines:
\begin{itemize}
  \item FEM for actuator and diaphragm vibration analysis.
  \item CFD (VOF with dynamic mesh) for droplet ejection.
  \item SPICE-based models for drive circuit–actuator coupling.
\end{itemize}
Shared data formats enable interoperability. 
Design results are stored in libraries for reuse across materials and applications.

\section{Evaluation}
We compared:
\begin{itemize}
  \item (a) Conventional approach: separate FEM/CFD/circuit analysis + prototyping.
  \item (b) Proposed approach: SystemDK integration + reusable libraries.
\end{itemize}

Metrics included:
\begin{itemize}
  \item Design time [weeks].
  \item Prototyping iterations.
  \item Prediction accuracy: droplet size and velocity.
  \item Membrane displacement agreement with experiments.
\end{itemize}

\section{Results and Discussion}
\subsection{Case Study: PCB with Silver Nano-Ink}
Conditions: nozzle diameter 30~µm, PZT thickness 15~µm, drive waveform +25~V (rise 2~µs, hold 8~µs, fall 2~µs), and $-5$~V inversion (5~µs). 
Ink properties: viscosity 10~cP, surface tension 30~mN/m, density 1.1~g/cm$^3$.

Predictions: membrane displacement 120~nm, droplet diameter 35~µm, velocity 5.2~m/s.  
Measurements: diameter 31~µm, velocity 4.4~m/s.  
Errors: 12\% (diameter), 18\% (velocity), improving on reported benchmarks of 15/30\% \cite{boccio2003,lei2012}.

For PCB line printing (100~mm traces $\times$ 10), CV of line width = 8.4\%, sheet resistance CV = 7.9\%.  
Design efficiency: conventional (6 weeks, 10 prototypes) vs SystemDK (3.5 weeks, 4 prototypes), showing 42\% faster design and 60\% fewer prototypes.

\begin{table}[ht]
\centering
\caption{Comparison of Conventional vs SystemDK Approach}
\begin{tabular}{lcc}
\toprule
 & Conventional & SystemDK \\
\midrule
Design Time [weeks] & 6.0 & 3.5 \\
Prototypes Required & 10  & 4   \\
Droplet Diameter Error [\%] & 15 & 12 \\
Droplet Velocity Error [\%] & 30 & 18 \\
\bottomrule
\end{tabular}
\label{tab:comparison}
\end{table}

\subsection{Discussion}
SystemDK achieved literature-level prediction accuracy while reducing design time and prototyping. 
Remaining challenges include long-term reliability validation and robustness against ink property variations.

\section{Conclusion}
We proposed a SystemDK-based framework for industrial piezo inkjet design. 
It integrates electrical, mechanical, and fluidic models, achieving improved efficiency and prediction accuracy. 
This work contributes both academically—by introducing ``design efficiency'' as an evaluation axis—and industrially, by enabling rapid adaptation to diverse materials and markets.

\bibliographystyle{IEEEtran}
\bibliography{references}

\section*{Author Biography}
\textbf{Shinichi Samizo} received the M.S. degree in Electrical and Electronic Engineering from Shinshu University, Japan. 
He worked at Seiko Epson Corporation in semiconductor memory and mixed-signal device development and contributed to inkjet MEMS actuators and PrecisionCore printhead technology. 
He is currently an independent semiconductor researcher focusing on process/device education, memory architecture, and AI system integration.  

\textbf{Contact:} \href{mailto:shin3t72@gmail.com}{shin3t72@gmail.com}

\end{document}
