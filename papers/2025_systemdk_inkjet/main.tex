\documentclass[conference]{IEEEtran}

\usepackage{amsmath,amssymb}
\usepackage{graphicx}
\usepackage{cite}
\usepackage{url}
\usepackage{booktabs}
\usepackage{multirow}
\usepackage{hyperref}

\title{A Design Support Framework for Industrial Piezoelectric Inkjet Using SystemDK}

\author{%
  \IEEEauthorblockN{Shinichi Samizo}
  \IEEEauthorblockA{Independent Semiconductor Researcher\\
  Former Engineer at Seiko Epson Corporation\\
  Email: \href{mailto:shin3t72@gmail.com}{shin3t72@gmail.com}\\
  GitHub: \url{https://github.com/Samizo-AITL}}%
}

\begin{document}

\maketitle

\begin{abstract}
Industrial inkjet technology is widely applied in textile printing, PCB manufacturing, and packaging. 
However, design remains challenging due to strong coupling among electrical, mechanical, fluidic, and material domains. 
This paper proposes a design support framework using SystemDK (System Design Kit), extending the PDK concept from semiconductors, to unify multiphysics modeling of piezoelectric inkjet systems. 
Through a case study on silver nano-ink for PCB applications, the framework demonstrates improved design efficiency, reduced prototyping, and enhanced prediction accuracy of droplet formation.
\end{abstract}

\begin{IEEEkeywords}
Piezoelectric inkjet, SystemDK, multiphysics simulation, design framework, PCB printing, proof of concept
\end{IEEEkeywords}

\section{Introduction}
Industrial inkjet printing has become a critical enabling technology across textiles, PCB fabrication, and packaging~\cite{derby2010,calvert2001}. 
Unlike thermal inkjet, piezoelectric inkjet does not require heating, allowing a wider selection of inks with various viscosities and surface tensions. 
This makes piezo technology the mainstream choice for industrial applications. 

\section{Related Work}
Boccio~\cite{boccio2003} reported 15\% error in droplet size and 30\% in velocity compared with experiments. 
Lei et al.~\cite{lei2012} optimized CFD models but residual errors remained. 
Kim et al.~\cite{kim2022} achieved 87\% agreement between simulations and experiments by considering supply pressure. 
Shin et al.~\cite{shin2025} developed a coupled fluid--structure model for OLED printing. 

\section{Proposed Framework}
SystemDK integrates domain-specific models into a unified design flow.  
% (図を後で追加)
%\begin{figure}[ht]
%\centering
%\includegraphics[width=0.48\textwidth]{figures/systemdk_flow.pdf}
%\caption{Proposed SystemDK-based design flow for piezo inkjet.}
%\end{figure}

\section{Results and Discussion}
Case study: PCB with silver nano-ink.  
Predicted droplet diameter 35 µm vs measured 31 µm.  
Velocity 5.2 m/s vs measured 4.4 m/s.  

\section{Conclusion}
We proposed a SystemDK-based framework for industrial piezo inkjet design.  

\begin{thebibliography}{99}
\bibitem{boccio2003} J. Boccio, ``Computational fluid dynamics study of droplet formation in a piezo inkjet printhead,'' Rochester Institute of Technology, 2003.
\bibitem{kim2022} S. Kim et al., ``The Effect of Ink Supply Pressure on Piezoelectric Inkjet,'' Micromachines, vol. 13, no. 4, p. 615, 2022.
\bibitem{lei2012} T. Lei et al., ``Numerical Analysis and Optimal CFD Model Verification of Piezoelectric Inkjet Printhead,'' J. Appl. Fluid Mech., 2012.
\bibitem{shin2025} D. Y. Shin et al., ``Simulation of OLED-based inkjet printing using a piezoelectric fluid-structure interaction model,'' Sci. Rep., 2025.
\bibitem{derby2010} B. Derby, ``Inkjet Printing of Functional and Structural Materials,'' Annu. Rev. Mater. Res., vol. 40, 2010.
\bibitem{calvert2001} P. Calvert, ``Inkjet Printing for Materials and Devices,'' Chem. Mater., vol. 13, no. 10, 2001.
\end{thebibliography}

\section*{Author Biography}
\textbf{Shinichi Samizo} received the M.S. degree in Electrical and Electronic Engineering from Shinshu University, Japan. 
He worked at Seiko Epson Corporation in semiconductor memory and mixed-signal device development and contributed to inkjet MEMS actuators and PrecisionCore printhead technology. 
He is currently an independent semiconductor researcher focusing on process/device education, memory architecture, and AI system integration.  

\textbf{Contact:} \href{mailto:shin3t72@gmail.com}{shin3t72@gmail.com}

\end{document}
