% ==== Fig. 3: Package cross-section (LPDDR+FeRAM Chiplet Integration) ====
\begin{figure}[t]
  \centering
  % 2段組の列幅に必ず収める
  \resizebox{\columnwidth}{!}{%
  \begin{tikzpicture}[node distance=8mm, font=\scriptsize, >=latex]

    % ---- 上段:CPU/Controller ----
    \node[draw, thick, fill=gray!15,
          minimum width=0.38\columnwidth, minimum height=9mm] (cpu) {CPU / Controller};

    % ---- 下段:LPDDR ----
    \node[draw, thick, fill=gray!15,
          minimum width=0.38\columnwidth, minimum height=9mm,
          below=of cpu] (lpddr) {LPDDR DRAM Chip};

    % ---- 右:FeRAM chiplet ----
    \node[draw, thick, fill=gray!15,
          minimum width=0.32\columnwidth, minimum height=9mm,
          right=13mm of lpddr.center, anchor=center] (feram) {FeRAM Chiplet};

    % ---- SystemDK supervision (fitで3ブロックを囲う) ----
    \node[draw, thick, dashed, rounded corners,
          inner sep=5mm, fit=(cpu) (lpddr) (feram), name=systemdk] {};

    \node[font=\scriptsize\itshape, anchor=north west, text=black]
      at ([xshift=1mm,yshift=1mm]systemdk.north west) {SystemDK Co-Design Supervision};

    % ---- 相互接続矢印 ----
    \draw[->, thick] (cpu.south) -- (lpddr.north) node[midway, left] {high bandwidth};
    \draw[->, thick] (cpu.east) -- (feram.west) node[midway, above] {checkpoint / state};
    \draw[<->, thick, dashed] (lpddr.east) -- ([yshift=-1mm]feram.south west)
      node[midway, below] {refresh offloading};

    % ---- Substrate / Interposer ----
    \node[draw, thick, fill=gray!10,
          minimum width=0.95\columnwidth, minimum height=4mm,
          below=10mm of systemdk.south] (substrate) {};
    \node[align=center] at (substrate)
      {Package Substrate / Interposer (SiP/PoP)};

  \end{tikzpicture}%
  }
  \vspace{-0.6ex}
  \caption{Package-level integration of LPDDR and FeRAM chiplet under SystemDK supervision.
  LPDDR serves as main working memory, while FeRAM provides checkpointing and refresh suppression via a common substrate (SiP/PoP).}
  \label{fig:package_lpddr_feram}
\end{figure}
