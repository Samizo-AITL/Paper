% ==== Fig. 1: Conceptual overview of LPDDR+FeRAM for Mobile Edge AI ====
\begin{figure}[t]
  \centering
  % 列幅に合わせて全体をスケール
  \begin{adjustbox}{width=\columnwidth,center}
  \begin{tikzpicture}[node distance=1.35cm, font=\small, >=Stealth]

    % Mobile Edge AI Device(列幅に対して安全な余白)
    \node[draw, thick, rounded corners, fill=gray!15,
          minimum width=0.94\linewidth, minimum height=4.2cm] (device) {};

    % SoC block
    \node[draw, thick, fill=blue!10, minimum width=3.2cm, minimum height=1.0cm]
         (soc) at (device.center) {SoC / AI Accelerator};

    % LPDDR
    \node[draw, thick, fill=cyan!20, minimum width=2.8cm, minimum height=0.9cm,
          above=1.1cm of soc] (lpddr) {LPDDR (main memory)};

    % FeRAM
    \node[draw, thick, fill=green!20, minimum width=2.6cm, minimum height=0.9cm,
          right=2.2cm of soc] (feram) {FeRAM Chiplet};

    % SystemDK supervision(ラベルの配置を安全側に)
    \node[draw, thick, dashed, red!60!black, rounded corners,
          fit=(soc) (lpddr) (feram), inner sep=6pt] (sysdk) {};
    \node[anchor=south west, font=\small\itshape, red!60!black]
         at (sysdk.north west) {SystemDK Co-Design Supervision};

    % Arrows
    \draw[->, thick] (soc.north) -- (lpddr.south) node[midway,left] {high BW};
    \draw[->, thick] (soc.east)  -- (feram.west) node[midway,above] {checkpoint / state};
    \draw[<->, thick, dashed] (lpddr.east) -- (feram.north west)
         node[midway,below] {refresh offload};

    % Label for device
    \node[below=1.9cm of soc, font=\small\bfseries] {Mobile Edge AI Device};

  \end{tikzpicture}
  \end{adjustbox}
  \vspace{-0.6ex}
  \caption{High-level concept of LPDDR+FeRAM integration for mobile edge AI.}
  \label{fig:concept_lpddr_feram}
\end{figure}
