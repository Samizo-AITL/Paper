% =========================
% sections/introduction.tex
% =========================
\section{Introduction}

Mobile edge AI platforms such as smartphones, wearables, and embedded accelerators require memory subsystems that balance
\emph{bandwidth}, \emph{energy efficiency}, and \emph{responsiveness}.
Low-power DRAM (LPDDR) has become the de facto main memory for these devices, delivering tens to hundreds of~GB/s bandwidth
with lower I/O energy than server-class high-bandwidth memory (HBM)~\cite{ChoiIEDM2022}.
Nevertheless, LPDDR remains \emph{volatile} and depends on periodic refresh, which incurs standby-power overhead
and constrains energy efficiency in always-connected modes.

Non-volatile memories (NVMs) such as ReRAM, MRAM, and FeRAM have been explored as replacements or complements to DRAM~\cite{NohedaNRM2023,WongProcIEEE2012,IkedaNature2010,WeebitIEDM2022}.
Among these, ferroelectric RAM (FeRAM) based on HfO$_2$ shows promise: it offers low-voltage switching, sub-$10$~ns-class rewriting, and long retention~\cite{MullerAPL2011,KimIEDM2021}.
However, direct monolithic integration of LPDDR and FeRAM is not feasible due to severe process--temperature mismatch:
DRAM capacitors require high-temperature anneals ($>700\,^\circ\mathrm{C}$), whereas ferroelectric crystallization in HfO$_2$ must remain near $400\,^\circ\mathrm{C}$.
This incompatibility motivates heterogeneous integration at the \emph{package level} rather than within a single process flow.

In this work, we propose \textbf{LPDDR+FeRAM integration via chiplet or system-in-package (SiP/PoP) assembly}.
LPDDR continues to serve as the primary working memory, while a small FeRAM die acts as an assistive checkpoint and refresh-offload memory.
The organization is supervised by the \emph{SystemDK} co-design framework, which coordinates policies across hardware, packaging, and runtime software.
Figure~\ref{fig:concept_lpddr_feram} illustrates the high-level concept:
LPDDR supplies high-bandwidth working memory, FeRAM chiplets retain state with negligible standby power,
and SystemDK supervision ensures seamless operation for mobile edge AI workloads.

% --- Figure 1: Concept overview ---
% ==== Fig. 1: Conceptual overview of LPDDR+FeRAM for Mobile Edge AI ====
\begin{figure}[t]
  \centering
  \begin{tikzpicture}[node distance=1.5cm, font=\small, >=stealth]

    % Mobile Edge AI Device
    \node[draw, thick, rounded corners, fill=gray!15, minimum width=8cm, minimum height=5cm] (device) {};

    % SoC block
    \node[draw, thick, fill=blue!10, minimum width=3.5cm, minimum height=1.2cm] (soc) at (device.center) {SoC / AI Accelerator};

    % LPDDR
    \node[draw, thick, fill=cyan!20, minimum width=3.0cm, minimum height=1.0cm, above=1.2cm of soc] (lpddr) {LPDDR (main memory)};

    % FeRAM
    \node[draw, thick, fill=green!20, minimum width=2.8cm, minimum height=1.0cm, right=2.5cm of soc] (feram) {FeRAM Chiplet};

    % SystemDK supervision
    \node[draw, thick, dashed, red!60!black, rounded corners, fit=(soc) (lpddr) (feram), inner sep=0.5cm] (sysdk) {};
    \node[above right=-0.3cm and -0.3cm of sysdk.north west, font=\small\itshape, red!60!black]
      {SystemDK Co-Design Supervision};

    % Arrows
    \draw[->, thick] (soc.north) -- (lpddr.south) node[midway,left] {high BW};
    \draw[->, thick] (soc.east) -- (feram.west) node[midway,above] {checkpoint / state};
    \draw[<->, thick, dashed] (lpddr.east) -- (feram.north west) node[midway,below] {refresh offload};

    % Label for device
    \node[below=2.4cm of soc, font=\small\bfseries] {Mobile Edge AI Device};

  \end{tikzpicture}
  \vspace{-1ex}
  \caption{High-level concept of LPDDR+FeRAM integration for mobile edge AI. 
  LPDDR provides main working memory, FeRAM acts as assistive checkpoint/storage, 
  and SystemDK supervises co-design across package and system levels.}
  \label{fig:concept_lpddr_feram}
\end{figure}

