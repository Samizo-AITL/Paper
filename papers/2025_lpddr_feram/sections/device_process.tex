% =========================
% sections/device_process.tex
% =========================
\section{Device and Process Integration}

\subsection{LPDDR Technology Background}
Low-power DRAM (LPDDR) is the de facto main memory for mobile systems, providing tens to a few hundreds of~GB/s bandwidth
at substantially lower I/O energy than HBM-class DRAM~\cite{ChoiIEDM2022}.
Despite architectural and I/O optimizations, LPDDR is \emph{volatile} and incurs standby power due to periodic refresh.

% --- Fig.2: 特性比較 ---
% ==== Fig. 2: Access time vs. retention (LPDDR vs FeRAM) ====
\begin{figure}[t]
  \centering
  \begin{tikzpicture}
    \begin{loglogaxis}[
      width=\columnwidth,
      height=0.62\columnwidth,
      xlabel={Access time (ns)},
      ylabel={Retention (s)},
      xmin=8, xmax=300,
      ymin=1e-3, ymax=1e9,
      xtick={10,20,50,100,200},
      ytick={1e-3,1e-2,1e-1,1,1e1,1e3,1e5,1e7,1e9},
      legend style={at={(0.97,0.03)},anchor=south east,fill=none,draw=none},
      tick align=outside,
      grid=both,
      grid style={line width=.1pt, draw=gray!40}
    ]

    % --- LPDDR (volatile / needs refresh): red squares ---
    % Access time ~15-60 ns, effective "retention" ~32-64 ms before refresh
    \addplot+[
      only marks,
      mark=square*,
      mark size=2.2pt
    ] table[row sep=\\]{
      x   y
      15  3.2e-2  % 32 ms
      20  6.4e-2  % 64 ms
      30  3.2e-2
      60  6.4e-2
    };

    % --- FeRAM: blue circles ---
    % Access time ~80-150 ns, retention years (1e7-1e8 s)
    \addplot+[
      only marks,
      mark=*,
      mark size=2.2pt
    ] table[row sep=\\]{
      x    y
      80   1.0e7   % ~0.3 year
      100  3.0e7   % ~1 year
      120  1.0e8   % ~3.2 years
      150  3.0e8   % ~9.5 years
    };

    \legend{LPDDR (typ.), FeRAM (typ.)}
    \end{loglogaxis}
  \end{tikzpicture}
  \vspace{-1ex}
  \caption{Access time vs. retention comparison between LPDDR (red squares) and FeRAM (blue circles). LPDDR offers fast access but requires frequent refresh (tens of ms), whereas FeRAM provides orders-of-magnitude longer retention with modest access latency.}
  \label{fig:access_retention_lpddr_feram}
\end{figure}


\subsection{FeRAM Device and Process}
Ferroelectric RAM (FeRAM) based on doped HfO$_2$ leverages polarization switching to store data with low write voltage and fast access~\cite{MullerAPL2011,KimIEDM2021,NohedaNRM2023}.
Process-wise, FeRAM/FeFET flows require \emph{low-to-mid} temperature stabilization ($\sim$350--450$\,^\circ\mathrm{C}$) to preserve the ferroelectric orthorhombic phase in HfZrO$_2$.

\subsection{Why Monolithic Co-Integration Is Impractical}
LPDDR DRAM arrays rely on high-temperature anneals ($>700\,^\circ\mathrm{C}$) to realize high-quality storage capacitors.
Such thermal budgets collapse the ferroelectric phase of HfO$_2$, whereas post-FeRAM low-temperature windows cannot support DRAM capacitor quality.
Therefore, monolithic LPDDR+FeRAM co-fabrication is \textbf{impractical}; a package-level approach is required.

\subsection{Package-Level Integration: Chiplet/SiP/PoP}
Figure~\ref{fig:package_lpddr_feram} shows our organization:
(1) LPDDR remains as a standard DRAM die/package optimized in its own process;
(2) a small FeRAM die (chiplet) is co-packaged on a common substrate (SiP/interposer or PoP);
(3) the SoC connects to both through short, low-parasitic interconnects.
This separation preserves each technology's process window while enabling system-level policies to exploit non-volatility.

% --- Fig.3: パッケージ統合 ---
% ==== Fig. 3: Package cross-section (LPDDR+FeRAM Chiplet Integration) ====
\begin{figure}[t]
  \centering
  % 2段組の列幅に必ず収める(文字も含めて一括スケール)
  \resizebox{\columnwidth}{!}{%
  \begin{tikzpicture}[node distance=8mm, font=\scriptsize, >=latex]

    % すべてのノードの余白を控えめに(はみ出し抑制)
    \tikzset{every node/.style={inner sep=2pt, outer sep=0pt}}

    % ---- 上段:CPU/Controller ----
    \node[draw, thick, fill=gray!15,
          minimum width=0.38\columnwidth, minimum height=9mm] (cpu) {CPU / Controller};

    % ---- 下段:LPDDR ----
    \node[draw, thick, fill=gray!15,
          minimum width=0.38\columnwidth, minimum height=9mm,
          below=of cpu] (lpddr) {LPDDR DRAM Chip};

    % ---- 右:FeRAM chiplet ----
    \node[draw, thick, fill=gray!15,
          minimum width=0.32\columnwidth, minimum height=9mm,
          right=13mm of lpddr] (feram) {FeRAM Chiplet};

    % ---- SystemDK supervision(3ブロックを囲う)----
    \node[draw, thick, dashed, rounded corners,
          inner sep=5mm, fit=(cpu) (lpddr) (feram), name=systemdk] {};

    % ラベルは枠の内側左上に
    \node[font=\scriptsize\itshape, anchor=north west]
      at ($(systemdk.north west)+(1mm,-1mm)$) {SystemDK Co-Design Supervision};

    % ---- 相互接続矢印 ----
    \draw[->, thick] (cpu.south) -- (lpddr.north)
      node[midway, left] {high bandwidth};

    \draw[->, thick] (cpu.east) -- (feram.west)
      node[midway, above] {checkpoint / state};

    \draw[<->, thick, dashed] (lpddr.east) -- ([yshift=-1mm]feram.south west)
      node[midway, below] {refresh offloading};

    % ---- Substrate / Interposer ----
    \path (systemdk.south) coordinate (baseS);
    \node[draw, thick, fill=gray!10,
          minimum width=0.95\columnwidth, minimum height=4mm,
          below=10mm of baseS] (substrate) {};
    \node[align=center] at (substrate)
      {Package Substrate / Interposer (SiP/PoP)};

  \end{tikzpicture}%
  }
  \vspace{-0.6ex}
  \caption{Package-level integration of LPDDR and FeRAM chiplet under SystemDK supervision.
  LPDDR serves as main working memory, while FeRAM provides checkpointing and refresh suppression via a common substrate (SiP/PoP).}
  \label{fig:package_lpddr_feram}
\end{figure}


\subsection{Interface and Policy Hooks}
The FeRAM chiplet exposes a narrow, reliable link (e.g., mailbox DMA or AXI-lite) for:
\begin{itemize}
  \item \textbf{Checkpoint Write/Read}: bulk DMA of model/activation checkpoints and OS state.
  \item \textbf{Refresh Offloading}: firmware migrates cold regions from LPDDR to FeRAM, suppressing refresh traffic.
  \item \textbf{Instant Resume}: fast restore path avoiding full DRAM warm-up.
\end{itemize}
These hooks are orchestrated by the \emph{SystemDK} co-design framework (policies spanning architecture, package, and OS).

\subsection{Key Technology Parameters}
Table~\ref{tab:tech_params} summarizes representative parameters used in our analysis (also reflected in Fig.~\ref{fig:access_retention_lpddr_feram}).
Values are order-of-magnitude estimates for policy exploration; silicon-specific tuning is straightforward.

\begin{table}[t]
  \centering
  \caption{Representative parameters for LPDDR and FeRAM used in evaluation.}
  \label{tab:tech_params}
  \vspace{2pt}
  \small
  \setlength{\tabcolsep}{5pt}
  \begin{tabular}{@{}lcc@{}}
    \toprule
    Parameter & LPDDR (typical) & FeRAM (typical) \\
    \midrule
    Access latency & 15--60~ns & 80--150~ns \\
    Retention & volatile (32--64~ms refresh) & $10^7$--$10^8$~s ($\sim$years) \\
    Write energy/bit & moderate & low \\
    Endurance & $>10^{15}$ accesses & $10^8$--$10^{12}$ writes \\
    Process temperature & capacitor anneal $>700\,^\circ\mathrm{C}$ & 350--450$\,^\circ\mathrm{C}$ \\
    Role & working memory & checkpoint/state \\
    \bottomrule
  \end{tabular}
\end{table}
