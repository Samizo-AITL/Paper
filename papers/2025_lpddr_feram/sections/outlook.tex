% =========================
% sections/outlook.tex
% =========================
\section{Outlook and Future Directions}

\subsection{Implementation Pathways}
The most practical short-term realization of LPDDR+FeRAM integration is through \emph{System-in-Package (SiP)} or \emph{Package-on-Package (PoP)} assembly. 
Standard LPDDR dies remain unchanged, while a small FeRAM die can be co-packaged using mature 2.5D/3D integration techniques.
As shown in Fig.~\ref{fig:package_lpddr_feram}, this organization introduces minimal process disruption and leverages existing packaging infrastructure widely used in mobile SoCs.

\subsection{Extension to Other NVM Options}
While FeRAM provides an effective proof-of-concept, alternative non-volatile memory (NVM) options can extend the approach:
\begin{itemize}
  \item \textbf{ReRAM}: CMOS-friendly BEOL integration with high scalability, though variability and endurance remain open issues.
  \item \textbf{FeFET}: Excellent CMOS compatibility by embedding ferroelectricity into the gate stack. 
  Critically, FeFETs can be fabricated in sub-10\,nm logic nodes, offering a realistic path toward \textbf{monolithic integration of logic and NVM}.
  \item \textbf{MRAM}: Strong endurance and speed, but process/material mismatch with CMOS logic makes it more suitable as a chiplet for high-performance domains.
\end{itemize}
The same architectural hooks (checkpoint, refresh suppression, instant resume) apply across these NVM types, allowing drop-in replacement in future generations.

\subsection{Mobile Edge AI Use Cases}
Mobile edge AI workloads emphasize \emph{energy efficiency, responsiveness, and always-on connectivity}. 
Representative scenarios include:
\begin{itemize}
  \item \textbf{On-device inference}: reduce standby energy when the accelerator is idle between bursts of activity.
  \item \textbf{Federated and continual learning}: enable frequent checkpointing of model updates without incurring DRAM refresh overhead.
  \item \textbf{Interactive AR/VR and sensor fusion}: support instant resume from standby to active state within sub-ms latency.
\end{itemize}
In each case, LPDDR+FeRAM integration provides measurable benefits while staying within the power and form-factor constraints of mobile SoCs.

\subsection{Broader Implications}
The proposed framework highlights a broader co-design philosophy:
\begin{enumerate}
  \item Retain standard, mass-produced DRAM as the main working memory.
  \item Add a small NVM chiplet for persistence and standby optimization.
  \item Coordinate at the system level via policies in \emph{SystemDK} to maximize efficiency.
\end{enumerate}
This division of labor between volatile and non-volatile memories offers a scalable and portable approach, aligning with both current packaging capabilities and future heterogeneous integration trends.

\subsection{Roadmap and Long-Term Vision}
\begin{itemize}
  \item \textbf{Short term (now--2025)}: LPDDR+FeRAM chiplet integration via SiP/PoP for smartphones, wearables, and embedded AI devices.
  \item \textbf{Mid term (2025--2028)}: Extension to HBM+FeRAM for edge servers and AI boxes, exploiting higher bandwidth packaging.
  \item \textbf{Long term (beyond 2028)}: Transition to \textbf{FeFET-based sub-10\,nm monolithic integration}, embedding non-volatility directly in the logic or DRAM stack. 
  This enables finer-grained checkpointing, lower standby power, and simpler packaging.
\end{itemize}

Until FeFET or scaled ReRAM reach production maturity, LPDDR+FeRAM chiplet integration stands as a \emph{practical near-term solution} that balances performance, energy efficiency, and manufacturability for mobile edge AI.
