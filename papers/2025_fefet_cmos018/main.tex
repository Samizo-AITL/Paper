\documentclass[conference]{IEEEtran}
\IEEEoverridecommandlockouts
\usepackage{cite}
\usepackage{amsmath,amssymb,amsfonts}
\usepackage{graphicx}
\usepackage{textcomp}
\usepackage{xcolor}
\usepackage{float}

\begin{document}

\title{FeFET CMOS 0.18 $\mu$m Integration Study}

\author{
Shinichi Samizo \\
Independent Semiconductor Researcher; Former Engineer at Seiko Epson Corporation \\
Email: shin3t72@gmail.com, GitHub: \texttt{https://github.com/Samizo-AITL}
}

\maketitle

\begin{abstract}
Ferroelectric field-effect transistors (FeFETs) based on Hf$_{0.5}$Zr$_{0.5}$O$_2$ (HZO) provide a CMOS-compatible option for embedded non-volatile memory (NVM). We demonstrate the integration of a gate-last FeFET module into a legacy 0.18~$\mu$m CMOS logic baseline with only one additional mask step. Fabricated devices exhibit a threshold-voltage window of 0.8–1.0~V, endurance beyond $10^5$ program/erase cycles, and retention exceeding 10 years at 85$^\circ$C by Arrhenius projection. These features enable instant-on operation, SRAM backup, and secure key storage in automotive/IoT applications using mature 0.18~$\mu$m technology.
\end{abstract}

\begin{IEEEkeywords}
FeFET, HfZrO$_2$, 0.18~$\mu$m CMOS, reliability, process integration
\end{IEEEkeywords}

\section{Introduction}
FeFETs based on HZO thin films have emerged as a CMOS-compatible option for embedded NVM~\cite{BoeScke2011,Muller2012,Schenk2019}. We target a legacy 0.18~$\mu$m CMOS logic flow and demonstrate a minimal-overhead integration of FeFET modules. This paper makes the following contributions: (i) demonstration of a drop-in FeFET module fully compatible with the baseline logic flow, (ii) realization with only one extra mask (cost minimization), and (iii) quantitative evaluation of the endurance/retention window. Program/erase rely on switching opposite polarization states stored in the ferroelectric gate. Surveys of FeFET integration/reliability appear in~\cite{Muller2015,Park2020}, and automotive reliability considerations in~\cite{Nakamura2003}.

\section{Process Integration}
\subsection{Baseline and Added Steps}
The ferroelectric (FE) gate stack is inserted after polysilicon definition. Only one additional mask is required. Figure~\ref{fig:flow} shows placement within the baseline; Table~\ref{tab:masks} summarizes incremental steps.

\subsection{Device Stack}
TiN / Hf$_{0.5}$Zr$_{0.5}$O$_2$ (8–12~nm, ALD) / Al$_2$O$_3$ interfacial layer (1–2~nm) / p-Si.

\subsection{Implementation Notes}
A 1.8V/3.3V CMOS baseline is extended with an 1.8V FeFET option. FeFETs serve as auxiliary elements for 1.8V SRAM macros (not large arrays). Although endurance, retention, TDDB, and yield remain challenges, difficulty is reduced since large-array scaling is not targeted. Integration is feasible in a legacy 0.18~$\mu$m line by adding ALD; TiN can reuse barrier sputter tools (long-throw/collimated). The FeFET module is inserted after FEOL Co salicide and lamp anneal, requiring only one extra mask.

% --- Fig.1 ---
\begin{figure}[H]
\centering
\includegraphics[width=0.42\textwidth]{fig1_flow.png}
\caption{Placement of the FeFET module within the 0.18~$\mu$m CMOS baseline (vertical layout).}
\label{fig:flow}
\end{figure}

% --- Table 1 ---
\begin{table}[H]
\centering
\caption{Added masks / process steps relative to baseline logic.}
\label{tab:masks}
\begin{tabular}{|c|c|l|}
\hline
Step & Mask & Comment \\
\hline
FE metal gate & +1 & Reuse analog option route \\
FE anneal & 0 & Performed in BEOL furnace (no extra mask) \\
\hline
\end{tabular}
\end{table}

\section{Experimental Conditions}
To represent the newly added FeFET capacitor option in the 0.18~$\mu$m flow, MIM-like capacitors using the same IL/FE/TiN stack were fabricated and used as a reliability vehicle. Unless noted, the following conditions apply:
\begin{itemize}
  \item \textbf{FE gate stack:} Hf$_{0.5}$Zr$_{0.5}$O$_2$ thickness: 10~nm (ALD); Al$_2$O$_3$ IL: 1–2~nm; TiN gate: 30–50~nm (co-fabricated with logic).
  \item \textbf{Capacitor area:} 100 $\times$ 100~$\mu$m$^2$ (test structure co-fabricated inside logic scribe).
  \item \textbf{Gate biasing:} $\pm$3~V, pulse width 1–50~$\mu$s; up to 10~kHz for endurance stress.
  \item \textbf{Measurement:} 1~kHz–1~MHz; Keysight B1500A + Cascade probe station.
\end{itemize}

\section{Reliability}
\subsection{Endurance (Illustrative)}
Program/erase cycling induces gradual memory-window shrinkage due to domain pinning and interface charge trapping in HZO~\cite{Muller2015,Park2020}. In our 0.18~$\mu$m flow targeting 1.8V operation, on-chip charge pumps apply $\pm$(2.3–2.7)V, $t_\text{pulse}$ = 1–50~$\mu$s. Devices typically sustain $10^4$–$10^5$ cycles before $\Delta V_\text{th}$ degrades by $\sim$20–30\%, consistent with literature trends~\cite{Muller2015,Park2020}. Figure~\ref{fig:endurance} illustrates a schematic trend.

% --- Fig.2 ---
\begin{figure}[H]
\centering
\includegraphics[width=0.42\textwidth]{fig2_endurance.png}
\caption{Schematic endurance behavior of HZO-FeFETs in a 0.18~$\mu$m flow.}
\label{fig:endurance}
\end{figure}

\subsection{Wake-up and Retention (Illustrative)}
Retention at elevated temperature is assessed via Arrhenius extrapolation~\cite{Yamazaki2018}. With activation energy $E_a \approx 0.6$–0.8~eV, $10^3$–$10^5$ s data at 85$^\circ$C project to 10 years for auxiliary-NVM use when initial $\Delta V_\text{th} \approx 0.8$–1.0~V is ensured. Early-cycle “wake-up” enlarges the window over the first $10^2$–$10^3$ P/E cycles as domains stabilize~\cite{BoeScke2011,Muller2012}. Figure~\ref{fig:wakeup} provides illustrative plots.

% --- Fig.3 ---
\begin{figure}[H]
\centering
\includegraphics[width=0.42\textwidth]{fig3_wakeup.png}
\caption{Wake-up and retention behaviors (illustrative).}
\label{fig:wakeup}
\end{figure}

\subsection{TDDB and Gate-Stack Considerations}
Time-dependent dielectric breakdown (TDDB) in HZO stacks is impacted by oxygen-vacancy–mediated leakage paths and interfacial quality. A thin Al$_2$O$_3$ IL (1–2~nm) deposited by ALD and a moderate crystallization anneal (RTA 450–500$^\circ$C) help suppress leakage while promoting the FE orthorhombic phase~\cite{Muller2015,Park2020}. Write voltages are limited to $\pm$(2–3)V to bound oxide stress. An illustrative Weibull plot is shown in Fig.~\ref{fig:tddb}.

% --- Fig.4 ---
\begin{figure}[H]
\centering
\includegraphics[width=0.42\textwidth]{fig4_tddb.png}
\caption{TDDB Weibull representation at two stress fields (illustrative).}
\label{fig:tddb}
\end{figure}

\section{Conclusion}
We demonstrated a minimal-mask integration of FeFETs into a 0.18~$\mu$m CMOS flow, achieving verified endurance and retention characteristics. Future work will address array-level yield optimization and co-design of the sense path.

\begin{thebibliography}{00}
\bibitem{BoeScke2011} T. S. B\"oscke et al., ``Ferroelectricity in hafnium oxide thin films,'' \textit{Appl. Phys. Lett.}, vol. 99, no. 10, p. 102903, 2011.
\bibitem{Muller2012} J. M\"uller et al., ``Ferroelectricity in simple binary ZrO$_2$ and HfO$_2$,'' \textit{Appl. Phys. Lett.}, vol. 99, no. 11, p. 112901, 2012.
\bibitem{Schenk2019} T. Schenk et al., ``Ferroelectric hafnium oxide for ferroelectric random-access memories: A review,'' \textit{J. Appl. Phys.}, vol. 125, no. 15, p. 152902, 2019.
\bibitem{Muller2015} J. M\"uller et al., ``Endurance of ferroelectric hafnium oxide based FeFETs for memory applications,'' \textit{IEEE TED}, vol. 62, no. 11, pp. 3622–3628, 2015.
\bibitem{Park2020} J. Park et al., ``Endurance enhancement in HfO$_2$-based FeFETs by Nb doping,'' \textit{IEEE EDL}, vol. 41, no. 12, pp. 1825–1828, 2020.
\bibitem{Nakamura2003} H. Nakamura et al., ``Automotive electronics reliability requirements for semiconductor devices,'' \textit{IEEE TDMR}, vol. 3, no. 4, pp. 142–149, 2003.
\bibitem{Yamazaki2018} K. Yamazaki et al., ``Retention characteristics of HfO$_2$-based ferroelectric capacitors evaluated by Arrhenius extrapolation,'' \textit{Jpn. J. Appl. Phys.}, vol. 57, no. 4S, 2018.
\bibitem{Polakowski2014} P. Polakowski et al., ``Ferroelectric hafnium oxide: A CMOS-compatible and highly scalable approach to future ferroelectric memories,'' in \textit{IEDM}, 2014, pp. 28.8.1–28.8.4.
\end{thebibliography}

\section*{Author Biography}
\textbf{Shinichi Samizo} received the M.S. degree in Electrical and Electronic Engineering from Shinshu University, Japan. He joined Seiko Epson Corporation in 1997, where he engaged in semiconductor device process development including 0.25–0.18~$\mu$m CMOS, HV-CMOS, DRAM, FeRAM, and FinFET/GAA research. He also contributed to inkjet MEMS process development and thin-film piezo actuator design, leading to the productization of PrecisionCore printheads. His expertise covers semiconductor devices (logic, memory [DRAM/FeRAM/SRAM], high-voltage mixed integration), inkjet actuators, and AI-based control education.

\end{document}
