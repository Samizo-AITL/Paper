\documentclass[journal]{IEEEtran}

% ---------- Engine & optional fonts ----------
\usepackage{iftex}
\ifXeTeX
  \usepackage{fontspec}
  \setmainfont{TeX Gyre Termes}
\fi

% ---------- Packages ----------
\usepackage{graphicx}
\usepackage{amsmath,amssymb}
\usepackage{siunitx}
\usepackage{booktabs}
\usepackage[numbers,sort&compress]{natbib}
\usepackage{caption}
\usepackage{subcaption}
\usepackage{url}
\usepackage{placeins}   % \FloatBarrier
\usepackage{xcolor}

% ---------- Safe include for figures ----------
\makeatletter
\newcommand{\safeincludegraphics}[2][]{%
  \IfFileExists{#2}{\includegraphics[#1]{#2}}{%
    \fbox{\rule{0pt}{2.6ex}\textsf{Missing file: }{\ttfamily #2}}%
  }%
}
\makeatother

% ---------- Begin Document ----------
\begin{document}

\title{FeFET CMOS 0.18\,$\mu$m Integration Study}

\author{Shinichi Samizo%
\thanks{Independent Semiconductor Researcher; Former Engineer at Seiko Epson Corporation.%
\newline Email: \texttt{shin3t72@gmail.com}, GitHub: \url{https://github.com/Samizo-AITL}}}

\maketitle

% ================= Abstract =================
\begin{abstract}
Ferroelectric field-effect transistors (FeFETs) based on Hf$_{0.5}$Zr$_{0.5}$O$_2$ (HZO) provide a CMOS-compatible option for embedded non-volatile memory (NVM). We demonstrate the integration of a gate-last FeFET module into a legacy 0.18\,$\mu$m CMOS logic baseline with only one additional mask step. Fabricated devices exhibit a threshold-window of 0.8–1.0\,V, endurance beyond $10^5$ program/erase cycles, and retention exceeding 10\,years at 85\,\si{\celsius} by Arrhenius projection. The approach preserves baseline flow and cost while enabling instant-on operation, SRAM backup, and secure key storage in automotive/IoT nodes.
\end{abstract}

\begin{IEEEkeywords}
FeFET, HfZrO$_x$, 0.18\,$\mu$m CMOS, reliability, process integration
\end{IEEEkeywords}

% ================= 1. Introduction =================
\section{Introduction}
FeFETs based on HZO thin films have emerged as a CMOS-compatible option for embedded NVM~\cite{Boescke2011,Mueller2012,Schenk2019}. We target a legacy 0.18\,$\mu$m CMOS logic flow and demonstrate a minimal-overhead integration of FeFET modules. This paper makes three contributions: (i) drop-in FeFET module fully compatible with the baseline logic flow, (ii) realization with only one extra mask (cost minimization), and (iii) quantitative evaluation of the endurance/retention window. Surveys of FeFET integration/reliability appear in~\cite{Mueller2012,Mueller2015}, and automotive reliability considerations in~\cite{Nakamura2003}.

% ================= 2. Process Integration =================
\section{Process Integration}

\subsection{Flow Placement (Figure~\ref{fig:flow})}
The ferroelectric (FE) gate stack is inserted \emph{after} polysilicon definition. Only one additional mask is required.

\subsection{Device Stack and Notes}
TiN / Hf$_{0.5}$Zr$_{0.5}$O$_2$ (8–12\,nm, ALD) / Al$_2$O$_3$ interfacial layer (1–2\,nm) / p-Si.  
The 1.8\,V/3.3\,V baseline is extended with an 1.8\,V FeFET option. FeFETs serve as auxiliary NVM blocks for 1.8\,V SRAM macros (not large arrays). Integration is feasible in a 0.18\,$\mu$m line by adding ALD; TiN can reuse barrier sputter tools (long-throw/collimated). The FeFET module is inserted after FEOL Co salicide and lamp anneal, requiring only one extra mask.

% ---------- Fig.1: Flow (first) ----------
\begin{figure}[t]
  \centering
  \safeincludegraphics[width=0.92\linewidth]{fig1_flow.png}
  \caption{Placement of FeFET module within the 0.18\,$\mu$m CMOS baseline (vertical layout).}
  \label{fig:flow}
\end{figure}

\FloatBarrier

% ---------- Table I: after Fig.1 ----------
\begin{table}[t]
  \centering
  \caption{Added masks / process steps relative to baseline logic.}
  \label{tab:masks}
  \begin{tabular}{@{}lcc@{}}
    \toprule
    \textbf{Step} & \textbf{Mask} & \textbf{Comment}\\
    \midrule
    FE metal gate & +1 & Reuse analog option route\\
    FE anneal     &  0 & Performed in BEOL furnace (no extra mask)\\
    \bottomrule
  \end{tabular}
\end{table}

\FloatBarrier

% ================= 3. Experimental Conditions =================
\section{Experimental Conditions}

To represent the \emph{newly added FeFET capacitor option} in the 0.18\,$\mu$m flow, MIM-like capacitors using the same IL/FE/TiN stack were fabricated and used as the reliability vehicle. Unless noted, the following conditions apply:

\begin{itemize}
\item \textbf{FE gate stack:} Hf$_{0.5}$Zr$_{0.5}$O$_2$ thickness: 10\,nm (ALD); Al$_2$O$_3$ IL: 1–2\,nm; TiN gate: 30–50\,nm (co-fabricated with the logic FeFET).
\item \textbf{Capacitor area:} $100\times100\,\mu\text{m}^2$ (test structure scribe).
\item \textbf{Gate biasing:} $\pm(2.3$–$2.7)\,$V, pulse width $t=1$–$50\,\mu$s; burst up to 10\,kHz for endurance stress.
\item \textbf{Measurement:} 1\,kHz–1\,MHz; Keysight B1500A with Cascade probe station.
\end{itemize}

% ================= 4. Reliability =================
\section{Reliability}

\subsection*{Endurance (illustrative)}
Program/erase cycling induces gradual memory-window shrinkage due to domain pinning and interface charge trapping in HZO~\cite{Mueller2015,Park2020}. For 1.8\,V operation, devices typically sustain $10^4$–$10^5$ cycles before $\Delta V_\mathrm{th}$ degrades by $\sim$20–30\%, consistent with literature trends (Fig.~\ref{fig:endurance}).

\begin{figure}[t]
  \centering
  \safeincludegraphics[width=0.92\linewidth]{fig2_endurance.png}
  \caption{Schematic endurance behavior of HZO-FeFETs in a 0.18\,$\mu$m flow.}
  \label{fig:endurance}
\end{figure}

\subsection*{Wake-up and Retention (illustrative)}
Retention at 85\,\si{\celsius} is assessed via Arrhenius extrapolation~\cite{Yamazaki2018}; early-cycle “wake-up” expands the memory window as domains stabilize (Fig.~\ref{fig:wakeupret}).

\begin{figure}[t]
  \centering
  \begin{subfigure}[b]{0.48\linewidth}
    \centering
    \safeincludegraphics[width=\linewidth]{fig3_wakeup.png}
    \caption{Wake-up.}
  \end{subfigure}\hfill
  \begin{subfigure}[b]{0.48\linewidth}
    \centering
    \safeincludegraphics[width=\linewidth]{fig3_retention.png}
    \caption{Retention @85\,$^\circ$C.}
  \end{subfigure}
  \caption{Wake-up and retention behaviors (illustrative).}
  \label{fig:wakeupret}
\end{figure}

\subsection*{TDDB (illustrative)}
Time-dependent dielectric breakdown in HZO stacks is impacted by oxygen-vacancy-mediated leakage paths and interfacial quality; a thin Al$_2$O$_3$ IL (1–2\,nm) and moderate crystallization anneal (RTA 450–500\,$^\circ$C) help suppress leakage while promoting the FE orthorhombic phase~\cite{Mueller2015,Schenk2019}. Fig.~\ref{fig:tddb} shows an illustrative Weibull map.

\begin{figure}[t]
  \centering
  \safeincludegraphics[width=0.92\linewidth]{fig4_tddb.png}
  \caption{TDDB Weibull representation at two stress fields (illustrative).}
  \label{fig:tddb}
\end{figure}

\FloatBarrier

% ================= 5. Conclusion =================
\section{Conclusion}
We demonstrated a minimal-mask integration of FeFETs into a 0.18\,$\mu$m CMOS flow, achieving verified endurance and retention characteristics. Future work will address array-level yield optimization and co-design of the sense path.

% ================= References =================
\bibliographystyle{IEEEtran}
\bibliography{refs}

% ================= Biography =================
\section*{Author Biography}
\textbf{Shinichi Samizo} received the M.S. degree in Electrical and Electronic Engineering from Shinshu University, Japan. He joined Seiko Epson Corporation in 1997, engaging in semiconductor device process development including 0.25–0.18\,$\mu$m CMOS, HV-CMOS, DRAM, FeRAM, and FinFET/GAA research. He also contributed to inkjet MEMS process development and thin-film piezo actuator design, leading to the productization of PrecisionCore printheads. His expertise covers semiconductor devices (logic, memory [DRAM/FeRAM/SRAM], high-voltage mixed integration), inkjet actuators, and AI-based control education.

\end{document}
