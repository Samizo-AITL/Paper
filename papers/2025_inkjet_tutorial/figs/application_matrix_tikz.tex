\documentclass[tikz,border=2pt]{standalone}
\usepackage{amsmath}
\usepackage{newtxtext,newtxmath}
\usetikzlibrary{calc}
\tikzset{>=Latex, line/.style={line width=0.5pt}}

\begin{document}
% Fig.6: Application matrix(重なり無し・1カラム幅)
\begin{tikzpicture}[font=\footnotesize, x=1cm, y=1cm]
  %---- 幅と格子 --------------------------------------------------
  \def\W{8.2} \def\H{4.8}
  % 列位置: Domain を 2.0cm、残り4列は等間隔
  \def\xA{0.0}
  \def\xB{2.0}
  \def\xC{3.55}
  \def\xD{5.10}
  \def\xE{6.65}
  \def\xF{8.20}
  % 行位置(4行)
  \def\yA{0.0} \def\yB{1.2} \def\yC{2.4} \def\yD{3.6} \def\yE{4.8}

  % 外枠と縦横罫
  \draw[line] (\xA,\yA) rectangle (\xF,\yE);
  \foreach \x in {\xB,\xC,\xD,\xE} \draw[line] (\x,\yA)--(\x,\yE);
  \foreach \y in {\yB,\yC,\yD}     \draw[line] (\xA,\y)--(\xF,\y);

  %---- ヘッダ(枠の“外”上側) ------------------------------------
  \def\yHead{5.15}
  \node[align=left, text width=2.0cm] at ({(\xA+\xB)/2}, \yHead) {Domain};
  \node at ({(\xB+\xC)/2}, \yHead) {Throughput};
  \node at ({(\xC+\xD)/2}, \yHead) {Uniformity};
  \node at ({(\xD+\xE)/2}, \yHead) {Alignment};
  \node at ({(\xE+\xF)/2}, \yHead) {Viability};

  %---- 行ラベル(左セル内・折返し可) ----------------------------
  \node[align=left, text width=2.0cm] at ({(\xA+\xB)/2}, {(\yD+\yE)/2}) {Bio};
  \node[align=left, text width=2.0cm] at ({(\xA+\xB)/2}, {(\yC+\yD)/2}) {Printing};
  \node[align=left, text width=2.0cm] at ({(\xA+\xB)/2}, {(\yB+\yC)/2}) {Electronics};
  \node[align=left, text width=2.0cm] at ({(\xA+\xB)/2}, {(\yA+\yB)/2}) {Semi\-conductor};

  %---- 列中心(指標列) ------------------------------------------
  \def\xThru{(\xB+\xC)/2}
  \def\xUnif{(\xC+\xD)/2}
  \def\xAlign{(\xD+\xE)/2}
  \def\xViab{(\xE+\xF)/2}

  % 行中心
  \def\yBio{(\yD+\yE)/2}
  \def\yPrint{(\yC+\yD)/2}
  \def\yElec{(\yB+\yC)/2}
  \def\ySemi{(\yA+\yB)/2}

  %---- ドット(セル中心に配置) -----------------------------------
  \newcommand{\dotat}[2]{\fill (#1,#2) circle[radius=0.06cm];}

  % Printing
  \dotat{\xThru}{\yPrint}
  \dotat{\xUnif}{\yPrint}
  % Electronics
  \dotat{\xAlign}{\yElec}
  % Semiconductor
  \dotat{\xUnif}{\ySemi}
  \dotat{\xAlign}{\ySemi}
  % Bio
  \dotat{\xViab}{\yBio}
\end{tikzpicture}
\end{document}
