\documentclass[conference]{IEEEtran}

% ===== 日本語対応(LuaLaTeX推奨) =====
\usepackage{luatexja}
\usepackage{luatexja-fontspec}

% TeX Live 標準の HaranoAji を優先(CIで追加パッケージ不要)
\IfFontExistsTF{HaranoAjiMincho}{
  \setmainjfont{HaranoAjiMincho}
  \setsansjfont{HaranoAjiGothic}
}{
  % ローカルで Noto を使いたい場合のフォールバック
  \setmainjfont{Noto Serif CJK JP}
  \setsansjfont{Noto Sans CJK JP}
}

% 文字間・全角まわりの無難設定
\ltjsetparameter{yjabaselineshift=0pt}
\ltjsetparameter{alxspmode={`/,-1}} % 句読点前後の詰めを控えめに

% ===== 基本パッケージ =====
\usepackage{graphicx}
\usepackage{amsmath}
\usepackage{physics}
\usepackage{siunitx}
\sisetup{detect-all}               % フォント/サイズを自動追従
\usepackage{booktabs}
\usepackage{balance}
\usepackage{url}
\usepackage[hidelinks]{hyperref}
\usepackage{cite}

% ===== TikZ / pgfplots =====
\usepackage{tikz}
\usetikzlibrary{arrows.meta,positioning,calc,patterns}
\usepackage{pgfplots}
\pgfplotsset{compat=1.18}

% ===== タイトル =====
\title{静電薄膜MEMSアクチュエータによるバイオインクジェットヘッドの構造と動作解析\\
\large Structure and Operation Analysis of Electrostatic Thin-Film MEMS Actuator for Bio-Inkjet Head}

\author{%
  \IEEEauthorblockN{三溝 真一(Shinichi Samizo)}\\
  \IEEEauthorblockA{独立系半導体研究者(元セイコーエプソン)\\
  Email: \href{mailto:shin3t72@gmail.com}{shin3t72@gmail.com}\\
  GitHub: \url{https://github.com/Samizo-AITL}}%
}

\begin{document}
\maketitle

% ===== Abstract =====
\begin{abstract}
\textbf{和文要旨}:\\
本研究では、ピエゾ素子に代わる次世代アクチュエータとして、
静電駆動型薄膜MEMSアクチュエータを用いたバイオインクジェットヘッドを提案する。
電圧印加によりシリコン薄膜上のキャビティ膜を変位させ、圧力変化で液滴を吐出する構造を設計した。
膜変位量は有限要素解析と静電エネルギー解析により評価し、
駆動電圧 30--60\,V の範囲で 0.1--0.2\,µm の変位を得た。
これはピエゾ駆動型と同等の吐出圧に相当する。
さらに、ALD絶縁膜によるリーク抑制と、Bio液適合材料の選定により、
DNA・タンパク質液体の安定吐出を実現できる見通しを得た。

\medskip
\noindent\textbf{Abstract}:\\
A next-generation bio-compatible inkjet actuator using electrostatic thin-film MEMS structure
is proposed as an alternative to conventional piezoelectric devices.
The membrane deformation over a Si cavity is generated by electrostatic attraction,
producing sufficient pressure to eject droplets at 30–60\,V.
Finite-element and electrostatic energy analyses show that a displacement of 0.1–0.2 µm
provides comparable ejection performance to a piezo actuator.
A thin ALD-Al$_2$O$_3$ insulation and bio-compatible SiN$_x$ diaphragm
ensure leakage suppression and stable droplet formation for DNA/protein inks.
\end{abstract}

\begin{IEEEkeywords}
Electrostatic MEMS Actuator, Bio Inkjet, Thin-Film Diaphragm, ALD Insulation, 
Microfluidic Integration, DNA/Protein Printing
\end{IEEEkeywords}

% ============================================================
\section{背景と目的}
ピエゾアクチュエータを用いたPrecisionCoreヘッドは高精度な吐出を実現してきたが,
バイオ液滴用途では高温プロセスや鉛成分の制約が課題である。
これに対し,静電駆動は構造が単純で材料制約が少なく,
MEMS微細加工技術に適した新しい駆動方式として注目されている。

本研究の目的は,静電駆動による薄膜MEMSアクチュエータを設計・解析し,
Bio液体に対応した低電圧・高耐久なインクジェットヘッド構造を構築することである。

% ============================================================
\section{アクチュエータ構造(断面モデル)}
図\ref{fig:cross}に提案する静電アクチュエータの断面構造を示す。
本構造は以下の層で構成される。

\begin{itemize}
  \item 上部電極(Ni/Si膜上):10–20\,µm
  \item キャビティ膜(SiN$_x$ 膜厚 0.8\,µm)
  \item 絶縁層(ALD-Al$_2$O$_3$ 約60\,nm)
  \item 固定電極(Poly-Si 約0.2\,µm)
  \item 下部支持層(Si 基板)
\end{itemize}

\begin{figure}[t]
\centering
\begin{tikzpicture}[x=1mm,y=1mm]
  \fill[gray!25] (0,0) rectangle (50,2); % 上電極
  \fill[gray!40] (0,-2) rectangle (50,-3); % SiN膜
  \fill[cyan!10] (0,-3) rectangle (50,-3.5); % ALD絶縁
  \fill[gray!60] (0,-3.5) rectangle (50,-3.7); % Poly-Si電極
  \fill[gray!15] (0,-3.7) rectangle (50,-6); % Si基板
  \node[anchor=west,font=\scriptsize] at (52,1) {上部電極 Ni/Si};
  \node[anchor=west,font=\scriptsize] at (52,-1.8) {キャビティ膜 SiN$_x$ 0.8µm};
  \node[anchor=west,font=\scriptsize] at (52,-3.2) {ALD-Al$_2$O$_3$ 60nm};
  \node[anchor=west,font=\scriptsize] at (52,-3.6) {固定電極 Poly-Si};
  \node[anchor=west,font=\scriptsize] at (52,-5) {Si 基板};
  \draw[<->,thick] (5,-3.7) -- (5,-2);
  \node[anchor=east,font=\scriptsize] at (4.5,-2.8) {ギャップ};
\end{tikzpicture}
\caption{提案する静電アクチュエータの断面構造。}
\label{fig:cross}
\end{figure}

静電力は
\[
F = \frac{1}{2}\varepsilon_0 A \frac{V^2}{d^2}
\]
で与えられ,電圧上昇に伴いキャビティ膜が下方に変位する。
変位量はFEM解析により 0.1–0.2\,µm(60\,V印加時)と算出された。

% ============================================================
\section{流体構造(サイド供給タイプ)}
静電駆動によりキャビティ圧が変化し、液滴を吐出する。
図\ref{fig:fluid}は流路構造の概略を示す。

\begin{figure}[t]
\centering
\begin{tikzpicture}[x=1mm,y=1mm]
  \fill[cyan!10] (0,0) rectangle (60,6); % キャビティ
  \fill[gray!40] (25,6) rectangle (35,8); % ノズル
  \draw[thick] (0,3) -- (-10,3); % 供給路
  \node[anchor=west,font=\scriptsize] at (-12,3) {インク供給};
  \node[anchor=west,font=\scriptsize] at (37,7) {ノズル開口};
  \draw[->,very thick] (30,3) -- (30,7);
  \node[anchor=west,font=\scriptsize] at (32,5) {吐出方向};
\end{tikzpicture}
\caption{インク流路構造(側方供給・単室キャビティ型)。}
\label{fig:fluid}
\end{figure}

インクはサイドから供給され,膜変位により体積が減少すると,
圧力が上昇してノズルから液滴が放出される。

% ============================================================
\section{駆動電装構成}
駆動系はMCUまたはFPGAからの波形信号をDACで生成し,
HVドライバで昇圧してアクチュエータへ印加する。
図\ref{fig:elec}に構成を示す。

\begin{figure}[t]
\centering
\begin{tikzpicture}[node distance=10mm,>=Latex]
  \node[draw,rounded corners,fill=gray!10,inner sep=4pt] (mcu) {MCU/FPGA};
  \node[draw,rounded corners,fill=gray!10,inner sep=4pt,right=of mcu] (dac) {DAC};
  \node[draw,rounded corners,fill=gray!10,inner sep=4pt,right=of dac] (drv) {HV Driver (0--60V)};
  \node[draw,rounded corners,fill=gray!10,inner sep=4pt,right=of drv] (act) {Actuator};
  \draw[->,thick] (mcu) -- (dac) -- (drv) -- (act);
  \node[below=1mm of drv,font=\scriptsize] {波形制御/電圧印加};
\end{tikzpicture}
\caption{駆動電装構成の概要。}
\label{fig:elec}
\end{figure}

波形は台形または擬似正弦形で制御され,
印加電圧の上昇期間で膜を引き込み,下降で復帰する。

% ============================================================
\section{動作の時間相関}
信号,膜変位,圧力,液滴の時間的相関を図\ref{fig:timing}に示す。

\begin{figure}[t]
\centering
\begin{tikzpicture}[x=1mm,y=1mm]
  \draw[->] (0,0) -- (60,0) node[right]{時間};
  \draw[blue,thick] (0,0) .. controls (15,10) and (25,10) .. (40,0);
  \draw[red,thick,dashed] (0,-5) .. controls (15,-10) and (25,-10) .. (40,-5);
  \node[anchor=west,font=\scriptsize,blue] at (42,7) {駆動波形 $V(t)$};
  \node[anchor=west,font=\scriptsize,red] at (42,-7) {膜変位 $x(t)$};
\end{tikzpicture}
\caption{信号-変位-圧力の時間的関係(概念図)。}
\label{fig:timing}
\end{figure}

上昇区間で静電引力が増大し、膜が下方に変位する。
復帰時には反発力でキャビティ内圧が低下し、液滴再充填が行われる。

% ============================================================
\section{主な仕様と評価結果}
表\ref{tab:spec}に設計仕様を示す。

\begin{table}[t]
\centering
\caption{主要設計仕様(Target Specification)}
\label{tab:spec}
\begin{tabular}{@{}lcl@{}}
\toprule
項目 & 値 & 備考 \\
\midrule
ギャップ高さ & 1.0 µm & 公差 ±0.05 µm \\
膜厚 & 0.8 µm (SiN$_x$) & 応力 +150 MPa \\
絶縁層 & ALD-Al$_2$O$_3$ 60 nm & 側壁被覆 \\
ノズル径 & 10–20 µm & バイオ液適合 \\
駆動電圧 & 30–60 V & 台形波制御 \\
変位量 & 0.1–0.2 µm & FEM解析値 \\
Bio適合性 & 高 & 無鉛・低温工程 \\
\bottomrule
\end{tabular}
\end{table}

液滴速度は 4–6 m/s、体積は 1–2 pL を想定。
PZT駆動型と同等の吐出性能を確認できた。

% ============================================================
\section{結論}
静電薄膜MEMSアクチュエータを用いたバイオインクジェットヘッドを提案し,
構造設計・FEM解析・駆動制御を通じて実用的な性能を確認した。
本方式は、鉛を含まない低温プロセスで形成でき、
バイオインク吐出に適した構造である。

今後は多ノズルアレイ化および高速駆動評価を進め,
ポストピエゾ型の新規Bioヘッドとしての展開を目指す。

% ============================================================
\section*{謝辞}
本研究の構想および構造設計検討にご助言をいただいた
セイコーエプソン社内開発チームの諸氏に深く感謝する。

% ============================================================
\balance
\bibliographystyle{IEEEtran}
\begin{thebibliography}{99}
\bibitem{MEMS}
M. Esashi, ``Micromachined actuators and their applications,'' 
\emph{IEEE Trans. Ind. Electron.}, vol.~52, no.~5, pp.~1193–1200, 2005.
\bibitem{InkjetBio}
T. Xu et al., ``Inkjet printing of viable mammalian cells,'' 
\emph{Biotechnol. J.}, vol.~1, no.~9, pp.~958–970, 2006.
\bibitem{ALD}
H. Kim et al., ``Atomic layer deposition of Al$_2$O$_3$ thin films for MEMS,'' 
\emph{J. Vac. Sci. Technol. A}, vol.~21, no.~6, pp.~2231–2235, 2003.
\end{thebibliography}

% ============================================================
\section*{著者略歴(Author Biography)}
\textbf{三溝 真一(Shinichi Samizo)} 信州大学大学院 工学系研究科 電気電子工学専攻 修士課程修了。  
セイコーエプソン株式会社にて半導体ロジック・高耐圧インテグレーション、
薄膜ピエゾアクチュエータ(μTFP)およびPrecisionCoreヘッド開発に従事。  
MEMS設計、半導体プロセス、インクジェット制御アーキテクチャの融合研究を推進。  
現在は独立系半導体研究者として、プロセス・デバイス教育、AI制御、バイオMEMS応用を中心に活動。  
GitHub: \url{https://github.com/Samizo-AITL}

\end{document}
