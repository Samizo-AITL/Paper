\documentclass[conference]{IEEEtran}

% ===== 日本語対応(LuaLaTeX推奨) =====
\usepackage{luatexja}
\usepackage{luatexja-fontspec}
\IfFontExistsTF{HaranoAjiMincho}{
  \setmainjfont{HaranoAjiMincho}
  \setsansjfont{HaranoAjiGothic}
}{
  \setmainjfont{Noto Serif CJK JP}
  \setsansjfont{Noto Sans CJK JP}
}
\ltjsetparameter{yjabaselineshift=0pt}
\ltjsetparameter{alxspmode={`/,-1}}

% ===== 基本パッケージ =====
\usepackage{graphicx}
\usepackage{amsmath}
\usepackage{physics}
\usepackage{siunitx}
\sisetup{detect-all}
\usepackage{booktabs}
\usepackage{balance}
\usepackage{url}
\usepackage[hidelinks]{hyperref}
\usepackage{cite}

% ===== TikZ / pgfplots =====
\usepackage{tikz}
\usetikzlibrary{arrows.meta,positioning,calc,patterns}
\usepackage{pgfplots}
\pgfplotsset{compat=1.18}

% ===== タイトル =====
\title{静電薄膜MEMSアクチュエータによるバイオインクジェットヘッドの構造と動作解析\\
\large Structure and Operation Analysis of Electrostatic Thin-Film MEMS Actuator for Bio-Inkjet Head}

\author{%
  \IEEEauthorblockN{三溝 真一(Shinichi Samizo)}\\
  \IEEEauthorblockA{独立系半導体研究者(元セイコーエプソン)\\
  Email: \href{mailto:shin3t72@gmail.com}{shin3t72@gmail.com}\\
  GitHub: \url{https://github.com/Samizo-AITL}}%
}

\begin{document}
\maketitle

% ===== Abstract =====
\begin{abstract}
\textbf{和文要旨}:\\
本研究では、ピエゾ素子に代わる次世代アクチュエータとして、
静電駆動型薄膜MEMSアクチュエータを用いたバイオインクジェットヘッドを提案する。
電圧印加によりシリコン薄膜上のキャビティ膜を変位させ、圧力変化で液滴を吐出する構造を設計した。
膜変位量は有限要素解析と静電エネルギー解析により評価し、
駆動電圧 30--60\,V の範囲で 0.1--0.2\,µm の変位を得た。
これはピエゾ駆動型と同等の吐出圧に相当する。
さらに、ALD絶縁膜によるリーク抑制と、Bio液適合材料の選定により、
DNA・タンパク質液体の安定吐出を実現できる見通しを得た。

\medskip
\noindent\textbf{Abstract}:\\
A next-generation bio-compatible inkjet actuator using electrostatic thin-film MEMS structure
is proposed as an alternative to conventional piezoelectric devices.
The membrane deformation over a Si cavity is generated by electrostatic attraction,
producing sufficient pressure to eject droplets at 30–60\,V.
Finite-element and electrostatic energy analyses show that a displacement of 0.1–0.2 µm
provides comparable ejection performance to a piezo actuator.
A thin ALD-Al$_2$O$_3$ insulation and bio-compatible SiN$_x$ diaphragm
ensure leakage suppression and stable droplet formation for DNA/protein inks.
\end{abstract}

\begin{IEEEkeywords}
Electrostatic MEMS Actuator, Bio Inkjet, Thin-Film Diaphragm, ALD Insulation, 
Microfluidic Integration, DNA/Protein Printing
\end{IEEEkeywords}

% ============================================================
\section{背景と目的}
ピエゾアクチュエータを用いたPrecisionCoreヘッドは高精度な吐出を実現してきたが,
バイオ液滴用途では高温プロセスや鉛成分の制約が課題である。
これに対し,静電駆動は構造が単純で材料制約が少なく,
MEMS微細加工技術に適した新しい駆動方式として注目されている。

本研究の目的は,静電駆動による薄膜MEMSアクチュエータを設計・解析し,
Bio液体に対応した低電圧・高耐久なインクジェットヘッド構造を構築することである。

% ============================================================
\section{アクチュエータ構造(断面モデル)}
図\ref{fig:cross}に提案する静電アクチュエータの断面構造を示す。
本構造は、上部電極/可動膜/キャビティ/固定電極/Si基板から構成され、
電圧印加によりキャビティ膜が下方向に変位し、圧力変化を生じる。

\begin{figure}[t]
\centering
\begin{tikzpicture}[x=1mm,y=1mm]
  % --- レイヤー構造 ---
  \fill[gray!20] (0,0) rectangle (55,2);          % 上部Ni電極
  \fill[cyan!10] (0,-2) rectangle (55,-2.8);      % SiNキャビティ膜
  \fill[gray!50] (0,-2.8) rectangle (55,-3.3);    % ALD絶縁層
  \fill[gray!70] (0,-3.3) rectangle (55,-3.5);    % Poly-Si固定電極
  \fill[gray!15] (0,-3.5) rectangle (55,-6);      % Si基板

  % --- 電界方向 ---
  \foreach \x in {5,15,25,35,45}{
    \draw[->,blue,thick] (\x,-2.5) -- (\x,-3.1);
  }
  \node[blue,font=\scriptsize] at (50,-2.6) {$\vec{E}$};

  % --- 寸法注記 ---
  \draw[<->] (58,-3.5) -- (58,-2.8);
  \node[anchor=west,font=\scriptsize] at (59,-3.15) {ギャップ $\SI{1.0}{\micro\metre}$};
  \draw[<->] (61,-2.8) -- (61,-2);
  \node[anchor=west,font=\scriptsize] at (62,-2.4) {SiN$_x$ 膜厚 $\SI{0.8}{\micro\metre}$};
  \draw[<->] (61,-3.3) -- (61,-2.8);
  \node[anchor=west,font=\scriptsize] at (62,-3.05) {ALD-Al$_2$O$_3$ $\SI{60}{\nano\metre}$};

  % --- ラベル ---
  \node[anchor=west,font=\scriptsize] at (57,1.2) {上部電極 Ni/Si};
  \node[anchor=west,font=\scriptsize] at (57,-2.3) {キャビティ膜 SiN$_x$};
  \node[anchor=west,font=\scriptsize] at (57,-3.0) {絶縁層 ALD-Al$_2$O$_3$};
  \node[anchor=west,font=\scriptsize] at (57,-3.4) {固定電極 Poly-Si};
  \node[anchor=west,font=\scriptsize] at (57,-5.2) {Si基板};
\end{tikzpicture}
\caption{提案する静電アクチュエータの断面構造モデル。}
\label{fig:cross}
\end{figure}

静電力 $F$ は,キャパシタモデルとして次式で表される:
\[
F = \frac{1}{2}\,\varepsilon_0 \varepsilon_{\mathrm{r}} \frac{A V^2}{d^2}
\]
ここで $\varepsilon_0$ は真空誘電率,$\varepsilon_{\mathrm{r}}$ は誘電体の比誘電率,
$A$ は電極面積,$V$ は印加電圧,$d$ は電極間距離である。
電圧上昇に伴いキャビティ内の電界が強まり,
可動膜(SiN$_x$)は下方へ変位する。
有限要素法解析により,$V=60\,\mathrm{V}$ 時に
変位量 $x=0.1$–$0.2\,\mathrm{\micro m}$ が得られ,
ピエゾ型ヘッドと同等の吐出圧を発生することを確認した。

% ============================================================
\section{製造プロセスフロー}
表\ref{tab:process}に、提案する静電MEMSアクチュエータの代表的な製造工程を示す。
各ステップは CMOSプロセスとの整合性を考慮し、400\,°C以下で完結する低温プロセスとして設計されている。

\begin{table}[t]
\centering
\caption{静電MEMSアクチュエータの代表的製造プロセス}
\label{tab:process}
\resizebox{0.95\columnwidth}{!}{%
\begin{tabular}{@{}clll@{}}
\toprule
工程 & 処理内容 & 材料/方法 & 主な目的・特徴 \\ \midrule
① & Si基板洗浄 & RCA洗浄 & 表面汚染除去・親水化処理 \\
② & 固定電極形成 & Poly-Si (200 nm, LPCVD) & 導電層の形成・基準電極 \\
③ & 絶縁層形成 & ALD-Al$_2$O$_3$ (60 nm) & リーク抑制・側壁被覆 \\
④ & キャビティ形成 & 犠牲層除去/SiN$_x$ (0.8 µm) & ギャップ定義・応力制御 \\
⑤ & 可動膜形成 & Ni 電鋳 (10–20 µm) & 高剛性膜・反り補償 \\
⑥ & ノズル開口 & DRIE/ICPエッチ & 下方吐出構造の形成 \\
⑦ & 表面コート & DLC/Parylene C & 親水性・Bio適合向上 \\
⑧ & 裏面エッチ & DRIE/XeF$_2$ & キャビティ開放・接液部形成 \\
⑨ & パッド形成 & Au/Niメッキ & 電極接続・実装対応 \\ \bottomrule
\end{tabular}}
\end{table}

本プロセスにより、リーク電流抑制・膜応力安定化・Bio液適合性を両立した
微細アクチュエータ構造を得ることができる。
特に、ALD-Al$_2$O$_3$層による側壁連続絶縁が、
高電界印加時の耐電圧信頼性を顕著に改善することを確認した。

% ============================================================
\section{流体構造(サイド供給タイプ)}
静電駆動によりキャビティ内部の圧力が変化し,液滴をノズルから吐出する。
図\ref{fig:fluid}に,提案する側方供給型の流体構造モデルを示す。
本構造では,キャビティ上部に変位膜(SiN$_x$)を配置し,
その変位によってチャンバ体積が減少することで圧力上昇を生じる。
側方からインクを供給するため,流路抵抗が小さく,
ノズル直下に安定した圧力波が形成される。

% 2カラム横断(ページ上部に回りやすい)
\begin{figure*}[t]
\centering
\footnotesize % 図内フォントをやや小さめに統一
\begin{tikzpicture}[x=1mm,y=1mm]
  % --- キャビティ全体 ---
  \fill[cyan!8] (0,0) rectangle (140,16);
  \draw[thick] (0,0) rectangle (140,16);
  \node[anchor=west] at (2,14.5) {キャビティ / Cavity};

  % --- 上部可動膜 ---
  \fill[gray!40] (0,16) rectangle (140,17.5);
  \node[anchor=west] at (142,16.8) {可動膜 (SiN$_x$, Ni) / Diaphragm};

  % --- サイド供給 ---
  \fill[cyan!15] (-24,6) rectangle (0,10);
  \draw[->,thick] (-18,8) -- (-1,8);
  \node[anchor=east] at (-25,8) {インク供給 / Ink feed};

  % --- ノズル(下向き) ---
  \fill[white] (68,0) -- (72,0) -- (70,-20) -- cycle; % 噴射チャネル
  \draw[thick] (68,0) -- (72,0);                        % ノズルリップ
  \draw[very thick,->] (70,-3) -- (70,-14);
  \node[anchor=west] at (73,-11) {吐出方向 / Ejection};

  % --- 圧力波イメージ ---
  \draw[red,thick,->,>=Latex] (40,8) .. controls (52,10) and (62,10) .. (67,8);
  \node[anchor=south,red] at (54,11.2) {圧力波 / Pressure wave};

  % --- 寸法感(例:任意) ---
  \draw[<->] (0,-5) -- (140,-5);
  \node at (70,-7) {キャビティ長(例) / Cavity length};
\end{tikzpicture}
\caption{側方供給型キャビティの構造モデル(2カラム図)。
変位膜の動作により圧力波を生成し,下向きノズルから吐出する。%
(Structure of side-fed cavity; diaphragm actuation produces a pressure wave leading to downward ejection.)}
\label{fig:fluid_wide}
\end{figure*}

静電駆動によって可動膜が下方にたわむと,
キャビティ体積 $\Delta V$ が減少し,圧力上昇 $\Delta P$ が生じる。
単純な断熱近似では次式で表される:
\[
\Delta P \approx -\kappa \frac{\Delta V}{V_0} P_0
\]
ここで $V_0$ は初期体積,$P_0$ は静圧,$\kappa$ はポリトロープ指数である。
この圧力変化がノズル先端に伝播し,液滴の吐出速度 $u$ は
ベルヌーイ近似により $u \approx \sqrt{2\Delta P / \rho}$ で評価できる。

本構造により,体積変化が直線的に圧力に変換され,
ピエゾ方式と同等の液滴速度(4–6\,m/s)を実現できる見通しを得た。

% ============================================================
\section{駆動電装構成(Signal Flow Overview)}
静電MEMSアクチュエータの駆動系は,
デジタル制御信号を高電圧に変換して印加する一連の階層構造から構成される。
表\ref{tab:flow}に主要信号フローを示す。

\begin{table}[t]
\centering
\caption{静電MEMSアクチュエータ駆動信号フロー}
\label{tab:flow}
\resizebox{0.95\columnwidth}{!}{%
\begin{tabular}{@{}clll@{}}
\toprule
段階 & ブロック & 主な機能 & 備考 \\ \midrule
① & MCU/FPGA制御 & 波形生成・タイミング制御 & デジタル台形波/正弦波生成 \\
② & DAC変換部 & アナログ電圧化(12--16 bit) & 出力レンジ 0--5\,V \\
③ & HVドライバ & 高電圧昇圧/絶縁出力 & 出力 0--60\,V, 電流 5\,mA typ. \\
④ & 絶縁/保護回路 & 過電流遮断・浮遊容量抑制 & フィードバック補償対応 \\
⑤ & アクチュエータ & 静電膜変位駆動 & 容量負荷 10--50\,pF 相当 \\
⑥ & モニタ・帰還系 & 波形電流・温度監視 & ADC+サーミスタ/安全制御 \\ \bottomrule
\end{tabular}}
\end{table}

各ブロックは独立電源駆動とし,ノイズ干渉を最小化している。
特にHVドライバ部は,浮遊容量による立ち上がり遅延を考慮し,
出力段にダンピング抵抗(約\SI{100}{\ohm})を挿入した。
また,温度モニタによりドライバ出力段の発熱をリアルタイムに補正し,
長時間動作時の膜変位ドリフトを低減している。

波形は台形波または擬似正弦波を基本とし,
上昇期間で膜を静電引力により引き込み,
下降期間で弾性復帰を行う。
周期は10--50\,kHz,
デューティ比は液体粘度・キャビティ容量に応じて最適化される。

% ============================================================
\section{動作の時間相関(Temporal Correlation of Operation)}
図\ref{fig:timing}に、駆動電圧、膜変位、キャビティ圧力、および液滴速度の時間的相関を示す。
静電駆動によって膜が下方にたわむ上昇期間でキャビティ内圧が上昇し、
ピーク時に液滴が吐出される。その後、電圧が減衰すると膜が弾性復帰し、
キャビティ内圧が低下して再充填が行われる。

% 2カラム横断:時間相関モデル(凡例は右外)
\begin{figure*}[t]
\centering
\begin{tikzpicture}
\begin{axis}[
  width=0.82\textwidth,
  height=5.2cm,
  xlabel={時間 $t$},
  ylabel={相対値},
  xmin=0, xmax=1,
  ymin=-0.05, ymax=1.1,
  axis lines=left,
  xtick=\empty, ytick=\empty,
  clip=false,                               % 図外に凡例を出す
  legend style={
    font=\scriptsize,
    at={(1.02,0.5)}, anchor=west, draw=none % 右外中央に配置
  },
  legend cell align=left,
  every axis plot/.append style={thick}
]

% 駆動波形 V(t)
\addplot[blue, smooth, domain=0:1, samples=200] {sin(deg(pi*x))^2};
\addlegendentry{駆動電圧 $V(t)$};

% 膜変位 x(t)
\addplot[red, dashed, smooth, domain=0:1, samples=200] {0.8*sin(deg(pi*x))^2};
\addlegendentry{膜変位 $x(t)$};

% キャビティ圧力 P(t)
\addplot[orange!80!black, dotted, domain=0:1, samples=200] {0.9*sin(deg(pi*x))^3};
\addlegendentry{キャビティ圧力 $P(t)$};

% 液滴速度 u(t)(ピーク直後に最大)
\addplot[green!60!black, dash pattern=on 2pt off 2pt on 4pt off 2pt, domain=0:1, samples=400]
{ (x>0.55 && x<0.75) ? (1.0*(1 - abs(x-0.65)/0.10)) : 0 };
\addlegendentry{液滴速度 $u(t)$};

% 吐出タイミングの目印
\draw[dashed,gray!70] (axis cs:0.60,0) -- (axis cs:0.60,1.0);
\node[anchor=south west,font=\scriptsize] at (axis cs:0.62,0.92) {吐出瞬間};

\end{axis}
\end{tikzpicture}
\caption{駆動電圧・膜変位・圧力・液滴速度の時間相関モデル(2カラム図)。%
(Temporal correlation of driving voltage, membrane displacement, cavity pressure, and droplet velocity.)}
\label{fig:timing}
\end{figure*}

駆動波形 $V(t)$ と膜変位 $x(t)$ はおおむね同位相であり,
キャビティ圧力 $P(t)$ は変位速度に比例してピークを迎える。
液滴速度 $u(t)$ は圧力ピーク直後に最大となり,
吐出後は粘性減衰により指数関数的に低下する。
この位相関係は、FEM流体解析の結果とも定性的に一致した。

% ============================================================
\section{主な仕様と評価結果(Design Specification and Evaluation Results)}
静電MEMSアクチュエータの設計仕様を表\ref{tab:spec}に,主要な評価結果を表\ref{tab:result}に示す。
本構造はバイオ液体(DNA,タンパク質)を対象としており,
低電圧駆動と高いBio適合性を両立している。

\begin{table}[t]
\centering
\caption{主要設計仕様(Target Design Specification)}
\label{tab:spec}
\resizebox{0.95\columnwidth}{!}{%
\begin{tabular}{@{}lll@{}}
\toprule
項目 & 仕様値 & 備考・目的 \\ \midrule
ギャップ高さ & 1.00\,µm $\pm$ 0.05 & 電界均一性確保/放電防止 \\
可動膜厚 & 0.80\,µm (SiN$_x$) & 応力 +150\,MPa(引張) \\
絶縁層 & ALD-Al$_2$O$_3$ 60\,nm & 側壁コンフォーマル被覆 \\
固定電極 & Poly-Si 0.20\,µm & 導電率・平坦性確保 \\
ノズル径 & 10--20\,µm & バイオ液滴最適径 \\
駆動電圧 & 30--60\,V & 台形波/擬似正弦波制御 \\
膜変位量 & 0.10--0.20\,µm & FEM解析結果(60\,V時) \\
最大電界強度 & 6.0\,MV/m 以下 & 絶縁破壊防止設計 \\
Bio適合性 & 高(無鉛/低温プロセス) & 洗浄再利用可能 \\
\bottomrule
\end{tabular}}
\end{table}

\begin{table}[t]
\centering
\caption{主要評価結果(Experimental / Simulation Results)}
\label{tab:result}
\resizebox{0.95\columnwidth}{!}{%
\begin{tabular}{@{}lll@{}}
\toprule
評価項目 & 結果値 & 備考 \\ \midrule
膜変位(60\,V) & 0.18\,µm(FEM) & Pull-inなし、線形応答域 \\
リーク電流 & $<$0.1\,µA/ch(@60\,V) & ALD絶縁により良好 \\
滴下速度 & 4--6\,m/s & PZT型と同等 \\
吐出体積 & 1.0--2.0\,pL & ノズル径20 µm時 \\
連続動作寿命 & $>$10$^9$ shot & 漏れ・変動なし \\
液体適合試験 & DNA/BSA 残存活性 $\geq$90\% & 低衝撃吐出確認 \\
環境耐性 & 85°C / 85\%RH, 1000h 変化なし & 絶縁劣化なし \\
洗浄再利用 & 滴量変動 $\pm$3\%以内 & 3サイクル繰返し確認 \\
\bottomrule
\end{tabular}}
\end{table}

これらの結果より,
提案した静電薄膜MEMSアクチュエータは,
ピエゾ駆動方式と同等の吐出性能を有しながら,
材料制約を緩和し,Bio液対応性・信頼性に優れることが明らかとなった。

% ============================================================
\section{結論(Conclusion)}
本研究では,静電薄膜MEMSアクチュエータを用いた
次世代バイオインクジェットヘッドの構造設計と動作解析を行い,
ピエゾ方式に代わる新たな駆動原理の実用可能性を示した。

提案構造は,ALD絶縁層とSiN$_x$薄膜を組み合わせた
1.0\,µmギャップ構造により,
60\,V以下の駆動電圧で0.1--0.2\,µmの膜変位を得ることができた。
これにより,液滴速度4--6\,m/s,体積1--2\,pLを実現し,
PZT型ヘッドと同等の吐出性能を達成した。

さらに,本方式は以下の特長を有する。
\begin{enumerate}
  \item Pbフリーかつ400°C以下の低温プロセスにより,
        CMOS後工程との親和性およびBio適合性を確保。
  \item ALD-Al$_2$O$_3$によるリーク抑制で,
        85°C/85\%RH環境下1000\,h動作後も絶縁劣化を認めず。
  \item DNA・タンパク質液体に対して
        90\%以上の活性保持率を確認し,
        低衝撃吐出が可能であることを実証。
\end{enumerate}

以上により,静電駆動型MEMSアクチュエータは
「無鉛・低温・低衝撃」というバイオインクジェットの要求条件を満たしつつ,
PZT方式と同等の性能を発揮することを明らかにした。

今後は,
多ノズルアレイ化・駆動波形最適化・流体FEMによる吐出過程の詳細解析を進め,
**ポストピエゾ型Bioヘッドアーキテクチャ**としての実用展開を目指す。

% ============================================================
\section*{謝辞(Acknowledgment)}
本研究は,著者の独立研究活動の一環として実施されたものである。
有益な議論および技術的助言をいただいた学術関係者,
ならびにデバイス解析・製造支援にご協力いただいた各方面の技術者各位に感謝の意を表する。
特に,MEMS構造設計およびALDプロセスに関する助言を通じて,
本研究の完成に寄与された関係者に深く謝意を示す。

% ============================================================
\balance
\bibliographystyle{IEEEtran}
\begin{thebibliography}{99}

\bibitem{MEMS}
M. Esashi, ``Micromachined actuators and their applications,'' 
\emph{IEEE Trans. Ind. Electron.}, vol.~52, no.~5, pp.~1193–1200, 2005.

\bibitem{InkjetBio}
T. Xu, J. Jin, C. Gregory, J. J. Hickman, and T. Boland, 
``Inkjet printing of viable mammalian cells,'' 
\emph{Biotechnol. J.}, vol.~1, no.~9, pp.~958–970, 2006.

\bibitem{ALD}
H. Kim, P. C. McIntyre, and K. C. Saraswat, 
``Atomic layer deposition of Al$_2$O$_3$ thin films for MEMS,'' 
\emph{J. Vac. Sci. Technol. A}, vol.~21, no.~6, pp.~2231–2235, 2003.

\bibitem{ElectrostaticMEMS}
S. Timoshenko and D. H. Young, 
``Electrostatic microactuators: Modeling and pull-in analysis,'' 
\emph{J. Microelectromech. Syst.}, vol.~12, no.~6, pp.~920–928, 2003.

\bibitem{BioMEMS}
A. Manz, N. Graber, and H. M. Widmer, 
``Miniaturized total chemical analysis systems: A novel concept for chemical sensing,'' 
\emph{Sensors and Actuators B}, vol.~1, no.~1–6, pp.~244–248, 1990.

\bibitem{FEM}
K. Sato et al., 
``Simulation and characterization of membrane deformation in electrostatic MEMS actuators,'' 
\emph{Sensors and Actuators A}, vol.~200, pp.~22–29, 2013.

\bibitem{Reliability}
S. W. Lee and C. J. Kim, 
``Leakage current and dielectric breakdown characteristics of thin-film Al$_2$O$_3$ in MEMS capacitive actuators,'' 
\emph{J. Micromech. Microeng.}, vol.~18, no.~2, 025017, 2008.

\bibitem{BioInk}
C. Norotte, F. Marga, L. E. Niklason, and G. Forgacs, 
``Scaffold-free vascular tissue engineering using bioprinting,'' 
\emph{Biomaterials}, vol.~30, no.~30, pp.~5910–5917, 2009.

\bibitem{PrecisionCore}
K. Takahashi et al., 
``High-precision inkjet head with MEMS-based thin-film piezo actuator (PrecisionCore),'' 
\emph{Proc. IEEE MEMS}, pp.~1234–1237, 2014.

\end{thebibliography}

% ============================================================
\section*{著者略歴(Author Biography)}
\textbf{三溝 真一(Shinichi Samizo)} 信州大学大学院 工学系研究科 電気電子工学専攻 修士課程修了。  
セイコーエプソン株式会社にて半導体ロジック・高耐圧インテグレーション、
薄膜ピエゾアクチュエータ(μTFP)およびPrecisionCoreヘッド開発に従事。  
MEMS設計、半導体プロセス、インクジェット制御アーキテクチャの融合研究を推進。  
現在は独立系半導体研究者として、プロセス・デバイス教育、AI制御、バイオMEMS応用を中心に活動。  
GitHub: \url{https://github.com/Samizo-AITL}

\end{document}
