\documentclass[conference]{IEEEtran}

% ===== 日本語対応(LuaLaTeX推奨) =====
\usepackage{luatexja}
\usepackage{luatexja-fontspec}
\IfFontExistsTF{HaranoAjiMincho}{
  \setmainjfont{HaranoAjiMincho}
  \setsansjfont{HaranoAjiGothic}
}{
  \setmainjfont{Noto Serif CJK JP}
  \setsansjfont{Noto Sans CJK JP}
}
\ltjsetparameter{yjabaselineshift=0pt}
\ltjsetparameter{alxspmode={`/,-1}}

% ===== 基本パッケージ =====
\usepackage{graphicx}
\usepackage{amsmath}
\usepackage{physics}
\usepackage{siunitx}
\sisetup{detect-all}
\usepackage{booktabs}
\usepackage{balance}
\usepackage{url}
\usepackage[hidelinks]{hyperref}
\usepackage{cite}

% ===== TikZ / pgfplots =====
\usepackage{tikz}
\usetikzlibrary{arrows.meta,positioning,calc,patterns}
\usepackage{pgfplots}
\pgfplotsset{compat=1.18}

% ===== タイトル =====
\title{静電薄膜MEMSアクチュエータによるバイオインクジェットヘッドの構造と動作解析\\
\large Structure and Operation Analysis of Electrostatic Thin-Film MEMS Actuator for Bio-Inkjet Head}

\author{%
  \IEEEauthorblockN{三溝 真一(Shinichi Samizo)}\\
  \IEEEauthorblockA{独立系半導体研究者(元セイコーエプソン)\\
  Email: \href{mailto:shin3t72@gmail.com}{shin3t72@gmail.com}\\
  GitHub: \url{https://github.com/Samizo-AITL}}%
}

\begin{document}
\maketitle

% ===== Abstract =====
\begin{abstract}
\textbf{和文要旨}:\\
本研究では、ピエゾ素子に代わる次世代バイオ適合アクチュエータとして、
静電駆動型薄膜MEMSアクチュエータを応用したインクジェットヘッドを提案する。
電圧印加によりシリコン上の薄膜ダイアフラムを静電的に変位させ、
キャビティ内圧を変化させて液滴を吐出する構造を設計した。
膜変位量は有限要素解析および静電エネルギー解析により評価し、
駆動電圧30--60\,Vの範囲で0.1--0.2\,µmの変位を得た。
これはピエゾ駆動型と同等の吐出圧に相当する。
さらに、ALD-Al$_2$O$_3$絶縁膜によるリーク電流の抑制と、
Bio液適合SiN$_x$膜の採用により、DNA・タンパク質液の安定吐出性を実証した。

\medskip
\noindent\textbf{Abstract}:\\
A next-generation bio-compatible inkjet actuator based on an electrostatic thin-film MEMS structure
is proposed as an alternative to conventional piezoelectric devices.
Electrostatic attraction between a top electrode and a Si-based diaphragm
induces membrane deformation, generating pressure variations sufficient for droplet ejection.
Finite-element and electrostatic energy analyses confirm that a displacement of 0.1–0.2\,µm
at 30–60\,V achieves ejection pressure equivalent to that of a piezo-driven head.
Furthermore, an ALD-deposited Al$_2$O$_3$ insulation layer effectively suppresses leakage current,
while a bio-compatible SiN$_x$ diaphragm ensures stable ejection of DNA and protein solutions.
\end{abstract}

\begin{IEEEkeywords}
Electrostatic MEMS Actuator, Bio-Inkjet, Thin-Film Diaphragm, ALD Insulation, 
Microfluidic Integration, DNA/Protein Printing
\end{IEEEkeywords}

% ============================================================
\section{背景と目的}
ピエゾアクチュエータを用いた PrecisionCore ヘッドは,
高精度かつ高信頼な液滴吐出技術として広く実用化されている。
しかし,その材料系には Pb を含む PZT 薄膜が用いられ,
高温成膜プロセス(650~700 °C)や環境負荷物質の使用が避けられない。
また,バイオ液滴用途では,熱・化学的ダメージの低減や,
生体分子の活性保持が重要であり,既存ピエゾ方式では適用範囲に限界がある。

これに対し,静電駆動方式は構造が単純で,
低温プロセスによる MEMS 一体化が容易であり,
かつ無鉛材料による環境・医療適合性を両立できる利点を有する。
静電力は電圧の二乗に比例するため,適切な電界設計と膜応力制御により,
高応答・長寿命動作を実現できる可能性がある。

本研究の目的は,この静電駆動原理を用いた
薄膜 MEMS アクチュエータ構造を設計・解析し,
低電圧(30–60 V)で駆動可能なバイオ適合インクジェットヘッドを構築することである。
さらに,ALD-Al$_2$O$_3$ 絶縁膜によるリーク電流抑制と,
SiN$_x$ 膜応力制御による安定駆動を検証し,
ポスト・ピエゾ型バイオインクジェットの実現性を明らかにする。

% ============================================================
\section{アクチュエータ構造(断面モデル)}
図\ref{fig:cross}に提案する静電アクチュエータの断面構造を示す。
本構造は,上部電極/可動膜(SiN$_x$)/静電ギャップ/絶縁層(ALD-Al$_2$O$_3$)/
固定電極(Poly-Si)/Si基板から構成される多層膜構造であり,
電圧印加によってキャビティ膜が下方向に変位し,圧力変化を生じる。
静電駆動に必要なギャップは 1.0\,µm と微小であるが,
絶縁信頼性確保のために ALD 成膜による均一膜厚化を行い,
リーク電流を抑制している。

% Fig.1(縦方向拡大型/重なり解消)
\begin{figure}[t]
\centering
\resizebox{\columnwidth}{!}{% ← 1カラム幅に収める
\begin{tikzpicture}[x=1mm,y=1mm]
  % ===== 図のスタイル定義 =====
  \tikzset{
    labelbox/.style={fill=white,draw=black!40,rounded corners,
                     inner sep=1.5pt,font=\scriptsize}
  }

  % ===== 描画範囲(上下に余裕を持たせる) =====
  \draw[white] (-2,10) rectangle (82,-42);

  % ==== 各層定義(縦方向拡張) ====
  % 上部電極 Ni/Si
  \fill[gray!25] (0,0) rectangle (80,-6);
  \node[labelbox,anchor=west] at (82,-3) {上部電極 Ni/Si};

  % 可動膜(SiN_x)
  \fill[gray!40] (0,-8) rectangle (80,-12);
  \node[labelbox,anchor=west] at (82,-10) {可動膜 SiN$_x$(0.8\,µm)};

  % 静電ギャップ(Dry cavity)
  \fill[white] (0,-14) rectangle (80,-18);
  \node[labelbox,anchor=west] at (82,-16) {静電ギャップ(1.0\,µm)};

  % 絶縁層 ALD-Al2O3
  \fill[orange!10] (0,-18) rectangle (80,-19);
  \node[labelbox,anchor=west] at (82,-18.6) {絶縁層 ALD-Al$_2$O$_3$(60\,nm)};

  % 固定電極 Poly-Si
  \fill[gray!60] (0,-19) rectangle (80,-20.5);
  \node[labelbox,anchor=west] at (82,-19.8) {固定電極 Poly-Si(0.2\,µm)};

  % Si基板
  \fill[gray!15] (0,-21) rectangle (80,-40);
  \node[labelbox,anchor=west] at (82,-30) {Si 基板};

  % 外枠
  \draw[thick] (0,0) rectangle (80,-40);

  % ===== 矢印・注記 =====
  % 電界方向
  \foreach \x in {10,25,40,55,70} {
    \draw[->,blue!60,thick] (\x,-6.5) -- (\x,-17.5);
  }
  \node[anchor=west,font=\scriptsize,blue!60] at (2,2) {電界方向 $E$};

  % 膜変位方向
  \draw[->,red!70!black,thick] (12,-8) -- (12,-13);
  \node[anchor=east,font=\scriptsize,red!70!black] at (11.5,-10.5) {膜変位 $x$};

  % 吐出方向(キャビティ下方)
  \draw[->,cyan!70!black,thick] (40,-25) -- (40,-38);
  \node[anchor=west,font=\scriptsize,cyan!70!black] at (41,-32) {吐出方向 / Ejection};

  % ギャップ寸法
  \draw[<->,thick] (6,-14) -- (6,-18);
  \node[anchor=east,font=\scriptsize] at (5.5,-16) {ギャップ $d$};
\end{tikzpicture}
}% end resizebox
\caption{静電MEMSアクチュエータの断面構造(縦方向拡大型,重なり解消版)。%
電界印加により可動膜が下向きに変位し,キャビティ圧を変化させて液滴を吐出する。}
\label{fig:cross}
\end{figure}

静電駆動力 $F$ はキャパシタンスモデルとして次式で表される:
\[
F = \frac{1}{2}\,\varepsilon_0 \varepsilon_{\mathrm{r}} \frac{A V^2}{d^2}
\]
ここで,$\varepsilon_0$ は真空誘電率,$\varepsilon_{\mathrm{r}}$ は誘電体の比誘電率,
$A$ は電極面積,$V$ は印加電圧,$d$ は電極間距離である。
電圧上昇に伴い電界強度 $E = V/d$ が増大し,
可動膜は静電引力によって下方へ変位する。
有限要素法(FEM)解析の結果,
$V=60\,\mathrm{V}$ 印加時に変位量 $x = 0.1$–$0.2\,\mathrm{\micro m}$ が得られ,
ピエゾ駆動型ヘッドと同等の吐出圧力を発生することを確認した。

% ============================================================
\section{製造プロセスフロー}
表\ref{tab:process}に、提案する静電MEMSアクチュエータの代表的な製造工程を示す。
本プロセスは、CMOS後工程(BEOL)との整合性を考慮し、
全工程を 400\,°C 以下で完結する低温プロセスとして設計した。
各ステップは、絶縁信頼性・膜応力制御・Bio液適合性を同時に満たすことを目的としている。

\begin{table}[t]
\centering
\caption{静電MEMSアクチュエータの代表的製造プロセス(400\,°C以下)}
\label{tab:process}
\resizebox{0.97\columnwidth}{!}{%
\begin{tabular}{@{}clll@{}}
\toprule
工程 & 処理内容 & 材料/方法 & 主な目的・特徴 \\ \midrule
① & Si基板洗浄 & RCA洗浄 & 表面汚染除去・親水化処理 \\
② & 固定電極形成 & Poly-Si(200\,nm, LPCVD) & 導電層形成・基準電位確立 \\
③ & 絶縁層形成 & ALD-Al$_2$O$_3$(60\,nm) & リーク抑制・側壁連続被覆 \\
④ & キャビティ形成 & 犠牲層/SiN$_x$(0.8\,µm) & ギャップ定義・応力安定化 \\
⑤ & 可動膜形成 & Ni電鋳(10–20\,µm) & 高剛性膜・反り補償構造 \\
⑥ & ノズル開口 & DRIE/ICPエッチ & 下方吐出経路の形成 \\
⑦ & 表面コート & DLC/Parylene-C & 親水制御・Bio液適合性向上 \\
⑧ & 裏面エッチ & DRIE/XeF$_2$ & キャビティ開放・流路形成 \\
⑨ & パッド形成 & Au/Niメッキ & 電極接続・COF実装対応 \\ \bottomrule
\end{tabular}}
\end{table}

本プロセスでは、ALDによるAl$_2$O$_3$層を全周に形成することで、
電極端部および側壁での電界集中を抑制し、リーク電流を大幅に低減した。
また、SiN$_x$膜の応力を +150\,MPa 程度に制御することで、
可動膜の反りを抑制しつつ変位の再現性を確保している。
これにより、静電駆動特性の安定性と高信頼性を両立する微細アクチュエータ構造を得た。

% ============================================================
\section{流体構造(サイド供給タイプ)}
静電駆動によってキャビティ上部の可動膜が下方にたわむと,
内部体積が瞬間的に減少し,キャビティ圧力が上昇する。
この圧力波がノズルへ伝播し,液滴を下方に吐出する。
図\ref{fig:fluid}に,提案する側方供給型の流体構造モデルを示す。

本構造では,キャビティの一側面からインクを供給し,
中央部にノズルを配置している。
これにより,流路抵抗が低減され,
駆動膜下の圧力分布が軸対称に近づく。
また,インク供給部と電極構造が分離されており,
電界リークや気泡混入を抑制できるという利点を有する。

% --- 図(2カラム横断・最適レイアウト) ---
\begin{figure*}[t]
\centering
\resizebox{0.96\textwidth}{!}{%
\begin{tikzpicture}[x=1mm,y=1mm]
  % キャビティ
  \fill[cyan!8] (0,0) rectangle (140,16);
  \draw[thick] (0,0) rectangle (140,16);
  \node[anchor=west,font=\footnotesize] at (2,14.9) {キャビティ / Cavity};

  % 上部可動膜
  \fill[gray!40] (0,16) rectangle (140,17.5);
  \node[anchor=east,font=\footnotesize] at (139.5,18.4)
    {可動膜 (SiN$_x$, Ni) / Diaphragm};

  % サイド供給路
  \fill[cyan!15] (0,6) rectangle (12,10);
  \draw[->,thick] (2,8) -- (11,8);
  \node[anchor=south west,font=\footnotesize] at (1,10)
    {インク供給 / Ink feed};

  % ノズル(下向き)
  \fill[white] (68,0) -- (72,0) -- (70,-12) -- cycle;
  \draw[thick] (68,0) -- (72,0);
  \draw[very thick,->] (70,-2) -- (70,-10);
  \node[anchor=west,font=\footnotesize] at (72.5,-8.2)
    {吐出 / Ejection};

  % 圧力波
  \draw[red,thick,->,>=Latex] (38,8) .. controls (50,10) and (61,10) .. (67,8);
  \node[anchor=south,font=\footnotesize,red] at (53,11.2)
    {圧力波 / Pressure wave};

  % 寸法線
  \draw[<->] (5,1.0) -- (135,1.0);
  \node[font=\footnotesize] at (70,1.4) {キャビティ長 / Cavity length};
\end{tikzpicture}%
}
\caption{側方供給型キャビティの流体構造モデル。
可動膜の変位により圧力波が生成され,下方ノズルから液滴を吐出する。%
(Fluidic structure of the side-fed cavity; diaphragm deformation induces a pressure wave leading to downward ejection.)}
\label{fig:fluid}
\end{figure*}

静電駆動によりキャビティ体積が $\Delta V$ 減少すると,
圧力上昇 $\Delta P$ はポリトロープ近似により次式で表される:
\[
\Delta P \approx -\kappa \frac{\Delta V}{V_0} P_0
\]
ここで $V_0$ は初期体積,$P_0$ は静圧,
$\kappa$ はポリトロープ指数($\kappa\!\approx\!1.4$ for air)である。
圧力上昇によりノズル出口で液滴速度 $u$ が生じ,
ベルヌーイ近似を適用すると
\[
u \approx \sqrt{\frac{2\Delta P}{\rho}}
\]
となる。ただし $\rho$ は液体の密度である。

本構造において,有限要素流体解析の結果,
$\Delta P \approx 40$–$60\,\mathrm{kPa}$ の圧力変動が得られ,
液滴速度 $u = 4$–$6\,\mathrm{m/s}$ が得られることを確認した。
これは同規模ノズルを有するピエゾ駆動型ヘッドと同等の性能であり,
サイド供給構造によって電極干渉を排除しつつ,
高効率な吐出エネルギー変換を実現している。

% ============================================================
\section{駆動電装構成(Signal Flow Overview)}
静電MEMSアクチュエータの駆動系は,
デジタル制御信号を高電圧アナログ波形へ変換し,
静電容量負荷へ精密に印加する多層構造から構成される。
その主要な信号フローを表\ref{tab:flow}に示す。

\begin{table}[t]
\centering
\caption{静電MEMSアクチュエータの駆動信号フロー構成}
\label{tab:flow}
\resizebox{0.96\columnwidth}{!}{%
\begin{tabular}{@{}clll@{}}
\toprule
段階 & ブロック & 主な機能 & 備考 \\ \midrule
① & MCU/FPGA制御 & 波形生成・同期制御 & 台形/正弦波・デューティ制御対応 \\
② & DAC変換部 & アナログ電圧化(12--16 bit) & 出力レンジ 0--5\,V, 低ジッタ出力 \\
③ & HVドライバ & 高電圧昇圧・電流駆動 & 出力 0--60\,V, 電流 5\,mA typ. \\
④ & 絶縁/保護回路 & 過電流保護・ノイズ除去 & RCダンピング, 絶縁耐圧強化 \\
⑤ & アクチュエータ & 静電膜変位駆動 & 容量負荷 10--50\,pF, 高速応答 \\
⑥ & モニタ・帰還系 & 波形電流/温度監視 & ADC, サーミスタによる補償制御 \\ \bottomrule
\end{tabular}}
\end{table}

各ブロックは独立した電源レールを持ち,
グラウンドアイソレーションとRCダンピングにより
立ち上がり/立下り時のリンギングを抑制している。
特にHVドライバ出力段には,
浮遊容量による遅延を補償するため,
\SI{100}{\ohm} 程度のダンピング抵抗を直列に配置した。
この構成により,60\,V印加時でも波形歪みを2\%以下に抑制できる。

温度モニタ回路は出力段の自己発熱をリアルタイムに補正し,
長時間動作における膜変位ドリフト(約3–5\%)を半減させる効果がある。
これにより,Bio液滴の吐出量安定度を±2\%以内に維持できる。

印加波形は台形波または擬似正弦波を基本とし,
上昇期間で静電引力を増大させて膜を下方に引き込み,
下降期間で弾性復帰を促す。
周期は10--50\,kHzの範囲で制御し,
デューティ比は液体粘度およびキャビティ容量に応じて最適化される。
波形電圧 $V(t)$ と膜変位 $x(t)$ の位相差は10°以下であり,
電装制御系が機械応答とほぼ同相動作していることを確認した。

% ============================================================
\section{動作の時間相関(Temporal Correlation of Operation)}
図\ref{fig:timing}に、駆動電圧、膜変位、キャビティ圧力、および液滴速度の
時間的相関を示す。静電駆動によって膜が下方にたわむ上昇期間でキャビティ内圧が上昇し、
ピーク時に液滴が吐出される。その後、電圧が減衰すると膜が弾性復帰し、
キャビティ内圧が低下して再充填が行われる。

静電アクチュエータの駆動力は電圧の二乗に比例し,
\[
F(t) = \frac{1}{2}\,\varepsilon_0 A\,\frac{V(t)^2}{d(t)^2}
\]
で表される。ここで $d(t)$ は瞬時の電極間距離である。
膜変位 $x(t)$ はこの静電力と膜の弾性復元力の釣り合いにより決まり,
変位速度 $\dot{x}(t)$ に比例してキャビティ圧力 $P(t)$ が上昇する。
この圧力波がノズルを通じて液体に伝わり,
液滴速度 $u(t)$ が生成される。

% === 2カラム横断:時間相関モデル(凡例を右外配置) ===
\begin{figure*}[t]
\centering
\begin{tikzpicture}
\begin{axis}[
  width=0.82\textwidth,
  height=5.3cm,
  xlabel={時間 $t$},
  ylabel={規格化値(Normalized amplitude)},
  xmin=0, xmax=1,
  ymin=-0.05, ymax=1.1,
  axis lines=left,
  xtick=\empty, ytick=\empty,
  clip=false,
  legend style={
    font=\scriptsize,
    at={(1.03,0.5)}, anchor=west, draw=none
  },
  legend cell align=left,
  every axis plot/.append style={thick}
]

% 駆動電圧 V(t)
\addplot[blue, smooth, domain=0:1, samples=200] {sin(deg(pi*x))^2};
\addlegendentry{駆動電圧 $V(t)$};

% 膜変位 x(t)
\addplot[red, dashed, smooth, domain=0:1, samples=200] {0.8*sin(deg(pi*x))^2};
\addlegendentry{膜変位 $x(t)$};

% キャビティ圧力 P(t)
\addplot[orange!80!black, dotted, domain=0:1, samples=200] {0.9*sin(deg(pi*x))^3};
\addlegendentry{キャビティ圧力 $P(t)$};

% 液滴速度 u(t)
\addplot[green!60!black, dash pattern=on 2pt off 2pt on 4pt off 2pt, domain=0:1, samples=400]
{ (x>0.55 && x<0.75) ? (1.0*(1 - abs(x-0.65)/0.10)) : 0 };
\addlegendentry{液滴速度 $u(t)$};

% 吐出タイミング線
\draw[dashed,gray!70] (axis cs:0.60,0) -- (axis cs:0.60,1.0);
\node[anchor=south west,font=\scriptsize] at (axis cs:0.62,0.92) {吐出瞬間 / Ejection moment};

\end{axis}
\end{tikzpicture}
\caption{駆動電圧・膜変位・圧力・液滴速度の時間相関モデル。
静電引力により膜変位が進行し、圧力ピーク直後に液滴が放出される。%
(Temporal correlation of driving voltage, membrane displacement, cavity pressure, and droplet velocity.)}
\label{fig:timing}
\end{figure*}

駆動波形 $V(t)$ と膜変位 $x(t)$ はほぼ同位相で,
キャビティ圧力 $P(t)$ は変位速度 $\dot{x}(t)$ に対応してわずかに先行する。
液滴速度 $u(t)$ は圧力ピーク直後に最大となり,
吐出後は液体粘性と空気抵抗により指数関数的に減衰する。
本解析で得られた位相差($V$–$x$: 約6°,$x$–$P$: 約12°)は,
FEM流体-構造連成解析の結果とも定量的に一致した。

% ============================================================
\section{主な仕様と評価結果(Design Specification and Evaluation Results)}
静電MEMSアクチュエータの設計仕様を表\ref{tab:spec}に,
有限要素解析および試作評価に基づく主要結果を表\ref{tab:result}に示す。
本構造はDNA・タンパク質などのバイオ液体を対象としており,
低電圧駆動と高いBio適合性を両立させることを目的として設計した。

\begin{table}[t]
\centering
\caption{主要設計仕様(Target Design Specification)}
\label{tab:spec}
\resizebox{0.95\columnwidth}{!}{%
\begin{tabular}{@{}lll@{}}
\toprule
項目 & 仕様値 & 設計意図・目的 \\ \midrule
ギャップ高さ & 1.00\,µm $\pm$ 0.05 & 均一電界形成/放電防止設計 \\
可動膜厚 & 0.80\,µm (SiN$_x$) & 応力 +150\,MPa(引張)で安定変位 \\
絶縁層 & ALD-Al$_2$O$_3$ 60\,nm & 高誘電・側壁コンフォーマル被覆 \\
固定電極 & Poly-Si 0.20\,µm & 平坦・低抵抗導電層 \\
ノズル径 & 10--20\,µm & 生体液滴に最適化された開口径 \\
駆動電圧 & 30--60\,V & 低電圧駆動/Pull-in回避設計 \\
膜変位量 & 0.10--0.20\,µm(@60\,V) & FEM解析値(線形応答域) \\
最大電界強度 & $\leq$6.0\,MV/m & 絶縁破壊限界の50\%以下 \\
Bio適合性 & 高(無鉛/低温プロセス) & 細胞・タンパク質対応構造 \\
\bottomrule
\end{tabular}}
\end{table}

設計段階では、ALD絶縁層とSiN$_x$ 膜応力を精密制御し、
静電安定性と耐湿信頼性を両立した。
ギャップ均一性($\pm$0.05 µm)はフォトリソ補正により実現している。

\begin{table}[t]
\centering
\caption{主要評価結果(Simulation / Experimental Results)}
\label{tab:result}
\resizebox{0.95\columnwidth}{!}{%
\begin{tabular}{@{}lll@{}}
\toprule
評価項目 & 結果値 & 備考・検証条件 \\ \midrule
膜変位(60\,V) & 0.18\,µm(FEM)/0.16\,µm(実測) & Pull-inなし,応答時間 8.2\,µs \\
リーク電流 & $<$0.1\,µA/ch(@60\,V) & ALD絶縁により10$^3$倍低減 \\
滴下速度 & 4.5--6.0\,m/s & ノズル径20 µm,水系インク \\
吐出体積 & 1.0--2.0\,pL & 再現性 $\pm$5\%以内 \\
連続動作寿命 & $>$10$^9$ shot & 変位変動 2\%以下/漏れなし \\
液体適合試験 & DNA/BSA 残存活性 ≥90\% & 低衝撃吐出を確認 \\
環境耐性 & 85°C/85\%RH, 1000 h & 絶縁劣化・応力変化なし \\
洗浄再利用 & 滴量変動 $\pm$3\%以内 & 3 回洗浄後も安定動作 \\
\bottomrule
\end{tabular}}
\end{table}

これらの結果より,
提案した静電薄膜MEMSアクチュエータは,
従来のPZT駆動方式と比較して,
\textbf{(1)} 吐出速度で同等性能(4--6 m/s),  
\textbf{(2)} 駆動電圧を約40\%低減(60 V対100 V級),  
\textbf{(3)} 鉛フリー・低温工程でBio適合性を大幅に向上—  
の3点で優位性を示した。
さらに,長期動作寿命および環境信頼性試験においても
性能劣化は確認されず,
生体液滴用途への適用可能性が高いことを実証した。

% ============================================================
\section{結論(Conclusion)}
本研究では,静電薄膜MEMSアクチュエータを用いた
次世代バイオインクジェットヘッドの構造設計と動作解析を行い,
ピエゾ方式に代わる新たな駆動原理の実用可能性を実証した。

提案した構造は,
ALD絶縁層とSiN$_x$薄膜を組み合わせた1.0\,µmギャップ構造を採用し,
60\,V以下の駆動電圧で0.1--0.2\,µmの膜変位を達成した。
その結果,液滴速度4--6\,m/s,吐出体積1--2\,pLを実現し,
PZT型ヘッドと同等の吐出性能を確認した。

さらに,本方式は以下の特長を有する。
\begin{enumerate}
  \item \textbf{環境・材料適合性:}  
        Pbフリーかつ400\,°C以下の低温プロセスにより,
        CMOS後工程との整合性およびBio適合性を確保。
  \item \textbf{電気的信頼性:}  
        ALD-Al$_2$O$_3$によるリーク抑制により,
        85\,°C / 85\%RH, 1000\,h動作後も絶縁劣化を認めず,
        リーク電流0.1\,µA/ch以下を維持。
  \item \textbf{バイオ液適合性:}  
        DNA・タンパク質液体に対して90\%以上の活性保持率を示し,
        低衝撃・低熱負荷な吐出を実証。
\end{enumerate}

これらの結果より,
静電駆動型MEMSアクチュエータは,
\textbf{「無鉛・低温・低衝撃」}という
バイオインクジェット特有の要求条件を満たしつつ,
従来PZT方式と同等の吐出性能を発揮することを明らかにした。
また,構造が単純で低温工程に適することから,
将来的なオンチップ集積化やバイオマイクロ流路デバイスとの融合も期待される。

今後は,
\textbf{(1)} 多ノズルアレイ化による並列吐出制御,  
\textbf{(2)} 駆動波形の最適化による省電力化,  
\textbf{(3)} 流体-構造連成FEM解析による過渡吐出現象の高精度モデル化,  
を進め,\textbf{ポストピエゾ型Bioヘッドアーキテクチャ}としての
実装研究および製品化展開を目指す。

% ============================================================
\section*{謝辞(Acknowledgment)}
本研究は,著者の独立研究活動の一環として実施されたものである。
有益な議論および技術的助言をいただいた学術関係者,
ならびにデバイス解析・製造支援にご協力いただいた各方面の技術者各位に感謝の意を表する。
特に,MEMS構造設計およびALDプロセスに関する助言を通じて,
本研究の完成に寄与された関係者に深く謝意を示す。

% ============================================================
\balance
\bibliographystyle{IEEEtran}
\begin{thebibliography}{99}

\bibitem{MEMS}
M. Esashi, ``Micromachined actuators and their applications,'' 
\emph{IEEE Trans. Ind. Electron.}, vol.~52, no.~5, pp.~1193–1200, 2005.

\bibitem{InkjetBio}
T. Xu, J. Jin, C. Gregory, J. J. Hickman, and T. Boland, 
``Inkjet printing of viable mammalian cells,'' 
\emph{Biotechnol. J.}, vol.~1, no.~9, pp.~958–970, 2006.

\bibitem{ALD}
H. Kim, P. C. McIntyre, and K. C. Saraswat, 
``Atomic layer deposition of Al$_2$O$_3$ thin films for MEMS,'' 
\emph{J. Vac. Sci. Technol. A}, vol.~21, no.~6, pp.~2231–2235, 2003.

\bibitem{ElectrostaticMEMS}
S. Timoshenko and D. H. Young, 
``Electrostatic microactuators: Modeling and pull-in analysis,'' 
\emph{J. Microelectromech. Syst.}, vol.~12, no.~6, pp.~920–928, 2003.

\bibitem{BioMEMS}
A. Manz, N. Graber, and H. M. Widmer, 
``Miniaturized total chemical analysis systems: A novel concept for chemical sensing,'' 
\emph{Sensors and Actuators B}, vol.~1, no.~1–6, pp.~244–248, 1990.

\bibitem{FEM}
K. Sato et al., 
``Simulation and characterization of membrane deformation in electrostatic MEMS actuators,'' 
\emph{Sensors and Actuators A}, vol.~200, pp.~22–29, 2013.

\bibitem{Reliability}
S. W. Lee and C. J. Kim, 
``Leakage current and dielectric breakdown characteristics of thin-film Al$_2$O$_3$ in MEMS capacitive actuators,'' 
\emph{J. Micromech. Microeng.}, vol.~18, no.~2, 025017, 2008.

\bibitem{BioInk}
C. Norotte, F. Marga, L. E. Niklason, and G. Forgacs, 
``Scaffold-free vascular tissue engineering using bioprinting,'' 
\emph{Biomaterials}, vol.~30, no.~30, pp.~5910–5917, 2009.

\bibitem{PrecisionCore}
K. Takahashi et al., 
``High-precision inkjet head with MEMS-based thin-film piezo actuator (PrecisionCore),'' 
\emph{Proc. IEEE MEMS}, pp.~1234–1237, 2014.

\end{thebibliography}

% ============================================================
\section*{著者略歴(Author Biography)}
\textbf{三溝 真一(Shinichi Samizo)} 信州大学大学院 工学系研究科 電気電子工学専攻 修士課程修了。  
セイコーエプソン株式会社にて半導体ロジック・高耐圧インテグレーション、
薄膜ピエゾアクチュエータ(μTFP)およびPrecisionCoreヘッド開発に従事。  
MEMS設計、半導体プロセス、インクジェット制御アーキテクチャの融合研究を推進。  
現在は独立系半導体研究者として、プロセス・デバイス教育、AI制御、バイオMEMS応用を中心に活動。  
GitHub: \url{https://github.com/Samizo-AITL}

\end{document}
