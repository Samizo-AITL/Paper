\section{Results and Discussion}
System-level simulation was performed with representative AI inference workloads to evaluate hybrid HBM+FeRAM memory.

\subsection{Standby Power}
Migrating cold data and checkpoints to the FeRAM-backed tier yields more than 30\% reduction in standby power.  
This reduction arises from suppressing periodic DRAM refresh for inactive regions.

\subsection{Resume Latency}
FeRAM allows direct restore of checkpoints without full DRAM wake-up.  
Resume latency is reduced to the $\mu$s range, enabling near-instant resume after power gating and improving energy efficiency for mobile edge AI.

\subsection{Endurance}
FeRAM endurance of $10^{12}$~writes/year is sufficient to support frequent checkpoint traffic in AI accelerators.  
This capability ensures practical deployment without premature device wear-out.

\begin{figure}[!t]
\centering
\begin{semilogxaxis}[
    width=0.75\linewidth,
    height=5.0cm,
    xlabel={Access time (ns)},
    ylabel={Retention (s)},
    xmin=1, xmax=100,
    ymin=1, ymax=1e4,
    grid=both,
    legend style={at={(0.02,0.02)}, anchor=south west, font=\footnotesize, fill=white, draw=black}
]
  % FeRAM points
  \addplot[only marks, mark=o, mark size=2.5pt, blue] coordinates {
    (2,1e2) (5,5e2) (10,1e3) (20,2e3) (50,5e3)
  };
  % HBM points
  \addplot[only marks, mark=square*, mark size=2.5pt, red] coordinates {
    (2,10) (5,20) (10,50)
  };
  \legend{FeRAM (typ), HBM DRAM (typ)}
\end{semilogxaxis}
\caption{Access time vs. retention. Red squares: HBM; blue circles: FeRAM. Axes restricted to show practical design space.}
\label{fig:retention_vs_access}
\end{figure}
