\section{Device and Process Integration}
HBM DRAM stacks are fabricated with high-temperature capacitor anneals ($>700~^\circ$C), 
while FeRAM/FeFET requires lower-temperature processing ($\sim$400~^\circ$C) to preserve ferroelectric o-phase in HfO$_2$.
This incompatibility makes monolithic integration difficult at present.

\subsection{Chiplet-based Integration (Practical Solution)}
The most feasible near-term path is chiplet integration:
HBM stacks and FeRAM/FeFET dies are fabricated in their respective optimized flows and co-integrated on a silicon interposer with $\mu$-bump connections.
This allows:
\begin{itemize}
  \item HBM to deliver bandwidth $>$300 GB/s,
  \item FeRAM to hold checkpoints, metadata, and cold data persistently,
  \item Refresh traffic reduction in DRAM.
\end{itemize}

\subsection{Monolithic Integration (Research Challenge)}
A future direction is embedding FeFET arrays inside the HBM logic base die.
Both DRAM capacitor HfO$_2$ and FeFET gate-stack HfO$_2$ coexist, but require conflicting anneals.
Possible research enablers include selective annealing, dopant modulation, or stress engineering.
Today, monolithic HBM+FeFET remains a research challenge.

% Fig.1
\begin{figure}[!t]
\centering
\includegraphics[width=0.9\linewidth]{figs/fig1_systemdk.pdf}
\caption{Chiplet integration: CPU/accelerator with HBM and FeRAM chiplet on an interposer, supervised by SystemDK for top-down architecture/interface/package co-design.}
\label{fig:system_schematic}
\end{figure}
