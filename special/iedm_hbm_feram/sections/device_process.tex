\section{Device and Process Integration}
HBM DRAM stacks are typically fabricated with high-temperature capacitor anneals ($>700~^\circ$C), 
whereas FeRAM/FeFET devices require lower-temperature processing ($\sim$400~^\circ$C) to stabilize the ferroelectric o-phase in HfO$_2$. 
This incompatibility between high- and low-temperature requirements currently hinders monolithic integration.

\subsection{Chiplet-based Integration (Practical Solution)}
The most practical near-term approach is chiplet-based integration:  
HBM stacks and FeRAM/FeFET dies are fabricated in their respective optimized flows and co-integrated on a silicon interposer using $\mu$-bump connections.  
This architecture enables:
\begin{itemize}
  \item High-bandwidth operation from HBM ($>$300~GB/s),
  \item Persistent storage of checkpoints, metadata, and cold data in FeRAM,
  \item Reduction of refresh-induced traffic in DRAM.
\end{itemize}

\subsection{Monolithic Integration (Research Challenge)}
A longer-term research direction is embedding FeFET arrays within the HBM logic base die.  
In principle, DRAM capacitor HfO$_2$ and FeFET gate-stack HfO$_2$ could coexist; however, their annealing requirements remain incompatible.  
Possible enablers include selective or dual-step annealing, dopant modulation, or stress engineering.  
At present, monolithic HBM+FeFET integration remains an open challenge for device and process research.

\begin{figure}[!t]
\centering
\begin{tikzpicture}[font=\footnotesize, >=Stealth]
  % Styles
  \tikzset{
    blk/.style={draw=black, rounded corners, fill=black!6, minimum width=2.8cm, minimum height=8mm, align=center},
    ctrl/.style={draw=black, rounded corners, fill=black!10, minimum width=9cm, minimum height=8mm, align=center},
    sysdk/.style={draw=black, thick, rounded corners, fill=black!4, align=center, inner sep=3pt, minimum width=9.5cm}
  }

  % SystemDK (top)
  \node[sysdk] (sdk) { \textbf{SystemDK: Memory Ctl. + Policies} };

  % CPU, HBM, FeRAM (row 2)
  \node[blk, below=1.4cm of sdk, xshift=-4.0cm] (cpu) {CPU / Accelerator};
  \node[blk, right=3.5cm of cpu] (hbm) {HBM (DRAM)\\High BW};
  \node[blk, right=3.5cm of hbm] (feram) {FeRAM Chiplet\\Persistent};

  % Controller (row 3)
  \node[ctrl, below=1.6cm of hbm] (ctrl) {Policy Engine: Tiering / Checkpoints / Refresh Mgmt};

  % Arrows
  \draw[->, thick] (cpu) -- (hbm);
  \draw[->, thick] (hbm) -- (feram);
  \draw[->, thick] (ctrl.north) -- (cpu.south);
  \draw[->, thick] (ctrl.north) -- (hbm.south);
  \draw[->, thick] (ctrl.north) -- (feram.south);

  % Supervision
  \draw[->, thick] (sdk.south) -- (hbm.north);
  \draw[->, thick] (sdk.south west) .. controls +(-1.5,-0.2) .. (cpu.north);
  \draw[->, thick] (sdk.south east) .. controls +(1.5,-0.2) .. (feram.north);
\end{tikzpicture}
\caption{Minimal chiplet integration view: CPU connects to HBM for bandwidth, FeRAM holds persistent data, and the policy engine manages tiering/checkpoints.}
\label{fig:minimal_chiplet}
\end{figure}
