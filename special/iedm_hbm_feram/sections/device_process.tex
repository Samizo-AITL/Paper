\section{Device and Process Integration}
HBM DRAM stacks are typically fabricated with high-temperature capacitor anneals ($>700~^\circ$C), 
whereas FeRAM/FeFET devices require lower-temperature processing ($\sim$400~^\circ$C) to stabilize the ferroelectric o-phase in HfO$_2$. 
This incompatibility between high- and low-temperature requirements currently hinders monolithic integration.

\subsection{Chiplet-based Integration (Practical Solution)}
The most practical near-term approach is chiplet-based integration:  
HBM stacks and FeRAM/FeFET dies are fabricated in their respective optimized flows and co-integrated on a silicon interposer using $\mu$-bump connections.  
This architecture enables:
\begin{itemize}
  \item High-bandwidth operation from HBM ($>$300~GB/s),
  \item Persistent storage of checkpoints, metadata, and cold data in FeRAM,
  \item Reduction of refresh-induced traffic in DRAM.
\end{itemize}

\subsection{Monolithic Integration (Research Challenge)}
A longer-term research direction is embedding FeFET arrays within the HBM logic base die.  
In principle, DRAM capacitor HfO$_2$ and FeFET gate-stack HfO$_2$ could coexist; however, their annealing requirements remain incompatible.  
Possible enablers include selective or dual-step annealing, dopant modulation, or stress engineering.  
At present, monolithic HBM+FeFET integration remains an open challenge for device and process research.

% ===== Fig.1: System-level schematic with SystemDK (column-safe) =====
\begin{figure}[!t]
\centering
\begin{tikzpicture}[font=\scriptsize, >=Stealth, node distance=8mm]
  % ===== styles (1-column safe sizes) =====
  \tikzset{
    blk/.style={draw, rounded corners, fill=black!6,
                minimum width=2.6cm, minimum height=6mm, align=center},
    ctrlblk/.style={draw, rounded corners, fill=black!6,
                minimum width=7.6cm, minimum height=6mm, align=center},
    note/.style={draw, rounded corners, fill=black!2, align=left,
                inner sep=3pt, text width=6.8cm},
    arrow/.style={->, thick},
    sysdk/.style={draw, thick, rounded corners, fill=black!4, align=center,
                inner sep=3pt, minimum width=7.8cm}
  }

  % ===== SystemDK box (top) =====
  \node[sysdk] (sdk) {\textbf{SystemDK Co-Design Framework}\\
  (Architecture / Interfaces / Package / OS Policies)};

  % ===== main blocks (中央基準で左右対称に) =====
  \node[blk, below=10mm of sdk, xshift=-2.8cm] (cpu)  {CPU / Accelerator\\(power-gated)};
  \node[blk, right=1.9cm of cpu]                  (hbm)  {HBM (DRAM)\\High bandwidth};
  \node[blk, right=1.9cm of hbm]                  (feram){FeRAM chiplet\\Persistent tier};

  % ===== controller =====
  \node[ctrlblk, below=10mm of hbm] (ctrl) {Memory Controller \& Policy Engine};

  % ===== short policies note =====
  \node[note, below=6mm of ctrl] (pol) {
    \textbf{Policies:} Tiering (Hot/Warm/Cold); checkpoints $\to$ FeRAM;
    DRAM refresh reduction; wear/ECC/telemetry management.
  };

  % ===== data arrows =====
  \draw[arrow] (cpu) -- node[above]{req/data} (hbm);
  \draw[arrow] (hbm) -- node[above]{ckpt/meta} (feram);
  \draw[arrow] (ctrl.north) -- ++(-2.6,0) |- (cpu.south);
  \draw[arrow] (ctrl.north) -- (hbm.south);
  \draw[arrow] (ctrl.north) -- ++(2.6,0) |- (feram.south);

  % ===== supervision arrows (欄外に出ない直線経路) =====
  \draw[arrow] (sdk.south) |- (cpu.north);
  \draw[arrow] (sdk.south) -- (hbm.north);
  \draw[arrow] (sdk.south) |- (feram.north);
\end{tikzpicture}
\caption{Chiplet-level integration supervised by \textbf{SystemDK}. CPU/Accelerator with HBM and FeRAM chiplet co-integrated on an interposer.}
\label{fig:system_schematic_sdk}
\end{figure}
