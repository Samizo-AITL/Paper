\section{Device and Process Integration}
HBM DRAM stacks are typically fabricated with high-temperature capacitor anneals ($>700~^\circ$C), 
whereas FeRAM/FeFET devices require lower-temperature processing ($\sim$400~^\circ$C) to stabilize the ferroelectric o-phase in HfO$_2$. 
This incompatibility between high- and low-temperature requirements currently hinders monolithic integration.

\subsection{Chiplet-based Integration (Practical Solution)}
The most practical near-term approach is chiplet-based integration:  
HBM stacks and FeRAM/FeFET dies are fabricated in their respective optimized flows and co-integrated on a silicon interposer using $\mu$-bump connections.  
This architecture enables:
\begin{itemize}
  \item High-bandwidth operation from HBM ($>$300~GB/s),
  \item Persistent storage of checkpoints, metadata, and cold data in FeRAM,
  \item Reduction of refresh-induced traffic in DRAM.
\end{itemize}

\subsection{Monolithic Integration (Research Challenge)}
A longer-term research direction is embedding FeFET arrays within the HBM logic base die.  
In principle, DRAM capacitor HfO$_2$ and FeFET gate-stack HfO$_2$ could coexist; however, their annealing requirements remain incompatible.  
Possible enablers include selective or dual-step annealing, dopant modulation, or stress engineering.  
At present, monolithic HBM+FeFET integration remains an open challenge for device and process research.

% ===== Fig.1: Minimal chiplet schematic =====
\begin{figure}[!t]
\centering
\resizebox{\columnwidth}{!}{%
\begin{tikzpicture}[font=\footnotesize, >=Stealth]
  \tikzset{
    blk/.style={draw, rounded corners, fill=black!7,
                minimum width=28mm, minimum height=8mm, align=center},
    big/.style={draw, rounded corners, fill=black!10,
                minimum width=86mm, minimum height=8mm, align=center},
    arr/.style={->, thick}
  }

  % 行1: SystemDK
  \node[big] (sysdk) { \textbf{SystemDK} (Architecture / Interfaces / Package / OS policies) };

  % 行2: CPU - HBM - FeRAM
  \matrix[column sep=14mm, row sep=10mm] at (sysdk.south) [below] {
    \node[blk] (cpu)  {CPU / Accelerator}; &
    \node[blk] (hbm)  {HBM (DRAM)\\High bandwidth}; &
    \node[blk] (nvm)  {FeRAM Chiplet\\Persistent tier}; \\
  };

  % 行3: Controller
  \node[big, below=12mm of hbm] (ctrl) {Memory Controller \& Policy Engine};

  % 矢印
  \draw[arr] (cpu) -- (hbm);
  \draw[arr] (hbm) -- (nvm);
  \draw[arr] (ctrl.north) -- (cpu.south);
  \draw[arr] (ctrl.north) -- (hbm.south);
  \draw[arr] (ctrl.north) -- (nvm.south);

  % SystemDK supervision
  \draw[arr] (sysdk.south) -- (hbm.north);
  \draw[arr] ([xshift=-28mm]sysdk.south) .. controls +(0,-6mm) .. (cpu.north);
  \draw[arr] ([xshift=+28mm]sysdk.south) .. controls +(0,-6mm) .. (nvm.north);
\end{tikzpicture}}
\caption{Minimal chiplet integration view: HBM provides bandwidth, FeRAM holds persistent data, and the controller enforces tiering/checkpoints under SystemDK supervision.}
\label{fig:minimal_chiplet}
\end{figure}
