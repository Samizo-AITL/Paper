\section{Device and Process Integration}
HBM DRAM stacks are typically fabricated with high-temperature capacitor anneals ($>700~^\circ$C),
whereas FeRAM/FeFET devices require lower-temperature processing ($\sim$400~^\circ$C) to stabilize
the ferroelectric o-phase in HfO$_2$. This thermal budget mismatch currently hinders monolithic integration.

\subsection{Chiplet-based Integration (Practical Solution)}
The most practical near-term approach is chiplet-based integration:
HBM stacks and FeRAM/FeFET dies are fabricated in their respective optimized flows
and co-integrated on a silicon interposer using $\mu$-bump connections. This architecture enables:
\begin{itemize}
  \item High-bandwidth operation from HBM ($>$300~GB/s),
  \item Persistent storage of checkpoints, metadata, and cold data in FeRAM,
  \item Reduction of refresh-induced traffic in DRAM.
\end{itemize}

\subsection{Monolithic Integration (Research Challenge)}
A longer-term research direction is embedding FeFET arrays within the HBM logic base die.
In principle, DRAM capacitor HfO$_2$ and FeFET gate-stack HfO$_2$ could coexist; however,
their annealing requirements remain incompatible. Potential enablers include selective/dual-step annealing,
dopant modulation, or stress engineering. At present, monolithic HBM+FeFET integration
remains an open challenge for device and process research.

% ===== Fig.1: Minimal chiplet view (no overlap) =====
\begin{figure}[t]
\centering
\begin{tikzpicture}[>=Stealth, font=\footnotesize]
  % styles
  \tikzset{
    blk/.style   ={draw, rounded corners, fill=black!6, minimum height=7mm, align=center},
    ctrl/.style  ={draw, rounded corners, fill=black!10, minimum height=6mm, align=center},
    arr/.style   ={->, thick}
  }

  % controller/policy (上段に独立配置)
  \node[ctrl, minimum width=7.5cm] (pol) {Memory Ctl \& Policy Engine};

  % 下段の3ブロック(左右に間隔を空けて固定座標で配置)
  \node[blk, below=7mm of pol, xshift=-3.2cm, minimum width=3.0cm] (cpu)  {CPU / Accelerator};
  \node[blk, below=7mm of pol, minimum width=3.0cm]                  (hbm)  {HBM (DRAM)};
  \node[blk, below=7mm of pol, xshift=+3.2cm, minimum width=3.0cm] (feram){FeRAM Chiplet};

  % データ/制御の矢印(重ならないよう一直線)
  \draw[arr] (cpu.east) -- node[above]{bandwidth} (hbm.west);
  \draw[arr] (hbm.east) -- node[above]{ckpt/metadata} (feram.west);

  % ポリシーから各ブロックへ(上から真下に)
  \draw[arr] (pol.south) |- (cpu.north);
  \draw[arr] (pol.south) -- (hbm.north);
  \draw[arr] (pol.south) |- (feram.north);
\end{tikzpicture}
\caption{Minimal chiplet integration view: CPU connects to HBM for bandwidth; FeRAM holds persistent data; policies manage tiering/checkpoints.}
\label{fig:system_minimal}
\end{figure}
