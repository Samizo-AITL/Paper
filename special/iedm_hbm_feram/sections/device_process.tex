\section{Device and Process Integration}
HBM DRAM stacks are typically fabricated with high-temperature capacitor anneals ($>700~^\circ$C), 
whereas FeRAM/FeFET devices require lower-temperature processing ($\sim$400~^\circ$C) to stabilize the ferroelectric o-phase in HfO$_2$. 
This incompatibility between high- and low-temperature requirements currently hinders monolithic integration.

\subsection{Chiplet-based Integration (Practical Solution)}
The most practical near-term approach is chiplet-based integration:  
HBM stacks and FeRAM/FeFET dies are fabricated in their respective optimized flows and co-integrated on a silicon interposer using $\mu$-bump connections.  
This architecture enables:
\begin{itemize}
  \item High-bandwidth operation from HBM ($>$300~GB/s),
  \item Persistent storage of checkpoints, metadata, and cold data in FeRAM,
  \item Reduction of refresh-induced traffic in DRAM.
\end{itemize}

\subsection{Monolithic Integration (Research Challenge)}
A longer-term research direction is embedding FeFET arrays within the HBM logic base die.  
In principle, DRAM capacitor HfO$_2$ and FeFET gate-stack HfO$_2$ could coexist; however, their annealing requirements remain incompatible.  
Possible enablers include selective or dual-step annealing, dopant modulation, or stress engineering.  
At present, monolithic HBM+FeFET integration remains an open challenge for device and process research.

% ===== Fig.1: Minimal chiplet view (no overlap) =====
\begin{figure}[!t]
\centering
\begin{tikzpicture}[font=\footnotesize, >=Stealth]
  % sizes
  \def\W{7.6cm}
  \def\H{0.75cm}
  % styles
  \tikzset{
    box/.style={draw, rounded corners, fill=black!7, minimum height=\H, align=center},
    head/.style={draw, fill=black!4, minimum height=0.55cm, align=center},
    arrow/.style={->, thick}
  }

  % header
  \node[head, minimum width=\W] (head) {Memory Ctrl \& Policies};

  % three blocks in a row (fixed coordinates to avoid overlap)
  \node[box, anchor=north west, minimum width=2.4cm] (cpu)  at ([yshift=-3mm]head.south west) {CPU / Accel};
  \node[box, anchor=north,      minimum width=2.8cm] (hbm)  at ([yshift=-3mm]head.south)       {HBM (DRAM)};
  \node[box, anchor=north east, minimum width=2.4cm] (feram)at ([yshift=-3mm]head.south east)  {FeRAM Chiplet};

  % arrows
  \draw[arrow] (cpu) -- (hbm) node[midway, above]{req/data};
  \draw[arrow] (hbm) -- (feram) node[midway, above]{ckpt/meta};
  \draw[arrow] (head.south west)++(0.3,0) |- (cpu.north);
  \draw[arrow] (head.south)                    -- (hbm.north);
  \draw[arrow] (head.south east)++(-0.3,0) |- (feram.north);
\end{tikzpicture}
\caption{Minimal chiplet view: HBM provides bandwidth, FeRAM holds persistent data; the controller enforces tiering/checkpoints.}
\label{fig:system_schematic_sdk}
\end{figure}
