\section{Device and Process Integration}
HBM DRAM stacks are typically fabricated with high-temperature capacitor anneals ($>700~^\circ$C), 
whereas FeRAM/FeFET devices require lower-temperature processing ($\sim$400~^\circ$C) to stabilize the ferroelectric o-phase in HfO$_2$. 
This incompatibility between high- and low-temperature requirements currently hinders monolithic integration.

\subsection{Chiplet-based Integration (Practical Solution)}
The most practical near-term approach is chiplet-based integration:  
HBM stacks and FeRAM/FeFET dies are fabricated in their respective optimized flows and co-integrated on a silicon interposer using $\mu$-bump connections.  
This architecture enables:
\begin{itemize}
  \item High-bandwidth operation from HBM ($>$300~GB/s),
  \item Persistent storage of checkpoints, metadata, and cold data in FeRAM,
  \item Reduction of refresh-induced traffic in DRAM.
\end{itemize}

\subsection{Monolithic Integration (Research Challenge)}
A longer-term research direction is embedding FeFET arrays within the HBM logic base die.  
In principle, DRAM capacitor HfO$_2$ and FeFET gate-stack HfO$_2$ could coexist; however, their annealing requirements remain incompatible.  
Possible enablers include selective or dual-step annealing, dopant modulation, or stress engineering.  
At present, monolithic HBM+FeFET integration remains an open challenge for device and process research.

% ===== Fig.1: System-level schematic with SystemDK (TikZ) =====
\begin{figure}[!t]
\centering
\begin{tikzpicture}[font=\scriptsize, >=Stealth, node distance=1.0cm]
  % Styles
  \tikzset{
    blk/.style={draw=black, rounded corners, fill=black!6, minimum width=3.0cm, minimum height=7mm, align=center},
    note/.style={draw=black, rounded corners, fill=black!2, align=left, inner sep=3pt, text width=6.0cm},
    arrow/.style={->, thick},
    sysdk/.style={draw=black, thick, rounded corners, fill=black!4, align=center, inner sep=3pt, minimum width=9.5cm}
  }

  % SystemDK box
  \node[sysdk] (sdk) {\textbf{SystemDK Co-Design Framework}\\
  (Architecture / Interfaces / Package / OS Policies)};

  % Main blocks
  \node[blk, below=1.0cm of sdk, xshift=-3.5cm] (cpu) {CPU / Accelerator\\(Power-gated)};
  \node[blk, right=2.0cm of cpu] (hbm) {HBM (DRAM)\\High Bandwidth};
  \node[blk, right=2.0cm of hbm] (feram) {FeRAM Chiplet\\Persistent Tier};

  % Controller
  \node[blk, below=1.2cm of hbm, minimum width=7.5cm] (ctrl) {Memory Controller \& Policy Engine};

  % Policies note(中央に短く表示)
  \node[note, below=0.8cm of ctrl] (pol) {
    \textbf{Policies:}\\
    • Tiering (Hot/Warm/Cold)\\
    • Checkpoints $\to$ FeRAM\\
    • Refresh reduction\\
    • Wear/ECC mgmt.
  };

  % Data arrows
  \draw[arrow] (cpu) -- node[above]{req/data} (hbm);
  \draw[arrow] (hbm) -- node[above]{ckpt/meta} (feram);
  \draw[arrow] (ctrl.north) -- ++(-3.0,0) |- (cpu.south);
  \draw[arrow] (ctrl.north) -- (hbm.south);
  \draw[arrow] (ctrl.north) -- ++(3.0,0) |- (feram.south);

  % SystemDK supervision arrows
  \draw[arrow] (sdk.south) |- (cpu.north);
  \draw[arrow] (sdk.south) -- (hbm.north);
  \draw[arrow] (sdk.east) .. controls +(0.8,-0.2) and +(0.6,0.6) .. (feram.north);
\end{tikzpicture}
\caption{Chiplet-level integration supervised by \textbf{SystemDK}. CPU/Accelerator with HBM and FeRAM chiplet co-integrated on an interposer.}
\label{fig:system_schematic_sdk}
\end{figure}
