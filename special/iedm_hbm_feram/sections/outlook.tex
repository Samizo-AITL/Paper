\section{Future Outlook: Toward HBM+FeFET}
In the near term, chiplet integration of HBM and FeRAM offers a practical solution for mobile edge AI, balancing bandwidth and persistence. 
Future prospects include replacing FeRAM with FeFET:
\begin{itemize}
  \item \textbf{Non-destructive read}, reducing wear-out,
  \item \textbf{Higher density}, fitting within HBM logic base,
  \item \textbf{CMOS compatibility}, easing scaling to advanced nodes.
\end{itemize}

Nevertheless, true monolithic integration faces process conflicts: DRAM capacitors require high-$T$ anneals, while FeFETs demand low-$T$ stabilization. 
Overcoming this is an active research challenge.

% ===== Fig.3: Package cross-section with SystemDK (TikZ) =====
\begin{figure}[!t]
\centering
\begin{tikzpicture}[font=\footnotesize, x=1cm, y=1cm, >=Stealth]
  \tikzset{
    layer/.style={draw=black, fill=black!5, rounded corners},
    die/.style={draw=black, fill=black!8, rounded corners},
    stack/.style={draw=black, fill=black!12},
    bump/.style={circle, draw=black, fill=black!40, minimum size=2pt, inner sep=0pt},
    tsv/.style={draw=black, line width=0.3pt},
    sysdk/.style={draw=black, rounded corners, fill=black!2, align=center, inner sep=3pt}
  }
  % SystemDK label
  \node[sysdk] (sdk) at (0,1.2) {\textbf{SystemDK Co-design Framework}\\
    \scriptsize Architecture / Interfaces / Package / OS policies};
  % substrate & interposer
  \draw[layer] (-4.5,-2.2) rectangle (4.5,-1.6);
  \node at (0,-2.35) {\scriptsize Package Substrate};
  \draw[layer] (-4.2,-1.6) rectangle (4.2,-1.1);
  \node at (0,-0.95) {\scriptsize Silicon Interposer};
  % micro-bumps
  \foreach \x in {-3.5,-3.2,...,-1.5} \node[bump] at (\x,-1.1) {};
  \foreach \x in {-0.5,-0.2,...,1.5}  \node[bump] at (\x,-1.1) {};
  \foreach \x in {2.5,2.8,...,3.9}     \node[bump] at (\x,-1.1) {};
  % dies
  \draw[die] (-3.8,-0.6) rectangle (-1.2,-1.1);
  \node[align=center] at (-2.5,-0.85) {\scriptsize CPU /\\ \scriptsize Controller};
  \draw[die] (-0.8,-0.6) rectangle (1.8,-1.1);
  \node at (0.5,-0.85) {\scriptsize HBM Base Die};
  % HBM stacks + TSVs
  \foreach \y in {0.0,0.25,0.50,0.75} {
    \draw[stack] (-0.6,\y-0.6) rectangle (1.6,\y-0.35);
  }
  \foreach \x in {-0.3, 0.2, 0.7, 1.2}{
    \draw[tsv] (\x,-0.6) -- (\x,0.15);
  }
  \node[align=center] at (0.5,0.05) {\scriptsize DRAM layers\\ \scriptsize (TSVs)};
  \draw[die] (2.1,-0.6) rectangle (4.0,-1.1);
  \node[align=center] at (3.05,-0.85) {\scriptsize FeRAM\\ \scriptsize Chiplet};
  % data direction
  \draw[->, thick] (-1.2,-0.85) -- (-0.8,-0.85);
  \draw[->, thick] (1.8,-0.85) -- (2.1,-0.85);
  % SystemDK arrows(右側から回す)
  \draw[->] (sdk.east) .. controls +(1.0,-0.1) and +(0.8,0.6) .. (3.05,-0.6); % to FeRAM
  \draw[->] (sdk.south) -- (0.5,0.2);                                        % to HBM
  \draw[->] (sdk.west) .. controls +(-1.0,-0.1) and +(-0.8,0.6) .. (-2.5,-0.6); % to CPU
\end{tikzpicture}
\caption{Package cross-section: CPU/Controller, HBM DRAM stack, and FeRAM chiplet co-integrated on an interposer with \textbf{SystemDK} supervision.}
\label{fig:package_cross_section_sdk}
\end{figure}
