\begin{abstract}
High-bandwidth memory (HBM) provides the throughput required by mobile edge AI accelerators but suffers from high standby power due to periodic refresh and volatility. 
Ferroelectric memories (FeRAM/FeFET), based on HfO$_2$, offer non-volatility and fast switching, though often at higher energy cost. 
This work explores hybrid integration of HBM and FeRAM/FeFET: a near-term solution based on chiplet co-packaging, and a long-term direction toward monolithic integration. 
Results indicate that FeRAM enables substantial standby power reduction and instant resume capability, while FeFET provides a pathway for future scalability of dense, non-volatile HBM.
\end{abstract}

\section{Introduction}
Mobile edge AI demands memory subsystems that simultaneously provide:
(1) multi-hundred GB/s bandwidth, 
(2) ultra-low standby power, 
(3) near-instant resume after power gating, 
and (4) sufficient endurance for frequent checkpoints.
HBM DRAM has proven effective for (1), but its reliance on periodic refresh causes significant standby power overhead and resume latency, increasingly limiting scalability \cite{ChoiIEDM2022,KimIEDM2021}.  

Ferroelectric memories based on HfO$_2$---including FeRAM and FeFET devices---naturally address (2) and (3), while also exhibiting high endurance of $10^{12}$–$10^{13}$ cycles \cite{MuellerIEDM2012,MartinVLSI2020}. 
Recent advances in HfO$_2$-based ferroelectrics further demonstrate improved retention, reliability, and compatibility with advanced CMOS nodes \cite{NohedaNature2023}.  

This paper evaluates hybrid HBM+FeRAM/FeFET integration, emphasizing practical chiplet-based co-packaging for near-term deployment, and assessing the challenges and opportunities of monolithic process integration for future high-density, non-volatile HBM.
