% sections/hybrid.tex
\section{Hybrid Perspectives and Future Memory Hierarchies}

Hybrid memory hierarchies aim to combine DRAM performance with FeRAM persistence. By placing FeRAM near the controller or adjacent to DRAM, systems can reduce refresh energy, enable instant-on features, and accelerate checkpointing and recovery for critical state.

% --- Fig.5: Hybrid hierarchy schematic (TikZ actual figure) ---
\begin{figure}[!t]
  \centering
  \begin{tikzpicture}[
    font=\footnotesize,
    box/.style={rounded corners, draw, minimum width=38mm, minimum height=9mm, align=center},
    >={Latex},
    node distance=6mm
  ]
    % Blocks
    \node[box, fill=gray!18] (cpu)  {CPU Registers / Cache};
    \node[box, fill=blue!12,  below=of cpu]  (dram) {DRAM\\(high-speed working set)};
    \node[box, fill=green!12, below=of dram] (feram){FeRAM / FeFET\\(near-memory, persistent)};
    \node[box, fill=yellow!18,below=of feram] (ssd)  {SSD / HDD\\(block storage)};

    % Vertical data path
    \draw[->] (cpu)  -- (dram);
    \draw[->] (dram) -- (feram);
    \draw[->] (feram) -- (ssd);

    % Side/bypass flows
    \draw[dashed,->] (cpu.east)  .. controls +(1,0.3)  and +(1,0.3)  .. (feram.east);
    \draw[dashed,->] (dram.east) .. controls +(1,0.3)  and +(1,-0.3) .. (ssd.east);

    % Notes
    \node[align=left, anchor=west] at ($(dram.east)+(1.2,0)$)
      {Refresh reduction\\Hot/Cold tiering};
    \node[align=left, anchor=west] at ($(feram.east)+(1.2,0)$)
      {Instant-on \& checkpoints\\Wear/retention mgmt.};
  \end{tikzpicture}
  \caption{Hybrid memory hierarchy where FeRAM near DRAM provides persistence and reduces refresh pressure.}
  \label{fig:hybrid_hierarchy}
\end{figure}

\subsection*{Benefits}
\begin{itemize}
  \item Reduced refresh overhead: cold pages and metadata can reside in FeRAM, cutting DRAM refresh traffic and standby power.
  \item Fast persistence: OS and application state can be checkpointed to FeRAM with microsecond-scale latency.
  \item Data resilience: FeRAM provides crash consistency for critical metadata and write-back buffers.
\end{itemize}

\subsection*{Constraints and trade-offs}
\begin{itemize}
  \item Endurance and variability: FeRAM endurance ($10^{12}$--$10^{13}$ cycles) is high but below effective DRAM activity levels.
  \item Write energy and latency: typically higher than DRAM; placement should bias read-mostly or cold data to FeRAM.
  \item Integration cost: ferroelectric layer/FeFET adoption adds process and reliability risks (e.g., high-field stress).
\end{itemize}

\subsection*{System directions}
\begin{itemize}
  \item Tiering policies using intensity/retention-aware placement and migration.
  \item Refresh co-optimization: shrink DRAM refresh for regions shadowed or backed by FeRAM.
  \item Controller/OS support for wear tracking, retention-aware placement, and error telemetry.
\end{itemize}
