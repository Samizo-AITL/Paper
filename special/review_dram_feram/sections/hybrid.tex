% sections/hybrid.tex
\section{Hybrid Perspectives and Future Memory Hierarchies}

Hybrid memory hierarchies aim to combine DRAM performance with FeRAM persistence. By placing FeRAM near the controller or adjacent to DRAM, systems can reduce refresh energy, enable instant-on features, and accelerate checkpointing and recovery for critical state.

% --- Fig.5: Hybrid hierarchy schematic (placeholder) ---
\begin{figure}[!t]
  \centering
  \fbox{\rule{0pt}{50mm}\rule{0.95\linewidth}{0pt}}
  \caption{Conceptual hybrid hierarchy integrating DRAM (working set) and FeRAM (persistent/near-memory).}
  \label{fig:hybrid_hierarchy}
\end{figure}

\subsection*{Benefits}
\begin{itemize}
  \item Reduced refresh overhead: cold pages and metadata can reside in FeRAM, cutting DRAM refresh traffic and standby power.
  \item Fast persistence: OS and application state can be checkpointed to FeRAM with microsecond-scale latency.
  \item Data resilience: FeRAM provides crash consistency for critical metadata and write-back buffers.
\end{itemize}

\subsection*{Constraints and trade-offs}
\begin{itemize}
  \item Endurance and variability: FeRAM endurance ($10^{12}$--$10^{13}$ cycles) is high but below effective DRAM activity levels.
  \item Write energy and latency: typically higher than DRAM; placement should bias read-mostly or cold data to FeRAM.
  \item Integration cost: ferroelectric layer/FeFET adoption adds process and reliability risks (e.g., high-field stress).
\end{itemize}

\subsection*{System directions}
\begin{itemize}
  \item Tiering policies using intensity/retention-aware placement and migration.
  \item Refresh co-optimization: shrink DRAM refresh for regions shadowed or backed by FeRAM.
  \item Controller/OS support for wear tracking, retention-aware placement, and error telemetry.
\end{itemize}
