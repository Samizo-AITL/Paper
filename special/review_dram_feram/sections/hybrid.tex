% sections/hybrid.tex  ← NOTE: セクション見出しは main.tex にのみ置く

% --- 本文 ---
Hybrid memory hierarchies combine the high capacity and speed of DRAM with the
non-volatility and instant-resume capability of FeRAM (including FeFET variants).
Placing FeRAM near the memory controller or as chiplets alongside DRAM can
reduce refresh energy, enable instant-on, and support fast checkpointing and recovery.

% --- Fig.3: Hybrid hierarchy schematic (SystemDK-aware, mono-friendly) ---
\begin{figure}[!t]
  \centering
  \begin{tikzpicture}[font=\footnotesize, >=Stealth]
    % styles
    \tikzset{
      layer/.style={rounded corners, draw=black, fill=black!6,
                    minimum width=7.6cm, minimum height=6mm},
      arrow/.style={->, thick}
    }

    % y positions
    \def\yssd{-0.6}
    \def\yferam{0.5}
    \def\ydram{1.6}
    \def\ycpu{2.7}
    \def\ysdk{-1.6}

    % boxes
    \node[layer] (cpu)  at (0,\ycpu)  {CPU Registers / Cache (volatile)};
    \node[layer] (dram) at (0,\ydram) {DRAM \textit{(high-speed, large capacity)}};
    \node[layer] (fram) at (0,\yferam){FeRAM / FeFET \textit{(non-volatile, instant resume)}};
    \node[layer] (ssd)  at (0,\yssd)  {SSD / HDD \textit{(block storage)}};
    \node[layer, minimum height=8mm, fill=black!10] (sdk) at (0,\ysdk)
      {\textbf{SystemDK}: Top-down co-design across chiplets, controllers, and OS};

    % vertical arrows (memory hierarchy)
    \draw[arrow] (cpu.south)  -- (dram.north);
    \draw[arrow] (dram.south) -- (fram.north);
    \draw[arrow] (fram.south) -- (ssd.north);

    % SystemDK arrows routed on the right side
    \path[arrow] (sdk.east) to[out=60,in=-60] (cpu.east);
    \path[arrow] (sdk.east) to[out=45,in=-45] (dram.east);
    \path[arrow] (sdk.east) to[out=30,in=-30] (fram.east);
    \path[arrow] (sdk.east) to[out=15,in=-15] (ssd.east);

  \end{tikzpicture}
  \caption{Hybrid memory hierarchy. DRAM provides high-speed capacity, while FeRAM
  supplies persistent, instant-resume storage close to the controller. SystemDK connects
  to all levels, enabling holistic optimization across chiplets, controllers, and OS.}
  \label{fig:hybrid_hierarchy}
\end{figure}

\subsection*{Benefits}
\begin{itemize}
  \item \textbf{Reduced refresh overhead}: Cold data/metadata reside in FeRAM, lowering DRAM refresh traffic and standby power.
  \item \textbf{Fast persistence}: OS/app state checkpointed to FeRAM with $\mu$s--ms latency.
  \item \textbf{Data resilience}: Crash-consistent metadata and write-back buffers.
\end{itemize}

\subsection*{Constraints and trade-offs}
\begin{itemize}
  \item \textbf{Endurance/variability}: FeRAM endurance ($10^{12}$--$10^{13}$) is high but below DRAM activity.
  \item \textbf{Energy/latency}: Writes costlier than DRAM; favor read-mostly/cold data for FeRAM.
  \item \textbf{Integration cost}: Ferroelectric layers/FeFET add process and reliability risks (e.g., high-field stress).
\end{itemize}

\subsection*{System-level directions}
\begin{itemize}
  \item \textbf{Tiering policies}: Intensity/retention-aware placement and migration.
  \item \textbf{Refresh co-optimization}: Shrink DRAM refresh for regions shadowed/backed by FeRAM.
  \item \textbf{Controller/OS support}: Wear tracking, retention-aware placement, error telemetry.
  \item \textbf{SystemDK co-design}: Holistic optimization from chiplets to OS in one flow.
\end{itemize}
