Memory hierarchies are central to modern computing systems. DRAM remains the dominant volatile memory owing to its speed, density, and scalability \cite{choi2022,kim2021_dram}. However, as cell capacitors shrink and aspect ratios rise, scaling faces physical limits and refresh overheads \cite{kim2021_dram,iedm2023_dram}.

In parallel, ferroelectric HfO2 based memories (FeRAM and FeFET) have re-emerged as promising non-volatile options with fast CMOS-compatible switching \cite{boscke2011,mueller2012,noheda2023}. This review synthesizes device trends and compares DRAM and FeRAM at the metric level, then highlights hybrid uses that may reshape the volatile--non-volatile boundary. Historically, DRAM traces to the 1T1C concept by Dennard \cite{dennard1966}, while modern ferroelectric memory concepts were surveyed by Scott \cite{scott1998}.
